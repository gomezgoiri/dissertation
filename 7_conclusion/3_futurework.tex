\section{Future Work}

% Limitations => Future Work
% ¿Poner lista de la compra o sólo las cosas que se destilen de las limitaciones del trabajo?

Through this dissertation, we have described several limitations and open issues.
They open the door to interesting future work which is detailed below.


\subsection{Close the Gap between our Middleware and third Applications}

One of the goals of the middleware is to achieve interoperability with third applications.
In other words, to reuse their data transparently to the user and enabling them to reuse our middleware's data.
This goal can be seen as a two-fold strategy: ease the access to the data and make them reusable.

\begin{description}
  \item[Access to the data.] It must be done by means of an \ac{api}.
			    In this thesis we have defended the need of using \ac{http} for that purpose.
			    An adaptation for the upcoming \ac{coap} would increase the potential device base. % "base de usuarios"
			    
			    However, the most challenging aspect is not to replicate it in other appropriate protocols.
			    The most challenging aspect is to automatize the consumption of heterogeneous third \acp{api}.
			    In this thesis we have ignored it on behalf of a common \ac{api} that needs to be implemented in any device willing to share content. % TODO data api vs domain api => mirar el capitulo 3 y explicar esto
			    According to the \ac{rest} principles, the \ac{resthateoas} constraint can help automated the clients to use any \ac{api} independently of its specific shape. % shape or implementation
			    
			    Nevertheless, an agent needs knowledge about the domain explored to reuse any \ac{api}.
			    And to reuse many \acp{api} without reimplementing the agent, the data types and states of a domain should be standardized. % poner cita del blog de fielding sobre la discussion esa
			    Therefore, any step in that direction would requires some degree standardization of at least for the \ac{wot} domain.
			    
			    The counter-pad of that would be to ensure that our \ac{tsc} \ac{api} complies with the \ac{resthateoas} constraint.
			    Some steps have been done to make the \ac{api} human discoverable with its \ac{html} representation.
			    However, more effort needs to be done to ease its automatic consumption regardless of its specific implementation. % which would depend on the version
			    % se pueden usar varias estrategias en paralelo
			    
			    Finally, \ac{resthateoas} can result in more network usage, which may not be desirable for devices with energy limitations.
			    Therefore, this trade-off between the interoperability and the network efficiency should be also assessed.
			    
  \item[Reusing the data.] The semantic knowledge contributes to that goal.
			  However, as \citet{barnaghi_semantics_2012} point out, the use of semantic descriptions alone does not ensure interoperability.
			  Additionally, they suggest the need of:
			  (1) standardizing ontologies and using them (or at least when using own ontologies, referencing an upper ontology); and
			  (2) interpreting the annotations effectively. % efectivamente: con razonadores que funcionen bien
			  While the first one requires the consensus of the community, the second demands for efficient reasoners for mobile and embedded platforms.
			  This efficiency is particularly important since it directly influences their energy consumption.
			  % TODO Mencionar que estamos trabajando en ello?
			  % further assessment of existing ones, creating news for platforms
			  % pero decir que lo nuestro es un primer paso en paralelo a otros que puedan solucionar los otros problemas
			  % semantica y cacharros pequeños, sigue siendo algo difícil. Necesidad de razonadores ligeros.
			  
\end{description}


\subsection{Increase Searching Expressibility and Efficiency}

% Searching: SPARQL y demás
% 1ero: mayor eficiencia
% reescritura puede mejorar interop
We did not consider using query languages such as SPARQL because of the difficulties constrained platforms may find in parsing them.
With this decision, we design a uniform \ac{api} for all the types of devices.
However, for the future work, one can adopt SPARQL in an optional \ac{api}.
To do that, all the nodes should be aware that some of them are only able to interpret triple pattern templates.

The use of SPARQL in some nodes would benefit the developers and improve the middleware performance. % searching
The users of the middleware would be able to made more expressive queries.
The performance could be enhanced by introducing optimization techniques from the distributed databases field \ac{schwarte_fedx_optimization_2011}. % see references from Section~\ref{background}) or reference them here?


\subsection{Coordination Space's Distribution, Replication or Migration} % no se me ocurre un título más sexy

The current design demands the coordination space to be accessible through a \ac{http} \ac{api}.
For the sake of clarity, at some points of the dissertation we have assumed that this space is centralized in a unique machine.
However, the access through a \ac{http} \ac{api} does not restrict that.
Some alternatives that could worth to explore are: % mencionar cloud para algo?
\begin{itemize}
  \item Replication of the space to increase the system performance (e.g. load balancing).
  \item Distribution of the content over different machines to enhance its scalability. % TODO decir que ya ha sido tratado por más soluciones? (e.g. TripCom)
  \item Migration from a machine to another to avoid a single point of failure.
	%No node is essential so the system will still work.
        This will allow, for example, to select the space holder dynamically easing the manageability and deployment.
        This would be similar to the \ac{wp} role of our searching architecture. % mencionar Chapter~\ref{cha:searching}?
\end{itemize}

% aquella idea que tenían los de Lancaster de que el middleware cambia de forma dependiendo del entorno
Besides, monitoring the middleware usage, one alternative or another can be dynamically chosen depending on the specific needs of each situation.


\subsection{Further work on the integration of the Proof-based Actuation Mechanism in the \Space{}}

In chapter~\ref{cha:actuate} we presented an alignment between a proof-based mechanism and actuation through the space.
However, its implementation and evaluation is left as future work.
To enable the usage of the proof-based mechanism, tasks from the latter must be translated into goals.
This translation must be generalizable to automatize it.
However, it is not clear whether it will always be possible this direct translation for complex goals and tasks.


Another interesting issue is to discern which path to follow to achieve a goal when two or more paths are available.
We believe that additional high-level rules or policies can be defined (e.g. the path which consumed less energy).
Storing them into the space can enable reusing this \emph{behaviors} by third applications' developers.


Finally, instead of reusing third \ac{wot} applications' actuation capacities in \ac{tsc}, the opposite question could be explored:
How to reuse actuation mechanisms of nodes using \ac{ts} patterns from \ac{wot} solutions?


\subsection{Actuation on different plans}

Let us imagine that at the end of the previous section more than a path is found for the same desired goal.
For instance, the light of a room can be increased to a certain level by turning on a lamp or by raising the blinds.
In that case, different behaviors could be applied to select the preferred path.
For example, one will want to save energy in the long term by raising the blinds.
On the other hand, another application will decide to reach the light level no matter of the outside conditions.

Furthermore, these behaviors might be reused among the applications. % además, seguimos con la idea de reutilizar
For that, we envision to share the behaviors in the space itself. % se infiere que semánticamente anotados
The behaviors could be defined by the facility administrators or the developers.
After that, the developers will be able to consume the behaviors through another middleware abstraction.
% Besides, it will monitor the state to check if the state of the space is effectively changed or other actions need to be taken.


\subsection{Security for the Middleware}

% si no, es un poco fiesta y cualquiera podría poner datos falsos
Two key aspect for \ac{ubicomp} deployments are
(1) to authenticate the parts involved in the communication; and
(2) to ensure the privacy of the message being transmitted.
% Ya hemos hecho algo para dispositivos limitados y móviles
Although both problems have been deeply discussed in the literature,
the nodes in an \ac{ubicomp} environment face severe computing and energy restrictions.
To solve that, we have already propose a solution appropriate for sensor networks and mobile devices.

The next step is to evaluate it in more demanding scenarios.
The final goal is to integrate it in our \ac{tsc} middleware.


%%% A partir de aquí, no son aspectos técnicos concretos


\subsection{Measuring Ease of Usage}

A subtle goal of any middleware is to ease the development of applications by encapsulating complex tasks (i.e. searching on a distributed space).
% Medir de forma objetiva aspectos del ser humano como la sencillez de uso o de implementación
Some objective indicators can be extracted from the code by implementing the same application with and without the middleware (i.e. number of lines).
However, from the human perspective this indicators do not necessarily express how easy, comfortable or pleasant was to use the middleware.
% facilitar modelado?? => el hecho de que sea difícil anotar, puede echar para atrás a desarrolladores: eso es problema de mi app o de quien?
Although the primitives are rather simple, the configuration or just the semantic annotation can complicate the learning process of any developer.
Future work could measure this impact on them.
As a result, some improvements could be proposed (e.g. to create a version for a specific domain which masks all the semantic issues).


% Al final, a modo de colofón
\subsection{Further Evaluation on Real Deployments}

% muchas de nuestras decisiones en la implementación del middleware fueron guiadas por la realidad
Rather than designing the best theoretical solution, in this dissertation we have tried to present a realizable solution.
For that, we have implemented most of the ideas in the \emph{Otsopack} middleware and we have promote its use in different research projects.
This has helped us to identify the key problems to solve.


Sometimes, to efficiently evaluate our solution under very different and extreme circumstances, we have simulated some scenarios. % con muchos cacharros
In this simulations we have tried to present always realistic assumptions.
% de simulaciones => a la realidad
However, there is always a gap between what is perceived as realistic and the reality.


To bridge both worlds, the next natural step is assessing some aspects of our middleware in a real world environment: % real world es académico???
\begin{itemize}
	% Big ecosystem of interconnected objects and devices
  \item The search architecture.
        For that, we should deploy an scenario with a number of heterogeneous devices running a task in a long term.
        % viendo cuales son las necesidades reales de quienes lo usan
	%    para justificar los parametros que se usaron
        Then, we could measure their network usage and energy consumption to either tune up some parameters of the architecture or redesign some aspects.
        % unbiased: alguien que no sea yo
        The difficulty of this deployment would be not to biased it design and enable it to really solve a problem.
  % relacionado con el punto anterior (interoperabilidad): reusar distintas applicaciones existentes y ver que realmente se cumple
  %    una forma de asegurarse, es haciendo que distintos usuarios las creen
  \item The interoperability.
        Trying to reuse data from already working \ac{wot} applications, trying to minimize manual development.
\end{itemize}


% Meter aquí la cita de Feynman???
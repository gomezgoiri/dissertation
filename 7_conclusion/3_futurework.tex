\section{Future Work}

% Limitations => Future Work
% ¿Poner lista de la compra o sólo las cosas que se destilen de las limitaciones del trabajo?

Through this dissertation, we have described several limitations and open issues.
They open the door to interesting future work which is detailed below.


\subsection{Close the Gap between our Middleware and third Applications}

One of the goals of the middleware is to achieve interoperability with third applications.
In other words, to reuse their data transparently to the user and enabling them to reuse our middleware's data.
This goal can be seen as a two-fold strategy: ease the access to the data and make them reusable.

\begin{description}
  \item[Access to the data.] It must be done by means of an \ac{api}.
			    In this thesis we have defended the need of using \ac{http} for that purpose.
			    An adaptation for the upcoming \ac{coap} would increase the potential device base. % "base de usuarios"
			    
			    However, the most challenging aspect is not to replicate it in other appropriate protocols.
			    The most challenging aspect is to automatize the consumption of heterogeneous third \acp{api}.
			    In this thesis we have ignored it on behalf of a common \ac{api} that needs to be implemented in any device willing to share content. % TODO data api vs domain api => mirar el capitulo 3 y explicar esto
			    According to the \ac{rest} principles, the \ac{rest_hateoas} constraint can help automated the clients to use any \ac{api} independently of its specific shape. % shape or implementation
			    
			    Nevertheless, an agent needs knowledge about the domain explored to reuse any \ac{api}.
			    And to reuse many \acp{api} without reimplementing the agent, the data types and states of a domain should be standardized. % poner cita del blog de fielding sobre la discussion esa
			    Therefore, any step in that direction would requires some degree standardization of at least for the \ac{wot} domain.
			    
			    The counter-pad of that would be to ensure that our \ac{tsc} \ac{api} complies with the \ac{rest_hateoas} constraint.
			    Some steps have been done to make the \ac{api} human discoverable with its \ac{html} representation.
			    However, more effort needs to be done to ease its automatic consumption regardless of its specific implementation. % which would depend on the version
			    % se pueden usar varias estrategias en paralelo
			    
			    Finally, \ac{rest_hateoas} can result in more network usage, which may not be desirable for devices with energy limitations.
			    Therefore, this trade-off between the interoperability and the network efficiency should be also assessed.
			    
  \item[Reusing the data.] The semantic knowledge contributes to that goal.
			  However, the use of semantic descriptions alone does not ensure interoperability.
			  \citeauthor{barnaghi_semantics_2012} points out two problems:
			  (1) the need of standardizing ontologies, or at least when using own ontologies, reference an upper ontology; and
			  (2) effectively interpreting the annotations.
			  While the first one requires the consensus of the community, the second demands for efficient reasoners for mobile and embedded platforms.
			  This efficiency is particularly important since it directly influences their energy consumption.
			  % TODO Mencionar que estamos trabajando en ello?
			  % further assessment of existing ones, creating news for platforms
			  % pero decir que lo nuestro es un primer paso en paralelo a otros que puedan solucionar los otros problemas
			  % semantica y cacharros pequeños, sigue siendo algo difícil. Necesidad de razonadores ligeros.
			  
\end{description}


\subsection{Increase Searching Expressibility and Efficiency}
% Searching: SPARQL y demás
% 1ero: mayor eficiencia
% reescritura puede mejorar interop


\subsection{Migration of the Coordination Space} % To avoid central point of failure
% cloud?
% balanceo de carga
% distribuído en distintos lados...


\subsection{Actuation: implement and test the coexistence of both mechanisms}

\subsection{Actuation on different plans}


\subsection{Security Mechanisms}
% ya hemos hecho algo para dispositivos limitados y móviles
% ahora falta integrarlo en el middleware, si no, es un poco fiesta y cualquiera podría poner datos falsos



%%% A partir de aquí, no son aspectos técnicos concretos


\subsection{Medir de forma objetiva aspectos del ser humano como la sencillez de uso o de implementación}

% a subtle goal of any middleware
% unmesured


% facilitar modelado?? => el hecho de que sea difícil anotar, puede echar para atrás a desarrolladores



% Al final, a modo de colofón
\subsection{Further Evaluation on Real Deployments}

% muchas de nuestras decisiones en la implementación del middleware fueron guiadas por la realidad
Rather than designing the best theoretical solution, in this dissertation we have tried to present a realizable solution.
For that, we have implemented most of the ideas in the \emph{Otsopack} middleware and we have promote its use in different research projects.
This has helped us to identify the key problems to solve.


Sometimes, to efficiently evaluate our solution under very different and extreme circumstances, we have simulated some scenarios. % con muchos cacharros
In this simulations we have tried to present always realistic assumptions.
% de simulaciones => a la realidad
However, there is always a gap between what is perceived as realistic and the reality.


To bridge both worlds, the next natural step is assessing some aspects of our middleware in a real world environment: % real world es académico???
\begin{itemize}
	% Big ecosystem of interconnected objects and devices
  \item The search architecture.
        For that, we should deploy an scenario with a number of heterogeneous devices running a task in a long term.
        % viendo cuales son las necesidades reales de quienes lo usan
	%    para justificar los parametros que se usaron
        Then, we could measure their network usage and energy consumption to either tune up some parameters of the architecture or redesign some aspects.
        % unbiased: alguien que no sea yo
        The difficulty of this deployment would be not to biased it design and enable it to really solve a problem.
  % relacionado con el punto anterior (interoperabilidad): reusar distintas applicaciones existentes y ver que realmente se cumple
  %    una forma de asegurarse, es haciendo que distintos usuarios las creen
  \item The interoperability.
        Trying to reuse data from already working \ac{wot} applications, trying to minimize manual development.
\end{itemize}


% Meter aquí la cita de Feynman???
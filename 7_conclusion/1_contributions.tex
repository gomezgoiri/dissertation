\section{Contributions}

% De una charla con Buján recuerdo que dijo que para O. Corcho...
%   + Contribuciones científicas: lo que nadie había hecho antes (innovación)
%   + Contribuciones técnicas: implementar o aplicar en otro campo un aspecto innovado por otro


This section summarizes the contributions of this dissertation.
\begin{itemize}
  \item Chapter~\ref{cha:stateoftheart} presented an in-depth state-of-the-art review.
    \begin{itemize}
      %\item It presents the research fields to which the dissertation contributes. % Background
      \item It portrayed this dissertation regarding the relevant works from related research fields.
	    Particularly, it intensified the analysis on the semantic space-based computing middleware. % deepen, intensify
	    This analysis showed that other works do not address the problems tackled in this dissertation.
    \end{itemize}
  
  \item Chapter~\ref{cha:tsc} depicted a two-space model which
	\begin{enumerate*}[label=\itshape(\arabic*\upshape)]
	  \item preserves \ac{tsc}'s decoupling properties and
	  \item respects \ac{ubicomp}'s distributed nature.
	\end{enumerate*}
    % designs a middleware
    \begin{itemize}
      %\item It scrutinizes the \ac{rest} style and its compatibility with \ac{tsc}.
      \item It described a dual space model. % compromise between
            The first space is a common place where all the participants can write and read.
            The second space is a \emph{read-only} virtual \Space{} formed by contents managed by the different participants.
            It is accessed by any node as a whole using the same primitives as the first one.
      \item It designed the middleware by presenting the adopted primitives and how the spaces can be accessed through an \ac{http} \ac{api}.
	    %Doing so, we retain some of \ac{rest}'s properties and enable interoperability with other elements.
      \item It analysed the properties of the designed middleware from different perspectives.
	    Particularly, it scrutinized which beneficial properties from \ac{tsc} or \ac{rest} are retained.
	    Besides, it remarked the middleware's suitability for resource constrained devices.
    \end{itemize}
  
  \item Chapter~\ref{cha:searching} presented a search architecture for semantic \ac{wot} solutions.
        This architecture is compatible with, but it is not only applicable to, the space model proposed.
    \begin{itemize}
      \item It presented a search architecture that:
      \begin{itemize}
	\item Balances the load that the architecture management generates.
	      This balance considers devices' computing and energy capacities.
	\item Structures the nodes in different dynamic roles with different responsibilities. % roles: WP, Provider y Consumer, dynamic: ppalmente WP
	\item Promotes end-to-end \ac{http} requests between constrained devices.
	      They do not need an intermediary to search (i.e. to select the appropriate providers to request). % sé que esto es tricky, pero no lo necesita necesariamente
      \end{itemize}
      \item It assessed different types of information summaries the nodes can use to improve their search.
      \item It evaluated the proposal with a simulated but yet realistic environment.
    \end{itemize}
    
  \item Chapter~\ref{cha:actuate} analysed how to actuate on the physical environment using a space. % uno direct y otro indirect
	Particularly, it showed how to interoperate with existing \ac{rest}-based actuators from space-based computing.
    \begin{itemize}
      \item It explained how to use \ac{tsc} to coordinate actuator nodes and nodes willing to actuate.
      \item It depicted the need of a subscription system and its key requirements.
      \item It compared the space-based actuation technique with a direct actuation one based on the \ac{rest} architectural style. % the requirements and benefits
      \item It contributed with a hybrid approach which seamlessly reuses \ac{rest} \acp{api} from our space model.
    \end{itemize}
  
  % contribuciones técnicas: implementaciones varias
  \item Besides, as a result of the theoretical work made on this dissertation, we have done the following technical contributions:
    \begin{itemize}
      \item \emph{Otsopack} \citeweb{otsopack}: a \ac{tsc} middleware which works over \ac{http} and implements most of the ideas presented in the dissertation.
            This middleware is publicly available for different computing platforms and has been used in several research projects.
      % Esto deberiamos incluirlo aunque sea brevemente en algun lugar...
      % \item Exploration of the use of semantic content in several embedded platforms.
      %  We have tested several platforms, semantic frameworks and reasoners to verify the feasibility of the ideas and assumptions presented in the dissertation.
      \item A fully parametrizable simulation framework to evaluate different communication strategies \citeweb{simulation}.
      \item Three different implementations of the same simple \ac{ubicomp} scenario which illustrate the ideas explained in Chapter~\ref{cha:actuate}.
	    Much of these implementations are fully reusable for further future evaluations \citeweb{actuation}.
    \end{itemize}
\end{itemize}


% el tema de haber trabajado con otros
Finally, parallel to this dissertation, but deeply influenced by it, the author of this dissertation has collaborated with other academic institutions, research centres and companies to make further scientific and technical contributions.
\begin{itemize}
  \item Lightweight user access control for resource constrained devices. % que puede ser adaptada a esto
  \item Apply the \emph{Otsopack} middleware in several scenarios:
    \begin{itemize}
      \item Homes: automating them according to the user's preferences. % y lo del podometro?
                                                                        % en los escenarios de distintos papers
                                                                        % lo que se hizo en TALIS+Engine en Alicante o no se donde
      \item Hospitals and residences: tracking the evolution of patients with cognitive impairments.  % TODO traducir centro de día mejor que hospital
      \item Supermarkets: aiding to buy and easing mobility using robots.
      \item Hotels: intelligently adapting them to the customers' needs.
    \end{itemize}
  % Poner CLIPS4Android???
\end{itemize}
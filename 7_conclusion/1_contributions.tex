\section{Contributions}

% De una charla con Buján recuerdo que dijo que para O. Corcho...
%   + Contribuciones científicas: lo que nadie había hecho antes (innovación)
%   + Contribuciones técnicas: implementar o aplicar en otro campo un aspecto innovado por otro


This section presents a summary of the contributions explained in this dissertation.
\begin{itemize}
  \item Chapter~\ref{cha:stateoftheart} presents an in-depth state-of-the-art review.
    \begin{itemize}
      %\item It presents the research fields to which the dissertation contributes. % Background
      \item It portrays this dissertation regarding the relevant works from related research fields.
	    Particularly, it intensifies the analysis on the semantic space-based computing middleware. % deepen, intensify
	    The analysis shows how they do not address the problems tacked by this dissertation.
    \end{itemize}
  
  \item Chapter~\ref{cha:tsc} depicts a two-space model which
	\begin{enumerate*}[label=\itshape(\arabic*\upshape)]
	  \item preserves \ac{tsc}'s decoupling properties and
	  \item respects \ac{ubicomp}'s distributed nature.
	\end{enumerate*}
    % designs a middleware
    \begin{itemize}
      %\item It scrutinizes the \ac{rest} style and its compatibility with \ac{tsc}.
      \item It describes a dual space model. % compromise between
            The first space is a commonplace where all the participants can write and read.
            The second space is a \emph{read-only} virtual \Space{} formed by contents managed by the different participants.
            It is accessed by any node as a whole using the same primitives as the first one.
      \item It designs the middleware by presenting the adopted primitives and how the spaces can be accessed through an \ac{http} \acp{api}.
	    %Doing so, we retain some of \ac{rest}'s properties and enable interoperability with other elements.
      \item It analyses the properties of the designed middleware from different perspectives.
	    Particularly, it scrutinizes which beneficial properties from \ac{tsc} or \ac{rest} are retained.
	    Besides, it remarks the middleware's suitability for resource constrained devices.
    \end{itemize}
  
  \item Chapter~\ref{cha:searching} presents a searching architecture for semantic \ac{wot} solutions.
        This architecture suits into, but it is not only applicable to, the space model proposed.
    \begin{itemize}
      \item It balances the load that the architecture management generates.
	    This balance considers devices' computing and energy capacities.
      \item It structures the nodes in different dynamic roles with different responsibilities. % roles: WP, Provider y Consumer, dynamic: ppalmente WP
      \item It promotes end-to-end \ac{http} requests between constrained devices.
            They do not need an intermediary to search (i.e. to select the appropriate providers to request). % sé que esto es tricky, pero no lo necesita necesariamente
      \item It presents and assesses different types of information summaries the nodes can use to improve their search.
      \item It evaluates the proposal with a simulated but yet realistic environment.
    \end{itemize}
    
  \item Chapter~\ref{cha:actuate} analyses and compares techniques to actuate over the physical environment using a space. % uno direct y otro indirect
    \begin{itemize}
      \item It explains how to use \ac{tsc} to coordinate actuator nodes and nodes willing to actuate.
      \item It depicts the need of a subscription system and its key requirements.
      \item It compares the space-based actuation technique with a direct actuation one based on the \ac{rest} architectural style. % the requirements and benefits
      \item It presents and discusses an alternative to seamlessly reuse \ac{rest} services from our space model.
    \end{itemize}
  
  % contribuciones técnicas: implementaciones varias
  \item Besides, as a result of the theoretical work made on this dissertation, we have done the following technical contributions:
    \begin{itemize}
      \item \emph{Otsopack} \citeweb{otsopack}: a \ac{tsc} middleware which works over \ac{http} and implements most of the ideas presented in the dissertation.
            This middleware is publicly available for different computing platforms and has been used in several research projects.
      % Esto deberiamos incluirlo aunque sea brevemente en algun lugar...
      % \item Exploration of the use of semantic content in several embedded platforms.
      %  We have tested several platforms, semantic frameworks and reasoners to verify the feasibility of the ideas and assumptions presented in the dissertation.
      \item A fully parametrizable simulation framework to evaluate different communication strategies \citeweb{simulation}.
      \item Three different implementations of the same simple \ac{ubicomp} scenario which illustrate the ideas explained in the Chapter~\ref{cha:actuate}.
	    Much of these implementations are fully reusable for further future evaluations \citeweb{actuation}.
    \end{itemize}
\end{itemize}


% el tema de haber trabajado con otros
Finally, parallel to this dissertation, but deeply influenced by it, the author of this dissertation has collaborated with other academic institutions, research centres and companies to make the further scientific and technical contributions.
\begin{itemize}
  \item Lightweight user access control for resource constrained devices. % que puede ser adaptada a esto
  \item Apply the \emph{Otsopack} middleware in several scenarios:
    \begin{itemize}
      \item Homes: automating them according to the user's preferences. % y lo del podometro?
                                                                        % en los escenarios de distintos papers
                                                                        % lo que se hizo en TALIS+Engine en Alicante o no se donde
      \item Hospitals and residences: tracking the evolution of patients with cognitive impairments.  % TODO traducir centro de día mejor que hospital
      \item Supermarkets: aiding to buy and easing mobility using robots.
      \item Hotels: intelligently adapting it to the customers' needs.
    \end{itemize}
  % Poner CLIPS4Android???
\end{itemize}
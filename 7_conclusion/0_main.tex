
% this file is called up by thesis.tex
% content in this file will be fed into the main document

%: ----------------------- introduction file header -----------------------
\begin{savequote}[50mm]
I tell you what, I speak slow so those of you with a Phd can understand.
\qauthor{Doug \textit{(Déjà Vu)}}
\end{savequote}


\chapter{Conclusions}
\label{cha:conclusions}
\newcommand{\pathchapseven}{7_conclusion}


% the code below specifies where the figures are stored
\ifpdf
    \graphicspath{{\pathchapseven/figures/PNG/}{\pathchapseven/figures/PDF/}{\pathchapseven/figures/}}
\else
    \graphicspath{{\pathchapseven/figures/EPS/}{\pathchapseven/figures/}}
\fi


%------------------------------------------------------------------------- 

% Presentar de qué ha ido la tesis
% 3 ptos con los que quiero que el lector se quede: searching, acting y compatibility with WoT

This dissertation explores the design of a completely distributed \ac{tsc} middleware.
In this middleware, each node is responsible of its own information, but together with other nodes they create a virtual space. % virtual, concept
Writing and reading from this space, the nodes communicate in a much more decoupled indirect style.

The distribution of the information avoids the dependency of the participants on other devices, but arises other challenges.
We have tackled two main challenges: how to search in a decentralized way and how to act on the environment through the space.
Besides, after an analysis we acknowledged the importance and benefits that the upcoming \ac{wot} has brought to the \acl{ubicomp}.
This has lead us to have make our final solution compatible with the \ac{wot} and other REST services.

To enhance the search process between devices limiting their dependencies in others we designed an architecture.
This architecture dynamically chooses an intermediary according to its capabilities to support the most limited devices.
The final result is that less requests are required in overall, specifically reducing the load on resource constrained devices.

% TODO refinar cuando esté hecha esa parte
We also presented two different acting mechanisms: X and Y.
Comparing them we reach to the conclusion that X proposes a less decoupled design, which in turn depends on a subscription mechanism.
This subscription mechanism is particularly challenging on dynamic environments.
The Y solution is more straightforward and more based in REST services consumption.
In contrast, it depends on reasoners and on some additional information about the services offered by the rest of the nodes.

Besides, we have guided our design to build the middleware over a REST interface.
Using that interface, each node exposes the semantic resources, i.e. RDF triples grouped in RDF graphs, it manages.
This enables the consumption of these contents by other semantic \ac{wot} solutions or REST services, even if they are not part of our middleware.

\section{Contributions}

% De una charla con Buján recuerdo que dijo que para O. Corcho...
%   + Contribuciones científicas: lo que nadie había hecho antes (innovación)
%   + Contribuciones técnicas: implementar o aplicar en otro campo un aspecto innovado por otro

% Contributions (por si necesito ideas)
% \item Analysis of strengths and weaknesses of both \ac{wot} and \ac{tsc} and comparison of both approaches.
% \item Design and implementation of a RESTful interface which makes possible to build a \ac{tsc} on top of it. % TODO revisar y poner todos los peros que haga falta para no traicionar la idea de Fielding
%       This resource-oriented interface enables the access to the the semantic knowledge provided by each device.
%       By doing so, we keep our middleware compliant with the \ac{wot} idea. % o integrable con otros servicios, o...
%       % es como explicar una dualidad: puedes usar nuestro middleware usando las primitivas de TSC, pero también puedes usarlo simplemente porque te viene bien para WoT semántico
% \item A search-aware architecture which promotes the end-to-end communication between devices. % and balances their load by administrative tasks according to their capacities.
%       This architecture tries to minimize the requests devices have to handle by enhancing their search mechanism.
%       The extra-tasks introduced by this enhancement are performed by nodes chosen according to their capacities.
% \item Design and comparison of two approaches which enable to act on the environment by writing in the fully distributed space.
% \item Exploration of the use of semantic content in several current embedded platforms.
%       We have tested several platforms, semantic frameworks and reasoners to verify the feasibility of the ideas presented in the dissertation.
% \item Design and implementation of most of the ideas presented in a middleware publicly available called \emph{Otsopack}. % al primo no le gustará, pero insisto
%       This middleware has been used in several projects for a range of scenarios.


\section{Relevant Publications}

% International JCR Journals
% International conferences
% Workshops % o si queda muy corto, incluirlo directamente en conclusions
\section{Future Work}

% Limitations => Future Work
\section{Final Remarks}



% ----------------------------------------------------------------------

% this file is called up by thesis.tex
% content in this file will be fed into the main document

%: ----------------------- introduction file header -----------------------
\begin{savequote}[50mm]
I tell you what, I speak slow so those of you with a Phd can understand.
\qauthor{Doug \textit{(Déjà Vu)}}
\end{savequote}


\chapter{Conclusions}
\label{cha:conclusions}
\newcommand{\pathchapseven}{7_conclusion}


% the code below specifies where the figures are stored
\ifpdf
    \graphicspath{{\pathchapseven/figures/PNG/}{\pathchapseven/figures/PDF/}{\pathchapseven/figures/}}
\else
    \graphicspath{{\pathchapseven/figures/EPS/}{\pathchapseven/figures/}}
\fi


%------------------------------------------------------------------------- 

% Presentar de qué ha ido la tesis
% 3 ptos con los que quiero que el lector se quede: searching, acting y compatibility with WoT

This dissertation explores the design of a completely distributed \ac{tsc} middleware.
In this middleware, each node is responsible of its own information, but together with other nodes they create a virtual space. % virtual, concept
Writing and reading from this space, the nodes communicate in a much more decoupled indirect style.

The distribution of the information avoids the dependency of the participants on other devices, but arises other challenges.
We have tackled two main challenges: how to search in a decentralized way and how to act on the environment through the space.
Besides, after an analysis we acknowledged the importance and benefits that the upcoming \ac{wot} has brought to the \acl{ubicomp}.
This has lead us to have make our final solution compatible with the \ac{wot} and other REST services.

To enhance the search process between devices limiting their dependencies in others we designed an architecture.
This architecture dynamically chooses an intermediary according to its capabilities to support the most limited devices.
The final result is that less requests are required in overall, specifically reducing the load on resource constrained devices.

% TODO refinar cuando esté hecha esa parte
We also presented two different acting mechanisms: X and Y.
Comparing them we reach to the conclusion that X proposes a less decoupled design, which in turn depends on a subscription mechanism.
This subscription mechanism is particularly challenging on dynamic environments.
The Y solution is more straightforward and more based in REST services consumption.
In contrast, it depends on reasoners and on some additional information about the services offered by the rest of the nodes.

Besides, we have guided our design to build the middleware over a REST interface.
Using that interface, each node exposes the semantic resources, i.e. RDF triples grouped in RDF graphs, it manages.
This enables the consumption of these contents by other semantic \ac{wot} solutions or REST services, even if they are not part of our middleware.

\input{\pathchapfive/2_contributions}
\input{\pathchapfive/3_publications} % o si queda muy corto, incluirlo directamente en conclusions
\section{Future Work}

% Limitations => Future Work
\input{\pathchapfive/5_finalremarks}


% ----------------------------------------------------------------------
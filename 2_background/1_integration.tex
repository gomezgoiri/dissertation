\section{Application Integration}
\label{sec:soa_integration}

% Definiciones de middleware en el libro de Coulouris:
%
% (pag 17):
% Middleware • The term middleware applies to a software layer that provides a
% programming abstraction as well as masking the heterogeneity of the underlying
% networks, hardware, operating systems and programming languages.
%
% (en otra parte):
% The task of middleware is to provide a higher-level
% programming abstraction for the development of distributed systems and, through
% layering, to abstract over heterogeneity in the underlying infrastructure to promote
% interoperability and portability.


The integration of two applications is driven by how they communicate.
To ease this communication the applications use middlewares.
A middleware is a software layer which provides a higher level of abstraction and masks the underlying heterogeneity.
\citet{coulouris_distributed_2012} define two communication styles on the upper layer of a middleware: % figura 4.1 modificada para incluir elementos
the remote invocation and the indirect communication (see Figure~\ref{fig:middleware_layers}).

\InsertFig{middleware_layers}{fig:middleware_layers}{Middleware layers}{Middleware layers according to \citet{coulouris_distributed_2012} classification.}{1}{}
% According to Coulouris et al. HTTP is An example of a request-reply protocol!


\medskip

The \emph{remote invocation} involves the most common two-way exchange between senders and receivers in distributed systems.
Among others, it comprehends request-reply protocols, remote procedure calls and remote method invocation.
Request-reply protocols are the simplest and most lightweight mechanisms for client-server computing. % lo dicen Coulouris, no lo digo yo ;-)
Within these protocols \ac{http} shines as one of the web pillars.


\ac{http} can be used as the baseline to design other \emph{remote invocation} paradigms (e.g. the WS-* \citep{alonso_web_2010} standards).
However, its design is intended to support the web, whose modern architecture follows the \ac{rest} architectural style \citep{fielding_architectural_2000}.
% me gustaba más decir sólo "de las propiedades de REST", porque REST no se cumple sólo
Consequently, one can see the pervasiveness of the web applications as a proof of \ac{rest}'s success.

\medskip


The \emph{indirect communication} style comprehends all the techniques with no direct coupling between the sender and the receiver.
% Otra forma:
% In contrast, the \emph{indirect communication} comprehends decoupled communications between senders and receivers.
The group communication, publish-subscribe systems, message queues or shared memory approaches are examples of indirect communication.
These paradigms are characterized by two key properties \citep{gelernter_generative_1985,coulouris_distributed_2012}
\footnote{
  We use the terminology of the  \aclp{ts}' seminal paper \citep{gelernter_generative_1985}. % alternatives to terminology: nomenclature or naming
  However, note that
  % Gelernter et al. stated that \emph{distributed sharing} was just a consequence of the these properties.
  the \emph{space uncoupling} property is referred as \emph{reference autonomy} by some authors \citep{fensel_triple-space_2004}. % no sólo Fensel, también los del STI Innsbruck
  % El primero fue un tal Angerer en el 2002, pero en un artículo en alemán.
  % En el 2003 hay un artículo suyo en Internet, pero no sé si mencionarlo.
  These same authors mention a third property confusingly called \emph{space autonomy} (or \emph{location autonomy}).
  According to \citet{fensel_triple-space_2004} this autonomy is achieved because:
  \begin{emph} % quote lo indexa
  "The processes can run in completely different computational environments as long as both can access the same space".
  \end{emph}
}
:

\begin{itemize}
 \item \emph{Space uncoupling}, which is achieved when the sender does not need to know the receiver or receivers and vice versa.
 \item \emph{Time uncoupling}, which happens when senders and receivers do not need to exist in the same time\footnote{
	  Although some authors \citep{fensel_triple-space_2004,krummenacher_www_2005} explain this property just in terms of communication asynchrony,
	  % mencionar a otros? o no porque simplemente siguen lo dicho por Fensel?
	  % a Bundler no lo cito, porque era una master thesis "sólo"
	  % en este caso cito a krummenacher porque en esa publicación lo define directamente como el no uso de comunicación sincrona
	  \citet{coulouris_distributed_2012} make a clear distinction between them.
	  In their words, a communication is asynchronous when ``\emph{a sender sends a message and then continues without blocking}'',
	  whereas time uncoupling adds an extra dimension: ``\emph{the sender and the receiver can have independent existences}''.
	  }.
\end{itemize}



This dissertation delves into a particular shared memory approach: \acl{ts} computing. % whose benefits etc. are described
However, as mentioned, \ac{rest} architectures' properties have made them massively accepted to integrate applications.
Consequently, we also take into consideration the latter mechanism in our solution conception.
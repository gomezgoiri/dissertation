% ----------------------------------------------------------------------

\begin{savequote}[50mm]
The beginning of wisdom is to desire it.
\qauthor{Solomon Ibn Gabirol}
\end{savequote}


\chapter{Background}
\label{cha:stateoftheart}

% the code below specifies where the figures are stored
\ifpdf
    \graphicspath{{\pathchaptwo/figures/PNG/}{\pathchaptwo/figures/PDF/}{\pathchaptwo/figures/}}
\else
    \graphicspath{{\pathchaptwo/figures/EPS/}{\pathchaptwo/figures/}}
\fi


%------------------------------------------------------------------------- 

% Megafrase de las que no gustan al primo, pero a mí me gusta su pomposidad ;-)
In order to clarify the foundations in which our work relies and to provide a starting point for understanding the rest of the thesis,
this chapter introduces, categorizes, and describes the related research topics. % o llamarle research areas, important concepts?
This topics are classified attending to their relation with a critical aspect for \ac{ubicomp}: interoperability.
% TODO valorar meter en otra sección descripción de IoT y UbiComp y sus características?

% Otra forma de ver la interop:
%  La European Telecommunication Standards Institute (ETSI) define cuatro capas:
%      technical interop: a nivel de comunicación (p.e. de acuerdo en las 7 capas de OSI)
%      syntactic interop
%      semantic interop
%      organizational interop


% no habría que definir de nuevo y de manera formal ubicomp o vale con la intro?
The IEEE defines \emph{interoperability} as ``\emph{the ability of two or more systems or components to exchange information and to use the information that has been exchanged}'' \citep{_ieee_1990}.
The heterogeneity of technologies present in \ac{ubicomp} environments makes this a key property to consider.
The definition clearly distinguishes between two requirements: % o goals o incremental requirements
\textbf{(1) to exchange} information; and
\textbf{(2) to use} that information. % to understand others data (i.e. \emph{interoperation}).

\medskip

% Buscar una referencia mejor: http://en.wikipedia.org/wiki/Interoperability
Exchanging information in distributed systems comprehends the communication between two systems.
% lo de ab-initio practicamente sólo lo he visto en la wikipedia
For the lower communication levels, we rely on standard and widely accepted communication protocols (i.e. interoperability \emph{ab-initio}). % e.g. HTTP
For higher-levels (i.e. application layer), this dissertation delves into the \emph{space-based computing} and \ac{rest} architectures.
Section~\ref{sec:soa_integration} categorizes both of them together with other integration approaches.
Then, we individually present both integration styles in sections \ref{sec:rest} and \ref{sec:tuplespaces_eoa}.

\medskip

% Sintáctica vs semántica
Regarding the second goal, it can be analyzed from two perspectives: syntactically and semantically.
On the one hand, \emph{syntactic interoperability} is associated with the format of the data (i.e. its syntax and encoding) \citep{van_der_veer_achieving_2006}. % e.g. high level: HTML, XML, etc.
On the other hand, \emph{semantic interoperability} is concerned with ensuring that the exchanged information has a precise meaning.
Its ultimate goal is to make the information ``\emph{understandable by any other application that was not initially developed for this purpose}'' \citep{_european_2004}.
Section~\ref{sec:soa_sw} describes a prominent movement which is focused on this goal: the \acl{sw}. % TODO qué es la SW??? un mecanismo, conjunto de estándares, ¿?


\section{Application Integration}
\label{sec:soa_integration}

% Definiciones de middleware en el libro de Coulouris:
%
% (pag 17):
% Middleware • The term middleware applies to a software layer that provides a
% programming abstraction as well as masking the heterogeneity of the underlying
% networks, hardware, operating systems and programming languages.
%
% (en otra parte):
% The task of middleware is to provide a higher-level
% programming abstraction for the development of distributed systems and, through
% layering, to abstract over heterogeneity in the underlying infrastructure to promote
% interoperability and portability.


The integration of two applications is driven by how they communicate.
To ease this communication the applications use middlewares.
A middleware is a software layer which provides a higher level of abstraction and masks the underlying heterogeneity.
\citet{coulouris_distributed_2012} define two communication styles on the upper layer of a middleware: % figura 4.1 modificada para incluir elementos
the remote invocation and the indirect communication (see Figure~\ref{fig:middleware_layers}).

\InsertFig{middleware_layers}{fig:middleware_layers}{Middleware layers}{Middleware layers according to \citet{coulouris_distributed_2012} classification.}{1}{}
% According to Coulouris et al. HTTP is An example of a request-reply protocol!


\medskip

The \emph{remote invocation} involves the most common two-way exchange between senders and receivers in distributed systems.
Among others, it comprehends request-reply protocols, remote procedure calls and remote method invocation.
Request-reply protocols are the simplest and most lightweight mechanisms for client-server computing. % lo dicen Coulouris, no lo digo yo ;-)
Within these protocols \ac{http} shines as one of the web pillars.


\ac{http} can be used as the baseline to design other \emph{remote invocation} paradigms (e.g. the WS-* \citep{alonso_web_2010} standards).
However, its design is intended to support the web, whose modern architecture follows the \ac{rest} architectural style \citep{fielding_architectural_2000}.
% me gustaba más decir sólo "de las propiedades de REST", porque REST no se cumple sólo
Consequently, one can see the pervasiveness of the web applications as a proof of \ac{rest}'s success.

\medskip


The \emph{indirect communication} style comprehends all the techniques with no direct coupling between the sender and the receiver.
% Otra forma:
% In contrast, the \emph{indirect communication} comprehends decoupled communications between senders and receivers.
The group communication, publish-subscribe systems, message queues or shared memory approaches are examples of indirect communication.
These paradigms are characterized by two key properties \citep{gelernter_generative_1985,coulouris_distributed_2012}
\footnote{
  We use the terminology of the  \aclp{ts}' seminal paper \citep{gelernter_generative_1985}. % alternatives to terminology: nomenclature or naming
  However, note that
  % Gelernter et al. stated that \emph{distributed sharing} was just a consequence of the these properties.
  the \emph{space uncoupling} property is referred as \emph{reference autonomy} by some authors \citep{fensel_triple-space_2004}. % no sólo Fensel, también los del STI Innsbruck
  % El primero fue un tal Angerer en el 2002, pero en un artículo en alemán.
  % En el 2003 hay un artículo suyo en Internet, pero no sé si mencionarlo.
  These same authors mention a third property confusingly called \emph{space autonomy} (or \emph{location autonomy}).
  According to \citet{fensel_triple-space_2004} this autonomy is achieved because:
  \begin{emph} % quote lo indexa
  "The processes can run in completely different computational environments as long as both can access the same space".
  \end{emph}
}
:

\begin{itemize}
 \item \emph{Space uncoupling}, which is achieved when the sender does not need to know the receiver or receivers and vice versa.
 \item \emph{Time uncoupling}, which happens when senders and receivers do not need to exist in the same time\footnote{
	  Although some authors \citep{fensel_triple-space_2004,krummenacher_www_2005} explain this property just in terms of communication asynchrony,
	  % mencionar a otros? o no porque simplemente siguen lo dicho por Fensel?
	  % a Bundler no lo cito, porque era una master thesis "sólo"
	  % en este caso cito a krummenacher porque en esa publicación lo define directamente como el no uso de comunicación sincrona
	  \citet{coulouris_distributed_2012} make a clear distinction between them.
	  In their words, a communication is asynchronous when ``\emph{a sender sends a message and then continues without blocking}'',
	  whereas time uncoupling adds an extra dimension: ``\emph{the sender and the receiver can have independent existences}''.
	  }.
\end{itemize}



This dissertation delves into a particular shared memory approach: \acl{ts} computing. % whose benefits etc. are described
However, as mentioned, \ac{rest} architectures' properties have made them massively accepted to integrate applications.
Consequently, we also take into consideration the latter mechanism in our solution conception.
\section{REST actuation}
\label{sec:direct_actuation}


The actuation technique presented in the previous section uses the \Space{} (i.e. it is indirect).
This technique implies that the actuator must be aware of the \Space{}'s content.
For instance, a heater must check the space to find if a new desired temperature was written.
% lo podrá reusar a través de un intermediario
% interop era una de las cosas que queriamos cuidar
% y qué pasa con las soluciones REST que quieren usar lo nuestro?


% por qué? motivación: permitir a otras app integrarse en nuestro espacio
In contrast, this section presents a second technique which directly actuates in the environment without using any \Space{}.
A consumer following this technique directly uses \ac{rest} \acp{api}. % i.e. it is the client of these \ac{api} <-- se entiende esto o hace falta dar el salto entre consumer y client?
These \acp{api} expose the devices' capabilities to make physical changes in the environment.

However, this apparently simple mechanism hides a difficulty: the consumer needs to discern how to use these \acp{api} to make such changes.
According to the \ac{rest} principles, a client should navigate through the resources of an \ac{api} with no prior knowledge of it.
% Copiar esta explicación mejor de algún lado:
The client should 
\begin{enumerate*}[label=\itshape(\arabic*\upshape)]
  \item interpret the representations provided by the server and then
  \item choose the appropriate state transition from the hypertext according to its intention. % y su conocimiento básico del protocolo: CRUD
\end{enumerate*}
To do this, this section advocates for a semantic description of these \acp{api}' resources.

Section~\ref{sec:actuation_rest_background} briefly compares different alternatives to build semantic \ac{rest} \acp{api}.
Section~\ref{sec:restdesc} presents the selected alternative and Section~\ref{sec:actuation_scn2} implements the baseline scenario using it.



\subsection{Background}
\label{sec:actuation_rest_background}

As Section~\ref{sec:network_properties} explained, semantic representations do not include a native way to express the hypertext. % TODO realmente se explica?
To solve this, three solutions can be adopted:
% Unos proponen extender con ontologías
\begin{enumerate}
  \item To use an ontology to represent the hypertext \citep{kjernsmo_necessity_2012}.
  \item To embed the hypertext independently to the representations on the \ac{http} headers \citep{mark_web_2010}.
  \item To provide a description of the state changes each \ac{http} request triggers \citep{verborgh_functional_2012,verborgh_ijcs_2014}.
\end{enumerate}


The latter two enable to discover resources and state transitions without adding metadata to the representations.
This allows not only to describe semantic representations, but any type of formats.


\citeauthor{mark_web_2010}'s \citep{mark_web_2010} approach is extended by \citet{erik_profile_2013} to define how to embed additional semantics to process a resource representation.
\citet{erik_profile_2013} calls these additional semantics \emph{profiles} and identifies them using \acsp{uri}.


\citet{verborgh_ijcs_2014} present a more expressive solution which goes beyond simply describing a resource type.
It also allows to semantically describe the knowledge needed to use a concrete \acs{http} verb on a resource, and the content this request returns. % o precondition
The materialization of this proposal is called \restdesc{} \citep{verborgh_functional_2012}.
\citet{mayer_semantic_2013} use \restdesc{} in an environment populated by web-powered devices. % i.e. the \ac{wot}
This environment is analogous to the ones envisioned by this dissertation. % analogous/equivalent


\bigskip


We consider \restdesc{} the best current solution which helps to achieve truly \ac{rest}ful \acsp{api}.
Therefore, this chapter assumes that the \ac{http} \acsp{api} whose capabilities we want to reuse in our space model describe their \acsp{api} with \restdesc{}.


\subsection{\restdesc{}}
\label{sec:restdesc}

% TODO mirar si tiene cabida la mención de otros enfoques para describir recursos que no son muy RESTful
%Different ways exist to describe \ac{rest} services. % mencionar WADL, etc.?
% citar a donde se hable de \restdesc{} y así ya se empieza a explicar la solución de forma discreta.
\restdesc{} describes \acs{http} methods using rules expressed in the \ac{n3} language \citeweb{n32011}.
% he evitado explicar que las reglas tienen premisa y conclusión, porque me parece demasiado obvio
A rule's \emph{premise} expresses the requirements to invoke a \ac{rest} service.
A rule's \emph{conclusion} expresses both the \ac{rest} call that needs to be made and the description of that invocation result.


\citet{verborgh_functional_2012} suggest three complementary alternatives to discover descriptions.
The first and second alternatives use \ac{http} mechanisms: content negotiation and the \ac{http} OPTIONS verb.
Using content negotiation a client can obtain a dereferenceable link's representation which corresponds to the description. % puede ser comments#algo
Requesting a resource using the \ac{http} OPTIONS verb a client can obtain the description in the response body.
Finally, the third alternative is not \ac{http}-centric and it depends on repositories to store and retrieve the descriptions.
% En el paper los autores no justifican bien porque es last resort, así que paso de meterme yo en berenjenales
% This alternative should only be used as a last resort because it requires out-of-band information (e.g. the repository address and how to use it) to learn how to use an \acs{api}.
% it separates the description of the \acp{api} functionality from the \acp{api} themselves. % dificultando el descubrimiento de las APIs usando sólo ellas


\citet{verborgh_ijcs_2014} propose a service composition mechanism for Web \acp{api} using \restdesc{}.
This mechanism uses as inputs:
\begin{enumerate*}[label=\itshape(\arabic*\upshape)]
  \item an initial state,
  \item a goal state,
  \item Web \ac{api} descriptions using \restdesc{}, and
  \item optional background knowledge.
\end{enumerate*}
Each of these inputs are semantically expressed and therefore, they can be processed by standard \ac{n3} reasoners.
These reasoners generate proofs about how to achieve the goal starting from the initial state and using the rest of the inputs.
These proofs can be seen as steps that need to be made to reach a desired state.


Additionally, \citeauthor{verborgh_ijcs_2014} distinguish between pre-proof and post-proof.
The first, are those which assume that the execution of all \acs{api} calls will behave as expected.
The latter, can be seen as a \emph{revision} of the pre-proof.
It executes the Web \acs{api} of the pre-proofs and uses actual execution results to generate a new proof.


\subsection{Baseline scenario: Implementation 2}
\label{sec:actuation_scn2}
\newcommand{\implRest}{\emph{Implementation 2}}

% poner un diagrama que presente el escenario?
% En ppio no, porque sería muy sencillote, y el flujo del consumidor que es el complejo ya ha sido explicado.
The proposed implementation for the baseline scenario using \restdesc{} presents the following nodes:
\begin{enumerate}[resume,label=\itshape(\Alph*\upshape)]
  \setcounter{enumi}{\theenumNodes}
  \item A node which exposes the lamp and its actuators through a \ac{rest} \ac{api}.
	This \ac{api} is described using \restdesc{}. % REST API seguro 100%? por si acaso decir HTTP API?
	To physically change the light value, any client must send an \acs{http} request to the resource which represents the light actuator.
	% Please note that \nodeProvRest{} is omitted because its operation is the usual for an \acs{http} server.
	
  \item A node which directly communicates with the desired provider.
	To discern which provider's resources has to call and how to do it, this implementation reasons to obtain a plan.
	This plan determines how to fulfil the node's initial goal invoking the needed \ac{http} \acsp{api}.
	Figure~\ref{fig:flow_rest_prov} shows the actions performed by this node in detail.
\end{enumerate}

% In advance, we will refer to these nodes as "C" and "D" using the following commands:
\newcommand{\nodeProvRest}{\emph{Node C}}
\newcommand{\nodeConsRest}{\emph{Node D}}


\InsertFig{flowRESTConsumer}{fig:flow_rest_prov}{Flow chart for the \nodeConsRest{}}{}{0.5}{}


The \acs{http} \acs{api} provided by the \nodeProvRest{} is modelled using the following resources:
\begin{itemize}
  \item \emph{/lamp}: It provides basic information about the lamp.
  \item \emph{/lamp/actuators}: It enumerates the actuators which compose the \emph{smart lamp}.
  \item \emph{/lamp/actuators/light}: It represents the unique actuator which composes the lamp in our simple example (i.e. the lamp's light).
  \item \emph{/lamp/actuators/light/01}: It represents a concrete preference to change the light.
\end{itemize}


% Aquí estoy explicando un poco el diagrama de flujo de la figura,
% dando detalles de cómo se ha implementado y a qué me refiero con cada pieza de información
% ¿Añadir descripciones y demás aquí o ponerlas como anexo?

To instruct consumers on how to use the services provided, they are annotated using \restdesc{}.
The \acs{http} OPTIONS returns listings~\ref{lst:light_descpost}~and~\ref{lst:measure_descget} for \emph{/lamp/actuators/light}.
Thanks to these descriptions and to the dereferenceable \acsp{uri} \citep{sauermann_cool_2008}, starting from \emph{/lamp} any client can crawl the \acs{api} to autonomously learn how to use it.

\begin{listing}
  \expandafter\def\csname PY@tok@err\endcsname{}
{\small
\begin{Verbatim}[commandchars=\\\{\},numbers=left,firstnumber=1,stepnumber=1]
\PY{err}{\PYZob{}}
\PY{n+nc}{  actuators:light }\PY{o}{ssn:madeObservation }\PY{err}{?}\PY{n+na}{l}\PY{n+na}{i}\PY{n+na}{g}\PY{n+na}{h}\PY{n+na}{t}\PY{n+na}{\PYZus{}}\PY{n+na}{o}\PY{n+na}{b}\PY{n+na}{s }.
\PY{err}{\PYZcb{}}\PY{err}{ }\PY{err}{=}\PY{err}{\PYZgt{}}\PY{err}{ }\PY{err}{\PYZob{}}
\PY{n+nc}{  \PYZus{}:request }\PY{o}{http:methodName }\PY{l+s}{\PYZdq{}GET\PYZdq{} };
            \PY{o}{http:requestURI }\PY{err}{?}\PY{n+na}{l}\PY{n+na}{i}\PY{n+na}{g}\PY{n+na}{h}\PY{n+na}{t}\PY{n+na}{\PYZus{}}\PY{n+na}{o}\PY{n+na}{b}\PY{n+na}{s };
            \PY{c}{\PYZsh{}http:body actuators:light ;}
\PY{o}{            http:resp }[\PY{o}{ http:body }\PY{err}{?}\PY{n+na}{l}\PY{n+na}{i}\PY{n+na}{g}\PY{n+na}{h}\PY{n+na}{t}\PY{n+na}{\PYZus{}}\PY{n+na}{o}\PY{n+na}{b}\PY{n+na}{s }].
  
  \PY{err}{?}\PY{n+nc}{light\PYZus{}obs }\PY{o}{a  }\PY{n+na}{s}\PY{n+na}{s}\PY{n+na}{n}\PY{n+na}{:}\PY{n+na}{O}\PY{n+na}{b}\PY{n+na}{s}\PY{n+na}{e}\PY{n+na}{r}\PY{n+na}{v}\PY{n+na}{a}\PY{n+na}{t}\PY{n+na}{i}\PY{n+na}{o}\PY{n+na}{n };
         \PY{o}{ssn:observedProperty  }\PY{n+na}{s}\PY{n+na}{w}\PY{n+na}{e}\PY{n+na}{e}\PY{n+na}{t}\PY{n+na}{:}\PY{n+na}{L}\PY{n+na}{i}\PY{n+na}{g}\PY{n+na}{h}\PY{n+na}{t };
         \PY{o}{ssn:observedBy }\PY{n+na}{a}\PY{n+na}{c}\PY{n+na}{t}\PY{n+na}{u}\PY{n+na}{a}\PY{n+na}{t}\PY{n+na}{o}\PY{n+na}{r}\PY{n+na}{s}\PY{n+na}{:}\PY{n+na}{l}\PY{n+na}{i}\PY{n+na}{g}\PY{n+na}{h}\PY{n+na}{t };
         \PY{o}{ssn:observationResult }\PY{err}{?}\PY{n+na}{s}\PY{n+na}{o }.
     
  \PY{err}{?}\PY{n+nc}{so }\PY{o}{ssn:hasValue }\PY{err}{?}\PY{n+na}{o}\PY{n+na}{v }.

  \PY{err}{?}\PY{n+nc}{ov }\PY{o}{a }\PY{n+na}{s}\PY{n+na}{s}\PY{n+na}{n}\PY{n+na}{:}\PY{n+na}{O}\PY{n+na}{b}\PY{n+na}{s}\PY{n+na}{e}\PY{n+na}{r}\PY{n+na}{v}\PY{n+na}{a}\PY{n+na}{t}\PY{n+na}{i}\PY{n+na}{o}\PY{n+na}{n}\PY{n+na}{V}\PY{n+na}{a}\PY{n+na}{l}\PY{n+na}{u}\PY{n+na}{e };
      \PY{o}{dul:isClassifiedBy  }\PY{n+na}{u}\PY{n+na}{c}\PY{n+na}{u}\PY{n+na}{m}\PY{n+na}{:}\PY{n+na}{l}\PY{n+na}{u}\PY{n+na}{x };
      \PY{o}{dul:hasDataValue }\PY{n+na}{\PYZus{}}\PY{n+na}{:}\PY{n+na}{v}\PY{n+na}{a}\PY{n+na}{l }.
\PY{err}{\PYZcb{}}\PY{err}{.}
\end{Verbatim}
}
  \caption{Rule which expresses that having a light sensor observation, one can obtain details about the observation through an \acs{http} GET.}
  \label{lst:measure_descget}
\end{listing}

\begin{listing}
  \expandafter\def\csname PY@tok@err\endcsname{}
\begin{Verbatim}[commandchars=\\\{\},numbers=left,firstnumber=1,stepnumber=1]
\PY{err}{\PYZob{}}
\PY{c}{  \PYZsh{} it is not \PYZsq{}just a measure\PYZsq{}}
\PY{err}{ }\PY{err}{ }\PY{err}{?}\PY{n+nc}{obsv }\PY{o}{a }\PY{n+na}{s}\PY{n+na}{s}\PY{n+na}{n}\PY{n+na}{:}\PY{n+na}{O}\PY{n+na}{b}\PY{n+na}{s}\PY{n+na}{e}\PY{n+na}{r}\PY{n+na}{v}\PY{n+na}{a}\PY{n+na}{t}\PY{n+na}{i}\PY{n+na}{o}\PY{n+na}{n}\PY{n+na}{V}\PY{n+na}{a}\PY{n+na}{l}\PY{n+na}{u}\PY{n+na}{e };
      \PY{c}{\PYZsh{} it is also a preference}
\PY{o}{      a }\PY{n+na}{f}\PY{n+na}{r}\PY{n+na}{a}\PY{n+na}{p}\PY{n+na}{:}\PY{n+na}{P}\PY{n+na}{r}\PY{n+na}{e}\PY{n+na}{f}\PY{n+na}{e}\PY{n+na}{r}\PY{n+na}{e}\PY{n+na}{n}\PY{n+na}{c}\PY{n+na}{e };
      \PY{o}{dul:isClassifiedBy  }\PY{n+na}{u}\PY{n+na}{c}\PY{n+na}{u}\PY{n+na}{m}\PY{n+na}{:}\PY{n+na}{l}\PY{n+na}{u}\PY{n+na}{x };
      \PY{o}{dul:hasDataValue }\PY{err}{?}\PY{n+na}{d}\PY{n+na}{e}\PY{n+na}{s}\PY{n+na}{i}\PY{n+na}{r}\PY{n+na}{e}\PY{n+na}{d}\PY{n+na}{\PYZus{}}\PY{n+na}{v}\PY{n+na}{a}\PY{n+na}{l}\PY{n+na}{u}\PY{n+na}{e }.
\PY{err}{\PYZcb{}}\PY{err}{ }\PY{err}{=}\PY{err}{\PYZgt{}}\PY{err}{ }\PY{err}{\PYZob{}}
\PY{n+nc}{  \PYZus{}:request }\PY{o}{http:methodName }\PY{l+s}{\PYZdq{}POST\PYZdq{}};
            \PY{o}{http:requestURI }\PY{n+na}{a}\PY{n+na}{c}\PY{n+na}{t}\PY{n+na}{u}\PY{n+na}{a}\PY{n+na}{t}\PY{n+na}{o}\PY{n+na}{r}\PY{n+na}{s}\PY{n+na}{:}\PY{n+na}{l}\PY{n+na}{i}\PY{n+na}{g}\PY{n+na}{h}\PY{n+na}{t };
            \PY{o}{http:body }\PY{err}{?}\PY{n+na}{d}\PY{n+na}{e}\PY{n+na}{s}\PY{n+na}{i}\PY{n+na}{r}\PY{n+na}{e}\PY{n+na}{d}\PY{n+na}{\PYZus{}}\PY{n+na}{v}\PY{n+na}{a}\PY{n+na}{l}\PY{n+na}{u}\PY{n+na}{e };
            \PY{o}{http:resp }[\PY{o}{ http:body }\PY{err}{?}\PY{n+na}{l}\PY{n+na}{i}\PY{n+na}{g}\PY{n+na}{h}\PY{n+na}{t}\PY{n+na}{O}\PY{n+na}{b}\PY{n+na}{s }].
  
  \PY{n+nc}{actuators:light }\PY{o}{ssn:madeObservation }\PY{err}{?}\PY{n+na}{l}\PY{n+na}{i}\PY{n+na}{g}\PY{n+na}{h}\PY{n+na}{t}\PY{n+na}{O}\PY{n+na}{b}\PY{n+na}{s }.
  
  \PY{err}{?}\PY{n+nc}{lightObs }\PY{o}{a }\PY{n+na}{s}\PY{n+na}{s}\PY{n+na}{n}\PY{n+na}{:}\PY{n+na}{O}\PY{n+na}{b}\PY{n+na}{s}\PY{n+na}{e}\PY{n+na}{r}\PY{n+na}{v}\PY{n+na}{a}\PY{n+na}{t}\PY{n+na}{i}\PY{n+na}{o}\PY{n+na}{n };
         \PY{o}{ssn:observedProperty }\PY{n+na}{s}\PY{n+na}{w}\PY{n+na}{e}\PY{n+na}{e}\PY{n+na}{t}\PY{n+na}{:}\PY{n+na}{L}\PY{n+na}{i}\PY{n+na}{g}\PY{n+na}{h}\PY{n+na}{t };
         \PY{o}{ssn:observedBy }\PY{n+na}{a}\PY{n+na}{c}\PY{n+na}{t}\PY{n+na}{u}\PY{n+na}{a}\PY{n+na}{t}\PY{n+na}{o}\PY{n+na}{r}\PY{n+na}{s}\PY{n+na}{:}\PY{n+na}{l}\PY{n+na}{i}\PY{n+na}{g}\PY{n+na}{h}\PY{n+na}{t };
         \PY{o}{ssn:observationResult }\PY{err}{?}\PY{n+na}{s}\PY{n+na}{o }.
     
  \PY{err}{?}\PY{n+nc}{so }\PY{o}{ssn:hasValue }\PY{err}{?}\PY{n+na}{o}\PY{n+na}{v }.

  \PY{err}{?}\PY{n+nc}{ov }\PY{o}{a }\PY{n+na}{s}\PY{n+na}{s}\PY{n+na}{n}\PY{n+na}{:}\PY{n+na}{O}\PY{n+na}{b}\PY{n+na}{s}\PY{n+na}{e}\PY{n+na}{r}\PY{n+na}{v}\PY{n+na}{a}\PY{n+na}{t}\PY{n+na}{i}\PY{n+na}{o}\PY{n+na}{n}\PY{n+na}{V}\PY{n+na}{a}\PY{n+na}{l}\PY{n+na}{u}\PY{n+na}{e };
      \PY{o}{dul:isClassifiedBy }\PY{n+na}{u}\PY{n+na}{c}\PY{n+na}{u}\PY{n+na}{m}\PY{n+na}{:}\PY{n+na}{l}\PY{n+na}{u}\PY{n+na}{x };
      \PY{o}{dul:hasDataValue }\PY{err}{?}\PY{n+na}{d}\PY{n+na}{e}\PY{n+na}{s}\PY{n+na}{i}\PY{n+na}{r}\PY{n+na}{e}\PY{n+na}{d}\PY{n+na}{\PYZus{}}\PY{n+na}{v}\PY{n+na}{a}\PY{n+na}{l}\PY{n+na}{u}\PY{n+na}{e }.
\PY{err}{\PYZcb{}}\PY{err}{.}
\end{Verbatim}

  \caption{Rule which expresses that having a preference which is measured in luxes, one can create a light observation using the \acs{http} POST.}
  \label{lst:light_descpost}
\end{listing}


In addition to the crawled content, the \nodeConsRest{} provides two extra pieces of information to the reasoner: a preference and a goal (see listings~\ref{lst:additional_information} and~\ref{lst:light_goal}).
The preference allows the consumer to express the interest in changing a resource, which may not always be feasible.
The goal drives the reasoning process, which tries to extract a plan to achieve it.

\begin{listing}
  \expandafter\def\csname PY@tok@err\endcsname{}
\begin{Verbatim}[commandchars=\\\{\},numbers=left,firstnumber=1,stepnumber=1]
\PY{k}{@prefix }\PY{n+nv}{frap:  }\PY{n+nn}{\PYZlt{}http://purl.org/frap/\PYZgt{} .}
\PY{k}{@prefix }\PY{n+nv}{dul:  }\PY{n+nn}{\PYZlt{}http://www.loa.istc.cnr.it/ontologies/DUL.owl\PYZsh{}\PYZgt{} .}
\PY{k}{@prefix }\PY{n+nv}{ssn:  }\PY{n+nn}{\PYZlt{}http://www.w3.org/2005/Incubator/ssn/ssnx/ssn\PYZsh{}\PYZgt{} .}
\PY{k}{@prefix }\PY{n+nv}{ucum:  }\PY{n+nn}{\PYZlt{}http://purl.oclc.org/NET/muo/ucum/\PYZgt{} .}
\PY{k}{@prefix }\PY{n+nv}{: }\PY{n+nn}{\PYZlt{}http://example.org/lamp/\PYZgt{}.}


\PY{c}{\PYZsh{} Description of the preference}
\PY{n+nc}{:obsv }\PY{o}{a }\PY{n+na}{s}\PY{n+na}{s}\PY{n+na}{n}\PY{n+na}{:}\PY{n+na}{O}\PY{n+na}{b}\PY{n+na}{s}\PY{n+na}{e}\PY{n+na}{r}\PY{n+na}{v}\PY{n+na}{a}\PY{n+na}{t}\PY{n+na}{i}\PY{n+na}{o}\PY{n+na}{n}\PY{n+na}{V}\PY{n+na}{a}\PY{n+na}{l}\PY{n+na}{u}\PY{n+na}{e}\PY{err}{,}\PY{n+na}{ f}\PY{n+na}{r}\PY{n+na}{a}\PY{n+na}{p}\PY{n+na}{:}\PY{n+na}{P}\PY{n+na}{r}\PY{n+na}{e}\PY{n+na}{f}\PY{n+na}{e}\PY{n+na}{r}\PY{n+na}{e}\PY{n+na}{n}\PY{n+na}{c}\PY{n+na}{e };
      \PY{o}{dul:isClassifiedBy  }\PY{n+na}{u}\PY{n+na}{c}\PY{n+na}{u}\PY{n+na}{m}\PY{n+na}{:}\PY{n+na}{l}\PY{n+na}{u}\PY{n+na}{x };
      \PY{o}{dul:hasDataValue }\PY{n+na}{1}\PY{n+na}{9 }. 
\end{Verbatim}

  \caption{A preference which expresses the interest in modifying the sensed value of a light.}
  \label{lst:additional_information}
\end{listing}

\begin{listing}
  \expandafter\def\csname PY@tok@err\endcsname{}
{\small
\begin{Verbatim}[commandchars=\\\{\},numbers=left,firstnumber=1,stepnumber=1]
\PY{p}{\PYZob{}}
  \PY{c}{\PYZsh{} More things could be specified.}
  \PY{c}{\PYZsh{} E.g. location.}
  
  \PY{n+nc}{actuators:light} \PY{n+nf}{ssn:madeObservation} \PY{n+nv}{?light} \PY{p}{.}
  
  \PY{n+nv}{?light} \PY{n+nf}{ssn:observedProperty}  \PY{n+na}{sweet:Light} \PY{p}{;}
         \PY{n+nf}{ssn:observationResult} \PY{n+nv}{?so} \PY{p}{.}
  
  \PY{n+nv}{?so} \PY{n+nf}{ssn:hasValue} \PY{n+nv}{?ov} \PY{p}{.}
  
  \PY{n+nv}{?ov} \PY{o}{a} \PY{n+na}{ssn:ObservationValue} \PY{p}{;}
      \PY{n+nf}{dul:isClassifiedBy}  \PY{n+na}{ucum:lux} \PY{p}{;}
      \PY{n+nf}{dul:hasDataValue} \PY{l+m+mi}{19} \PY{p}{.}
\PY{p}{\PYZcb{}} \PY{o}{=\PYZgt{}} \PY{p}{\PYZob{}}
  \PY{n+nv}{?ov}  \PY{n+nf}{dul:hasDataValue}  \PY{n+nv}{?val} \PY{p}{.}
\PY{p}{\PYZcb{}}\PY{p}{.}
\end{Verbatim}
}
  \caption{A goal which expresses the interest in modifying the value for a light.}
  \label{lst:light_goal}
\end{listing}

With this plan, the consumer just needs to call to the different \acs{http} resources that it defines.
If more than a resource needs to be called, the plan may also indicate how to use the information obtained from one to use it in the next call.
\section{\acl{ts}}
\label{sec:tuplespaces_eoa}

% Que un middleware TS sea time-uncoupled depende de la implementación
% por otro lado, asíncronismo != time uncoupling

\acf{ts} computing, also called space-based computing, offers an improvement over traditional distributed shared memory approaches.
% Colouris se refiere a ellos como distributed shared memory antes de presentar TS
Whereas the latters work at byte-level and accessing to memory addresses,
the \acl{ts} works with semi-structured data which is accessed in an associative manner.
In other words, in \ac{ts} the participants read data specifying patterns of interest.


\ac{ts} has its roots in the Linda parallel programming language \citep{gelernter_generative_1985}.
In this communication model different processes read and write pieces of information so-called tuples in a common space.
Tuples are composed by one or more typed data fields (e.g. $<"aitor",1984>$ or $<3,7,21.0>$).
The tuples are accessed associatively using a template.
A template provides either a value or type for different fields (e.g. $<String,1984>$ or $<Integer, Integer, Float>$).
The operations over the space are defined by different primitives.
Although the primitives may change from implementation to implementation, the most common ones allow to:

\begin{itemize}
  \item \emph{Access the tuples non-destructively}, using a primitive which is usually called \emph{read} or \emph{rd}.
	This primitive returns a tuple from the space which matches the given template without changing the space.
  \item \emph{Access the tuples destructively}, using a primitive which is usually called \emph{take} or \emph{in}.
	This primitive extracts a tuple which matches the given template from the space.
  \item \emph{Write a new tuple into the space}. This primitive is usually called \emph{write} or \emph{out}.
\end{itemize}
\section{The \acl{sw}}
\label{sec:soa_sw}

A problem of the initial view of the Web was that it was human-centred.
Regardless of whether the contents were machine processable, a human needed to interpret them to give them a meaning.
% mirar a ver cuando se empezó a usar \ac{sw} de verdad, porque en el artículo de 2001 dan a enterder que se llevaba un tiempo usando
In the early 2000s, \citet{berners-lee_semantic_2001} proposed a solution to this problem: the \acf{sw}.
%The \ac{sw} was conceived as an extension of the ordinary web which would bring structure to its content.
%As the ordinary web, the \ac{sw} benefits from the universality the hypertext provides by linking \emph{anything with anything}.
%Besides, like the Internet, the \ac{sw} is intended to be as decentralized as possible.
The \emph{World Wide Web Consortium} defines the \ac{sw} and its key features as follows \citep{semanticWeb-FAQ}:
\begin{quote}
The vision of the \acl{sw} is to extend principles of the Web from documents to data.
Data should be accessed using the general Web architecture using, e.g., \acs{uri}-s;
data should be \emph{related to one another} just as documents (or portions of documents) are already.
This also means creation of a common framework that allows data to be \emph{shared and reused} across application, enterprise, and community boundaries,
to be \emph{processed automatically} by tools as well as manually, including revealing possible \emph{new relationships} among pieces of data.
\end{quote}


In the \ac{sw}, triples are the most basic information units.
A triple is composed by a subject, a predicate and an object like in a normal sentence (see Figure~\ref{fig:triples_example}).
As \citeauthor{berners-lee_semantic_2001} explained, using triples ``\emph{a document can assert that things (people, web pages or whatever) have properties (such as 'is a sister of', 'is the author of') with certain values (another person, another Web page)}''.
A key difference with a normal sentence is that each concept is unambiguously defined by an \acs{uri}.
These \acsp{uri} form links between different triples as shown in Figure~\ref{fig:triples_example}.
% This set of triples can be expressed in multiple formats such as RDF, Turtle, Notation3 or NTriples. % TODO citas


\begin{savenotes}
\InsertFig{triples_example}{fig:triples_example}{
  Sample triples using different ontologies.
}{
  All the triples are represented graphically and some of them also textually.
  They describe academic and personal details about the author of this thesis.
  The knowledge is expressed using four different ontologies: FOAF \citeweb{foaf}, DC \citeweb{dc}, SWRC \citeweb{swrc} and CiTO \citeweb{cito}.
  For the sake of clarity, some of the \acs{uri}s in the figure are shortened using aliases or prefixes. % una o la otra
}{1}{}
\end{savenotes}


A problem with the information described so far is that two different databases may use different \acs{uri}s to express the same concept.
To overcome this the \ac{sw} offers collections of information called ontologies.
An ontology is a document which expresses the relations between terms commonly using a taxonomy and a set of rules to infer content.
The taxonomy defines the classes of a given domain and how they relate with each other.

% W3C on Ontology vs Vocabulary: http://www.w3.org/standards/semanticweb/ontology
% There is no clear division between what is referred to as “vocabularies” and “ontologies”.
% The trend is to use the word “ontology” for more complex, and possibly quite formal collection of terms,
% whereas “vocabulary” is used when such strict formalism is not necessarily used or only in a very loose sense.

Of course, different content providers may provide similar data described according to different ontologies.
To overcome this problem, ontologies can be easily mapped by providing equivalence relations within them.
Similarly, an ontology can be extended to adapt it to different application domains.
In any case, the reuse of the same models is beneficial to ease the interoperability.
This reuse is promoted by the community through the standardization of vocabularies.

% Remarkably, the use of the \ac{sw} has been promoted in the last years with the Linked Open Data (LOD) initiative.
% The LOD are datasets which follow a series of principles on how to open and publish data.
% The ultimate goal of the LOD is to publish linked terms using full semantics.


% ----------------------------------------------------------------------


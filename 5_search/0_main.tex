% ----------------------------------------------------------------------

\begin{savequote}[50mm]
I woke up one morning thinking about wolves and realized that wolf packs function as families.
Everyone has a role, and if you act within the parameters of your role, the whole pack succeeds, and when that falls apart, so does the pack.
\qauthor{Jodi Picoult}
\end{savequote}


\chapter{Searching in a Distributed Space}
% ó Energy-aware Architecture for Information Search in the Semantic WoT
\label{cha:searching}


\newcommand{\Space}{\emph{Space}} % en mayúsculas porque \space ya existe
\newcommand{\Spaces}{\emph{Spaces}} % en mayúsculas porque \spaces ya existe
\newcommand{\consumer}{\emph{Consumer}}
\newcommand{\consumers}{\emph{Consumers}}
\newcommand{\provider}{\emph{Provider}}
\newcommand{\providers}{\emph{Providers}}
\newcommand{\clue}{\emph{clue}}
\newcommand{\clues}{\emph{clues}}


% the code below specifies where the figures are stored
\ifpdf
    \graphicspath{{\pathchapfive/figures/PNG/}{\pathchapfive/figures/PDF/}{\pathchapfive/figures/}}
\else
    \graphicspath{{\pathchapfive/figures/EPS/}{\pathchapfive/figures/}}
\fi


%------------------------------------------------------------------------- 

% Mini-intro para enmarcar esta sección en el capítulo
In Chapter~\ref{cha:tsc} we presented a \ac{tsc} model for \ac{ubicomp} environments which is composed of two spaces: the \coordspace{} and the \outerspace{}.
The \outerspace{} is formed by the semantic data provided (\selfgraphs{}) by any participating device.
Searching for a graph in that \outerspace{} is not trivial because it is composed by dynamic and unreliable devices.

This chapter presents an architecture to enable searching for semantic content in a decentralized energy efficient manner.
This searching mechanism can be generalized to other \ac{wot} solutions and scenarios.
To empathize this, during the rest of the chapter we avoid explicitly mentioning the \ac{tsc} middleware or related concepts (e.g. \outerspace{}).

% TODO ver si se joden las propiedades de TSC expuestas anteriormente! (e.g. stateless o lo que sea)


This chapter is organized as follows:
Section~\ref{sec:search_intro} gives an overview of the problem addressed.
Section~\ref{background} discusses the related work.
Section~\ref{proposal} presents in detail our energy-aware architecture.
Section~\ref{sec:clues} describes the information that devices exchange to maintain this architecture.
Section~\ref{environment} presents our experimental environment and Section~\ref{results} evaluates our solution.
Finally, Section~\ref{sec:search_summary} states the conclusions of this chapter.


%------------------------------------------------------------------------- 

% comprobar y aplicar los ultimos cambios y ajustes introducidos en el svn de IJWGS!
%\section{Introduction}

% Originally, the Internet was basically composed of a small number of computers that were physically connected to a wired network.
% Over the years, the popularity of the Internet grew and connecting computers became easier and cheaper.
% Thanks to wireless technologies, devices can connect to the Internet without having to be connected to a network.
% 
% Nowadays, everyday objects like cars or washing machines start to be connected to the Internet to exchange information.
% This is what is currently known as the \emph{Internet of Things (IoT)} \citep{atzori_internet_2010}.
% We can organize the IoT into \emph{Spaces} or islands which cover a particular knowledge as proposed by \citet{abdulrazakCGF10}.
% Using Spaces, we can ease the creation of different \emph{Ambient Intelligence (AmI)} environments.
% For example, we can create a Space for our home where we connect our devices to form an AmI environment.
% Thus, the washing machine could plan to finish washing the clothes two minutes before the car gets home.

Integrating mobile or embedded devices is not trivial as they usually communicate using different protocols.
To solve this problem, the \acl{wot} initiative proposes to use well-established web standards to ease their communication.
However, the format of the data they exchange is also multifarious and application domain dependent.
This implies that data will not be meaningful in other domains unless a specialized system converts and reinterprets them.
A way to solve this problem is annotating the data semantically as proposed by the \ac{www} \citep{ChuaG10,kimKC11}.

Adding semantics to the \ac{iot} works well for devices with high computational capacity but it adds too much overhead for most of the devices composing the \ac{iot}.
To reduce this overhead in such devices, part of this computation is usually delegated to intermediaries \citep{honkola_smart-m3_2010}.
This approach reduces the overhead of semantically annotated data but brings other problems.
(1) When devices rely on others to provide information, it is not guarantee that the information accessed will accurately represent the last information available in the data providers (e.g., the sensors).
(2) Once we rely on those intermediaries, it is required that they are available at all time.
Otherwise, the devices would not be able to talk to each other.

We propose a solution where we use intermediaries to release some workload from the less powerful devices, but at the same time, we promote the direct communication between the devices.
In particular, our system uses intermediaries to search where the data is located in the \Space{} and queries the final devices directly.

In our solution, the devices can become or stop being intermediaries dynamically.
To decide which device will be an intermediary, we evaluate the state of a device (i.e., energy and computation capacity).
Using this dynamic architecture, the absence of a particular intermediary does not collapse the system.
In addition, our system is flexible enough to support a wide range of scenarios.

To enable the search for information in this system, intermediaries need to know the information available in the \Space{}.
To do that, they aggregate summaries sent by each device.
We propose alternatives to summarize and to aggregate this information taking into account the payload of the shared information and the accuracy of the search.

We compare our approach against other common approaches.
In particular, we focus on the energy aspect, demonstrating that our approach helps energy constrained devices to face fewer unnecessary requests.
We also evaluate other important metrics like the number of messages a device has to exchange to perform a request, and the accuracy of the search results provided by the intermediaries.
We perform this evaluation under different scenarios to prove the flexibility of our proposal.

In summary, we make the following contributions:
\begin{itemize}
\item We present a new architecture for managing semantics on the \ac{wot} which reduces the load for resource-constrained devices.
\item We propose and evaluate different approaches to aggregate and summarize the semantic information available in the \Space{} and enable semantic search in this architecture.
\item We demonstrate the advantages of our approach compared to other typical searching approaches under different scenarios.
\end{itemize}

The remainder of this chapter is organized as follows:
Section~\ref{background} discusses the related work.
Section~\ref{proposal} presents in detail our energy-aware architecture.
Section~\ref{sec:clues} describes the information that devices exchange to maintain this architecture.
Section~\ref{environment} presents our experimental environment and Section~\ref{results} evaluates our solution.
Finally, Section~\ref{conclusion} states the conclusions of our work.
\section{Background}
\label{background}

Querying over the semantic content provided by independent sources transparently is a problem which often appears in the \acf{lod}.
\citet{gorlitz_federated_2011} classify the possible \ac{lod} infrastructures according to
\begin{enumerate*}[label=\itshape(\arabic*\upshape)]
  \item how they store data,
  \item whether the index used to search is distributed, and
  \item whether data sources cooperate.
\end{enumerate*}
Table~\ref{tab:infrastructure_lod} summarizes the resulting infrastructure types: central repository, federation and \ac{p2p} data management.


\InsertTab{tab:infrastructure_lod}{Infrastructure paradigms according to their characteristics \citep{gorlitz_federated_2011}}{}{
  \begin{tabular}{l|c|c|c|c|}
    \cline{2-5}
    ~ & \multicolumn{2}{c|}{Central Data Storage} & \multicolumn{2}{c|}{Distributed Data Storage} \\
    \hline
    \multicolumn{1}{|l|}{Independent} & \multirow{2}{*}{n/a} & ~ & \multirow{4}{*}{Federation} & \multirow{2}{*}{n/a} \\
    \multicolumn{1}{|l|}{Data Sources} & ~ & Central &~ & ~ \\
    \cline{1-2} \cline{5-5}
    \multicolumn{1}{|l|}{Cooperative} & \multirow{2}{*}{n/a} & Repository & ~ & P2P Data \\
    \multicolumn{1}{|l|}{Data Sources} & ~ & ~ & ~ & Management \\
    \hline
    ~ & Distr. Index & \multicolumn{2}{c|}{Central Index} & Distr. Index \\
    \cline{2-5}
  \end{tabular}
}{htbp}


% Lo nuestro se parece más a LOD de dispositivos que a distributed triple store:
%     grafos semánticos de fuentes de datos independientes
The solutions which distribute the semantic content are federation and \ac{p2p} data management.
While the latter distributes the index to look for content in multiple machines, federation does not.
%The difference among them resides in whether the index to look for content in the different machines is distributed or not consecutively.
%A federated \ac{lod} infrastructure maintains the meta information about data sources in the machine in charge of searching (called \emph{federator}).
%On the other hand, the P2P approach maintains the meta information distributed.
Our solution proposes a federation infrastructure with some particularities:
\begin{itemize}
  % Esto es, los índices están replicados, pero quien los usa siempre los coge de la misma máquina.
  % No se necesita coordinación entre quienes guardan los índices para balancear o obtener algunos índices que no tienen...
  \item There is a unique node in charge of managing the main version of the metadata needed to create indexes.
	However, it is dynamically chosen among all the participants and can change over the time.
	Consumers hold copies of the manager's metadata.
  \item Each consumer builds its own index from its metadata replica.
	This allows consumers not to critically depend on the availability of any other node to search. % no depender => MUCHO
	%In other words, in our solution there are as many \emph{federators} as data consumers. % Ya no puedo poner esto, porque no he explicado qué es un federator
\end{itemize}


\bigskip


% presentamos 3 soluciones: FedX, ARQ y XXX.
FedX \citep{schwarte_fedx_federation2011,schwarte_fedx_optimization_2011}, DARQ \citep{quilitz_querying_2008} and SemWIQ \citep{langegger_semantic_2008} are the most relevant solutions to perform federated queries over independent semantic data sources.
%Both in these solutions and in ours relevant sources are know beforehand (i.e. they use a \emph{top-down} strategy).
%\emph{Bottom-up} strategies discover data sources by following the links on the content provided by other sources during the query process \citep{schwarte_fedx_optimization_2011}.
%Although our solution also uses a \emph{top-down} strategy, we explicitly avoid human intervention to discover new data sources or providers. % a diferencia de FedX, por lo que ví en la demo % o si, en realidad no nos metemos mucho con eso
%Instead, we rely on automatic discovery mechanisms which additionally must provide some additional information about the data providers. % BTW, esto es un problema para interoperabilidad!
% esto está sacado sobre todo del estado del arte de schwarte_fedx_optimization_2011
DARQ and SemWIQ require locally preprocessed metadata about data sources, while Fedex uses on-demand queries together with a caching mechanism. % SPARQL ASK
In that aspect, our solution resembles DARQ and SemWIQ since it also requires this metadata (called \clues{} in our solution).
In fact, we propose two types of \clues{} which are equivalent to the metadata used by DARQ and SemWIQ.
DARQ maintains predicate indexes to find relevant data sources.
SemWIQ maintains a local catalog with the type information of RDF entities provided by data sources. % el tipo de cada sujeto se debe conocer de alguna forma
In contrast with our solution, DARQ and SemWIQ also use statistics about data sources to optimize their queries.


%    por lo tanto se podría implementar sobre lo nuestro lo suyo
%            (y permitir optimizaciones si introducimos SPARQL en algunos nodos)
% no hay ninguna solución global con una arquitectura que tenga en cuenta le energía
% cómo el problema difiere: más fuentes de datos, menos info en cada una y dinamicas
A key difference with the mentioned systems is the context where we have applied our solution.
It is intended to serve in \ac{ubicomp} environments.
This means that there will be many more data sources and with less information to process in each of them.
Furthermore, their nature also differs from the usual \ac{lod} endpoints' one: the mobile and embedded devices which may provide data are less reliable, more dynamic and more restricted in computation and energy.
Therefore, we designed and evaluated our solution considering this nature and restrictions.
To the best of our knowledge, this is the first approximation to the problem within the explained context.
% además, abarcamos más parte del proceso?? => discovery, la replica, etc.


% pros y contras:
Considering the constrained architecture of the data sources, we acknowledge that some of them will not be able to process \acs{sparql} \citeweb{sparql2008}.
Consequently, we require the lowest common denominator: basic triple patterns.
This causes the main weakness of our solution compared to FedX, DARQ and SemWIQ: a less sophisticated querying mechanism.
%This is because all these solutions assume that all data sources can process SPARQL and offer optimization techniques on top of it. % todo cita a SPARQL
%In any case, SPARQL can always be decomposed into triple patterns.
For our future work, we are considering using \acs{sparql} just with the endpoints able to manage it as a halfway solution.
This will allow us to introduce some of the optimization techniques presented by other solutions.


\bigskip


% mencionar que se ha hablado de soluciones IoT que usen semántica anteriormente y que to the best of our knowledge no existe nada parecido
% como se ha visto en TAL y TAL sección no existen soluciones de busqueda de contenidos semánticos distribuido para WoT. (o al menos no para funcionar en cacharrillos)
In sections~\ref{sec:sw_intermediaries} and \ref{sec:sw_providers} we have analyzed other solutions for \ac{ubicomp} which use semantics.
However, none of them have addressed federated distributed search in resource constrained devices.
% meter alguna referencia a Cabilmonte y demás?
% Es más de cómo consultar datos semánticos en streaming a una base de datos relacional...


% Mencionar SPARQL SERVICES y demás ??
% No es transparente, pero permiten hacer queries a distintas fuentes.


% mencionar fokoue con lo de los resumenes ?
% meter enlace con lo de summaries de ABox en la parte correspondiente?






% La acepción de Name resolution cambia depende del contexto!
% http://en.wikipedia.org/wiki/Name_resolution

% Name resolution vs recognition
% http://stackoverflow.com/questions/8589005/difference-between-named-entity-recognition-and-resolution

% Mirar en el punto 6.3 para ver unos pasos de consulta en la LOD que parece que pueden ser interesantes para esto
% http://www.unifr.ch/webnews/content/20/attach/4765.pdf
\section{Energy-aware super-peer architecture} % pensar titulo más sexy
\label{proposal}

% In this section, we discuss the content the clues should contain to make them suitable to the \ac{wot};
% we explain in detail the roles the nodes of our solution can have;
% we overview the deployment issues related to these roles, such as discovery or role transition;
% and we discuss how our solution can adapt multiple scenarios.

% AG: ¿node, peer y/o device? ¿Aclarar qué es un node?
% IG: personalment peer no lo veo, y entre node y device, pues creo que usarlos indistintamente, pero aclarando eso en algun punto
% AG: Crear comandos para asegurarnos de llamar a las cosas igual siempre (e.g. clues, acronimos, etc.)
% IG: comandos? esa es muy de crack xD

% En algún momento de la sección: CoAP vs HTTP
% AG: Igual habría que comentar que asumimos HTTP por su estado de madurez, pero que podría aplicarse a CoAP en cuanto fuese un estandar y con implementaciones que lo implementasen por completo. Esto tendría ventajas a nivel de uso de Multicast para agrupar WP, consumidores, etc.
%     Podríamos valorar su uso más minuciosamente en cada sección, pero igual en una primera versión no le daría mucho bombo.
% IG: influye en algo de la solucion es algo particular del experimental environment?
% AG: En su mayor parte no, pero habría algunas cosas que se simplificarían (e.g. comunicación con un grupo de nodos)

%   + Vinculo rapido con la intro (1 o 2 frases) % igual también con el related work
In a semantic \ac{wot}, nodes are part of a network where they share semantically described information.
%   + Mencionar el uso de espacios
These nodes gather their information in \Spaces{} which are useful to create \ac{ami} environments.

%   + Introducir ejemplo de hotel
For example, a hotel can create its own semantic \Space{} with the semantically described information of its services and the data provided by devices (e.g., sensors) spread around the hotel.
Thus, clients can use their own personal devices (e.g., smartphones) to interact with the devices of the hotel and search useful information.
Using this approach, a client, for example, can check the current swimming pool occupancy rate using his personal mobile phone.

However, not all devices can afford the costs of fully managing the additional overhead of semantics.
In this section, we present our architecture to manage such environments and deal with the limitations of resource-constrained devices.

\subsection{Basic roles}
As we introduced before, devices can be part of one or multiple \Spaces{}.
A device can provide data (\providers{}), consume data (\consumers{}) or both at the same time.

\begin{description}
\item[Providers:]
This is the simplest role which any device in our system can carry out, even the smallest sensor.
\providers{} must manage their own semantic information.
In particular, they must organize triples in RDF graphs.
\item[Consumers:]
This is the role a device must take to get information from the system.
These devices need to be able to use semantics to find the data they require.
\end{description}

Note that a device could hold just one or both roles at the same time or switch between them.
For example, a \provider{} could become a \consumer{} to get data from the system or a \consumer{} could stop asking for data.
% Cambios de roles:
% To Consumer: try to read something from the network. => evidente
% To not-consumer: "t" without querying anything (it does not worth to update the clues if we won't query).


\subsection{Use of Intermediaries: \acl{wp}}
%   + Intermediaries
%   + Necesidad de tener white pages, mención de roles. => subsección N
Tasks involving the use of semantics can be expensive for some devices.
To reduce the load on such resource-constrained devices, we need to use intermediaries to carry out some of these tasks.
% No hacen tareas por ellos, más bien les evitan hacerlas.
%   + Por que queremos que ellos administren su info?
However, we do not want devices to completely rely on intermediaries for three reasons:
(1) data consumers need to get fresh data and we can only get this by direct access to the actual \provider{},
% We aim to promote the direct communication between devices to ensure that freshness of the information shared.
(2) we must support mobile scenarios where personal devices carry their own data,
%   + Explicar un poco la naturaleza dinámica de los escenarios (sensores que entran y salen del sistema)
% In particular, we consider the dynamicity of WoT environments, where nodes can join and leave a space making their information available or unavailable at any time.
and (3) we need to reduce the maintenance complexity (nodes may join or leave at any time).

For these reasons, we cannot rely on intermediaries to host and provide \emph{all} the semantic information.
However, we use intermediaries as searching enablers in the semantic \Space{}.
This kind of intermediary is what we call \acl{wp}.

\acp{wp} manage metadata or \clues{} about the information shared by others.
The \consumers{} rely on \acp{wp} to enhance the search process and reduce the number of requests generated.
Consequently, the \providers{} process less semantic data and reduce their overall overhead.
Thus, our architecture enables an intermediary-aided energy-aware information search.


\InsertFig{architecture}{fig:solutionArchitecture}{
  Role of White Pages in our proposal
}{}{0.8}{}


Figure~\ref{fig:solutionArchitecture} shows the role of \acp{wp} in our proposal.
In particular, \acp{wp} manage \clues{} which summarize the semantic information shared by the \providers{}.
These \clues{}, which are described in detail in Section~\ref{sec:clues}, are pieces of information useful to determine which nodes can answer to a certain query.
\consumers{} use these \clues{} to directly access the semantic information on the \providers{}.
Summarizing, the main tasks of a \ac{wp} are:
\begin{itemize}
 \item Manage the \clues{} sent by all the nodes in the system.
 \item Aggregate clues sent by \providers{} in a unique response.
 \item Reply to requests from \consumers{} with aggregated clues or a list of nodes (details in Section~\ref{sec:interacting}).
\end{itemize}
 % debería haber un único \emph{WP} o varios? AG: Optaría por 1 para simplificar.
 % Dependiendo de eso, puede ser más complejo mantener una visión común de los clues de todos
 % (con algún algoritmo de gossiping?)
 % Para simplificar el merging de los differentes gossipings agregados se podría hacer: 1 master, N slaves.

\acp{wp} only store \clues{} for a period and they expire afterwards.
Using this approach, we avoid storing information from no longer available nodes.

% Si hubiese que dar detalles de implementación, había pensado en Redis para el \emph{WP} a poner en un server o cluster
% De forma que la gestión distribuida de dicho gossiping no fuese problema mio en esos casos
% Veo 2 casos:
%    + Un cluster de servidores hacen de WP: que usen algo que ya exista y me deje en paz.
%    + Dos/n móviles hacen de \emph{WP} conjuntamente: entonces sí que debería mojarme y decir como mantienen esa visión
%      conjunta.
% El problema es que al haber cambiado la explicación de tipos de cacharros a roles, esta consideración es difícil de
%  hacer.


\subsection{Versioning Clues}
As previously explained, the \ac{wp} aggregates clues sent by \providers{}.
This aggregation is versioned using a \emph{version number} and a \emph{generation number}.
The \emph{version number} increases each time the \ac{wp} receives a new clue.
The \emph{generation number} ($g_{id}$) is a timestamp that represents the moment a new \ac{wp} was chosen.
This requires the clocks to be synchronized in all the nodes to avoid problems.
In any case, the \ac{wp} must guarantee that the $g_{id}$ is higher than the one used by a previous \ac{wp}.

The \ac{wp} maintains two version numbers:
(1) the one used during the setup to do the first load (\emph{setup version}) and
(2) the version of the last aggregated clue.
The first one is shared through the discovery mechanism.
The second one is shared with both \providers{} and \consumers{} in \acs{http} responses.
The versioning is used to improve the \ac{wp} selection process explained in Section~\ref{sec:selection}.


\subsection{Discovering a \acl{wp}}
% He matizado la frase para que no se entienda que los Consumer SIEMPRE dependen de los WP. Sólo para actualizar clues.
% AG: te has empeñado en quitar el matiz de que "sólo" al principio está vendido a la existencia de un WP :-)
In our architecture, when a \consumer{} joins the \Space{} it relies on a \ac{wp} to find information.
Hence, the first thing a node needs to do afterwards is discovering who is the \ac{wp} in that \Space{}.

To run our proposal, we require a discovery system able to
(1) get the \Spaces{} that a particular node belongs to,
(2) identify the \ac{wp} in the system and its \emph{setup version}, and
(3) provide additional information about nodes to decide which one can be the next \ac{wp}.
To implement our architecture, the information about the nodes must include:
% Estos parámetros se vuelven a explicar en la sección de "Selección de WP"!
battery level, available memory, storage space and the approximate time passed since the node joined the \Space{} for the last time (to estimate its reliability).

How to discover the nodes in the \Space{} is transversal to this paper.
We could either extend approaches like \ac{upnp}, \ac{mdns}\footnote{\url{http://tools.ietf.org/html/rfc6762}} or lmDNS \citep{jara_light-weight_2012} or use \acs{http}, \ac{coap}\footnote{\url{https://datatracker.ietf.org/doc/draft-ietf-core-coap/}} and \ac{dns} together as proposed by \citet{ishaq_facilitating_2012}.
In particular, in Section~\ref{sec:mdns} we use \ac{mdns} to evaluate our solution.
% En la primera ronda de revisiones hubo un revisor que dijo:
%  "How does the system receive the information about the battery level, available memory?
%   The authors justify that this is out the scope of this paper, but it is important to know
%   the potential overhead introduced in these processes."
% Es una forma de responderle. Con esa frase quería decir:
%  " Esto no solo funciona en el plano teorico, usando un sistema como mDNS es perfectamente
%    posible implementarlo porque compartir esa pizca de info, pese a que te llegue un poco
%    desactualizada, no es un problema. "

% - Which is the topology of the network?
% It is hard to evaluate the proposed solution in terms of overhead and energy consumption for performing necessary tasks (WP discovery, WP selection, clues updates,...)
% without any knowledge of the network. Are there wired/fixed nodes? Are WP infrastructural nodes? Are the wireless nodes connected through multihop paths?
Note that we assume that any node in the \Space{} can reach the other nodes.
In particular, we assume IP addressability (no matter whether the node is wired or wireless).
As a consequence, to use devices from Wireless Sensor Networks (e.g., Zigbee or 6LoWPAN) as \providers{}, one should rely on gateways.

% IG el CoAP multicast no tiene una utilidad diferente a lo q se habla?
% Se podría usar UPnP, mDNS o  incluso lo propuesto por Jara (lmDNS).
% También se podría usar multicast CoAP (e.g. los que respondan a /\ac{wp} o /consumers)


% Partimos de la base de que el sistema de descubrimiento nos ha descubierto a otros nodos y los siguientes datos acerca de ellos:
%    - Battery (B): valor [1 a N] y unidad [min, hora, dia] -> siendo -1 infinito o cargando.
%         Problema: puede que no sea fácil hacer ese cálculo.
%         Solución: mejor eso a carga relativa o medida de carga, cuando no sabes cuanto consume el cacharro de normal.
%    - Memory (M): Valor [1 a 1023] y unidad [KB, MB, GB]
%    - Persistence (P): Valor [1 a 1023] y unidad [MB, GB, TB]
%    - Joined since (JC): Tiempo desde su unión al espacio (para permitirnos medir su estabilidad): medido en ciclos de media hora
%    - Stores aggregated clue (SAC): [T o F] ¿Tienen una versión del clue agregado?
%
% Hemos tratado de no crear mucho tráfico en el mecanismo de descubrimiento almacenando datos que varíen con poca frecuencia.
% Los campos más variables: la batería y JC, pueden no actualizarse en media hora sin que el algoritmo se vea sustancialmente alterado.
%
% Comprobar que no son datos que costase mucho embeber en cualquier sistema (registro TXT, lo que sea que use UPnP...)
% Jara habla de 80 bytes en 6LoWPAN, probablemente tampoco nos orientemos a cacharros tan peques, pero oye, por comentarlo...


\subsection{Interacting with a \acl{wp}}
\label{sec:interacting}
When a node needs to access a piece of information, it asks the \ac{wp} for the information about the other devices in the \Space{}.
Using this information, the node can find the owner of the required data and ask it for this information.

In our architecture, \providers{} and \consumers{} have additional duties.

\noindent\textbf{Providers.}
They manage their own semantic information and generate \clues{} about the information they host.
Then, they send these \clues{} to the \Space{} \ac{wp}.
As a response, they receive the last version of the aggregated clue they have contributed to.
\providers{} will send their \clues{} to the \ac{wp}
(1) every time a \emph{clue} is updated,
(2) before the lifetime of a \emph{clue} expires, or
(3) whenever there is a new \ac{wp} in the \Space{} with a lower \emph{setup version} than the one in the \provider{}.
Note that \clues{} do not change frequently since they represent the type of information the nodes host rather than the specific data which is constantly generated.

\noindent\textbf{Consumers.}
When a \consumer{} needs to get information from the system, it first needs to find the \ac{wp} in the \Space{} and use this \ac{wp} to obtain an aggregated clue of all the nodes.
Then, it processes this aggregated clue to decide which are the nodes to query.
Finally, it sends the query through \acs{http} requests.

They perform this process synchronously for the first query and periodically in an asynchronous manner for the following ones.
% cada cuanto? arbitrario? que lo decida en base a parámetros de funcionamiento? => Mi propuesta abajo
This period should have an upper limit to ensure a fresh view of the \Space{} and a lower limit to guarantee that the \ac{wp} is not flooded.
To adjust the update frequency within these two limits, we evaluate the frequency of the last 10 requests to that node.
Thus, the view of the \Space{} will be fairly up to date when the \consumer{} processes the next query.
% TODO Añadir mejora: cuando se detecta un nuevo nodo por el mecanismo de descubrimiento, esperas un poco y actualizas clue para ver si ya ha enviado su info al WP.
% Mejora de la mejora: exclusivamente preguntas por la clue de el/los nuevo/s nodo/s detectado/s. Asi envias un JSON más peque.

\noindent\textbf{Optimizations for resource constrained devices.}
Note that \providers{} will update the \ac{wp} with new information and \consumers{} will periodically check the \ac{wp}.
Using this approach, the \ac{wp} reduces the network load and in particular, it decreases the load for resource constrained devices.

In addition, \emph{aggregated clues} can be too long for some really limited devices.
As an optimization for such devices, the \ac{wp} is able to answer specifically the node to address a query.
Thus, these nodes will only maintain a list of the nodes they should ask for several predefined queries.

% Limitar la información enviada también a otros nodos?
% Si tenemos 1024 nodos y cada gossiping es de 100KBs, vamos a tener 100MBs de gossipings :-S
% Aunque esto es justo lo que hay que discutir en el apartado de evaluación de clues...


\subsection{Selecting a \acl{wp}}
\label{sec:selection}
%   + Que queremos solucionar => subsección K (tipos de sistemas)
Depending on the setup, having a dedicated \ac{wp} (e.g., a high-end server) can be too expensive.
For example, in a domestic environment, it is not worth dedicating a full server to be a \ac{wp}.
However, when we have a large setup (e.g., a hotel), it is convenient to dedicate a few servers to decrease the load in small devices.
%   + Necesidad de que esos roles sean algo dinámico y el problema que ello presenta
%      (descubrimiento, pero sobre todo selección) => subsección M

We have to be able to manage the complexity of a system composed by heterogeneous devices.
For example, making a small device a \ac{wp} may be inappropriate in highly populated environments.
On the contrary, having multiple dedicated servers implies a high management overhead which is unnecessary in simple environments.
Our architecture is flexible and any node can be a \provider{}, a \consumer{}, a \ac{wp} or any of them at the same time.

\noindent\textbf{When does the \ac{wp} selection start?}
The selection process can start (1) when no \ac{wp} is available or (2) when the current \ac{wp} gets a worse score than other nodes.
In the first case, the first node to realize there is no \ac{wp} starts the selection process.
In the second case, the current \ac{wp} periodically checks if there is another node with a better score.
If this actually happens, it starts the process.
We limit the frequency for this checking to avoid the overhead associated with this change.

\noindent\textbf{How to select a \acs{wp}?}
The node in charge of selecting a \ac{wp} retrieves information of available nodes using the discovery mechanism.
It ranks them, selects the top one, and informs it that it should become a \ac{wp}.
If a selected node rejects the new role, the selector chooses the next one on the ranking.

The node in charge of the selection sends its aggregated clue to the new \ac{wp}.
If this aggregated clue is fresher than the one in the \ac{wp}, the \ac{wp} will use this one.
Otherwise, the \ac{wp} will use its own.
In case there is no previous aggregated clue, the new \ac{wp} will start from scratch (version to -1).
Once the node becomes the new \ac{wp}, it notifies the version used to initialize the discovery system.
Doing so, \providers{} will realize if they must send their \clue{} again because it is not included in the aggregation provided by the \ac{wp}.

% aclarado que facilitará esa información de cada nodo activo en el espacio
Note that the discovery mechanism must provide the following information about each node in the \Space{} to the selection algorithm:
(1) memory of the device,
(2) storage capacity of the device,
(3) time since it joined the \Space{}, and % esto no debería cambiar a menudo para no generar mucho trafico
(4) its battery charge.
We use the algorithm detailed in Listing~\ref{list:selectionAlgorithm} to rank the nodes.

\begin{listing}
  \begin{algorithmic}[1]
  \State $nodes \gets filter_{threshold}(nodes, ``memory'', threshold_{memory})$
  \State $nodes \gets filter_{threshold}(nodes, ``storage'', total_{nodes} \times storage\_needed_{avg})$
  \If { $anyWith(nodes, battery_{infinite})$ }
    \State $nodes \gets filter(nodes, ``battery'', battery_{infinite})$
    \State $nodes \gets orderBy(nodes, ``battery'')$
  \Else
    \If { $anyGreaterThan(nodes, ``joined\_since'', joined_{threashold})$ }
      \State $nodes \gets filter_{score}(nodes, ``joined\_since'')$
    \EndIf
    \State $nodes \gets filter_{score}(candidates, ``battery'')$
    \State $nodes \gets orderBy(nodes, ``memory'')$
  \EndIf
  \State \Return $nodes$
  \end{algorithmic}
  \caption{\acl{wp} selection algorithm.}
  \label{list:selectionAlgorithm}
\end{listing}

Lines 1 and 2 of the algorithm apply filters based on thresholds.
In particular, the second threshold is variable and depends on the number of nodes (more nodes will generate more \clues{} to store).
After that, we check the power availability of the devices.
We first select the nodes connected to the power grid (represented by $Battery_{infinite}$).
If there are devices connected to the electrical grid, we select devices which are steady enough to apply a filter by $joined\_since$ field.
Filters in lines 8 and 10 choose nodes with z-scores higher than one for the specified fields.
If no node fits that filter, it returns the nodes with values higher or equal to the mean.

% AG: Intentando aclarar si un móvil puede ser WP o no (a un revisor no le había quedado claro)
Summarizing, the algorithm prioritizes energy autonomous nodes and within the nodes with battery limitations, it prefers steady ones.
% IG pues si quieres aclararlo, remata xD
In both cases, it finally selects a node with enough storage and memory (including mobile devices).

% AG: Problema a mencionar: y si no hay ningún \ac{wp} candidato? Broadcasting entre cacharros?

% \noindent\textbf{\ac{wp} conflict resolution}
Due to out-of-dated information, two nodes may become \acp{wp} at the same time.
Our solution would eventually correct this situation thanks to a conflict resolution algorithm.
When a \emph{\acs{wp} A} detects another \emph{\acs{wp} B} in the \Space{}, it has to check which one has a better score according to Listing~\ref{list:selectionAlgorithm}.
If \emph{B} has a better rank, \emph{A} simply resigns as \ac{wp} and notifies it to the discovery system.
Otherwise, it forces \emph{B} to resign through a \acs{http} invocation.
Other nodes will be aware of these changes through the discovery mechanism.




\section{Shared Clues}
\label{sec:clues}
As we introduced before, \acp{wp} host \clues{} about the information in the \Space{}.
Using a \emph{clue}, a \consumer{} can find which node (or nodes) is the \provider{} of a piece of information.
\providers{} generate these \clues{} by digesting the semantic information they store.

Thanks to these \clues{}, resource constrained devices do not have to process unnecessary requests.
We also limit the length of the \clues{} to reduce the bandwidth, the memory, and the storage overhead on these devices.

In this section, we describe in detail these \clues{}, the information they contain and their format.


\subsection{Querying Basics} % Querying assumptions

% Por qué una nueva sección?
% Porque estaba quedando muy abstracto todo y cosas básicas como cómo un consumidor consulta a un proveedor podía no quedar claro.
% O qué es un template y para qué se usa.

A \consumer{} directly queries the information to the selected \providers{}.
This selection is done using the \clues{} described in the rest of the Section.
However, in this section we explain how \consumers{} directly query for information to the selected \providers{}.

We assume that \providers{} offer a HTTP endpoint which accepts queries.
Whether this endpoint redirects the requests to the RDF graph which describes a specific resource,
it creates a response with RDF triples belonging to different resources or
it offers hyperlinks to the relevant graphs is transversal to our solution.
The design of this endpoint is also out of the scope of this paper, but it should be the same in all the providers.

% explicar query templates
To perform the queries, which enable the selection of a subset of the semantic content hold in a given \Space{}, we require a \textbf{template}.
We assume the use of wildcard templates, which are special triples with optional wildcard subject, predicate and/or object (see Figure~\ref{fig:semanticExample}). % TODO referencia al ejemplo o a figura de ejemplo
We could use more sophisticated query languages (e.g. SPARQL) % TODO footnote?
but processing them may be too demanding for devices with constrained capabilities.
On the other hand, simple wildcard templates can be managed by any node able to manage RDF triples.
In any case, complex queries can be decomposed into wildcard templates and our solution would still apply.


\InsertFig{semantic_example}{fig:semanticExample}{
  Sample triples and query templates
}{
  The top of the figure shows some sample triples.
  It highlights the different parts used for each type of \clue{}:
  the information used in class-based \clues{} in orange, the prefix-based \clues{} in blue, and the predicate-based \clues{} in brown.
  The bottom shows query template examples for these triples.
}{0.8}{}



\subsection{Content of a Clue}
To find out which is the most appropriate solution for \ac{ami} scenarios, we have to consider scenarios populated by mobile devices and sensors. % for \ac{iot} scenarios?
Mobile devices usually share data which rarely changes and is described using a few ontologies (e.g., the user profile and his preferences).
On the other hand, sensors are constantly generating new instances of the same ontology (also called individuals).
In both cases, the data shared by each node is described according to one or few vocabularies or taxonomies.

At this point, it is important to define the \emph{TBox} and \emph{ABox} concepts following the definition of \citet{nardi2003introduction}.
\emph{TBox} contains knowledge describing general properties of concepts or terminology and
\emph{ABox} contains knowledge specific to the individuals of the domain of discourse.
An example of \emph{TBox} information is the device type or the elements it is made of
while \emph{ABox} can describe the mobile phone brand or the temperature sensed by a thermometer.

Each mobile device or sensor usually generates \emph{ABox} according to the same \emph{TBox}.
Using this information, we propose avoiding the use of URIs which represent \emph{ABox} information in general terms.
Specifically, we propose and evaluate three possible type of \clues{} to share.
%The first two are inspired by early works on peer-to-peer semantic web $[CITAEDUTELLA,devadithya2007index]$.

\medskip

\noindent\textbf{Prefix-based clue.}
%A coarse-grained step to find relevant nodes for a query is to ignore those which do not have \emph{ABox} for a specific \emph{TBox}. % no sólo aplicable al caso de la ontologias
The individuals (\emph{ABox}) are described according to different ontologies (\emph{TBox}).
Each ontology usually employs a common \emph{namespace}.
This means that the URIs of the concepts in these ontologies share a unique \emph{prefix}.
Besides, each \provider{} also uses few \emph{prefixes} to generate the URIs of its individuals.
In this type of clue, we propose a coarse-grained method to filter nodes which do not have individuals with the prefixes used by the query template.
To create prefix-based clues, the \providers{} have to extract the prefixes used in their graphs. % obvio?

Following the example in Figure~\ref{fig:semanticExample}, a \provider{} \emph{P} may have that graph.
Thanks to the \clues{}, the \consumer{} knows that \emph{P} has URIs starting with \emph{sweet}, \emph{ssn}, \emph{ex} and \emph{weather}.
Consequently, it will send to \emph{P} a query which uses any of the templates shown in Figure~\ref{fig:semanticExample}.
On the contrary, if the template is \emph{?s pref2:sth ?o}, it will avoid sending the template to \emph{P}.


\medskip

\noindent\textbf{Predicate-based clue.}
The predicates relate subjects with other subjects or literals.
These predicates are defined in the \emph{TBox} (e.g., to state they relate concept \emph{A} with concept B) and used in each triple of the \emph{ABox}.
In this approach, we propose extracting the set of predicates used in the graphs stored by each node.
Using this information, they can be simply matched with the predicate defined in the query template.
% AG: Posible pega, se ha explicado antes lo que es un "template"?
% IG: nope, de hecho en la anterior tb me ha surgido la duda de q era

\medskip

\noindent\textbf{Class-based clue.}
In the third approach, we propose sharing the classes of concepts (\textit{rdf:type}) provided by the nodes.
Using this information and the \emph{TBox}, each node can check if the information matching certain query template is susceptible to be stored in other nodes.
For this kind of clues, we assume that each node should have (or be able to obtain) the \emph{TBox} related to the template used for querying.
This is a reasonable assumption since ontologies describing \emph{TBox} are usually accessible on the URL described by its prefix.

To understand how it works, consider that, according to the \emph{TBox}, a predicate \emph{p} relates the concept \emph{A} with another concept.
We assume that the nodes storing instances of the class \emph{A} are more likely to use this predicate in their graphs.
Therefore, if a query template uses that predicate, we will send it to all nodes storing instances of the class \emph{A}.


% IG: esto de referirte a una ontologia del futuro... no me parece muy asines
% AG: Exactamente mismo caso que arriba, pero con ejemplos concretos.
%     La idea era que no quedase muy abstracto. Lo comento porque ahora puede quedar redundante.
% For instance, consider that the following template is queried: \textit{?s~ssn:observes~dbpedia:CO2}.
% According to the SSN ontology, which will be presented later on, only an instance of the class \textit{ssn:Sensor} \textit{observes} something.
% Therefore, the querying node will send the template to the nodes which have instances of the class \textit{ssn:Sensor}.


\subsection{Reasoning to expand clues} % Esto se relaciona principalmente con class-based, pero igual también puede comentarse para Predicate-based.
Through a reasoning process one can know unstated knowledge.
Using this knowledge, we can detect more relevant nodes.
For example, let us consider the previous example where the class \emph{A} defines the predicate \emph{p}.
Providing that \emph{C} is a subclass of \emph{A}, a node which has many instances of the class \emph{C}, may also use the predicate \emph{p}.
Thanks to the reasoning, we can discover that the node has knowledge of the type \emph{A} and therefore, it may be relevant.
% AG: Comentado por lo mismo que arriba
%(in the example, instances of \textit{ssn:Sensor} class' subclasses such as \textit{Accelerometer}).
The drawback, is that reasoning consumes a lot of resources.
This limitation will be further analysed in Section~\ref{sec:clues_eval}.
% TODO Many authors solve this limitation by transforming the queries in the consumer side \cite{}.


\subsection{Use of ABox in clues}
\label{sec:aboxinclues}
% se podrían usar individuals que no cambian mucho
% location, individual en base al cual se generan todos los elementos.
As stated before, in general terms, we want to avoid the use of \emph{ABox} URIs (individuals) in our clues.
Thus, we fulfil two goals:
(1) we generate smaller \clues{} and
(2) the \clues{} will not change too frequently and therefore, we will require less communication to update clues.
However, in some cases, the use of \emph{ABox} content in the \clues{} may be beneficial.
For instance, assume a URI that refers to the specific location \emph{L}.
If we want to search for devices in location \emph{L}, we cannot deduce anything about it using the proposed \clues{}.

% AG: Atencion, a partir de aqui viene una propuesta que me acabo de sacar de la manga que habra que discutir.
%     El problema de esto es que no está en la evaluación y para el 15 de Octubre, lo veo ajustado de incluir.
For this reason, we need to consider sharing the \emph{N} most queried individuals in our clues.
To do that, the \ac{wp} needs to store a list with the statistics about the information collected by each \consumer{}.
\consumers{} can send this information together with the request to update a clue.
\providers{} can obtain a list of the current most popular URIs before sending their updated \clues{} to the \ac{wp}.
Using this list, \providers{} can know if they have these URIs and include them in the \clue{} to be sent to the \ac{wp}.
% TODO el proceso no me acaba de convencer, y cuando hay una nueva URI? Como saben los proveedores que deben actualizarla?
Note that this process would imply an extra request per \provider{} before each update.

This simple approach implies sending not only \emph{TBox} but also \emph{ABox}.
The amount of extra information added to each \clue{} will depend on the size of this list (\emph{N}).
The effectiveness of this method will depend on the number of queries using one of the \emph{N} URIs in their subjects or objects.
% Propuesta: uso de un porcentaje de las queries realizadas
% Problema: no vas a raspar a cada providers con 1000 comprobaciones tampoco!
% Solución: ¿merece la pena evaluar eso? ¿como?


\subsection{Format}
% Tras explicarlas, poner algunos fragmentos con las clues (JSON) para que se vea de que hablamos
We can use many formats to represent the content of a clue.
% pongo esto de EXI porque se que a la gente del mundillo de IoT se le hace el culo chupicola con sus estandares
One option is the ongoing Efficient XML Interchange (EXI) \footnote{\url{http://www.w3.org/XML/EXI}}. % o cite?
EXI is designed to efficiently interchange XML data and therefore, we could obtain better compression rates than with JSON. % aseveracion cierta o solo para XML?
However, we have chosen JSON for its simplicity and its wide adoption in the WWW.

% TODO no lo he acabado de pillar...
The prefix-based \clue{} is the easiest one to represent in JSON since it is formed by a set of URIs.
We can also represent the predicate-based and the class-based \clues{} using a set of URIs.
However, the prefixes of those URIs are usually repeated and for this reason, we do not use plain URI transmission for these clues.
We show an example in Listing \ref{list:oneClue}.
It first defines the prefixes used and gives them a name and then, it specifies the URI endings for each prefix.

\begin{listing}
  \begin{minted}[frame=single, framesep=3mm, linenos=true, xleftmargin=21pt, tabsize=4]{js}
  {
    "s": [[
      "so",
      "http://knoesis.wright.edu/ssw/ont/sensor-observation.owl#"
    ]],
    "p": {
      "so": ["result", "procedure",
		  "observedProperty", "samplingTime"]
  }  }
  \end{minted}
  \caption{
    Representation of a predicate-based \clue{} in JSON.
    The node sending the \clue{} has RDF triples which use the predicates
    \emph{so:result}, \emph{so:procedure}, \emph{so:observedProperty} and \emph{so:samplingTime}.
  }
  \label{list:oneClue}
\end{listing}

\medskip

These are isolated \clues{} sent from a \provider{} to the \ac{wp}.
However, the \ac{wp} gathers all these \clues{} in an aggregated clue which is sent to the \consumer{}.
We show an example of an aggregated clue in Listing \ref{list:aggregatedClue}.
As one can see, the aggregated clue defines the type of \clues{} wrapped through the numeric field \emph{i}.
\emph{G} and \emph{v} form the version of this aggregated clue.
The field \emph{s} defines the prefixes.
Finally, for each node, each prefix is related with the URI endings.

% TODO sacar un ejemplo real? poner esquematicamente?
\begin{listing}
  \begin{minted}[frame=single, framesep=3mm, linenos=true, xleftmargin=21pt, tabsize=4]{js}
  {
    "i": 1,
    "g": 2435467,
    "v": 556,
    "s": [
	  ["dc", "http://purl.org/dc/elements/1.1/"],
	  ["dul", "http://www.loa.istc.cnr.it/ontologies/DUL.owl#"],
	  ["ssn", "http://purl.oclc.org/NET/ssnx/ssn#"] ],
    "p": {
      "node1": {
	"ssn": ["observedBy", "observationResult"],
	"dul": ["isClassifiedBy"] },
      "node0": {
	"ssn": ["observes"],
	"dc": ["description"]
  } } }
  \end{minted}
  \caption{
    Representation of an aggregated clue in JSON.
    Line 2 defines that it embeds predicate \clues{} (i.e. type 1).
    Lines 3 and 4, contain the version of the aggregated clue.
    The remaining lines express the predicates used by two nodes.
    For example, \emph{Node1} has at least a RDF triple which uses the predicate \emph{ssn:observedBy}.
  }
  \label{list:aggregatedClue}
\end{listing}

% Mencionar: Note that we can gossip sleeps periods to know when to query them...
% Comentar que se podrían gossipear los sleeps de los nodos embebidos para saber cuando interrogarles.
% AG: Esto es un aspecto que puede ser guay comentar, de cara a IoTificar la propuesta.
%     Pero pensandolo mejor, debería ir en la sección 3, a la hora d explicar el formato de las
%     clues.
\section{Experimental Environment}
\label{environment}
We have used simulation to study the performance of our solution and compare it against a flooding-based one.
Using a simulation, we can evaluate multiple scenarios and repeat experiments for different approaches under the same conditions.

\subsection{Methodology}
Table~\ref{tab:configurationParameters} shows the main parameters of our simulator.
We vary these parameters to simulate a wide range of scenarios.
% Dado que los revisores parece que no hicieron cado de este aviso, mejor quitarlo.
% We omit topology considerations as we assume devices have IP connectivity.


\begin{table}
  \centering
    \begin{tabular}{l p{7cm}}
      \hline
      Name & Description \\
      \hline
      Network size & The number of nodes in a network. In the simulations conducted, all the nodes are \providers{}. \\
      Number of writes & Amount of writes performed during the simulation period. \\
      Number of queries & Amount of queries performed during the simulation period. \\
      Number of \consumers{} & Amount of nodes querying to other nodes in the \Space{}. \\
      Distribution strategy & Our solution or \ac{nb}. \\
      & In \ac{nb} nodes write locally and spread the queries to the rest of the nodes in the \Space{}. \\
      Drop interval & At the beginning of this interval a node abruptly leaves the network. \\
      & At the end, the node joins the network and a new one is chosen to leave it. \\
      % FUTURE WORK: añadir métrica de accuracy [ http://en.wikipedia.org/wiki/Recall_(information_retrieval) ]
      \hline
    \end{tabular}
    \caption {Configuration parameters.}
  \label{tab:configurationParameters}
\end{table}

% Since we wanted to simulate the Triple Space Computing paradigm over HTTP, the communication between the nodes was point to point and the data exchanged were RDF Triples.
% The node discovery process was ignored since it will show similar additional overhead for each strategy.
% It can be considered transversal to what was being measured.
% IG: version del parrafo anterior
% AG: Ok, pero esta es una de las cosas que en el uptade debería volar o cambiar. Sobre todo, porque no hablamos de TSC
%     en ningún otro lado. Puede ser más fácil aún, ya que como asumimos que queremos semántica en WoT, el HTTP está
%     justificado per se.
As we simulate \ac{http}, we assume a point to point communication between devices which exchange \ac{rdf} Triples.
We discuss how the discovery process affects the solution in Section~\ref{sec:dynamic}.
% TODO ENERO
In the rest of the sections we omit the node discovery process as it represents the same overhead for all strategies.

%To represent the data managed by each node, we first considered using LUBM\footnote{\url{http://swat.cse.lehigh.edu/projects/lubm/}}, a synthetic benchmark.
%Unfortunately, it creates instances from very few classes for each node, which makes all the nodes to have the same TBox.
%In our opinion, this does not faithfully represent Internet of Things scenarios with heterogeneous devices sharing disparate information.

%In our second attempt, we found a dataset which could represent this heterogeneity.
To represent the data managed by each node, we used data from diverse sensor network environments.
These data follow the \textit{Semantic Sensor Network Ontology} (SSN)\footnote{\url{http://www.w3.org/2005/Incubator/ssn/XGR-ssn-20110628/}}.
% created by the W3C Semantic Sensor Network Incubator Group. to represent diverse sensor network environments.
SSN has been used in many projects and scenarios to describe semantically the data provided by heterogeneous sensors.
% TODO IG hasta aqui hemos llegado
Specifically, we used data from the following datasets:
AEMET metereological dataset\footnote{\url{http://datahub.io/dataset/aemet}},
University of Luebeck Wisebed Sensor Readings\footnote{\url{http://thedatahub.org/dataset/university-of-luebeck-wisebed-sensor-readings}},
\emph{Kno.e.sis} Linked Sensor Data\footnote{\url{http://wiki.knoesis.org/index.php/LinkedSensorData}}
and Bizkaisense\footnote{\url{http://helheim.deusto.es/bizkaisense/}}.%http://dev.morelab.deusto.es/bizkaisense
These datasets contain descriptions about the sensing stations and the data sensed by them during certain periods.
The analogy between stations which have different sensors and the \ac{iot} devices is reasonable.
The datasets have been adapted to provide just one measure of each sensor at each moment (to emulate the storage restrictions from embedded devices) and to use as many stations as nodes has the network (depending on the network size).

However, sensors can not only be found in AmI environments.
Personal devices such as mobile phones usually populate them also.
To represent such circumstance, we added semantic data of people to represent their profiles\footnote{\url{http://www.morelab.deusto.es/joseki/articles}}. % TODO TODO TODO actualizar con la nueva plataforma que pongan!

% poner disponible el código
We use SimPy\footnote{\url{http://simpy.sourceforge.net}} to simulate each scenario.
SimPy is a process-based discrete-event simulation language for Python.
To accurately simulate the time needed by each node to provide a response, we considered measures taken from
real embedded web servers running on a \textit{ConnectPort X2} IP gateway\footnote{\url{http://tinyurl.com/connectportx2}} for \textit{Digi}'s \textit{XBee sensors}\footnote{\url{http://tinyurl.com/xbee-sensors}} (\textit{XBee} from now on),
on a FoxG20\footnote{\url{http://www.acmesystems.it}} and on a Samsung Galaxy Tab.
Besides, we also provide measures taken from a regular computer.
Table~\ref{tab:measures_embedded} shows the measures used for the parametrization.

% TODO TODO TODO poner o referenciar las características de los dispositivos cogidas de WoT 2012

\begin{table}
  \begin{center}
	\begin{tabular}{p{2.5cm} r r r r}
	  \hline
	  & \multicolumn{4}{c}{Devices} \\
	  \cline{2-5}
	  Concurrent & \multirow{2}{*}{XBee} & \multirow{2}{*}{FoxG20} & Regular  & Samsung \\
	  requests   &  ~    &   ~     & computer & Galaxy tab \\
	  \hline
	  1  &  77 (1)	&  17 ~(0)  &   5 ~(1)  &  223 (349) \\
	  5  & 392 (8)	&  97 (16) &   8 ~(4)  &  256 ~(76)  \\
	  10 & 775 (8)	& 174 (28) &  13 ~(8)  &  372 (171) \\
	  15 &  -   	& 282 (43) &  18 (13) &  497 (191) \\
% Tampoco creo que haga falta tanto detalle, ¿cuando se van a dar tantas peticiones concurrentes?
%	  20 &  -	    & 375 (30) &  23 (13) &  661 (444) \\
%	  25 &  -	    & 460 (30) &  30 (18) &  748 (288) \\
%	  30 &  -	    & 540 (35) &  38 (22) &  929 (805) \\
%	  35 &  -	    & 632 (29) &  38 (20) & 1029 (672) \\
	  \hline
	\end{tabular}
  \end{center}
  \caption{Mean of the measurements taken in different devices with the standard deviation in parenthesis (milliseconds).}
  \label{tab:measures_embedded}
\end{table}


\subsection{Performance Metrics}
To evaluate the fundamental properties of the strategies, we use the following metrics:
\begin{itemize}
  \item \textit{Precision}: the fraction of nodes which answered relevant results (those responses which were not \textit{not found} responses).
                            It measures the exactness of the results.
  \item \textit{Recall}: the fraction of relevant answers that are returned.
                         It measures the completeness of the results.% overload, failed search, etc.
  \item \textit{Size}: the size of each type of clue.
%  \item \textit{Throughput}: the average rate of successful message delivery over a communication channel.
%  \item \textit{Idle time}: the average time each node is in an idle state and therefore is consuming less energy.
%  \item \textit{Failed requests}: the amount of request which could not be answered due to the physical limitations of
% the nodes (i.e. both due to timeouts or connection rejections when the server is overloaded).
  \item \textit{Total requests}: the number of \acs{http} requests performed during a simulation.
  \item \textit{Response time}: the average time needed to obtain an \acs{http} response.
  \item \textit{Active time}: the total time spent by each node either querying other nodes or handling a query.
\end{itemize}
\section{Results}
\label{results}

\subsection{Types of Clues Shared}
\label{sec:clues_eval}
As presented in Section~\ref{sec:clues}, the type of clue used will affect
(1) the \emph{precision} and \emph{recall} to find the nodes with the appropriate information; % explained in section 4
and (2) the amount of information to transfer over the network (both requests and responses) and nodes have to process.
Increasing \emph{precision} reduces the number of unsuccessful requests to handle and thus, it reduces the energy consumption. 
Similarly, sending more information over the network implies more processing time and more energy consumed.

\medskip

\noindent\textbf{Precision and recall.}
We evaluate the \emph{precision} and the \emph{recall} of the proposed algorithm in a network of 470 nodes issuing the query templates shown in Table~\ref{tab:evaluationTemplates}.
In average, the nodes manage instances belonging to 6.34 different classes ($SD=1.31$) among a total of 17 distinct classes in the space.
%When these classes are expanded with a reasoning process, the nodes store 20 different concepts among a total of 113 classes in the space ($SD=4$). % TODO actualizar
The distinct predicates managed by each node in average are 16.01 ($SD=1.53$) out of 68 different predicates in the space.

\begin{table}[h!tbp]
\centering
\begin{tabular}{| c | c |}
\hline
Name & Template \\
\hline
t1 & ?s~~~rdf:type~~~ssn-weather:RainfallObservation \\
t2 & ?s~~~wsg84:long~~~?p \\
t3 & ?s~~~ssn:observedProperty~~~?o \\
t4 & bizkaisense:ABANTO~~~?p~~~?o \\
t5 & ?s~~~dc:identifier~~~?o \\
%t9 & ?s~~~foaf:family_name~~~?o \\ % TODO futura version del paper con perfiles de usuarios, para meter mas variabilidad en contenidos
\hline
\end{tabular}
\caption{Templates used in the evaluation.}
\label{tab:evaluationTemplates}
\end{table}
% Las consultas se podrían sacar también de bizkaisense o alguna app que hayamos hecho para darle mayor verosimilitud

% Explicar class based
In Figures~\ref{fig:recall_measures}~and~\ref{fig:precision_measures}, class-based clue shows a good \emph{precision} and \emph{recall} for \emph{t1} and \emph{t2}.
\emph{T1} asks exactly for the information this type of clues define (i.e. nodes having instances of a certain class).
\emph{T2} evaluates which nodes have instances in the domain of the \emph{long} predicate (\emph{SpatialThing}).
Note that this works thanks to the RDFS inference because some nodes in the space only write \emph{Point} instances (a subclass of \emph{SpatialThing}).
The domain of \emph{t3} and \emph{t5}'s predicates could not be inferred just using RDFS inference. % (some properties of OWL are used => inverseof).
Even solving this limitation, we would expect a bad \emph{precision} since both predicates relate very general concepts.
In addition, when a class-based clue has no enough information to predict the nodes, it simply floods the query.
This is why the \emph{recall} of \emph{t4} is high.



\InsertFig{clues_recall}{fig:recall_measures}{
  \emph{Recall} for each type of clue used.
}{
}{1}{}

\InsertFig{clues_precision}{fig:precision_measures}{
  \emph{Precision} for each type of clue used.
}{
}{1}{}
% IG: TODO mencionar en el texto de referencia o en la caption algo como: The higher the better.

% Explicar predicate-based
We can see a bad prediction for \emph{t1} and \emph{t4} for predicate-based clues.
\emph{T1} defines a very common predicate and therefore, it cannot discriminate any possible node.
\emph{T4} suffers the same problem explained for the class-based clues.
We proposed a possible solution for this problem in Section~\ref{sec:aboxinclues}.

% Explicar schema-based
Finally, schema-based clue shows a slightly better \emph{precision} for \emph{t4}, since it can discriminate some nodes not using the \emph{bizkaisense} prefix.
On the other hand, it obtains marginally worse \emph{precision} than predicate-based clues for \emph{t3} and \emph{t5}.
This worsening could be greater if few nodes using the schemas \emph{ssn} and \emph{dc} will use the predicates defined in both templates.

\medskip

\noindent\textbf{Verbosity.}
% Una frase para retomar lo anterior, y al grano.
The clues verbosity is also a critical aspect for resource constrained devices.
Figure~\ref{fig:clueSize} shows a higher variance for schema based clues' length and lower verbosity of class based clues.
This is because the nodes virtually have a different number of sensors.
In addition, the links to concepts of other ontologies varies within the two datasets used in the parametrization.
In any case, the diagram shows a similar verbosity for all the clues for the semantic content considered in this evaluation. % TODO analizar si la media varía significativamente
% No es adecuado para cacharros pequeños: poner evaluación de WoT de inferencia


\InsertFig{clues_length}{fig:clueSize}{
  Length of the clues alternatives
}{
}{0.7}{}

% Añadir nuevas barras: Predicate+schema y predicate+schema+MostCommonsIndividuals
% Poner otra barra para saber cuanto contenido semántico guarda un nodo de media en el experimento?
% Comentar que cabe en MTU de ethernet y UDP en caso de querer enviarlo por CoAP


% TODO Añadir nuevo diagrama para el gossiping agregado y ver como crece a más elementos añadidos

\medskip

\noindent\textbf{Summary.}
Class-based clues are useful for templates asking for a specific type of content.
However, they still require inference to obtain a good \emph{precision}.
In \citet{gomez-goiri_restful_2012}, we tested the inference process on the devices and data used in this simulation. % ref a WoT2012 para más detalles
We could not run any reasoner in the ConnectPort X2 Gateway.
Actually, we could only run RDFS reasoners in more powerful embedded and mobile devices such as the FoxG20 and the Samsung Galaxy Tab.
In the FoxG20, it took 48.9 seconds the first load of all the ontologies used and 1.4 seconds to reason over each measurement written.
In a Samsung Galaxy Tab, it took 17.3 seconds and 0.2 the following measurement writings.
Considering these results, we can conclude that there is a clear need for efficient embedded reasoners.
Therefore, the class-based approach is promising but it is impossible to adopt in current embedded and mobile devices.

Between the predicate-based and schema-based clue, we propose to use the predicate-based clue since it subsumes much of the information provided by the schema-based clue.
The rest of the schemas, are referred in the subjects or the objects. % más que schemas son prefixes :-S
They could be easily added to predicate-based clues on the prefixes field.
In addition to the use of predicate-based clues, we could implement the solution for the specific individual search proposed in Section~\ref{sec:aboxinclues}.

% Posible TODO
% GRAFICO 3: \emph{precision} y \emph{recall} comparando con y sin de cada uno

% GRAFICO 4: tamaño de gossiping individual y agregado comparando con y sin cada uno
%            (barra con barra superpuesta encima con cuanto más añade)
%            Según hay más nodos, cómo crece el gossiping a manejar?


% Discussion: ¿elegir un tipo u otro dependiendo del modo de operación?
% (i.e. si hay que perder precisión a costa de no intercambiar MBs...)
% AG: Interesante, pero yo no complicaría una sección ya de por si bastante liosa.


% TODOs importantes que me gustaria hacer para la siguiente version:
%   - Ver cómo crecen los clues agregados.
%   - Proponer una clue híbrida que mezcle a las anteriores.
%   - Evaluar de alguna forma la mejora propuesta para ABox.



\subsection{Network usage} % IG: network role
\label{sec:NetworkUsage}

% Donde ``explicar'' negative broadcasting y/o centralizado? En seccion 4?
% Explicar porque no se pone centralizado en la comparación
%    No es directamente comparable dado que depende directamente de otro factor distinto: frecuencia de escritura.

We have conducted a simulation to evaluate the benefits of our solution against a flooding-based approach (i.e., \emph{negative broadcasting}).
The simulated environment has nodes that join to the same space.
All the nodes are \emph{Providers} and write new information periodically in the space.
During one hour, 1, 10 and 100 \emph{Consumers} perform 1000 queries in total using the templates described in Table~\ref{tab:evaluationTemplates}.

As expected, our solution scales much better than the \emph{negative broadcasting} (see Figure~\ref{fig:requestsByStrategies}).
In particular, the increase on requests per consumer in the space is not appreciable.


\InsertFig{requests_by_strategies}{fig:requestsByStrategies}{
  Required requests for negative broadcasting (\emph{nb}) and our solution with 1, 10 and 100 consumers
}{
}{0.7}{}


In Figure~\ref{fig:requestsByRoles}, we take a closer look to the origin of the traffic of our approach in a space with 100 consumers.
The communication between the \emph{Providers} and the \emph{WP} is much more infrequent than the other communication types.
The reason is that writing to a node only results in a clue update when the structure of the managed information changes.
The first time the metadata about the node (sensor) is written, the second time the first measure and following writings, just add or replace a measure.
Therefore, the clue does not change after the second step.
This matches with the assumption made to share \emph{TBox} information in our clues.

The communication between \emph{Consumers} and \emph{WP} is in between the other two communication patterns.
It is greater than the one from \emph{Providers} to \emph{WP} because \emph{Consumers} need to maintain an updated view of the space.
Recall that the update time depends on the query frequency of each \emph{Consumer}.
The maximum and minimum updating frequency were set to 10 and 1 minute(s) respectively.

The communications between \emph{Consumers} and \emph{Providers} is most of the total communications.
This shows that the overhead added by the use of \emph{WP} on our solution is not significant and it is justified by the reduction of the total number of communications shown in Figure~\ref{fig:requestsByStrategies}.


\InsertFig{requests_by_roles}{fig:requestsByRoles}{
  Requests between roles in our solution in a space with 100 consumers
}{
}{0.7}{}



\subsection{Energy consumption}
\label{sec:energyConsumption}
% Idea: Lo de arriba está muy bien, pero específicamente, cómo afecta a los cacharros?
Our solution tries to save energy by making \emph{Providers} handle fewer requests from \emph{Consumers}.
These savings contrast to the overhead added by the communication with the \emph{WP}.
However, our results have demonstrated that this overhead is small in comparison to the total number of communications.

The energy consumption in mobile and embedded devices increases each time a device needs to process something or communicate with another node (see Figure~\ref{fig:energy_consumption}).
To analyse how communications impact their energy autonomy, we have to consider not only the number of communications but also their time length (see Table~\ref{tab:measures_embedded}).
For example, a mobile phone will consume less energy asking clues to a server than asking them to an embedded device as it has to wait longer for the response.


\InsertFig{energy_consumption}{fig:energy_consumption}{
  Energy consumption for FoxG20 during different activity periods (miliwatts)
}{
}{0.8}{}


The experiment consists of 300 nodes joined to a space running on 1 server, 30 galaxy tabs, 75 FoxG20 and 194 Xbees. % TODO marca del xbee!
We increase the number of devices as their price and capacity decrease.
Using this approach, we mimic a typical space where cheap devices are more common.

As shown in Figure~\ref{fig:activity_measures}, our solution reduces the activity of each device by more than 5 times compared to \emph{negative broadcasting}.
The diagram on the right details the average activity for each type of device.

In our solution, we can check how the load moves from the embedded devices (XBee and FoxG20) to the server (which is indeed chosen as a \emph{WP}).
The exceptional activity registered by the Galaxy Tabs is caused by their extremely high response time.
% This implies that handling a request in the Galaxy Tab takes much longer than the average.
However, future work will cover the change of the HTTP library used in our Android implementation to reduce this response time.


\InsertFig{activity_measures}{fig:activity_measures}{
  Activity time for each strategy
}{
  In the first diagram we appreciate the average active time a node spends in each strategy. In the second one, the active time is classified by the type of device each node is running on.
}{0.8}{}
% TODO añadir a la derecha los valores en NB?

%%%%%%%%%%%%%%%%%%%%%%%%%%%%%%%%%%% QUIZA para 2da VERSION %%%%%%%%%%%%%%%%%%%%%%%%%%%%%%%%%%%%%%%%%
%  + Vamos a medir y comparar distintas situaciones entre sí:
%   Nota: cuando hablo de servidor, movil o dispositivo embebido, ejemplifico capacidades.
%         dispositivo embebido puede almacenar menos cosas y hace consultas más concretas.
%
%      - Situación 1: Negative broadcasting (o todo dispositivos embebidos)
%      - Situación 2: 1 servidor (WP), 20 moviles, 120 dispositivos embebidos (proporción 1:20:120)
%      - Situación 3: 20 moviles, 120 dispositivos embebidos (proporción 1:6, entre móviles 5 con el cargador enchufado)
%      - Situación 4: todo dispositivos embebidos
%  + Eje Y: Tiempo medio en ejecución
%  + Eje X: A parte de las situaciones (ver esquema de \emph{recall}) algún otro aspecto que también afecte al consumo de energía:
%      - número de consultas?
%      - frecuencia de las mismas?
%      - nodos consultores?



\subsection{Performance in Dynamic Environments}
\label{sec:dynamic}
We have evaluated the network usage of our solution in ordinary situations in Sections~\ref{sec:NetworkUsage} and \ref{sec:energyConsumption}.
Nevertheless, we have not conducted simulations where the nodes frequently join and leave the space.
In such situation, the communication needed to manage the clues might be a burden.


\InsertFig{dynamism}{fig:dynamic_situations}{
  Effects of dynamic scenarios in our solution
}{
  Note that the last interval in the x-axis represents a simulation with no drops.
}{0.8}{}


To assess the effect of  dynamic networks on the performance of our solution, we used the scenario presented in Section~\ref{sec:energyConsumption},
Then, we simulate nodes joining and leaving the space at different intervals: 30 seconds, 1 minute, 5 minutes, 10 minutes, 20 minutes, 30 minutes, 45 minutes.
Particularly, for our solution, we tested the most harmful situation: the node leaving the space abruptly is always the \emph{WP}.
We also added an scenario with no drops as a baseline.
Note that we represent this scenario by configuring the drop-interval with a greater value than the simulation time.

In Figure~\ref{fig:dynamic_situations}, we see the results of these simulations.
These results show that even in such dynamic situations, our solution requires fewer communications than negative broadcasting.
In our solution, most of the communications are between \emph{Consumers} and \emph{Providers}.
To evaluate the overhead added by our solution, the right graphic shows the communications involving \emph{WPs}.

We can appreciate that the updates on \emph{Consumers} are independent of the number of times the \emph{WP} changes.
We can also see a minimal change between the scenario where the \emph{WP} is always available and the one with 5 minutes drop-interval.
In this case, when the \emph{WP} drops, many \emph{Consumers} have the latest version of the aggregated clue.
% Dado que se parte del que tenia el WP o del que le ha dado el consumer que le ha elegido. Ambos tienen más opciones de tener una versión actualizada.
This situation increases the chances of getting an updated version for the initialization of the new \emph{WP}.
Thus, it reduces the number of messages from the \emph{Providers} to the new \emph{WP}.


\subsection{Effects on Discovery Mechanisms}
\label{sec:mdns}
We want to prove the feasibility of our solution using a common discovery mechanism.
To that end, we simulate the behavior of the \emph{Multicast DNS} (mDNS)\footnote{http://tools.ietf.org/html/draft-cheshire-dnsext-multicastdns-15}
and \emph{DNS service discovery} (DNS-SD)\footnote{http://tools.ietf.org/html/draft-cheshire-dnsext-dns-sd-11} protocols.
Both protocols are based on the well known and widely accepted Domain Name System protocol (DNS).

On the one hand, DNS-SD proposes the use of the data field of TXT records to share key/values.
We use these records to share the information needed by the selection algorithm among the nodes.

On the other hand, mDNS defines how this and other records are shared through UDP multicast (or unicast in certain situations).
We ignore the cost of browsing the nodes to discover new nodes because is equal for both strategies.
% TODO ENERO IG este frase del however no enlaza demasiado bien con la anterior
However, as explained above, our strategy does differ from \emph{Negative Broadcasting} in the use of TXT records.

The nodes announce records during the start up or whenever they have a resource record with new data.
Therefore, each time a record is updated, we send a multicast message that increases the network traffic.
In our solution the TXT record may change 
(1) when a new \emph{WP} is selected or
(2) when we update the time elapsed since it joined the space and its battery charge level.
The last two parameters need to be updated to select an appropriate \emph{WP} but they do not need to change too frequently.

% Regarding the first case,  % no pillo la introduccion del first case
In the most static scenario the TXT record is only written once.
The more dynamic scenario from the previous section, one on the contrary, updates that record 126 times after writing it for the first time.
This demonstrates that the overhead generated on the discovery system by our solution is minimal even in the worst-case scenario.

\section{Summary and Future Work}
\label{sec:actuation_summary}

This chapter presented two ways to actuate on the physical environment.
The first is the usual way to operate through spaces and provides a higher degree of decoupling.
However, it requires participants to use our middleware's primitives. % requiere la cooperación de los proveedores...
In other words, our middleware is not able to reuse third applications \ac{rest} services.


The second actuation mechanism directly consumes \ac{rest}ful \acs{http} \acsp{api}.
This mechanism relies in the semantic description of the services, additional knowledge and in a reasoning process. % additional knowledge: background + initial
With that information, it is able to generate executions plans towards a goal.
Following these plans implies different calls to the different services.


We implemented the same scenario using these two actuation mechanisms.
Besides, since interoperability is one of our middleware's guiding principles, we sketched how to reuse these \ac{rest}ful \acs{http} \acsp{api} in our \Space{} model in a third implementation.
This reuse avoids any alteration on the space-based consumer or the \ac{http} provider.
Instead, it improves the \Space{} implementation with an agent in charge of generating execution plans.
This agent reuses the information from the space-based actuation not to require any additional information to the developer.


This implementation alignment between our space-based actuation and the direct web \acsp{api} consumption one presents some limitations.
For some of these limitations, we discuss other design alternatives and compare them with the chosen one.
The rest of the limitations only appear in more complex scenarios.


For our future work, we will implement these complex scenarios where advanced conflicts between the \ac{rest} and space-based computing worlds can arise.
Besides, this dissertation does not answer how to reuse actuation mechanisms of the nodes using \ac{ts} patterns from \ac{rest}ful \acs{http} \acsp{api}.
Specifically, it would be interesting to experimentally test what would be necessary to seamlessly use our middleware's capacities from  third \ac{wot} solutions.


% ----------------------------------------------------------------------


\section{Energy-aware super-peer architecture} % pensar titulo más sexy
\label{proposal}

% This section discusses the content the clues should contain to make them suitable to the \ac{wot};
% we explain in detail the roles the nodes of our solution can have;
% we overview the deployment issues related to these roles, such as discovery or role transition;
% and we discuss how our solution can adapt multiple scenarios.

% AG: ¿node, peer y/o device? ¿Aclarar qué es un node?
% IG: personalment peer no lo veo, y entre node y device, pues creo que usarlos indistintamente, pero aclarando eso en algun punto
% AG: Crear comandos para asegurarnos de llamar a las cosas igual siempre (e.g. clues, acronimos, etc.)
% IG: comandos? esa es muy de crack xD

% En algún momento de la sección: CoAP vs HTTP
% AG: Igual habría que comentar que asumimos HTTP por su estado de madurez, pero que podría aplicarse a CoAP en cuanto fuese un estandar y con implementaciones que lo implementasen por completo. Esto tendría ventajas a nivel de uso de Multicast para agrupar WP, consumidores, etc.
%     Podríamos valorar su uso más minuciosamente en cada sección, pero igual en una primera versión no le daría mucho bombo.
% IG: influye en algo de la solucion es algo particular del experimental environment?
% AG: En su mayor parte no, pero habría algunas cosas que se simplificarían (e.g. comunicación con un grupo de nodos)

%   + Vinculo rapido con la intro (1 o 2 frases) % igual también con el related work
In a semantic \ac{wot}, nodes are part of a network where they share semantically described information.
%   + Mencionar el uso de espacios
These nodes gather their information in \Spaces{}
\footnote{
Note that in this chapter, the term \Space{} refers to a group of devices usually co-located which form an intelligent environment.
It has an obvious correspondence with our solution's \outerspace{}.
However, note that despite of this correspondence, the solution presented in this chapter transcends space-based computing.
}.

%   + Introducir ejemplo de hotel
For example, a hotel can create its own semantic \Space{} with the semantically described information of its services and the data provided by its devices (e.g., sensors) spread around the hotel.
Thus, clients can use their own personal devices (e.g., smartphones) to interact with the devices of the hotel and search useful information.
Using this approach, a client, for example, can check the current swimming pool occupancy rate using his personal mobile phone.

However, not all devices can afford the costs of fully managing the additional overhead of semantics.
This section presents our architecture to manage such environments and deal with the limitations of resource-constrained devices.

\subsection{Basic roles}
Devices can be part of one or multiple \Spaces{}.
A device can provide data (\providers{}), consume data (\consumers{}) or both at the same time.

\begin{description}
\item[Providers:]
This is the simplest role which any device in our system can carry out, even the smallest sensor.
\providers{} must manage their own semantic information.
In particular, they must organize triples in RDF graphs.
\item[Consumers:]
This is the role a device must take to get information from the system.
These devices need to be able to use semantics to find the data they require.
\end{description}

Note that a device could hold just one or both roles at the same time or switch between them.
For example, a \provider{} could become a \consumer{} to get data from the system or a \consumer{} could stop asking for data.
% Cambios de roles:
% To Consumer: try to read something from the network. => evidente
% To not-consumer: "t" without querying anything (it does not worth to update the clues if we won't query).


\subsection{Use of Intermediaries: \acl{wp}}
%   + Intermediaries
%   + Necesidad de tener white pages, mención de roles. => subsección N
Tasks involving the use of semantics can be expensive for some devices.
To reduce the load on such resource-constrained devices, we need to use intermediaries to carry out some of these tasks.
% No hacen tareas por ellos, más bien les evitan hacerlas.
%   + Por que queremos que ellos administren su info?
However, we do not want devices to completely rely on intermediaries for three reasons:
\begin{enumerate}
  \item Data consumers need to get fresh data and we can only get this by direct access to the actual \provider{}.
	% We aim to promote the direct communication between devices to ensure that freshness of the information shared.
  \item We must support mobile scenarios where personal devices carry their own data.
	% + Explicar un poco la naturaleza dinámica de los escenarios (sensores que entran y salen del sistema)
	% In particular, we consider the dynamicity of WoT environments, where nodes can join and leave a space making their information available or unavailable at any time.
  \item We need to reduce the maintenance complexity (nodes may join or leave at any time).
\end{enumerate}

For these reasons, we cannot rely on intermediaries to host and provide \emph{all} the semantic information.
However, we use intermediaries as searching enablers in the semantic \Space{}.
This kind of intermediary is what we call \acl{wp}.

\acp{wp} manage metadata or \clues{} about the information shared by others.
The \consumers{} rely on \acp{wp} to enhance the search process and reduce the number of requests generated.
Consequently, the \providers{} process less semantic data and reduce their overall overhead.
Thus, our architecture enables an intermediary-aided energy-aware information search.


\InsertFig{architecture}{fig:solutionArchitecture}{Role of White Pages in our proposal}{}{1}{}


Figure~\ref{fig:solutionArchitecture} shows the role of \acp{wp} in our proposal.
In particular, \acp{wp} manage \clues{} which summarize the semantic information shared by the \providers{}.
These \clues{}, which are described in detail in Section~\ref{sec:clues}, are pieces of information useful to determine which nodes can answer to a certain query.
\consumers{} use these \clues{} to directly access the semantic information on the \providers{}.
Summarizing, the main tasks of a \ac{wp} are:
\begin{itemize}
 \item Manage the \clues{} sent by all the nodes in the system.
 \item Aggregate clues sent by \providers{} in a unique response.
 \item Reply to requests from \consumers{} with aggregated clues or a list of nodes (details in Section~\ref{sec:interacting}).
\end{itemize}
 % debería haber un único \emph{WP} o varios? AG: Optaría por 1 para simplificar.
 % Dependiendo de eso, puede ser más complejo mantener una visión común de los clues de todos
 % (con algún algoritmo de gossiping?)
 % Para simplificar el merging de los differentes gossipings agregados se podría hacer: 1 master, N slaves.

\acp{wp} only store \clues{} for a period of time, expiring afterwards.
Using this approach, we avoid storing information from no longer available nodes.

% Si hubiese que dar detalles de implementación, había pensado en Redis para el \emph{WP} a poner en un server o cluster
% De forma que la gestión distribuida de dicho gossiping no fuese problema mio en esos casos
% Veo 2 casos:
%    + Un cluster de servidores hacen de WP: que usen algo que ya exista y me deje en paz.
%    + Dos/n móviles hacen de \emph{WP} conjuntamente: entonces sí que debería mojarme y decir como mantienen esa visión
%      conjunta.
% El problema es que al haber cambiado la explicación de tipos de cacharros a roles, esta consideración es difícil de
%  hacer.


\subsection{Versioning Clues}
As previously explained, the \ac{wp} aggregates clues sent by \providers{}.
This aggregation is versioned using a \emph{version number} and a \emph{generation number}.
The \emph{version number} increases each time the \ac{wp} receives a new clue update.
The \emph{generation number} ($g_{id}$) is a timestamp that represents the moment a new \ac{wp} was chosen.
This requires the clocks to be synchronized in all the nodes to avoid problems.
In any case, the \ac{wp} must guarantee that the $g_{id}$ is higher than the one used by a previous \ac{wp}.

The \ac{wp} maintains two version numbers:
\begin{enumerate*}[label=\itshape(\arabic*\upshape)]
  \item the one used during the setup to do the first load (\emph{setup version}) and
  \item the version of the last aggregated clue.
\end{enumerate*}
The first one is shared through the discovery mechanism.
The second one is shared with both \providers{} and \consumers{} in \acs{http} responses.
The versioning is used to improve the \ac{wp} selection process explained in Section~\ref{sec:selection}.


\subsection{Discovering a \acl{wp}}
% He matizado la frase para que no se entienda que los Consumer SIEMPRE dependen de los WP. Sólo para actualizar clues.
% AG: te has empeñado en quitar el matiz de que "sólo" al principio está vendido a la existencia de un WP :-)
In our architecture, when a \consumer{} joins the \Space{} it relies on a \ac{wp} to find information.
Hence, the first thing a node needs to do afterwards is discovering who is the \ac{wp} in that \Space{}.

To run our proposal, we require a discovery system able to
\begin{enumerate*}[label=\itshape(\arabic*\upshape)]
  \item get the \Spaces{} that a particular node belongs to,
  \item identify the \ac{wp} in the system and its \emph{setup version}, and
  \item provide additional information about nodes to decide which one can be the next \ac{wp}.
\end{enumerate*}
To implement our architecture, the information about the nodes must include:
% Estos parámetros se vuelven a explicar en la sección de "Selección de WP"!
battery level, available memory, storage space and the approximate time passed since the node joined the \Space{} for the last time (to estimate its reliability).

How to discover the nodes in the \Space{} is transversal to this dissertation.
We could either extend approaches like \ac{upnp} \citeweb{upnp}, \ac{mdns} \citeweb{mdns2013} or lmDNS \citep{jara_light-weight_2012} or use \acs{http}, \ac{coap} \citeweb{coap2013} and \ac{dns} together as proposed by \citet{ishaq_facilitating_2012}.
In particular, in Section~\ref{sec:mdns} we use \ac{mdns} to evaluate our solution.
% En la primera ronda de revisiones hubo un revisor que dijo:
%  "How does the system receive the information about the battery level, available memory?
%   The authors justify that this is out the scope of this paper, but it is important to know
%   the potential overhead introduced in these processes."
% Es una forma de responderle. Con esa frase quería decir:
%  " Esto no solo funciona en el plano teorico, usando un sistema como mDNS es perfectamente
%    posible implementarlo porque compartir esa pizca de info, pese a que te llegue un poco
%    desactualizada, no es un problema. "

% - Which is the topology of the network?
% It is hard to evaluate the proposed solution in terms of overhead and energy consumption for performing necessary tasks (WP discovery, WP selection, clues updates,...)
% without any knowledge of the network. Are there wired/fixed nodes? Are WP infrastructural nodes? Are the wireless nodes connected through multihop paths?
Note that we assume that any node in the \Space{} can reach to the other nodes.
In particular, we assume IP addressability (no matter whether the node is wired or wireless).
As a consequence, to use devices from Wireless Sensor Networks (e.g., Zigbee or 6LoWPAN) as \providers{}, one should rely on gateways.

% IG el CoAP multicast no tiene una utilidad diferente a lo q se habla?
% Se podría usar UPnP, mDNS o  incluso lo propuesto por Jara (lmDNS).
% También se podría usar multicast CoAP (e.g. los que respondan a /\ac{wp} o /consumers)


% Partimos de la base de que el sistema de descubrimiento nos ha descubierto a otros nodos y los siguientes datos acerca de ellos:
%    - Battery (B): valor [1 a N] y unidad [min, hora, dia] -> siendo -1 infinito o cargando.
%         Problema: puede que no sea fácil hacer ese cálculo.
%         Solución: mejor eso a carga relativa o medida de carga, cuando no sabes cuanto consume el cacharro de normal.
%    - Memory (M): Valor [1 a 1023] y unidad [KB, MB, GB]
%    - Persistence (P): Valor [1 a 1023] y unidad [MB, GB, TB]
%    - Joined since (JC): Tiempo desde su unión al espacio (para permitirnos medir su estabilidad): medido en ciclos de media hora
%    - Stores aggregated clue (SAC): [T o F] ¿Tienen una versión del clue agregado?
%
% Hemos tratado de no crear mucho tráfico en el mecanismo de descubrimiento almacenando datos que varíen con poca frecuencia.
% Los campos más variables: la batería y JC, pueden no actualizarse en media hora sin que el algoritmo se vea sustancialmente alterado.
%
% Comprobar que no son datos que costase mucho embeber en cualquier sistema (registro TXT, lo que sea que use UPnP...)
% Jara habla de 80 bytes en 6LoWPAN, probablemente tampoco nos orientemos a cacharros tan peques, pero oye, por comentarlo...


\subsection{Interacting with a \acl{wp}}
\label{sec:interacting}
When a node needs to access a piece of information, it asks the \ac{wp} for the information about the other devices in the \Space{}.
Using this information, the node can find the owner of the required data and ask it for this information.

In our architecture, \providers{} and \consumers{} have additional duties.

\noindent\textbf{Providers.}
They manage their own semantic information and generate \clues{} about the information they host.
Then, they send these \clues{} to the \Space{} \ac{wp}.
As a response, they receive the last version of the aggregated clue they have contributed to.
\providers{} will send their \clues{} to the \ac{wp}
\begin{enumerate*}[label=\itshape(\arabic*\upshape)]
  \item every time a \emph{clue} is updated,
  \item before the lifetime of a \emph{clue} expires, or
  \item whenever there is a new \ac{wp} in the \Space{} with a lower \emph{setup version} than the one in the \provider{}.
\end{enumerate*}
Note that \clues{} do not change frequently since they represent the type of information the nodes host rather than the specific data which is constantly generated.

\noindent\textbf{Consumers.}
When a \consumer{} needs to get information from the system, it first needs to find the \ac{wp} in the \Space{} and use this \ac{wp} to obtain an aggregated clue of all the nodes.
Then, it processes this aggregated clue to decide which are the nodes to query.
Finally, it sends the query through \acs{http} requests, one per provider identified in the previous step.

They perform this process synchronously for the first query and periodically in an asynchronous manner for the following ones.
% cada cuanto? arbitrario? que lo decida en base a parámetros de funcionamiento? => Mi propuesta abajo
This period should have an upper limit to ensure a fresh view of the \Space{} and a lower limit to guarantee that the \ac{wp} is not flooded.
To adjust the update frequency within these two limits, we evaluate the frequency of the last 10 requests to that node.
Thus, the view of the \Space{} will be fairly up to date when the \consumer{} processes the next query.
% TODO Añadir mejora: cuando se detecta un nuevo nodo por el mecanismo de descubrimiento, esperas un poco y actualizas clue para ver si ya ha enviado su info al WP.
% Mejora de la mejora: exclusivamente preguntas por la clue de el/los nuevo/s nodo/s detectado/s. Asi envias un JSON más peque.

\noindent\textbf{Optimizations for resource constrained devices.}
Note that \providers{} will update the \ac{wp} with new information and \consumers{} will periodically check the \ac{wp}.
Using this approach, the \ac{wp} reduces the network load and in particular, it decreases the load for resource constrained devices.

In addition, \emph{aggregated clues} can be too long for some really limited devices.
As an optimization for such devices, the \ac{wp} is able to answer specifically the node to address a query.
Thus, these nodes will only maintain a list of the nodes they should ask for several predefined queries.

% Limitar la información enviada también a otros nodos?
% Si tenemos 1024 nodos y cada gossiping es de 100KBs, vamos a tener 100MBs de gossipings :-S
% Aunque esto es justo lo que hay que discutir en el apartado de evaluación de clues...


\subsection{Selecting a \acl{wp}}
\label{sec:selection}
%   + Que queremos solucionar => subsección K (tipos de sistemas)
Depending on the setup, having a dedicated \ac{wp} (e.g., a high-end server) can be too expensive.
For example, in a domestic environment, it is not worth dedicating a full server to be a \ac{wp}.
However, when we have a large setup (e.g., a hotel), it is convenient to dedicate a few servers to decrease the load in small devices.
%   + Necesidad de que esos roles sean algo dinámico y el problema que ello presenta
%      (descubrimiento, pero sobre todo selección) => subsección M

We have to be able to manage the complexity of a system composed by heterogeneous devices.
For example, making a small device a \ac{wp} may be inappropriate in highly populated environments.
On the contrary, having multiple dedicated servers implies a high management overhead which is unnecessary in simple environments.
Our architecture is flexible and any node can be a \provider{}, a \consumer{}, a \ac{wp} or any of them at the same time.

\noindent\textbf{When does the \ac{wp} selection start?}
The selection process can start
\begin{enumerate*}[label=\itshape(\arabic*\upshape)]
  \item when no \ac{wp} is available or
  \item when the current \ac{wp} gets a worse score than other nodes.
\end{enumerate*}
In the first case, the first node to realize there is no \ac{wp} starts the selection process.
In the second case, the current \ac{wp} periodically checks if there is another node with a better score.
If this actually happens, it starts the process.
We limit the frequency for this checking to avoid the overhead associated with this change.

\noindent\textbf{How to select a \acs{wp}?}
The node in charge of selecting a \ac{wp} retrieves information of available nodes using the discovery mechanism.
It ranks them, selects the top one, and informs it that it should become a \ac{wp}.
If a selected node rejects the new role, the selector chooses the next one on the ranking.

The node in charge of the selection sends its aggregated clue to the new \ac{wp}.
If this aggregated clue is fresher than the one in the \ac{wp}, the \ac{wp} will use this one.
Otherwise, the \ac{wp} will use its own.
In case there is no previous aggregated clue, the new \ac{wp} will start from scratch (version to -1).
Once the node becomes the new \ac{wp}, it notifies the version used to initialize the discovery system.
Doing so, \providers{} will realize if they must send their \clue{} again because it is not included in the aggregation provided by the \ac{wp}.

% aclarado que facilitará esa información de cada nodo activo en el espacio
Note that the discovery mechanism must provide the following information about each node in the \Space{} to the selection algorithm:
\begin{enumerate*}[label=\itshape(\arabic*\upshape)]
  \item memory of the device,
  \item storage capacity of the device,
  \item time since it joined the \Space{}, and % esto no debería cambiar a menudo para no generar mucho trafico
  \item its battery charge.
\end{enumerate*}
We use the algorithm detailed in Listing~\ref{list:selectionAlgorithm} to rank the nodes.

\begin{listing}
  \begin{algorithmic}[1]
  \State $nodes \gets filter_{threshold}(nodes, ``memory'', threshold_{memory})$
  \State $nodes \gets filter_{threshold}(nodes, ``storage'', total_{nodes} \times storage\_needed_{avg})$
  \If { $anyWith(nodes, battery_{infinite})$ }
    \State $nodes \gets filter(nodes, ``battery'', battery_{infinite})$
    \State $nodes \gets orderBy(nodes, ``battery'')$
  \Else
    \If { $anyGreaterThan(nodes, ``joined\_since'', joined_{threashold})$ }
      \State $nodes \gets filter_{score}(nodes, ``joined\_since'')$
    \EndIf
    \State $nodes \gets filter_{score}(candidates, ``battery'')$
    \State $nodes \gets orderBy(nodes, ``memory'')$
  \EndIf
  \State \Return $nodes$
  \end{algorithmic}
  \caption{\acl{wp} selection algorithm.}
  \label{list:selectionAlgorithm}
\end{listing}

Lines 1 and 2 of the algorithm apply filters based on thresholds.
In particular, the second threshold is variable and depends on the number of nodes (more nodes will generate more \clues{} to store).
After that, we check the power availability of the devices.
We first select the nodes connected to the power grid (represented by $Battery_{infinite}$).
If there are not devices connected to the electrical grid, we select devices which are steady enough to apply a filter by $joined\_since$ field.
Filters in lines 8 and 10 choose nodes with z-scores higher than one for the specified fields.
If no node fits that filter, it returns the nodes with values higher or equal to the mean.

% AG: Intentando aclarar si un móvil puede ser WP o no (a un revisor no le había quedado claro)
Summarizing, the algorithm prioritizes energy autonomous nodes and within the nodes with battery limitations, it prefers steady ones.
% IG pues si quieres aclararlo, remata xD
In both cases, it finally selects a node with enough storage and memory (including mobile devices).

% AG: Problema a mencionar: y si no hay ningún \ac{wp} candidato? Broadcasting entre cacharros?

% \noindent\textbf{\ac{wp} conflict resolution}
Due to out-of-date information, two nodes may become \acp{wp} at the same time.
Our solution would eventually correct this situation thanks to a conflict resolution algorithm.
When a \emph{\acs{wp} A} detects another \emph{\acs{wp} B} in the \Space{}, it has to check which one has a better score according to Listing~\ref{list:selectionAlgorithm}.
If \emph{B} has a better rank, \emph{A} simply resigns as \ac{wp} and notifies it to the discovery system.
Otherwise, it forces \emph{B} to resign through an \acs{http} invocation.
Other nodes will be aware of these changes through the discovery mechanism.




\section{Shared Clues}
\label{sec:clues}
As we introduced before, \acp{wp} host \clues{} about the information in the \Space{}.
Using a \emph{clue}, a \consumer{} can find which node (or nodes) is the \provider{} of a piece of information.
\providers{} generate these \clues{} by digesting the semantic information they store.

Thanks to these \clues{}, resource constrained devices do not have to process unnecessary requests.
We also limit the length of the \clues{} to reduce the bandwidth, the memory, and the storage overhead on these devices.

This section describes in detail these \clues{}, the information they contain and their format.


\subsection{Querying Basics} % Querying assumptions

% Por qué una nueva sección?
% Porque estaba quedando muy abstracto todo y cosas básicas como cómo un consumidor consulta a un proveedor podía no quedar claro.
% O qué es un template y para qué se usa.

A \consumer{} directly queries the information to the selected \providers{}.
This selection is done using the \clues{} described in the rest of this Section.
However, in this section we explain how \consumers{} directly query for information to the selected \providers{}.

We assume that \providers{} offer an \ac{http} \ac{api} which accepts queries. % API o endpoint?
Whether this \ac{api} redirects the requests to the \ac{rdf} graph which describes a specific resource,
it creates a response with \ac{rdf} triples belonging to different resources or
it offers hyperlinks to the relevant graphs is transversal to our solution.
There is only one requirement regarding this \ac{api}: all the \consumers{} and \providers{} must agree on it.
In the context of this thesis, this \ac{api} corresponds with the one defined in Chapter~\ref{cha:tsc}.

% explicar query templates
To perform the queries, which enable the selection of a subset of the semantic content hold in a given \Space{}, we require a \textbf{template}.
We assume the use of wildcard templates, which are special triples with optional wildcard subject, predicate and/or object (see Figure~\ref{fig:semanticExample}). % TODO referencia al ejemplo o a figura de ejemplo
We could use more sophisticated query languages (e.g. \ac{sparql} \citeweb{sparql2008}),
but processing them may be too demanding for devices with constrained capabilities.
On the other hand, simple wildcard templates can be managed by any node able to manage \ac{rdf} triples.
In any case, complex queries can be decomposed into wildcard templates and our solution would still apply.


\InsertFig{semantic_example}{fig:semanticExample}{
  Sample triples and query templates
}{
  The top of the figure shows some sample triples.
  It highlights the different parts used for each type of \clue{}:
  the information used in class-based \clues{} in orange, the prefix-based \clues{} in blue, and the predicate-based \clues{} in brown.
  The bottom shows query template examples for these triples.
}{0.8}{}



\subsection{Content of a Clue}
To find out which is the most appropriate solution for \ac{ubicomp} scenarios, we have to consider scenarios populated by mobile devices and sensors. % for \ac{iot} scenarios?
Mobile devices usually share data which rarely changes and is described using a few ontologies (e.g., the user profile and his preferences).
On the other hand, sensors are constantly generating new instances of the same ontology (also called individuals).
In both cases, the data shared by each node is described according to one or few vocabularies or taxonomies.

At this point, it is important to define the \emph{TBox} and \emph{ABox} concepts following the definition of \citet{nardi2003introduction}.
\emph{TBox} contains knowledge describing general properties of concepts or terminology and
\emph{ABox} contains knowledge specific to the individuals of the domain of discourse.
An example of \emph{TBox} information is the device type or the elements which is made of,
while \emph{ABox} can describe the mobile phone brand or the temperature sensed by a thermometer.

Each mobile device or sensor usually generates \emph{ABox} according to the same \emph{TBox}.
Using this information, we propose avoiding the use of URIs which represent \emph{ABox} information in general terms.
Specifically, we propose and evaluate three possible type of \clues{} to share.
%The first two are inspired by early works on peer-to-peer semantic web $[CITAEDUTELLA,devadithya2007index]$.

\medskip

\noindent\textbf{Prefix-based clue.}
%A coarse-grained step to find relevant nodes for a query is to ignore those which do not have \emph{ABox} for a specific \emph{TBox}. % no sólo aplicable al caso de la ontologias
The individuals (\emph{ABox}) are described according to different ontologies (\emph{TBox}).
Each ontology usually employs a common \emph{namespace}.
This means that the URIs of the concepts in these ontologies share a unique \emph{prefix}.
Besides, each \provider{} also uses a few \emph{prefixes} to generate the URIs of its individuals.
In this type of clue, we propose a coarse-grained method to filter nodes which do not have individuals with the prefixes used by the query template.
To create prefix-based clues, the \providers{} have to extract the prefixes used in their graphs. % obvio?

Following the example in Figure~\ref{fig:semanticExample}, a \provider{} \emph{P} may have that graph.
Thanks to the \clues{}, the \consumer{} knows that \emph{P} has URIs starting with \emph{sweet}, \emph{ssn}, \emph{ex} and \emph{weather}.
Consequently, it will send to \emph{P} a query which uses any of the templates shown in Figure~\ref{fig:semanticExample}.
On the contrary, if the template is \emph{?s pref2:sth ?o}, it will avoid sending the template to \emph{P}.


\medskip

\noindent\textbf{Predicate-based clue.}
The predicates relate subjects with other subjects or literals.
These predicates are defined in the \emph{TBox} (e.g., to state they relate concept \emph{A} with concept B) and used in each triple of the \emph{ABox}.
In this approach, we propose extracting the set of predicates used in the graphs stored by each node.
Using this information, they can be simply matched with the predicate defined in the query template.
% AG: Posible pega, se ha explicado antes lo que es un "template"?
% IG: nope, de hecho en la anterior tb me ha surgido la duda de q era

\medskip

\noindent\textbf{Class-based clue.}
In the third approach, we propose sharing the classes of concepts (\textit{rdf:type}) provided by the nodes.
Using this information and the \emph{TBox}, each node can check if the information matching certain query template is susceptible to be stored in other nodes.
For this kind of clues, we assume that each node should have (or be able to obtain) the \emph{TBox} related to the template used for querying.
This is a reasonable assumption since the ontologies which mainly describe the \emph{TBox} are usually accessible on the URL described by its prefix.

To understand how it works, consider that, according to the \emph{TBox}, a predicate \emph{p} relates the concept \emph{A} with another concept.
We assume that the nodes storing instances of the class \emph{A} are more likely to use this predicate in their graphs.
Therefore, if a query template uses that predicate, we will send it to all nodes storing instances of the class \emph{A}.


% IG: esto de referirte a una ontologia del futuro... no me parece muy asines
% AG: Exactamente mismo caso que arriba, pero con ejemplos concretos.
%     La idea era que no quedase muy abstracto. Lo comento porque ahora puede quedar redundante.
% For instance, consider that the following template is queried: \textit{?s~ssn:observes~dbpedia:CO2}.
% According to the SSN ontology, which will be presented later on, only an instance of the class \textit{ssn:Sensor} \textit{observes} something.
% Therefore, the querying node will send the template to the nodes which have instances of the class \textit{ssn:Sensor}.


\subsection{Reasoning to expand clues} % Esto se relaciona principalmente con class-based, pero igual también puede comentarse para Predicate-based.
Through a reasoning process one can know unstated knowledge.
Using this knowledge, we can detect more relevant nodes.
For example, let us consider the previous example where the class \emph{A} defines the predicate \emph{p}.
Providing that \emph{C} is a subclass of \emph{A}, a node which has many instances of the class \emph{C}, may also use the predicate \emph{p}.
Thanks to the reasoning, we can discover that the node has knowledge of the type \emph{A} and therefore, it may be relevant.
% AG: Comentado por lo mismo que arriba
%(in the example, instances of \textit{ssn:Sensor} class' subclasses such as \textit{Accelerometer}).
The drawback, is that reasoning consumes a lot of resources.
This limitation will be further analysed in Section~\ref{sec:clues_eval}.
% TODO Many authors solve this limitation by transforming the queries in the consumer side \cite{}.


\subsection{Use of ABox in clues}
\label{sec:aboxinclues}
% se podrían usar individuals que no cambian mucho
% location, individual en base al cual se generan todos los elementos.
As stated before, in general terms, we want to avoid the use of \emph{ABox} \acp{uri} (individuals) in our clues.
Thus, we fulfil two goals:
\begin{enumerate*}[label=\itshape(\arabic*\upshape)]
  \item generate smaller \clues{} and
  \item the \clues{} will not change too frequently and therefore, less communication to update clues will be required.
\end{enumerate*}
However, in some cases, the use of \emph{ABox} content in the \clues{} may be beneficial.
For instance, assume a \ac{uri} that refers to the specific location \emph{L}.
If we want to search for devices in location \emph{L}, we cannot deduce anything about it using the proposed \clues{}.

% AG: Atencion, a partir de aqui viene una propuesta que me acabo de sacar de la manga que habra que discutir.
%     El problema de esto es que no está en la evaluación y para el 15 de Octubre, lo veo ajustado de incluir.
For this reason, we need to consider sharing the \emph{N} most queried individuals in our clues.
To do that, the \ac{wp} needs to store a list with the statistics about the information collected by each \consumer{}.
\consumers{} can send this information together with the request to update a clue.
\providers{} can obtain a list of the current most popular \acp{uri} before sending their updated \clues{} to the \ac{wp}.
Using this list, \providers{} can know if they have these \acp{uri} and include them in the \clue{} to be sent to the \ac{wp}.
% TODO el proceso no me acaba de convencer, y cuando hay una nueva URI? Como saben los proveedores que deben actualizarla?
Note that this process would imply an extra request per \provider{} before each update.

This simple approach implies sending not only \emph{TBox} but also \emph{ABox}.
The amount of extra information added to each \clue{} will depend on the size of this list (\emph{N}).
The effectiveness of this method will depend on the number of queries using one of the \emph{N} \acp{uri} in their subjects or objects.
% Propuesta: uso de un porcentaje de las queries realizadas
% Problema: no vas a raspar a cada providers con 1000 comprobaciones tampoco!
% Solución: ¿merece la pena evaluar eso? ¿como?


\subsection{Format}
% Tras explicarlas, poner algunos fragmentos con las clues (JSON) para que se vea de que hablamos
Many formats to represent the content of a clue can be used.
% pongo esto de EXI porque se que a la gente del mundillo de IoT se le hace el culo chupicola con sus estandares
One option is the ongoing \acf{exi} format \citeweb{exi2011}. % o cite?
\ac{exi} is designed to efficiently interchange XML data and therefore, we could obtain better compression rates than with \ac{json}. % aseveracion cierta o solo para XML?
However, we have chosen \ac{json} for its simplicity and its wide adoption in the \ac{www}.

% TODO no lo he acabado de pillar...
The prefix-based \clue{} is the easiest one to represent in \ac{json} since it is formed by a set of \acp{uri}.
We can also represent the predicate-based and the class-based \clues{} using a set of \acp{uri}.
However, the prefixes of those \acp{uri} are usually repeated and for this reason, we do not use plain \acp{uri} transmission for these clues.
We show an example in Listing \ref{list:oneClue}.
It first defines the prefixes used and gives them a name and then, it specifies the \ac{uri} endings for each prefix.

\begin{listing}
  \begin{minted}[fontsize=\scriptsize, frame=single, framesep=3mm, linenos=true, xleftmargin=21pt, tabsize=4]{js}
{
  "s": [
    [ "so", "http://knoesis.wright.edu/ssw/ont/sensor-observation.owl#" ]
  ],
  "p": {
    "so": ["result", "procedure", "observedProperty", "samplingTime"]
  }
}
  \end{minted}
  \caption{
    Representation of a predicate-based \clue{} in \acs{json}.
    The node sending the \clue{} has \acs{rdf} triples which use the predicates
    \emph{so:result}, \emph{so:procedure}, \emph{so:observedProperty} and \emph{so:samplingTime}.
  }
  \label{list:oneClue}
\end{listing}

\medskip

These are isolated \clues{} sent from a \provider{} to the \ac{wp}.
However, the \ac{wp} gathers all these \clues{} in an aggregated clue which is sent to the \consumer{}.
We show an example of an aggregated clue in Listing \ref{list:aggregatedClue}.
As one can see, the aggregated clue defines the type of \clues{} wrapped through the numeric field \emph{i}.
\emph{G} and \emph{v} form the version of this aggregated clue.
The field \emph{s} defines the prefixes.
Finally, for each node, each prefix is related with the URI endings.

% TODO sacar un ejemplo real? poner esquematicamente?
\begin{listing}
  \begin{minted}[fontsize=\footnotesize, frame=single, framesep=3mm, linenos=true, xleftmargin=21pt, tabsize=4]{js}
{
  "i": 1,
  "g": 2435467,
  "v": 556,
  "s": [
    ["dc", "http://purl.org/dc/elements/1.1/"],
    ["dul","http://www.loa.istc.cnr.it/ontologies/DUL.owl#"],
    ["ssn", "http://purl.oclc.org/NET/ssnx/ssn#"] ],
  "p": {
    "node1": {
      "ssn": ["observedBy", "observationResult"],
      "dul": ["isClassifiedBy"]
    },
    "node0": {
      "ssn": ["observes"],
      "dc": ["description"]
    }
  }
}
  \end{minted}
  \caption{
    Representation of an aggregated clue in \ac{json}.
    Line 2 defines that it embeds predicate \clues{} (i.e. type 1).
    Lines 3 and 4, contain the version of the aggregated clue.
    The remaining lines express the predicates used by two nodes.
    For example, \emph{Node1} has at least a RDF triple which uses the predicate \emph{ssn:observedBy}.
  }
  \label{list:aggregatedClue}
\end{listing}

% Mencionar: Note that we can gossip sleeps periods to know when to query them...
% Comentar que se podrían gossipear los sleeps de los nodos embebidos para saber cuando interrogarles.
% AG: Esto es un aspecto que puede ser guay comentar, de cara a IoTificar la propuesta.
%     Pero pensandolo mejor, debería ir en la sección 3, a la hora d explicar el formato de las
%     clues.
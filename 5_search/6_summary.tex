\section{Summary}
\label{sec:search_summary}

This chapter presented a dynamic architecture to enhance the search of semantic contents in the \acl{wot}.
%  by any node part of the Web of Things.
In particular, this architecture chooses an intermediary according to its capabilities to support resource constrained devices.
Intermediaries can use different types of clues to summarize the information in the semantic \Space{}.
% managed by the nodes belonging to a space.
Thanks to this support, the devices can directly interrogate others to obtain fresh information while reducing their semantic overhead.

% Esto lo he cambiado porque Ipiña decía que habría que volver a aclarar porqué lo he comparado con flooding en la conclusión (por si alguien no se ha leído el paper ;-))
The main characteristic of our solution is the ability of the devices to share semantic data directly, no matter how simple they are. % o algo como guides our design
Under this assumption, a flooding-based strategy would obtain a high recall. % or a high fraction of relevant answers (por si no se entiende aún que es recall)
However, our evaluation shows that our solution requires fewer messages between devices than a flooding-based strategy (i.e., is far more precise).
% he puesto more para que no tenga la acepción peyorativa de "many" (no se requieren muchísimos más para que sea mejor)
Even if we use caching strategies to alleviate these effects, our solution performs better for scenarios with more data consumers and query types.
In addition, our approach reduces the workload of mobile and embedded devices which indirectly results into energy savings.

% Para trabajo futuro, se podría considerar: 
% query optimization y citar a alguna solución de distributed triple stores
For our future work, we will consider allowing more expressive query languages such as SPARQL. % use acronym
This would force \consumers{} to discriminate between the \providers{} able to process it and the rest.
%The latters, could implement a trade off solution to send multiple triple pattern templates at once. % optimizando la red
On top of these queries, we could use query optimization techniques in the \consumers{} \citep{schwarte_fedx_optimization_2011}.
Using these techniques, we could transform the queries before sending them to
\begin{enumerate*}[label=\itshape(\alph*\upshape)]
  \item obtain better results or
  \item use the network more efficiently.
\end{enumerate*}
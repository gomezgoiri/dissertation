\section{\acl{sw} for \acl{ubicomp}}
\label{sec:soa_sw_ubicomp}

% Sin entrar en distinciones tipo: REST (o WoT) + SW y Ubicomp + SW
% Sólo complicarían el índice y la redacción

In this section we analyze some notable works for \ac{ubicomp} environments which use the \ac{sw} to represent the contextual information.
Rather than presenting an in-depth analysis of different architectures or their applications to concrete \ac{ubicomp} scenarios,
we focus on the resource constrained devices' perspective.
We scrutinize the role of the mobile and embedded devices in the systems which use the \ac{sw}.


\subsection{\acl{sw} using Intermediaries}
\label{sec:sw_intermediaries}

% como los smart environments describen contexto usando web semántica

% uso concreto por parte de soluciones significativas: siempre centralizando el uso de semántica en cacharros grandes

% intro a que ahora se va a hablar de soluciones IoT que usen semántica

Adding semantics works well for devices with high computational capacity but may add too much overhead for limited devices.
To reduce this overhead, part of this computation is usually delegated to an intermediary.
Some noteworthy example is the one proposed by \citet{broring_semantic_2009}.
% buscar otros ejemplos de enjundia
% The Context Broker Architecture (CoBrA)[Che04] 
% Gu et al. (2007)
% AlarmNet (2008)


% TODO meter alguna referencia a Cabilmonte sobre SPARQL en streaming
% Trata de consultar datos semánticos en streaming a una base de datos relacional.
% Por lo que entiendo, la comunicación sensor-DB se no es semántica.


\InsertFig{venn-sec2}{fig:venn_semantics_ubicomp}{
  The \acl{sw} for \acl{ubicomp}.
}{
  Scope of this section.
}{0.6}{}


These intermediaries or \emph{Semantic Gateways} are in charge of managing the semantic annotation.
The devices send raw data (which can be compressed) to the intermediaries and the gateways annotate the content semantically.
Thus, the devices do not have to care about any semantic aspect and just collect the data as they did before.

These \emph{Semantic Gateways} reduce the load to embedded devices with limited resources by decreasing the number of requests they have to provide.
In addition, a centralized intermediary can gather all the information and thus, reduce the complexity of managing a distributed environment.

However, using intermediaries to store the semantic data of resource constrained devices also has some drawbacks.
On the one hand, centralization does not faithfully represent mobility situations were individuals carry their own semantic information in their personal devices.
In addition, the data obtained from an intermediary will always be less fresh than the one obtained where it is generated (i.e., sensors).
On the other hand, the servers are critical in centralized systems and therefore, their availability determines the operation of these solutions.
They also impose a burden on the maintenance which may be worthless in some simple scenarios.



\subsection{\acl{sw} in Providers}
\label{sec:sw_providers}

% pero ahora los cacharros cada vez son más potentes y no es difícil imaginar un mundo poblado por ellos blablah
Lately, the computing capabilities of some mobile and embedded devices have improved enough to afford semantic processing. % poner ejemplos de FoxG20, RaspberryPi, Androids varios?
As a consequence, some solutions have arisen to semantically annotate data where it is generated. % in the provider
% justificar por qué todas las soluciones que presento son HTTP y no bluetooth o lo que sea? => mejor no meterme en lios, que se sobreentienda que es lo que mola


% Móviles y semántica como paso previo a WoT
\citet{daquin_enabling_2011} presents an architecture to deploy an \acs{sparql} \citeweb{sparql2008} endpoint in Android devices \citeweb{android}.
In their solution the mobile devices store and manage their own semantic knowledge by means of an embedded semantic repository.
On top of these endpoints, \citeauthor{daquin_enabling_2011} envision a query federation mechanism.
This dissertation shares this view of independent data sources sharing their own managed semantic content. % evitando entrar otra vez en el rollo de que también hay TS y blablablah


In the \acl{wot}, multiple solutions have considered using semantics to enrich the data definition in a machine processable manner.
Some solutions embed the metadata in HTML using microdata \citeweb{microdata2012}, microformats \citeweb{microformats} or RDFa \citeweb{rdfa2013}.
% \citep{mayer_extensible_2011} propone mapear estrategias de descubrimiento de recursos (RDFa, JSON, etc.) a un meta-formato interno basado en microformatos
These contents are returned by the Internet connected objects and are used to enhance the search-ability of the data by existing third party search engines.
% Ontologías no se pueden definir, tiene que haber consenso en su uso por parte de las máquinas de búsqueda. => eso en RDFa no es así!
% ojo que RDFa y microdata en realidad es rollo RDF-based, podría llevar a confusión el inicio del siguiente parrafo
% en vez de eso, queremos mejorar la busqueda por parte de los cacharros, no de los search engines
% This approach does not focus on making the devices able to search and interact with others.
% end-to-end search
% Instead, it considers them as mere human-oriented information providers which should be indexed by third party search services.
% aquí se puede enganchar también el discurso de Doulkeridis2007desent: coverage and scalability, freshness y monopoly
However, these search engines do not consider the mobile nature of the data providers in the \ac{wot}. % They can move from one context to another frequently.
To solve this problem \citet{trifa_leveraging_2011} propose a hierarchical discovery and lookup infrastructure.
This infrastructure extracts the semantic annotations from different representations and converts them to an internal data model.
This model is based on existing microformats, but limited to certain concepts.
Besides, the particularities of the semantic annotations are hidden to searchers in their custom search \acs{api}. % custom and non-standard


A more general way to represent semantic data is using RDF \citeweb{rdf2004} based representations (i.e., full semantics). % más general que los microformatos que usan Guinard y compañía
SPITFIRE European project \citeweb{spitfire} represents the most remarkable effort on gathering full semantics and the \ac{wot}.
It focuses on fully integrating sensor data with the \ac{lod}. 
The \ac{lod} are datasets which follow a series of principles on how to open and publish data.
The goal of the \ac{lod} is to publish linked terms using full semantics.

SPITFIRE shares with our solution the vision of a world populated by devices acting as semantic data providers no matter how small they are \citep{hasemann_rdf_2012}.
Therefore, many of their efforts are complimentary to this work.
%This is done by defining an ontology for mapping other common ontologies and providing semi-automatic generation of semantic annotations from raw data.
%They also propose an abstraction to represent real-world entities (virtual sensors) using data provided by low-level sensors.
%Finally, and more specific to this work, they propose a searching model which predicts the current state of things by computing their periodic patterns in past states.
To search within these providers, \citet{pfisterer_spitfire:_2011} proposes a model which predicts the current state of things by computing their periodic patterns in past states.
Again, the goal of this method is to adapt search engines to the new fashion of data provided on the \emph{\acl{sw} of Things}.


\subsection{Discussion}

Resource constrained devices have been extensively used to provide information to represent the physical environments.
When this information has been semantically annotated, traditional solutions delegate on intermediaries the semantization of the raw data.
However, recent advances in mobile and embedded devices' capabilities have enabled to increase their responsibilities.
Some of this devices are now able to personally manage and share their own semantic content.
This completely distributed approach tries to simplify the maintenance burden and promote the access to the most updated data.


This dissertation aims to explore this path.
Specifically, the search architecture presented in Chapter~\ref{cha:searching}, tries to enhance searching capability of web-connected devices.
It leads to environments where Internet-connected objects, mobile devices or regular computers directly interact and collaborate with each other. % que se note que no sólo hablamos de WoT
This contrast with the searching mechanisms presented by equivalent solutions in Section~\ref{sec:sw_providers}.
They assume that these limited devices need more powerful machines to search, which is not always true. % o third party search services.
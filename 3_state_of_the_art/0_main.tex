% ----------------------------------------------------------------------

\begin{savequote}[50mm]
I was born not knowing and have had only a little time to change that here and there.
\qauthor{Richard P. Feynman}
\end{savequote}


\chapter{State of the Art}
\label{cha:stateoftheart}

% the code below specifies where the figures are stored
\ifpdf
    \graphicspath{{\pathchapthree/figures/PNG/}{\pathchapthree/figures/PDF/}{\pathchapthree/figures/}}
\else
    \graphicspath{{\pathchapthree/figures/EPS/}{\pathchapthree/figures/}}
\fi


%------------------------------------------------------------------------- 

% TODO IG yo creo que deberias orientarlo a un state of the art, ahora suena a related work. Tambien creo que se deberia introducir un poco mas, del estilo de: este capitulo presenta el related work para dar X y Y.
% TODO uniformizar labels!
After having presented the general research topics in which this works fits, we analyze the specific works closely related to ours.
This thesis focuses on the intersection of the three topics represented in the Figure~\ref{fig:venn_ubicomp_tuple_semantic}. % puede liar que REST no aparezca y UbiComp si? (diff con Chapter 2)
It also stresses in which areas the following sections focus on.


% TODO definir en algún puto lado a qué nos referimos con Ubicomp
%      para mostrar porque no incluimos otros trabajos
Firstly, Section~\ref{sec:soa_ubicomp} explores relevant works on the \ac{ubicomp} field.
Rather than presenting an exhaustive (and exhausting) survey on middlewares for \ac{ubicomp},
we analyze those which follow the \ac{rest} architectural style.
Then, we remark how we can solve the main limitations of these works by using a \ac{ts} middleware.


Secondly, Section~\ref{sec:soa_ts_ubicomp} scrutinizes relevant \ac{ts} proposals for the \ac{ubicomp} field.
Specifically, we study their space model and, for the distributed proposals, how they distribute the tuples.
Then, we present and justify our model's similarities and divergences with theirs.


Thirdly, Section~\ref{sec:soa_sw_ubicomp} presents some semantic works for \ac{ubicomp}.
From these works we derive two main approaches to semantically annotate content provided by limited devices:
\begin{enumerate*}[label=\itshape\alph*\upshape)]
  \item do it in more powerful intermediaries; and
  \item directly involve all the devices in the management of their content.
\end{enumerate*}
This work is interested in the second approach.
Regarding works following this approach, we inspect their complementary proposals but we distinguish the unaddressed need covered by Chapter~\ref{cha:searching}. %: direct search between devices.


Finally, Section~\ref{sec:soa_tsc} analyzes and compares in detail other semantic \ac{ts} middlewares independently of their application domain.



\InsertFig{venn-summary}{fig:venn_ubicomp_tuple_semantic}{
  Background areas of the thesis, which focuses on their intersection.
}{}{0.7}{}


\section{The \acl{wot} and other \ac{rest} solutions for \acs{ubicomp}}
\label{sec:soa_ubicomp}

The \ac{ubicomp} idea is characterized by the presence and collaboration of a number of heterogeneous and often limited computing platforms.
% REST está en todos los lados, pero...
Dealing with this heterogeneity is probably the biggest benefit which \ac{rest} has brought to this field.
A representative example of how \ac{rest} has influenced \ac{ubicomp} is the \acf{wot}.
% \ac{wot} is probably the most prominent architectural solution used for smart-objects. % probably porque no tengo datos que lo defiendan


\InsertFig{venn-sec1_1}{fig:venn_ubicomp}{
  Non-semantic works for \ac{ubicomp} apart from \aclp{ts}.
}{
  This subsection focuses on the solutions using \ac{rest}.
}{0.6}{}


The \acl{wot} initiative encourages the use of \acs{rest}-based solutions embedding web servers in daily objects \citep{guinard_internet_2011,guinard_thesis_2011}.
In this way, the things can seamlessly integrate with the \ac{www} as first-class citizens \citep{gupta_network_2011}. % decir ``f-c cit of the Web" es reiterativo ya que WWW==web
This integration brings the following benefits:
\begin{itemize}
  \item The smart-things can be linked to enable its discovery by browsing.
	This involves using the tool most users are familiar with: the browser.
	% TODO The pervasive presence of web tools and libraries!!!
  \item They can be bookmarked or shared through social networks \citep{guinard_sharing_2010}.
  % explicar qué es un mashup?
  \item They can be integrated with other web applications through mash-ups \citep{guinard_towards_2009,ostermaier_webplug:_2010,pintus_anatomy_2011,blackstock_wotkit:_2012,stirbu_towards_2008}.
  \item Existing mechanisms such as searching, caching, load-balancing and indexing can be used over the objects to achieve the scalability of the web \citep{gupta_early_2010}.
\end{itemize}


% TODO TODO TODO Vestir un poco más esta sección con trabajos relevantes para mi tesis.
% TODO TODO TODO citar los trabajos de WoT que en su momento me molasen, porque si, bien por ellos :-P


\subsection{Discussion}


In the last decade, probably because of its perceived simplicity \citep{guinard_search_2011}, \ac{rest} has gained adopters in the \ac{iot}.
Particularly, the \ac{wot} has emerged as a simple mechanism to integrate smart objects with each other, but also with existing web applications.

However, as any remote invocation style \ac{rest} introduces coupling between senders and receivers.
% de johanson_extending_2004 he sacado esas características, no la particularización a WoT
These complicates the application changes both in the short-term and in the long-term \citep{johanson_extending_2004}.
In the short-term because nodes constantly join and leave the environment due to mobility or to failures.
In the long-term because the space is used to solve new problems and the obsolete devices are replaced with new technology.

In this dissertation, we propose a middleware to reduce this coupling.
This middleware ensures that, from the developer perspective, the communication is always driven by the data.
In other words, in our middleware the devices are unaware of others' presence (i.e. they are \emph{space uncoupled}). % REST + ubicomp (particularmente WoT)
% Ignorar esta sección? Es realmente necesaria cuando ya he puesto la comparativa de TSC?
\section{\acl{ts} on \acs{ubicomp}}
\label{sec:soa_ts_ubicomp}

So far, \acl{ts} has been adapted to \acl{ubicomp} by different authors.
This section presents the most relevant ones. % que sean más relevantes o no, es algo subjetivo...


\InsertFig{venn-sec1_2}{fig:venn_tuplespaces_ubicomp}{
  \aclp{ts} solutions for \acs{ubicomp}.
}{
  Scope of this subsection.
}{0.6}{}


% TODO leer artículos que me faltan y corregir
The \emph{event heap} \citep{johanson_extending_2004} is a system used for a specific \ac{ubicomp} sub-domain: interactive workspaces.
In this scenario, there are rooms with different devices deployed and where mobile devices can enter.
Each room has its own space where the devices exchange tuples to cooperate.
% For example, a video can be presented in a display and through a remote controller, the user can place the tuple for pausing it.
% The display consumes these kind of tuples, so it pauses the video whenever somebody places that tuple on the space.
% This system is simple to implement but is limited since it centralizes the space in a single machine per room.
This work merely identifies these environments' requirements and the properties to cover them. % to cover the reqs
Then, it discusses how these properties can be satisfied using \ac{ts} or some extensions.
Finally they compare their implementation both with other \acp{ts} and other coordination infrastructures. % e.g. RMI, MOM, Pub/sub


\emph{L$^2$imbo} \citep{davies_l2imbo:_1998,friday_experiences_1999} replicates the tuples to avoid a single point of failure.
Each node joined to a space uses an IP multicast address to exchange messages with other nodes in that space.
Writing into a space involves sending a multicast message to inform to the rest of the nodes of the tuple written.
Reading operation can usually be satisfied locally.
Destructive reading of a tuple is more complex as it requires a global withdrawal.
In \emph{L$^2$imbo} only the owner of a tuple can remove it from the space.
The ownership of a tuple initially belongs to its creator, but can be transferred. % TODO ver qué añadió Friday


\emph{LIME} (Linda in a Mobile Environment) \citep{picco_lime:_1999} is a \ac{ts} solution for mobile systems.
In LIME each mobile device has its own space where it generally writes its tuples.
This space is shared with other devices creating federated spaces, i.e. the aggregation of different shared spaces.
In this way, each mobile can access to tuples in other mobiles whenever they become available.
They also proposed a new writing primitive to insert tuples in remote spaces.
% ejemplo de service discovery?
\emph{LIME} has subsequently been adapted to limited platforms \citep{murphy_transiently_2006} with TinyLIME \citep{curino_tinylime:_2005} and TeenyLIME \citep{costa_programming_2007}.

\begin{sloppypar}
However, \citet{coulouris_distributed_2012} complain about the unrealistic assumptions LIME's authors make to simplify the problem. % cita al libro
These assumptions are the uniform multicast connectivity between devices whose tuple spaces are aggregated and the serialized and ordered connections and disconnections.
\end{sloppypar}


In the \emph{TOTA} (Tuples On The Air) Project \citep{mamei_programming_2009} tuples are disseminated to different devices.
To that end, each tuple has 3 fields:
\begin{enumerate*}[label=\itshape(\arabic*\upshape)]
  \item the content of the tuple;
  \item a rule which defines how it should be propagated; and
  \item a rule to define its maintenance.
\end{enumerate*}

For instance, they consider a museum where a visitor writes a query tuple describing a piece of art he wants to see.
The propagation rule defines that it should be propagated to all nodes in the vicinity, increasing the distance by one each time.
The tuple is configured to be deleted after a time-to-live period using its maintenance rule.
When it reaches the room where the piece of art is located, this art work writes a response tuple.
This response tuple jumps from a device to another until it reaches the device which queried for it.


% TODO Añadir tablita a modo de resumen?
% \begin{table}
%   \centering
%   \begin{tabular}{ c | c c c c }%p{5cm}}
%       ~ & The event heap & L2imbo & LIME & TOTA \\
%       \hline
%       Distributed & No & Yes & Yes & Yes \\
%       Objective & - & Availability & Federation & Ad-hoc \\
%       % Scalability?
%       % Coger propiedades del libro de sistemas distribuidos?
%   \end{tabular}
%   \caption{Comparison of the most prominent \acl{ts} alternatives for \acl{ubicomp}.} % no columpiarme, o poner todas o ninguna?
%   \label{tab:ubicomp_ts_comparison}
% \end{table}


% TODO Analizar otras como TOTAM o CRIME?


\subsection{Discussion}

% vemos el espacio necesario para:
The space can be used to
\begin{enumerate*}[label=\itshape(\arabic*\upshape)]
  % escribir y extraer para coordinarse con otros
  \item coordinate with other devices by writing and extracting content and
  % consultar que pasa en un entorno
  \item check what happens in the environment by reading.
\end{enumerate*}
% sin embargo, hay veces donde esa escritura y extraccion es mejor delegarla a los responsables de ella
The solutions presented integrate both uses in the same space, which can be distributed or not.
However, we argue that these uses face different needs and, therefore, should be treated separately.


% Teorema de CAP
The first usage demands consistency to avoid the unexpected consequences of two devices extracting the same tuple.
For this usage, we propose a typical coordination space which will be accessed through a \emph{resource oriented} \ac{http} \ac{api}.
% the space itself: el systema es distribuído (CS), otra cosa es que el espacio lo sea
Whether the space itself is distributed and how, is out of the scope of this dissertation and left up to the implementer.
%However, following the CAP theorem, we believe that the availability is more necessary than the partition tolerance.
%Otherwise the nodes using the space, will not be able to coordinate. % disponibilidad del espacio o de los nodos que pertenecen al mismo?


% 2do problema:
%    queriamos hacer a los dispositivos todo lo dependientes que fuese posible del espacio
For the second usage type, we aim to integrate as many external data sources as possible.
This integration demands to acknowledge their independence, and therefore limit the collaboration requirements imposed to them.
%The rationale behind this decision is that we believe that some content may be better managed by the device which creates them.
% ejemplo
%For instance, measures of a sensor or a user profile in a smartphone.
Consequently, for the second usage we propose a distributed space which resembles more to \emph{LIME}'s \emph{federated spaces} than to \emph{TOTA} or \emph{L$^2$imbo}.


In other words, each device locally manages its own part of the space and only collaborates with others to share content.
This means that writing does not imply replication or migration of the \emph{tuples}.
The main drawback of this decision is that the availability or the data depends only on its manager's availability.
In other words, this second space sacrifices the \emph{time uncoupling}.
However, this space reflects the current state of the environment and reduces the dependency of two devices in third ones.
Besides, for those cases where \emph{time uncoupling} or \emph{data availability} are important, the first space can still be used. % TS para ubicomp
\section{The \acl{sw} on \acs{ubicomp}}
\label{sec:soa_sw_ubicomp}

% Sin entrar en distinciones tipo: REST (o WoT) + SW y Ubicomp + SW
% Sólo complicarían el índice y la redacción

This section analyses some notable works for \ac{ubicomp} environments which use the \acf{sw} to represent the contextual information.
Rather than presenting an in-depth analysis of different architectures or their applications to concrete \ac{ubicomp} scenarios,
we focus on the resource constrained devices' perspective.
We scrutinize the role of the mobile and embedded devices in the systems which use the \ac{sw}.


\InsertFig{venn-sec2}{fig:venn_semantics_ubicomp}{The \acl{sw} for \acl{ubicomp}}{Scope of section~\ref{sec:soa_sw_ubicomp}.}{0.6}{}


\subsection{The \acl{sw} using intermediaries}
\label{sec:sw_intermediaries}

% como los smart environments describen contexto usando web semántica

% uso concreto por parte de soluciones significativas: siempre centralizando el uso de semántica en cacharros grandes

% intro a que ahora se va a hablar de soluciones IoT que usen semántica

Adding semantics works well for devices with high computational capacity but may add too much overhead for limited devices.
To reduce this overhead, part of this computation is usually delegated to an intermediary.
Some noteworthy example is the one proposed by \citet{broring_semantic_2009}.
% buscar otros ejemplos de enjundia
% The Context Broker Architecture (CoBrA)[Che04] 
% Gu et al. (2007)
% AlarmNet (2008)


% TODO meter alguna referencia a Cabilmonte sobre SPARQL en streaming
% Trata de consultar datos semánticos en streaming a una base de datos relacional.
% Por lo que entiendo, la comunicación sensor-DB se no es semántica.


These intermediaries or \emph{Semantic Gateways} are in charge of managing the semantic annotation.
The devices send raw data (which can be compressed) to the intermediaries and the gateways annotate the content semantically.
Thus, the devices do not have to care about any semantic aspect and just collect the data as they did before.

These \emph{Semantic Gateways} reduce the load to embedded devices with limited resources by decreasing the number of requests they have to provide.
In addition, a centralized intermediary can gather all the information and thus, reduce the complexity of managing a distributed environment.

However, using intermediaries to store the semantic data of resource constrained devices also has some drawbacks.
On the one hand, centralization does not faithfully represent mobility situations were individuals carry their own semantic information in their personal devices.
In addition, the data obtained from an intermediary will always be less fresh than the one obtained where it is actually being generated (i.e., sensors).
On the other hand, the servers are critical in centralized systems and therefore, their availability determines the operation of these solutions.
They also impose a burden on the maintenance which may be worthless in some simple scenarios.



\subsection{The \acl{sw} applied to data providers}
\label{sec:sw_providers}

% pero ahora los cacharros cada vez son más potentes y no es difícil imaginar un mundo poblado por ellos blablah
Lately, the computing capabilities of some mobile and embedded devices have improved enough to afford semantic processing. % poner ejemplos de FoxG20, RaspberryPi, Androids varios?
As a consequence, some solutions have arisen to semantically annotate data where it is generated. % in the provider
% justificar por qué todas las soluciones que presento son HTTP y no bluetooth o lo que sea? => mejor no meterme en lios, que se sobreentienda que es lo que mola


% Móviles y semántica como paso previo a WoT
\citet{daquin_enabling_2011} presents an architecture to deploy an \acs{sparql} \citeweb{sparql2008} endpoint in Android devices \citeweb{android}.
In their solution the mobile devices store and manage their own semantic knowledge by means of an embedded semantic repository.
On top of these endpoints, \citeauthor{daquin_enabling_2011} envision a query federation mechanism.
This dissertation shares this view of independent data sources sharing their own managed semantic content. % evitando entrar otra vez en el rollo de que también hay TS y blablablah


In the \acl{wot}, multiple solutions have considered using semantics to enrich the data definition in a machine processable manner.
Some solutions embed the metadata in HTML using microdata \citeweb{microdata2012}, microformats \citeweb{microformats} or RDFa \citeweb{rdfa2013}.
% \citep{mayer_extensible_2011} propone mapear estrategias de descubrimiento de recursos (RDFa, JSON, etc.) a un meta-formato interno basado en microformatos
These contents are returned by the Internet connected objects and are used to enhance the search-ability of the data by existing third-party search engines.
% Ontologías no se pueden definir, tiene que haber consenso en su uso por parte de las máquinas de búsqueda. => eso en RDFa no es así!
% ojo que RDFa y microdata en realidad es rollo RDF-based, podría llevar a confusión el inicio del siguiente parrafo
% en vez de eso, queremos mejorar la busqueda por parte de los cacharros, no de los search engines
% This approach does not focus on making the devices able to search and interact with others.
% end-to-end search
% Instead, it considers them as mere human-oriented information providers which should be indexed by third party search services.
% aquí se puede enganchar también el discurso de Doulkeridis2007desent: coverage and scalability, freshness y monopoly
However, these search engines do not consider the mobile nature of the data providers in the \ac{wot}. % They can move from one context to another frequently.
To solve this problem \citet{trifa_leveraging_2011} propose a hierarchical discovery and lookup infrastructure.
This infrastructure extracts the semantic annotations from different representations and converts them to an internal data model.
This model is based on existing microformats, but limited to certain concepts.
Besides, the particularities of the semantic annotations are hidden to searchers in their custom search \acs{api}. % custom and non-standard


A more general way to represent semantic data is using RDF \citeweb{rdf2004} based representations (i.e., full semantics). % más general que los microformatos que usan Guinard y compañía
The SPITFIRE European project \citeweb{spitfire} represents the most remarkable effort on gathering full semantics and the \ac{wot}.
It focuses on fully integrating sensor data with the \ac{lod}. 
The \ac{lod} are datasets which follow a series of principles on how to open and publish data.
The goal of the \ac{lod} is to publish linked terms using full semantics.

SPITFIRE shares with our solution the vision of a world populated by devices acting as semantic data providers no matter how small they are \citep{hasemann_rdf_2012}.
Therefore, many of their efforts are complimentary to this work.
%This is done by defining an ontology for mapping other common ontologies and providing semi-automatic generation of semantic annotations from raw data.
%They also propose an abstraction to represent real-world entities (virtual sensors) using data provided by low-level sensors.
%Finally, and more specific to this work, they propose a searching model which predicts the current state of things by computing their periodic patterns in past states.
To search within these providers, \citet{pfisterer_spitfire:_2011} propose a model which predicts the current state of things by computing their periodic patterns in past states.
Again, the goal of this method is to adapt search engines to the new fashion of data provided on the \emph{\acl{sw} of Things}.


\subsection{Discussion}

Resource constrained devices have been extensively used to provide information to represent the physical environments.
When this information has been semantically annotated, traditional solutions delegate on intermediaries the semantization of the raw data.
However, recent advances in mobile and embedded devices' capabilities have enabled to increase their responsibilities.
Some of these devices are now able to personally manage and share their own semantic content.
This completely distributed approach tries to simplify the maintenance burden and promote the access to the most updated data.


This dissertation aims to explore this path.
Specifically, the search architecture presented in Chapter~\ref{cha:searching}, tries to enhance the searching capability of web-connected devices.
It leads to environments where Internet-connected objects, mobile devices or regular computers directly interact and collaborate with each other. % que se note que no sólo hablamos de WoT
This contrast with the searching mechanisms presented by equivalent solutions in Section~\ref{sec:sw_providers}.
They assume that these limited devices need more powerful machines to search, which is not always true. % o third party search services.
\section{Semantic \aclp{ts}}
\label{sec:soa_tsc}

% TODO
% Añadir  Citron (2005)? (eoa de Almeida)
% Tanasescu
% swarm
% decir que tiene mucho que ver con distributed triple stores
% analizarlos también??? vaya pereza! -> igual comentar diferencia y dejar el análisis para la sección tosca de IGWS

Semantic \aclp{ts} aim to join \aclp{ts} with the \acl{sw} to propose a more uncoupled solution.
Particularly, it benefits from the autonomies introduced by both \ac{ts} and the \ac{sw}:

\begin{itemize}
  \item Space uncoupling.
  \item Time uncoupling.
  \item Data-schema uncoupling. % TODO TODO TODO explicar!!!! No se menciona en ningún otro lado!!!
% 	enhances the interoperability of the
% 	applications built on top of TSC. This way, two applications using standard ontologies can interact among them
% 	automatically enriching one each other, as long as they use the same space and standard and linked ontologies.
\end{itemize}


% lo de state-of-the-art es para dejar claro que la nuestra es más completa o nueva que la de la cita
This section provides a state-of-the-art review of the existing Semantic \aclp{ts} solutions \citep{nixon_tuplespace-based_2008}.
% así no nos repetimos como el ajo
Instead of describing each of these solutions individually and then include a comparison,
the review is divided in sections which describe the main aspects of a Semantic \ac{ts}.


\InsertFig{venn-sec3}{fig:venn_tuplespaces_semantics}{Semantic \aclp{ts}}{Scope of Section~\ref{sec:soa_tsc}.}{0.6}{}


% Definir constantes para referirnos a los distintos middlewares analizados
\newcommand{\midtsc}{\acs{tsc} \citep{fensel_tsc_2007}}
\newcommand{\midsws}{\acs{sws} \citep{tolksdorf_coordination_2006}}
\newcommand{\midstuples}{\acs{stuples} \citep{khushraj_stuples:_2004}}
\newcommand{\midcspaces}{\acs{cspaces} \citep{martinrecuerda_towards_2005}}
\newcommand{\midtscpp}{tsc++ \citep{krummenacher_open_2009,blunder_distributed_2009}}
\newcommand{\midtripcom}{\acs{tripcom} \citep{simperl_coordination_2007}}
\newcommand{\midsmartmt}{Smart-M3 \citep{honkola_smart-m3_2010}}
\newcommand{\midnardini}{\citet{nardini_semantic_2013}}


\subsection{Use of semantics}

The works analysed present two strategies to semantically annotate data: the use of \ac{sw} standards or the definition of their own language.
\ac{cspaces} \citep{martinrecuerda_towards_2005} and \citet{nardini_semantic_2013} define their own language-independent up to \ac{fol} notation.
% theoretical counterpart of OWL DL [7]. Since OWL is the W3C standard ontology description language for the Semantic Web,
% and the standard de-facto for semantic applications in general, we adopt OWL as the ontology language for the domain
% ontologies associated to ReSpecT semantic tuple centres in TuCSoN. While for the details of OWL we forward interested
However, these works face an inconvenience: most of the libraries and tools available are based on the \ac{sw} standards. % concretamente cuando quieren usar reasoners!
Therefore, they have to internally translate between both worlds in a not-straightforward and resource consuming manner. % son cosas distintas!


% TODO explicar diferencias, citar o evitar verengenales que no comprendo y que sólo me complican la vida?
% http://www.cs.ox.ac.uk/ian.horrocks/Seminars/download/Horrocks_Ian_pt1.pdf
%     DL == Decidable fragments of First Order Logic
% http://en.wikipedia.org/wiki/Description_logic#First_order_logic
%     Many Description Logic models (DLs) are decidable fragments of first order logic (FOL).
%     Some DLs now include operations (for example, transitive closure of roles) that allow efficient inference but cannot be expressed in FOL.


On the contrary, the majority of the semantic \acp{ts} use the metadata data model in which the \ac{sw} relies: \ac{rdf}. % también lo llaman language
We also opt for using the widely accepted \ac{rdf} to ensure interoperability \emph{ab-initio}.
On top of \ac{rdf}, the \ac{sw} defines new layers to increase the data expressiveness.
First, \ac{rdfs} \citeweb{rdfs2004} describes a series of classes and properties to define vocabularies in \ac{rdf}.
On top of \ac{rdfs}, \ac{owl}\citeweb{owl2004} offers additional vocabulary for describing properties and classes.
% 3 niveles: Lite, DL (que se corresponde con DL) y Full
These properties and classes allow to reason over the knowledge both to validate the data inserted or to infer implicit (unstated) knowledge. % implicit y unstated son redundantes, pero para que se entienda mejor


Although reasoning may be desirable, its use is tangential to the model proposed in this thesis.
In any case, it is by no means mandatory in our solution because:
\begin{itemize}
  \item The devices which may provide information to our \emph{federated space} are autonomous.
	They share a minimal contract with our middleware where we cannot define how they manage/provide their information.
  \item It is limited by the availability of reasoners, and nowadays there is no light enough reasoner to run in a reasonable time span in small devices.
        % TODO citar algo? quizá nuestro propio trabajo?
\end{itemize}
%For our solution, describing the data semantically is a first step towards semantically interpreting it.
%The Table~\ref{tab:comparisonSemantics} shows different features regarding the semantic model proposed by each solution.



\begin{table}[htbp]
\caption{Use of semantics in the analyzed solutions.}

% TODO En Languages decir si se basa en RDF o FOL, el hecho de que luego se use OWL para definir conocimiento, se sobreentiende
% TODO añadir reglas?
% TODO añadir Validity and consistency?
% TODO añadir semantic clustering

\begin{tabular}{ l p{3cm} p{3cm} p{3cm} }
\hline 
  & Languages  & Reasoning  & Semantic matching \tabularnewline
\hline 
 TSC & RDF  & -  & Graph pattern templates and N3QL resolution \tabularnewline
 SWS & RDF(s) extendable to OWL, SWRL  & Yes (to match and validate)  & Subsumption-based \tabularnewline
 sTuples & DAML+OIL  & Tuples with DAML+OIL field  & Subsumption-based \tabularnewline
 CSpaces & Language-independent, up to FOL  & Yes (used for query answering, rewriting and consistency checking)  & Based on query engines \tabularnewline
 tsc++ & RDF & No  & No \tabularnewline % comprobar %  lo de para query answering lo hice yo en su día
 TripCom & RDF  & Yes (query rewriting) & Yes \tabularnewline % TODO rehacer esta columna comprobandolo
 Smart-M3 & RDF  & ?  & G. patterns and complex \tabularnewline
 Nardini et al. & Language-independent, up to FOL & ?  & ? \tabularnewline
\hline 
\end{tabular}
\label{tab:comparisonSemantics}
\end{table}

% TODO añadir tabla con análisis de las autonomías?


\subsection{Tuple model}

% presentamos distintos modelos
On the one hand, several works embed semantic content in one of the fields of a common tuple.
\ac{cspaces} \citep{martinrecuerda_towards_2005} defines seven-field tuples.
\ac{stuples} \citep{khushraj_stuples:_2004} extend the \emph{JavaSpace} \citep{freeman_javaspaces_1999} middleware adding a field with semantic content associated to the tuple object.
This content follows the DAML+OIL language (the \ac{owl} precursor).
\citet{nardini_semantic_2013} use the concept of \emph{semantic tuples} which are expressed in logic terms.


On the other hand, the most common tuple model is the three-field tuple which corresponds with a \ac{rdf} triple.
That is, a tuple with a field for the \emph{subject}, another for the \emph{predicate} and a third one for the \emph{object}.
% Sacado de la descripción de TSC que tenía antes:
% In a later work, \citet{krummenacher2006specification} proposed to use identifiers just for a set of triples (i.e., \ac{rdf} graphs).
% bundler cita el de "WWW: or what's wrong with the web" para explicar eso último
However, a triple by itself cannot express much information \citep{krummenacher2006specification}.
% también para reducir la complejidad del storage (TSC dixit)
% decir que al primero que se le ocurrió fue a TSC?
To solve this limitation, semantic \acp{ts} have usually adopted the concept of \ac{rdf} graphs too.
% En SWS usan algo muy parecido pero a lo que llaman subspace.
A \ac{rdf} graph is a set of \ac{rdf} triples identified by an \ac{uri}. % TSC distinguen entre NamedGraph (grafo con URI) y Graph (conjunto de grafos)
Although they can be accessed by their \ac{uri}, more interestingly, these middlewares also guarantee associative access. % see the next section


\begin{savenotes}
  \begin{table}[htbp]
    \caption{Information units used by the different semantic \ac{ts} middlewares.}
    \centering
    \begin{tabular}{ l c c c }
      \hline 
	& Tuples with & \ac{rdf} triple- & \multirow{2}{*}{\ac{rdf} Graphs} \\
	& semantic field & like tuples & \\
      \hline 
      \midtsc{} & & $\checkmark$ & $\checkmark$ \\ % en una adición posterior!
      \midsws{} & & $\checkmark$ & $\checkmark$\footnote{\ac{sws}'s subspaces are conceptually equivalent to \ac{rdf} Graphs: an abstraction to work with a set of \ac{rdf} triples.} \\
      \midstuples{} & $\checkmark$ & & \\
      \midcspaces{} & $\checkmark$ & & \\ %7-field tuple
      \midtscpp{} & & $\checkmark$ & $\checkmark$ \\
      \midtripcom{} & & $\checkmark$ & $\checkmark$ \\
      \midsmartmt{} & & $\checkmark$ & $\checkmark$ \\
      \midnardini{} & $\checkmark$ & & \\
      \hline 
    \end{tabular}
    \label{tab:tuple_comparison}
  \end{table}
\end{savenotes}


\subsection{Query model}

In \ac{ts}, originally tuples were selected using special tuples where wildcard values were allowed in the fields. % simplificado, a veces también tipos
All the studied works which use a \ac{rdf} triple as a tuple follow this approach.
Beside, they provide access to the \ac{rdf} graphs by their \acp{uri}.

Most of these works also offer advanced query languages (e.g., \acs{sparql} \citeweb{sparql2008}) as a more expressive way to match graphs.
These languages can be decomposed in plain triple patterns.
However, this requires a parser which may not be available for resource constrained platforms.
This dissertation focuses on queries based on graph patterns and leaves the adoption of more complex querying languages as a future enhancement.

\ac{cspaces}, \ac{stuples} and \citeauthor{nardini_semantic_2013} use less standard querying approaches.
\ac{cspaces} \citep{martinrecuerda_towards_2005} and \citet{nardini_semantic_2013} offer a formal language to select appropriate tuples.
\ac{stuples}\citep{khushraj_stuples:_2004} extends JavaSpace's template by adding assertional axioms that can be used to match semantic tuples.

\begin{savenotes}
  \begin{table}[htbp]
    \caption{Querying units for semantic \ac{ts}s.}
    \centering
    \begin{tabular}{l c c c}
      \hline 
	& Graph  & Advanced query  & \multirow{2}{*}{Other} \\
	& patterns  & languages  & ~ \\
      \hline
      \midtsc{} & $\checkmark$ & & \\ % en una adición posterior!
      \midsws{} & $\checkmark$ & & \\
      \midstuples{} & & & $\checkmark$ \\
      \midcspaces{} & & & $\checkmark$ \\ %7-field tuple
      \midtscpp{} & $\checkmark$ & $\checkmark$\footnote{It completely depends on the underlying data store selected.} & \\ % but it can be SeRQL, SPARQL and N3QL
      \midtripcom{} & $\checkmark$ & $\checkmark$~~ & \\
      \midsmartmt{} & $\checkmark$ & $\checkmark$~~ & \\
      \midnardini{} & & & $\checkmark$ \\
      \hline
    \end{tabular}
    \label{tab:query_comparison}
  \end{table}
\end{savenotes}

% TODO añadir los lenguajes específicos en queries


\subsection{Space model}
\label{sec:soa_tsc_space}

The \emph{flat} model offers independent disjoint spaces.
\ac{stuples}, \citet{nardini_semantic_2013}, TSC, tsc++ and Smart-M3 use this model.
% sTuples:
%    Bundler dice que es "Centralized nested", pero no sé de dónde lo saca
%    Está basado en Javaspaces, que por lo que puedo ver es flat
%    En nixon2008 sólo dicen que es centralized
% decir que quiere decir: Tuple Centres Spread over the Network ?
Within them, \citet{nardini_semantic_2013} present the most singular model.
It extends TuCSoN \cite{omicini_tucson:_1998}, which presents an evolution of the \ac{ts} called \emph{tuple centre}.
A \emph{tuple centre} can be adapted to the application needs through reactions to communication operations.
These reactions allow to trigger behaviours in response to any primitives or to define new ones.
% TuCSoN follows a flat model where different independent tuple centres can coexist at the same time. % o created?
% The agents can access to remote \emph{tuple centres} and migrate to them, but these interactions need to be programmed.
% Besides, the latter three works identify each space with an \ac{uri}.


% TODO Nosotros proponemos flat spaces enriquecidos con federated spaces
% Nested
More sophisticated models allow to create hierarchies of spaces.
Three examples are \ac{sws}, \ac{tripcom} \citep{simperl_coordination_2007} and \ac{cspaces}.
% SWS -> Disjoinnt nested
% cosas relativas al despliegue
\ac{sws} \citep{tolksdorf_coordination_2006} proposes two ways to partition the spaces: sub-spaces and contexts.
Sub-spaces are disjoint partitions of the main space.
Contexts enable to virtually divide the space into overlapping partitions.
These partitions are used to enable particular clients' views of the space.


However, probably \ac{sws}'s most distinctive feature is that it virtually divides the spaces into two views.
The \emph{data view} stores syntactically valid \ac{rdf} and it is accessed using Linda-like primitives.
The \emph{information view} stores consistent and satisfiable data which are managed using new primitives.
The latter view takes into account the knowledge defined by ontologies to perform semantic matching over inferred triples.


% TripCom
\ac{tripcom} shares some similarities with \ac{sws}'s model.
It uses subspaces to form nested multiple spaces.
Doing so it restricts the communication to a part of the whole space leading to scalability and completeness. % según ellos
Besides, it offers a mechanism to overlap spaces called \emph{scopes}.
Using \emph{scopes} a client can create a temporary copy of some tuples.
However, any insertion and deletion would not apply to the whole Triple Space.

% CSpaces
% On the contrary of C04 where modifications need to be approved by all
% subscribers, the updates proposed by the members of a Shared CSpace are
% automatically included, and versioning mechanisms are in charge to track
% changes and provide rollback features if one of the members disagrees with
% the included updates.
% To join a Shared CSpaces and publish and retrieve data on it the new
% members should first complete a registration procedure in which one of the
% main tasks is to provide a semantic and alignment specification between the
% data that each new candidate want to share and the data that previous
% members have published beforehand.
\ac{cspaces} \citep{martinrecuerda_towards_2005} proposes two types of spaces: individual and shared.
An \emph{individual space} belongs to a single process.
Two participants can agree on how to represent the knowledge to share their individual spaces forming a \emph{shared space}.
Shared spaces can join to others forming a tree structure.
In a shared space the updates are versioned and can be revoked by any member.
However, neither the registration process needed for the agreement or the revocation process are detailed.
Furthermore, to the best of our knowledge, this conceptual exercise never went beyond a rather limited prototype.


\InsertTab{tab:space_comparison}{Space model used by the different works}{}{
  \begin{tabular}{lccc}
    \hline
    ~ & \multirow{2}{*}{Flat}  & Nested & Overlapping \\
    ~ & ~  & Disjoint  & views \\
    \hline 
    \midtsc{} & $\checkmark$ & & \\ % en una adición posterior!
    \midsws{} & & $\checkmark$ & \\
    \midstuples{} & $\checkmark$ & & \\
    \midcspaces{} & & $\checkmark$ & $\checkmark$ \\ %7-field tuple
    \midtscpp{} & $\checkmark$ & & \\
    \midtripcom{} & & $\checkmark$ & $\checkmark$ \\
    \midsmartmt{} & $\checkmark$ & & \\
    \midnardini{} & $\checkmark$ & & \\
    \hline
  \end{tabular}
}{htbp}



\subsection{Distribution} % o después de space model o integrado!
\label{sec:soa_tsc_distribution}
%	2. centralizado / distribuído
%	4. basado en clientes tontos

\begin{sloppypar}
Participant nodes usually access semantic spaces on client/server basis.
% no puedo usar \ac{} por la mayúscula
Tsc++\citep{krummenacher_open_2009,blunder_distributed_2009} proposes an exception to the client/server access to the space.
It relies on the Jxta P2P framework \citeweb{jxta} to propagate queries using different strategies.
In tsc++, spaces correspond to groups of nodes which locally manage their data.
\end{sloppypar}


In client/server spaces, the back-end of the server can be distributed or centralized in a single machine.
Centralized \aclp{ts} are much simpler and easier to implement.
Therefore, they usually offer more features than the distributed ones.
However, they also impose a single-point-of-failure.


Within the distributed approaches we can distinguish those works which replicate data and those which do not.
\acs{tsc} belongs to the first group, and replicates all the triples in each deployed kernel.
In \ac{tripcom} each kernel stores one or more subspaces and can contact other kernels responsible for different spaces.
To do that, if the space's \ac{url} is provided, it simply resolves this \ac{url} using \ac{dns} and contacts the other kernel.
Otherwise, the kernel uses three additional strategies:
\begin{itemize}
  \item Triple Provider.
	It uses shortcuts to know who answered a query in the past.
  \item Recommender.
	It uses shortcuts to know which kernel successfully routed a query in the past.
  \item Indexing - \ac{dht}.
	It creates indexes using a hash function over the subject, predicate, object and space \ac{url}.
	Then, it stores these indexes in a distributed database which relies in a structured \ac{p2p} system. % mencionar PGrid???
\end{itemize}
\ac{cspaces} uses a similar but vaguely described super-peer network \citep{martinrecuerda_application_2006}.


As discussed in Section~\ref{sec:soa_ts_ubicomp}, we also deal with distribution of data in one of the spaces of our hybrid model.
As detailed in the following chapters, we locally manage the content and we distribute the queries.
This resembles to the \emph{tsc++}'s strategy.
However, instead of using a \ac{p2p} framework to access the contents,
we individually access them using several \ac{http} requests. % añadir que son "parallel" o sólo liará?
In other words, each client may access various servers to obtain a result for a given primitive.

% TODO TODO TODO Smart-M3
% De: http://www.diem.fi/files/KP_reference%20implementation.pdf
%  The Semantic Information Broker is the information repository of the Smart 
%  Environment. In theory, the Smart-M3 Smart Space can consist of one or more SIB 
%  entities. However, the SIB reference implementation does not currently support any 
%  kind of interaction between different SIBs, which would enable distribution of the 
%  Smart Space. On the IOP, the SIB is implemented as a NoTA SN.
\citet{honkola_smart-m3_2010} defend that Smart-M3's space can be distributed using the \emph{distributed deductive closure protocol}.
 % TODO volver a asegurarse 100% de esto que voy a decir!
However, to the best of our knowledge this idea has never been implemented or evaluated, making Smart-M3's space de facto centralized.
For the communication between the clients and the space, Smart-M3 defines a stateful protocol called \emph{Smart Access Protocol (SSAP)}. % decir que está basado en XML?
The authors defend that this protocol is communication agnostic because it can be implemented on top of different communication mechanisms
(e.g., WS-* web services, XMPP \citeweb{xmpp}, Bluetooth \citeweb{bluetooth} or TCP/IP).
\citet{kiljander_knowledge_2012} propose an enhanced stateless access protocol designed to fit the needs of low capacity devices.


\InsertTab{tab:distribution_comparison}{Distribution of the spaces}{}{
  \begin{tabular}{ l c c p{5.5cm} }
    \hline 
    & C/S & Distributed & Distribution \\
    & access & space & strategy \\
    \hline 
    \midtsc{} & $\checkmark$ & $\checkmark$ & Replication \\ % Possitive broadcasting
    \midsws{} & $\checkmark$ & × & - \\
    \midstuples{} & $\checkmark$ & × & - \\
    \midcspaces{} & $\checkmark$ & $\checkmark$ & Not detailed \\ % además no implementado :-S
    \midtscpp{} & × & $\checkmark$  & Local writing, different query strategies \\ % Flooding, RW, etc.
    \midtripcom{} & $\checkmark$ & $\checkmark$ & Structured network, different strategies \\
    \midsmartmt{} & $\checkmark$ &  $\checkmark$ & Theoretical \\ % decidir si me convence que no haya sido implementado, TODO citar al tipo que dijo como hacer SIBs distribuidos
    \midnardini{} & $\checkmark$ & × & - \\
    \hline 
  \end{tabular}
}{}


\subsection{Discussion}
\label{sec:soa_tsc_discussion}

% Model: Triple Space
% TODO TODO TODO el haber pasado de hablar de TSC sin explicar que de ahí viene el nombre puede hacer que ahora te explote la cabeza al hablar de TSC como modelo!
% Poner footnote o algo!
The use of standard semantic protocols and \ac{rdf} triple-like tuples characterizes \acl{tsc}. % restringido a subconjunto de los trabajos analizados
% Originalmente TSC estaba basado en REST
\ac{tsc} was born to realign web services (WS-*) with the web.
To ensure this alignment, it was based on \ac{rest} architectural style's principles \citep{fensel_triple-space_2004,hernandez_formal_2010}. % se pueden poner otras intermedias: riemer2006tsc o fensel2007tsc
However, \ac{tsc} has never been true to all these principles. % porque según ellos había algunas cosas que ni de palo, de otras como HATEOAS simplemente se olvidaron
%no deliveradamente!
%fensel_tsc_2007
% Luego ha ido derivando e implementando funcionalidades más complejas alejandose del diseño inicial
%    => el interés por añadir features, hizo que la gente olvidase los ppios que guiaban la simplicidad de REST
Furthermore, the more features are added to the \ac{tsc} design (e.g., subscriptions or transactions), the more difficult it is to reconciliate both worlds.

%\subsubsection{Additional features}
%% transacciones
% subscripciones
% etc.

\begin{table}[htbp]
\caption{Features offered by the analyzed solutions.}

\begin{tabular}{ l p{3cm} l p{4cm} }
\hline 
  & Subscriptions  & Transactions  & Other features \tabularnewline
\hline 
 TSC & Yes  & Yes  &  \tabularnewline
 SWS & No  & No  &  \tabularnewline
 sTuples & Yes  & No  &  \tabularnewline
 CSpaces & Yes  & No  & Multiple read and writes \tabularnewline % bussler dice que tiene algún tipo de transaccionalidad, yo no lo creo
 tsc++ & Yes  & No  &  \tabularnewline
 TripCom & Yes  & Yes  &  \tabularnewline
 Smart-M3 & Yes  & No  &  \tabularnewline
 Nardini et al. & No \footnote{But it has been implemented in TuCSoN before \cite{ricci_extending_2002}.} & No  &  \tabularnewline
\hline 
\end{tabular}
\label{tab:compAdds} 
\end{table}




Regardless of their incompatibility with other features or practical technical difficulties, % e.g., conseguir HATEOAS o effective caching
we defend that, \emph{per se}, \ac{tsc} does not contradict in any sense the \ac{rest} principles described in Chapter~\ref{cha:background}:
% Ver cómo TSC puede cumplir con todos los principios de diseño
\begin{description}
 \item[\acf{restcs}.] Accessing to a space through a server in a \ac{restcs} fashion is completely feasible.
		      Indeed, this does not prevent to use a distributed solution in the \emph{back end} (e.g., a distributed semantic repository).
 \item[\acf{rests}.] The primitives to access the space imply simple reads and writes which do not store any state in the server.
 \item[\acf{restcache}. ] Despite of the difficulties detected by \citet{fensel_tsc_2007}, nothing prevents the semantic content stored in the space to be cached.
                          However, the dynamism of the knowledge can make effective caching challenging to achieve.
 %\item[\ac{rest_u}:] % lo quito porque si no, este punto sin texto con otros subpuntos queda feo
    %\begin{description}
	\item[\acf{restid}.]
			 In \ac{tsc}, there can be up to three type of resources which can be identified by an URI : spaces, \ac{rdf} Graphs and certain elements of the \ac{rdf} Triples.
	                 The space can be seen as a coarse-grained view of the underlying graphs.
	                 The graphs have sets of triples which are usually related and describe a unit of knowledge. % A RDF triple by itself cannot transmit too much information
	                 Self-identified triple's subjects, predicates or objects are the source of concept linking in the \ac{sw}.
	\item[\acf{restrep}.] The \ac{rdf} graphs and triples mentioned above can be represented using different standard serializations.
			      % TODO TODO TODO citar el paper que habla de cómo alinear LOD y REST
	\item[\acf{restdesc}.] The messages derived from the primitives are self-describing since they are expressed on standard \ac{rdf}-based languages. % including meta-data
				Therefore the server and clients know how to process the content according to its language and certain vocabularies.
				These vocabularies specifications (i.e., ontologies) are referenced in the content.
	\item[\acf{resthateoas}.] The lack of native hypermedia support in \ac{rdf}-based representations makes this the most challenging property to achieve \citep{amundsen_api_2010, page_rest_2011}. % compara LOD y REST
				  However, some recent works propose means to fulfil this property.
				  \citet{kjernsmo_necessity_2012} proposes a vocabulary for hypermedia \ac{rdf}.
				  \citet{steiner_fulfilling_2011} and \citet{verborgh_functional_2012} propose to enrich the \ac{http} header with hypermedia information.
				  While the first changes representations, the latter is a more general way to provide hypermedia.
    %\end{description}
 \item[\acf{restl}.] Encapsulation of functionalities can be achieved through a layered system.
                     For example, to balance the load to a space replicated in two machines.
 % TODO TODO repensar si ontologías, reglas y demás (personalizar behaviour) se puede entender cómo COD:
 % In the code-on-demand style [50], a client component has access to a set of resources, but
 %  not the know-how on how to process them. It sends a request to a remote server for the
 %  code representing that know-how, receives that code, and executes it locally.
 \item[\acf{restcod}.] There are several cases where the \emph{know how} can be downloaded from the server in \ac{tsc}: % ok, no tan común como scripting
            \begin{enumerate}
	      \item Modelling scripts using appropriate ontologies. % TODO cite
	      \item A semantic reasoner can be considered an interpreter for the content downloaded from the server.
	            It is used to extract unstated information from the content received from the server.
	            Therefore, the client downloads know-how to process the resource in the following situations:
		    \begin{itemize}
		      \item Through the taxonomies defined in ontology files. % e.g., para hacer un mapeo; en analogía a JS, da igual que tu tengas previamente el script o no
		            These taxonomies are modelled using standard semantic languages. % hablar de distintos niveles de expresividad?
		      \item Through semantically expressed rules \citep{berners-lee_n3logic:_2008}. % (e.g., N3 rules)
		    \end{itemize}
	      % con nuevo TBox que te bajas??
            \end{enumerate}
\end{description}


Table~\ref{tab:rest_principles} shows the conflicts that the analysed works present with these principles.
None of them achieve the \ac{resthateoas} principle.
In fact, its use in the \ac{sw} is subject of current research.
Most of them guarantee the access to the space in the \ac{restcs} basis.
The only exception is \midtscpp{}, where each participant of the space is a peer in a \ac{p2p} network.
Finally, the statefulness of these middlewares conflicts with additional features such as transactionality or subscriptions/notification systems.
While the transactionality need in \ac{tsc} can be argued,
subscriptions are useful to ensure certain level of asynchrony in the system (see Section~\ref{sec:notification}).
% explicar que lo nuestro al dividir en dos espacio en uno nos libramos de esto y en el otro es opcional para mejorar la eficiencia


% tabla de potenciales problemas con REST de los trabajos presentados:
\InsertTab{tab:rest_principles}{\ac{tsc}'s middlewares potential conflicts with \ac{rest} style's principles}{
  The crosses with an asterisk denote mainly \ac{restcs} middlewares which at some level break this property through asynchronous notifications to the clients.
  
}{
  % Tablita de cómo hereda propiedades de los estilos anteriores?
  % (y si quieren más información, que miren en la tesis de Fielding)
  \begin{tabular}{lccc}
      \hline
      ~ &
      \ac{restcs} &
      \ac{rests} &
      \ac{resthateoas} \\
      \hline
      \midtsc{} & ×$^*$ & × & × \\
      \midsws{} & ~ & ~ & × \\
      \midtscpp{} & ×~ & ~ & × \\
      \midtripcom{} & ×$^*$ & × & × \\
      \midsmartmt{} & ×$^*$ & ~ & × \\
      \hline
  \end{tabular}
}{}


% Nosotros proponemos una vuelta a los origenes para recuperar una simplicidad de la que limited devices pueden beneficiarse
%    => la clave para facilitar y en algunos casos posibilidar que los dispositivos constrained lo implementen
This dissertation proposes a return to the origins to recover the simplicity loss in the previous works. % ¿Explicar en qué consiste esto de simplicidad y por qué no es stateful?
The rationale behind this decision is that resource constrained devices will benefit from this simplicity.
In the proposed middleware they will be able not only to write and read knowledge into an external space, but also to enrich it providing their own managed data.
% In the rest of the cases, the nodes are head toward the use of the space as simple clients.
% none of them has been specifically designed to be run in devices with constrained capabilities apart from our solution.
%      => la prueba de eso podría ser un apéndice con cómo lo nuestro se ha implementado en distintos cacharros... ¿?
Table~\ref{tab:summary_own} summarizes the features of this middleware.

\InsertTab{tab:summary_own}{Features of the solution presented in this dissertation}{}{
  % Tablita de cómo hereda propiedades de los estilos anteriores?
  % (y si quieren más información, que miren en la tesis de Fielding)
  \begin{tabular}{ l @{\hskip 0.3in} | @{\hskip 0.3in} l}
    Use of SW standards & $\checkmark$  \\
    Tuple model & \acs{rdf} triple-like tuples \& \acs{rdf} Graphs \\
    Query model & Graph patterns \\
    Space model & Flat \\
    C/S access & $\checkmark$ \\
    Distributed space & $\checkmark$ \\
    Distribution strategy & Local management, distributed query \\
  \end{tabular}
}{htbp}



% ----------------------------------------------------------------------




\section{Conclusions and further work}

In this paper we have presented our work towards a lightweight flexible Triple Space Computing-based solution called
Otsopack. This framework is specifically designed to adapt to AmI and IoT environments and aims to distribute the
information among different types of nodes in a dynamic way. This forces the design a modular interface, a flexible
networking approach and a simple but yet trustful security layer.

The results show that firstly, the effort required to develop particular features of Otsopack on different platforms is
low, due to its simplicity and the use of standard HTTP capabilities. Secondly, the networking layer shows its ability
to adapt to a wide range of scenarios both in simulation and real environments. Finally, the distributed security schema
allows to automatically expose and share the information managed by the applications so multiple applications that did
not know each other can still interact.

To detail future directions, we aim to implement a notifications system that permits other nodes to know that an event
has occurred. Regarding the security of the information, work is being done to ensure security in certain deployments,
adapting research already done on secure multicast \cite{naranjo2011suite} techniques.

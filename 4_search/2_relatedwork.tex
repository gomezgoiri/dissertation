\section{Background}
\label{background}

Querying over semantic content provided by independent sources is a problem addressed in the \acl{lod} field.
\citet{gorlitz_federated_2011} classifies the possible \ac{lod} infrastructures according to three characteristics:
1) how they store data,
2) whether the index is distributed and
3) whether the data sources cooperate.
Table~\ref{tab:infrastructure_lod} summarizes the resulting infrastructure types: central repository, federation and \ac{p2p} data management.


%  \multicolumn{7}{c}{\multirow{2}{*}{


\begin{table}[htbp]
  \caption{Infrastructure paradigms according to their characteristics \citep{gorlitz_federated_2011}.}
  \begin{center}
    \begin{tabular}{l|c|c|c|c|}
      \cline{2-5}
      ~ & \multicolumn{2}{c|}{Central Data Storage} & \multicolumn{2}{c|}{Distributed Data Storage} \\
      \hline
      \multicolumn{1}{|l|}{Independent} & \multirow{2}{*}{n/a} & ~ & \multirow{4}{*}{Federation} & \multirow{2}{*}{n/a} \\
      \multicolumn{1}{|l|}{Data Sources} & ~ & Central &~ & ~ \\
      \cline{1-2} \cline{5-5}
      \multicolumn{1}{|l|}{Cooperative} & \multirow{2}{*}{n/a} & Repository & ~ & P2P Data \\
      \multicolumn{1}{|l|}{Data Sources} & ~ & ~ & ~ & Management \\
      \hline
      ~ & Distr. Index & \multicolumn{2}{c|}{Central Index} & Distr. Index \\
      \cline{2-5}
    \end{tabular}
  \end{center}
  \label{tab:infrastructure_lod}
\end{table}


% Lo nuestro se parece más a LOD de dispositivos que a distributed triple store:
%     grafos semánticos de fuentes de datos independientes
Within the solutions which distribute the semantic content are federation and \ac{p2p} data management.
The difference between them resides in whether the index to look for content in the different machines is distributed or not.
In other words, a federated \ac{lod} infrastructure maintains the meta information about data sources in an element called \emph{federator} \citep{gorlitz_federated_2011}.
This meta information is used to delegate queries to data sources.

Therefore, according to this classification, our solution is a query federation solution.
However, the solution presented in this chapter presents some particularities:
\begin{itemize}
  \item This node which manages the \emph{indexes} is dynamically chosen between all the participants and can change over the time.
  \item The \emph{indexes} are replicated in each data consumer.
	This replication allows consumers not to critically depend on the availability of a this manager. % no depender => MUCHO
	This implies that some components of a \emph{federator} are implemented by more than a node at a time.
\end{itemize}


\bigskip


% presentamos 3 soluciones: FedX, ARQ y XXX.
Querying over federated databases is a well-addressed problem which provides relevant searching mechanisms.
However, there are significant differences when it comes to search over semantic content \citep{gorlitz_federated_2011}. % estoy citando estados del arte de otros, un poco gitaner :-P Ya puestos, citar el de FedX Optimization
Although it is a more unexplored field, federated distributed query over independent semantic data sources has some remarkable proposals are: FedX, DARQ and SemWIG.
% (usually justified by \ac{lod} solution)


% pros y contras:
% No asumimos SPARQL, pero al final todos estos mecanismos descomponen SPARQL en triple patterns como los nuestros
%    por lo tanto se podría implementar sobre lo nuestro lo suyo
%            (y permitir optimizaciones si introducimos SPARQL en algunos nodos)




% TODO meter enlace con lo de summaries de ABox en la parte correspondiente


% mencionar fokoue con lo de los resumenes?

% presentar las siguientes subsecciones

% intro a que ahora se va a hablar de soluciones IoT que usen semántica
% Having analyzed non-specific solutions for distributed semantic architectures, the next step is to evaluate the use of semantics from the \ac{iot} perspective.

% \label{sec:sw_providers}

% La acepción de Name resolution cambia depende del contexto!
% http://en.wikipedia.org/wiki/Name_resolution

% Name resolution vs recognition
% http://stackoverflow.com/questions/8589005/difference-between-named-entity-recognition-and-resolution

% Mirar en el punto 6.3 para ver unos pasos de consulta en la LOD que parece que pueden ser interesantes para esto
% http://www.unifr.ch/webnews/content/20/attach/4765.pdf


% no hay ninguna solución global con una arquitectura que tenga en cuenta le energía


% como se ha visto en TAL y TAL sección, to the best of our knowledge, no existen soluciones de busqueda de contenidos semánticos distribuido para WoT. (o al menos no para funcionar en cacharrillos)
% meter alguna referencia a Cabilmonte y demás?
\section{Conclusion}
\label{conclusion}

In this chapter, we have presented a dynamic architecture to enhance the search of semantic contents in the Web of Things.
%  by any node part of the Web of Things.
In particular, this architecture chooses an intermediary according to its capabilities to support resource constrained devices.
Intermediaries can use different types of clues to summarize the information in the semantic space.
% managed by the nodes belonging to a space.
Thanks to this support, the devices can directly interrogate others to obtain fresh information while reducing their semantic overhead.

Our evaluations show that our solution requires less communication messages between devices than a flooding-based strategy.
In addition, our approach reduces the workload of mobile and embedded devices which indirectly results into energy savings.

% Para trabajo futuro, se podría considerar: 
% query optimization y citar a alguna solución de distributed triple stores
For our future work, we intent to further assess the proposed solution in a real deployment.
In addition, we will consider using query optimization techniques and reconciliation of query results in the \emph{Providers}.
Using these techniques, we could a) transform the queries before sending them to obtain more results
and b) refine the results.
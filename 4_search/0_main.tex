% ----------------------------------------------------------------------

\begin{savequote}[50mm]
I woke up one morning thinking about wolves and realized that wolf packs function as families.
Everyone has a role, and if you act within the parameters of your role, the whole pack succeeds, and when that falls apart, so does the pack.
\qauthor{Jodi Picoult}
\end{savequote}


\chapter{Searching in a Distributed Space}
% ó Energy-aware Architecture for Information Search in the Semantic WoT
\label{cha:searching}
\newcommand{\pathchapfour}{4_search}


\newcommand{\Space}{\emph{Space}} % en mayúsculas porque \space ya existe
\newcommand{\Spaces}{\emph{Spaces}} % en mayúsculas porque \spaces ya existe
\newcommand{\consumer}{\emph{Consumer}}
\newcommand{\consumers}{\emph{Consumers}}
\newcommand{\provider}{\emph{Provider}}
\newcommand{\providers}{\emph{Providers}}
\newcommand{\clue}{\emph{clue}}
\newcommand{\clues}{\emph{clues}}


% the code below specifies where the figures are stored
\ifpdf
    \graphicspath{{\pathchapfour/figures/PNG/}{\pathchapfour/figures/PDF/}{\pathchapfour/figures/}}
\else
    \graphicspath{{\pathchapfour/figures/EPS/}{\pathchapfour/figures/}}
\fi


%------------------------------------------------------------------------- 

% Mini-intro para enmarcar esta sección en el capítulo
In Chapter~\ref{cha:tsc} we presented a \ac{tsc} model for \ac{ubicomp} environments which is composed by two areas: coordination space and the outer space.
The \emph{outer space} is formed by the semantic data provided (\emph{self-managed graphs}) by any participant device.
Searching for a graph in that \emph{outer space} is not trivial because it is composed by dynamic and unrealiable devices.

In this chapter, we present an architecture to enable searching for semantic content in a decentralized energy efficient manner.
This searching mechanism can be generalized to other \ac{wot} solutions and scenarios.
To empathize this, during the rest of the chapter we avoid explicitly mentioning the \ac{tsc} middleware or related concepts (e.g. \emph{outer space}).

% TODO IMAGEN de DÓNDE encaja el searching mechanism?
% TODO ver si se joden las propiedades de TSC expuestas anteriormente! (e.g. stateless o lo que sea)



The remainder of this chapter is organized as follows:
Section~\ref{sec:search_intro} gives an overview of the problem addressed.
Section~\ref{background} discusses the related work.
Section~\ref{proposal} presents in detail our energy-aware architecture.
Section~\ref{sec:clues} describes the information that devices exchange to maintain this architecture.
Section~\ref{environment} presents our experimental environment and Section~\ref{results} evaluates our solution.
Finally, Section~\ref{conclusion} states the conclusions of our work.


%------------------------------------------------------------------------- 

% comprobar y aplicar los ultimos cambios y ajustes introducidos en el svn de IJWGS!
%\section{Introduction}

% Originally, the Internet was basically composed of a small number of computers that were physically connected to a wired network.
% Over the years, the popularity of the Internet grew and connecting computers became easier and cheaper.
% Thanks to wireless technologies, devices can connect to the Internet without having to be connected to a network.
% 
% Nowadays, everyday objects like cars or washing machines start to be connected to the Internet to exchange information.
% This is what is currently known as the \emph{Internet of Things (IoT)} \citep{atzori_internet_2010}.
% We can organize the IoT into \emph{Spaces} or islands which cover a particular knowledge as proposed by \citet{abdulrazakCGF10}.
% Using Spaces, we can ease the creation of different \emph{Ambient Intelligence (AmI)} environments.
% For example, we can create a Space for our home where we connect our devices to form an AmI environment.
% Thus, the washing machine could plan to finish washing the clothes two minutes before the car gets home.

Integrating mobile or embedded devices is not trivial as they usually communicate using different protocols.
To solve this problem, the \emph{Web of Things (WoT)} initiative proposes to use well-established web standards to ease their communication.
However, the format of the data they exchange is also multifarious and application domain dependent.
This implies that data will not be meaningful in other domains unless a specialized system converts and reinterprets them.
A way to solve this problem is annotating the data semantically as proposed by the WWW \citep{ChuaG10,kimKC11}.

Adding semantics to the IoT works well for devices with high computational capacity but it adds too much overhead for most of the devices composing the IoT.
To reduce this overhead in such devices, part of this computation is usually delegated to intermediaries \citep{honkola_smart-m3_2010}.
This approach reduces the overhead of semantically annotated data but brings other problems.
(1) When devices rely on others to provide information, it is not guarantee that the information accessed will accurately represent the last information available in the data providers (i.e., the sensors).
(2) Once we rely on those intermediaries, it is required them to be available at all time.
Otherwise, the devices would not be able to talk to each other.

We propose a solution where we use intermediaries to release some workload from the less powerful devices, but at the same time, we promote the direct communication between the devices.
In particular, our system uses intermediaries to search for data in the Space and queries the final devices directly.

In our solution, the devices can become or stop being intermediaries dynamically.
To decide which device will be an intermediary, we evaluate the state of a device (i.e., energy and computation capacity).
Using this dynamic architecture, the absence of a particular intermediary does not collapse the system.
In addition, our system is flexible enough to support a wide range of scenarios.

% hablar de las multiples clues
% IG: he reescrito un poco el tema, a ver que tal. Lo de payload a que se refiere?
% TODO AG: a la cantidad de informacion enviada
To enable the search for information in this system, intermediaries need to semantically aggregate the information in the Space.
We propose alternatives to aggregate this information taking into account the payload of the shared information and accuracy of the search.

We compare our approach against other common approaches.
In particular, we focus on the energy aspect, demonstrating that our approach helps energy constrained devices to face fewer unnecessary requests.
We also evaluate other important metrics like the number of messages a device has to exchange to perform a request, and the accuracy of the search results provided by the intermediaries.
We perform this evaluation under different scenarios to prove the flexibility of our proposal.

In summary, we make the following contributions:
\begin{itemize}
\item We present a new architecture for managing semantics on the WoT which reduces the load for resource-constrained devices.
% IG: he anadido esto basandome en el cambio anterior, hace falta verificar.
% TODO AG: Te refieres a distintos tipos de clues, no? Me parece correcto.
\item We propose and evaluate different approaches to aggregate semantic information and enable semantic search in this architecture.
\item We demonstrate the advantages of our approach compared to other typical searching approaches under different scenarios.
\end{itemize}

The remainder of this chapter is organized as follows:
Section~\ref{background} discusses the related work.
Section~\ref{proposal} and \ref{sec:clues} describe in detail our solution.
Section~\ref{environment} presents our environment and Section~\ref{results} evaluates our solution.
Finally, Section~\ref{conclusion} states the conclusions of our work.
\section{Background}
\label{background}

Querying over semantic content provided by independent sources is a problem addressed in the \acl{lod} field.
\citet{gorlitz_federated_2011} classifies the possible \ac{lod} infrastructures according to three characteristics:
1) how they store data,
2) whether the index is distributed and
3) whether the data sources cooperate.
Table~\ref{tab:infrastructure_lod} summarizes the resulting infrastructure types: central repository, federation and \ac{p2p} data management.


%  \multicolumn{7}{c}{\multirow{2}{*}{


\begin{table}[htbp]
  \caption{Infrastructure paradigms according to their characteristics \citep{gorlitz_federated_2011}.}
  \begin{center}
    \begin{tabular}{l|c|c|c|c|}
      \cline{2-5}
      ~ & \multicolumn{2}{c|}{Central Data Storage} & \multicolumn{2}{c|}{Distributed Data Storage} \\
      \hline
      \multicolumn{1}{|l|}{Independent} & \multirow{2}{*}{n/a} & ~ & \multirow{4}{*}{Federation} & \multirow{2}{*}{n/a} \\
      \multicolumn{1}{|l|}{Data Sources} & ~ & Central &~ & ~ \\
      \cline{1-2} \cline{5-5}
      \multicolumn{1}{|l|}{Cooperative} & \multirow{2}{*}{n/a} & Repository & ~ & P2P Data \\
      \multicolumn{1}{|l|}{Data Sources} & ~ & ~ & ~ & Management \\
      \hline
      ~ & Distr. Index & \multicolumn{2}{c|}{Central Index} & Distr. Index \\
      \cline{2-5}
    \end{tabular}
  \end{center}
  \label{tab:infrastructure_lod}
\end{table}


% Lo nuestro se parece más a LOD de dispositivos que a distributed triple store:
%     grafos semánticos de fuentes de datos independientes
Within the solutions which distribute the semantic content are federation and \ac{p2p} data management.
The difference between them resides in whether the index to look for content in the different machines is distributed or not.
In other words, a federated \ac{lod} infrastructure maintains the meta information about data sources in an element called \emph{federator} \citep{gorlitz_federated_2011}.
This meta information is used to delegate queries to data sources.

Therefore, according to this classification, our solution is a query federation solution.
However, the solution presented in this chapter presents some particularities:
\begin{itemize}
  \item This node which manages the \emph{indexes} is dynamically chosen between all the participants and can change over the time.
  \item The \emph{indexes} are replicated in each data consumer.
	This replication allows consumers not to critically depend on the availability of a this manager. % no depender => MUCHO
	This implies that some components of a \emph{federator} are implemented by more than a node at a time.
\end{itemize}


\bigskip


% presentamos 3 soluciones: FedX, ARQ y XXX.
Querying over federated databases is a well-addressed problem which provides relevant searching mechanisms.
However, there are significant differences when it comes to search over semantic content \citep{gorlitz_federated_2011}. % estoy citando estados del arte de otros, un poco gitaner :-P Ya puestos, citar el de FedX Optimization
Although it is a more unexplored field, federated distributed query over independent semantic data sources has some remarkable proposals are: FedX, DARQ and SemWIG.
% (usually justified by \ac{lod} solution)


% pros y contras:
% No asumimos SPARQL, pero al final todos estos mecanismos descomponen SPARQL en triple patterns como los nuestros
%    por lo tanto se podría implementar sobre lo nuestro lo suyo
%            (y permitir optimizaciones si introducimos SPARQL en algunos nodos)




% TODO meter enlace con lo de summaries de ABox en la parte correspondiente


% mencionar fokoue con lo de los resumenes?

% presentar las siguientes subsecciones

% intro a que ahora se va a hablar de soluciones IoT que usen semántica
% Having analyzed non-specific solutions for distributed semantic architectures, the next step is to evaluate the use of semantics from the \ac{iot} perspective.

% \label{sec:sw_providers}

% La acepción de Name resolution cambia depende del contexto!
% http://en.wikipedia.org/wiki/Name_resolution

% Name resolution vs recognition
% http://stackoverflow.com/questions/8589005/difference-between-named-entity-recognition-and-resolution

% Mirar en el punto 6.3 para ver unos pasos de consulta en la LOD que parece que pueden ser interesantes para esto
% http://www.unifr.ch/webnews/content/20/attach/4765.pdf


% no hay ninguna solución global con una arquitectura que tenga en cuenta le energía


% como se ha visto en TAL y TAL sección, to the best of our knowledge, no existen soluciones de busqueda de contenidos semánticos distribuido para WoT. (o al menos no para funcionar en cacharrillos)
% meter alguna referencia a Cabilmonte y demás?
\section{Energy-aware super-peer architecture} % pensar titulo más sexy
\label{proposal}

% In this section, we discuss the content the clues should contain to make them suitable to the WoT;
% we explain in detail the roles the nodes of our solution can have;
% we overview the deployment issues related to these roles, such as discovery or role transition;
% and we discuss how our solution can adapt multiple scenarios.

% AG: ¿node, peer y/o device? ¿Aclarar qué es un node?
% IG: personalment peer no lo veo, y entre node y device, pues creo que usarlos indistintamente, pero aclarando eso en algun punto
% AG: Crear comandos para asegurarnos de llamar a las cosas igual siempre (e.g. clues, acronimos, etc.)
% IG: comandos? esa es muy de crack xD

% En algún momento de la sección: CoAP vs HTTP
% AG: Igual habría que comentar que asumimos HTTP por su estado de madurez, pero que podría aplicarse a CoAP en cuanto fuese un estandar y con implementaciones que lo implementasen por completo. Esto tendría ventajas a nivel de uso de Multicast para agrupar WP, consumidores, etc.
%     Podríamos valorar su uso más minuciosamente en cada sección, pero igual en una primera versión no le daría mucho bombo.
% IG: influye en algo de la solucion es algo particular del experimental environment?
% AG: En su mayor parte no, pero habría algunas cosas que se simplificarían (e.g. comunicación con un grupo de nodos)

%   + Vinculo rapido con la intro (1 o 2 frases) % igual también con el related work
In a semantic WoT, nodes are part of a network where they share semantically described information.
%   + Mencionar el uso de espacios
These nodes gather their information in spaces which are useful to create AmI environments.

%   + Introducir ejemplo de hotel
For example, a hotel can create its own semantic space with the semantically described information of its services and the data provided by devices (e.g., sensors) spread around the hotel.
Thus, clients can use their own personal devices (e.g., smartphones) to interact with the devices of the hotel and search useful information.
Using this approach, a client can check the current swimming pool occupancy rate using his personal mobile phone.
% TODO este ejemplo anterior, como lo harias mas molon con semantica? search?

However, not all devices can afford the costs of fully managing the additional overhead of semantics.
In this section, we present our architecture to manage such environments and deal with the limitations of resource-constrained devices.

\subsection{Basic roles}
As we introduced before, devices can be part of one or multiple spaces.
A device can provide data (\emph{Providers}) consume data (\emph{Consumers}) or both at the same time.

\begin{description}
\item[Providers:]
This is the simplest role which any device in our system can carry out, even the smallest sensor.
Providers must manage their own semantic information.
In particular, they must organize triples in RDF graphs.
\item[Consumers:]
This is the role a device must take to get information from the system.
These devices need to be able to use semantics to find the data they require.
\end{description}

Note that a device could hold just one or both roles at the same time or switch between them.
For example, a \emph{Provider} could become a \emph{Consumer} to get data from the system or a \emph{Consumer} could stop asking for data.
% Cambios de roles:
% To Consumer: try to read something from the network. => evidente
% To not-consumer: "t" without querying anything (it does not worth to update the clues if we won't query).


\subsection{Use of Intermediaries: White Pages}
%   + Intermediaries
%   + Necesidad de tener white pages, mención de roles. => subsección N
Tasks involving the use of semantics can be expensive for some devices.
To reduce the load on such resource-constrained devices, we need to use intermediaries to carry out some of these tasks.
% No hacen tareas por ellos, más bien les evitan hacerlas.
%   + Por que queremos que ellos administren su info?
However, we do not want devices to completely rely on intermediaries for three reasons:
(1) data consumers need to get fresh data and we can only get this by direct access to the actual \emph{Provider},
% We aim to promote the direct communication between devices to ensure that freshness of the information shared.
(2) we must support mobile scenarios where personal devices carry their own data,
%   + Explicar un poco la naturaleza dinámica de los escenarios (sensores que entran y salen del sistema)
% In particular, we consider the dynamicity of WoT environments, where nodes can join and leave a space making their information available or unavailable at any time.
and (3) we need to reduce the maintenance complexity (nodes may join or leave at any time).

For these reasons, we cannot rely on intermediaries to host and provide \emph{all} the semantic information.
However, we use intermediaries as searching enablers in the semantic space.
This kind of intermediary is what we call \emph{White Pages (WPs)}.

\emph{WPs} manage indices or \emph{clues} about the information shared by others.
The \emph{Consumers} rely on \emph{WPs} to enhance the search process and reduce the number of requests generated.
Consequently, the \emph{Providers} process less semantic data and reduce their overall overhead.
Thus, our architecture enables an intermediary-aided energy-aware information search.


\InsertFig{architecture}{fig:networkTraffic}{
  Role of White Pages in our proposal
}{}{0.8}{}


Figure~\ref{fig:networkTraffic} shows the role of \emph{WPs} in our proposal.
In particular, \emph{WPs} manage clues which summarize the semantic information shared by the \emph{Providers}.
These \emph{clues} are pieces of information useful to determine which nodes can answer to a certain query.
\emph{Consumers} use these \emph{clues} to directly access the semantic information on the \emph{Providers}.
Summarizing, the main tasks of a \emph{WP} are:
\begin{itemize}
 \item Manage the clues sent by all the nodes in the system.
 \item Aggregate clues sent by \emph{Providers} in a unique response.
 \item Reply to requests from \emph{Consumers} with aggregated clues or a list of nodes (details in Section~\ref{sec:interacting}).
\end{itemize}
 % debería haber un único \emph{WP} o varios? AG: Optaría por 1 para simplificar.
 % Dependiendo de eso, puede ser más complejo mantener una visión común de los clues de todos
 % (con algún algoritmo de gossiping?)
 % Para simplificar el merging de los differentes gossipings agregados se podría hacer: 1 master, N slaves.

\emph{WPs} only store clues for a period and they expire afterwards.
Using this approach, we avoid storing information from no longer available nodes.

% Si hubiese que dar detalles de implementación, había pensado en Redis para el \emph{WP} a poner en un server o cluster
% De forma que la gestión distribuida de dicho gossiping no fuese problema mio en esos casos
% Veo 2 casos:
%    + Un cluster de servidores hacen de WP: que usen algo que ya exista y me deje en paz.
%    + Dos/n móviles hacen de \emph{WP} conjuntamente: entonces sí que debería mojarme y decir como mantienen esa visión
%      conjunta.
% El problema es que al haber cambiado la explicación de tipos de cacharros a roles, esta consideración es difícil de
%  hacer.


% TODO ENERO nueva version
% TODO lo he movido para que no rompa el flujo
\subsection{Versioning Clues}
As previously explained, the \emph{WP} aggregates clues sent by \emph{Providers}.
This aggregation is versioned using a \emph{version number} and a \emph{generation number}.
The \emph{version number} increases each time the \emph{WP} receives a new clue.
The \emph{generation number} ($g_{id}$) is a timestamp that represents the moment a new \emph{WP} was chosen.
This requires the clocks to be synchronized in all the nodes to avoid problems.
In any case, the \emph{WP} must guarantee that the $g_{id}$ is higher than the one used by a previous \emph{WP}.

The \emph{WP} maintains two version numbers:
(1) the one used during the setup to do the first load (\emph{setup version}) and
(2) the version of the last aggregated clue.
The first one is shared through the discovery mechanism.
The second one is shared with both \emph{Providers} and \emph{Consumers} in HTTP responses.
The versioning is used to improve the \emph{WP} selection process explained in Section~\ref{sec:selection}.


\subsection{Discovering a White Page}
% He matizado la frase para que no se entienda que los Consumer SIEMPRE dependen de los WP. Sólo para actualizar clues.
% AG: te has empeñado en quitar el matiz de que "sólo" al principio está vendido a la existencia de un WP :-)
In our architecture, when a \emph{Consumer} joins the space it relies on a \emph{WP} to find information.
Hence, the first thing a node needs to do afterwards is discovering who is the \emph{WP} in that space.

To run our proposal, we require a discovery system able to
(1) get the spaces which a particular node belongs,
(2) identify the \emph{WP} in the system and its \emph{setup version}, and
(3) provide additional information about nodes to decide which one can be the next \emph{WP}.
To implement our architecture, the information about the nodes must include:
battery level, available memory, storage space, connection time, and if they have been active in the last period.

How to discover the nodes in the space is transversal to the architecture presented.
We could either extend approaches like UPnP, mDNS or lmDNS \citep{jara_light-weight_2012} or use HTTP, CoAP and DNS together as proposed by \citet{ishaq_facilitating_2012}.
In particular, in Section~\ref{sec:mdns} we use mDNS to evaluate our solution.
% IG no entiendo lo del discovery strategy, que se intenta decir
% to demonstrate its feasibility with a particular discovery strategy.
% TODO ENERO habia un revisor que dijo:
%  "How does the system receive the information about the battery level, available memory?
%   The authors justify that this is out the scope of this paper, but it is important to know
%   the potential overhead introduced in these processes."
% Es una forma de responderle. Con esa frase quería decir:
%  " Esto no solo funciona en el plano teorico, usando un sistema como mDNS es perfectamente
%    posible implementarlo porque compartir esa pizca de info, pese a que te llegue un poco
%    desactualizada, no es un problema. "

% - Which is the topology of the network?
% It is hard to evaluate the proposed solution in terms of overhead and energy consumption for performing necessary tasks (WP discovery, WP selection, clues updates,...)
% without any knowledge of the network. Are there wired/fixed nodes? Are WP infrastructural nodes? Are the wireless nodes connected through multihop paths?
Note that we assume that any node in the space can reach the other nodes.
In particular, we assume IP addressability (no matter whether the node is wired or wireless).
As a consequence, to use devices from Wireless Sensor Networks (e.g., Zigbee or 6LoWPAN) as \emph{Providers}, one should rely on gateways.

% IG el CoAP multicast no tiene una utilidad diferente a lo q se habla?
% Se podría usar UPnP, mDNS o  incluso lo propuesto por Jara (lmDNS).
% También se podría usar multicast CoAP (e.g. los que respondan a /\emph{WP} o /consumers)


% Partimos de la base de que el sistema de descubrimiento nos ha descubierto a otros nodos y los siguientes datos acerca de ellos:
%    - Battery (B): valor [1 a N] y unidad [min, hora, dia] -> siendo -1 infinito o cargando.
%         Problema: puede que no sea fácil hacer ese cálculo.
%         Solución: mejor eso a carga relativa o medida de carga, cuando no sabes cuanto consume el cacharro de normal.
%    - Memory (M): Valor [1 a 1023] y unidad [KB, MB, GB]
%    - Persistence (P): Valor [1 a 1023] y unidad [MB, GB, TB]
%    - Joined since (JC): Tiempo desde su unión al espacio (para permitirnos medir su estabilidad): medido en ciclos de media hora
%    - Stores aggregated clue (SAC): [T o F] ¿Tienen una versión del clue agregado?
%
% Hemos tratado de no crear mucho tráfico en el mecanismo de descubrimiento almacenando datos que varíen con poca frecuencia.
% Los campos más variables: la batería y JC, pueden no actualizarse en media hora sin que el algoritmo se vea sustancialmente alterado.
%
% Comprobar que no son datos que costase mucho embeber en cualquier sistema (registro TXT, lo que sea que use UPnP...)
% Jara habla de 80 bytes en 6LoWPAN, probablemente tampoco nos orientemos a cacharros tan peques, pero oye, por comentarlo...


\subsection{Interacting with a White Page}
\label{sec:interacting}
When a node needs to access a piece of information, it asks the \emph{WP} for the information about the other devices in the space.
Using this information, the node can find the owner of the required data and ask it for this information.

In our architecture, \emph{Providers} and \emph{Consumers} have additional duties.

\noindent\textbf{Providers.}
They manage their own semantic information and generate \emph{clues} about the information they host.
Then, they send these \emph{clues} to the space \emph{WP}.
As a response, they receive the last version of the aggregated clue they have contributed to.
\emph{Providers} will send their \emph{clues} to the \emph{WP}
(1) every time a \emph{clue} is updated,
(2) before the lifetime of a \emph{clue} expires, or
(3) whenever there is a new \emph{WP} in the space with a lower \emph{setup version} than the one in the \emph{Provider}.
Note that \emph{clues} do not change frequently since they represent the type of the information the nodes host rather than the specific data which is constantly generated.

\noindent\textbf{Consumers.}
When a \emph{Consumer} needs to get information from the system, it first needs to find the \emph{WP} in the space and use this \emph{WP} to obtain an aggregated clue of all the nodes.
Then, they process this aggregated clue to decide which are the nodes to query.
They use a RESTful approach to send this query.

They perform this process synchronously for the first query and periodically in an asynchronous manner for the following ones.
% cada cuanto? arbitrario? que lo decida en base a parámetros de funcionamiento? => Mi propuesta abajo
This period should have an upper limit to ensure a fresh view of the space and a lower limit to guarantee that the \emph{WPs} are not flood.
To adjust the update frequency within these two limits, we evaluate the frequency of the last 10 requests to that node.
Thus, the view of the space will be fairly up to date when the \emph{Consumer} processes the next query.
% TODO Añadir mejora: cuando se detecta un nuevo nodo por el mecanismo de descubrimiento, esperas un poco y actualizas clue para ver si ya ha enviado su info al WP.
% Mejora de la mejora: exclusivamente preguntas por la clue de el/los nuevo/s nodo/s detectado/s. Asi envias un JSON más peque.

\noindent\textbf{Optimizations for resource constrained devices.}
Note that \emph{Providers} will update the \emph{WPs} with new information and \emph{Consumers} will periodically check the \emph{WPs}.
Using this approach, the \emph{WPs} reduce the network load and in particular, they decrease the load for resource constrained devices.

In addition, \emph{aggregated clues} can be too long for some really limited devices.
As an optimization for such devices, the \emph{WP} is able to answer specifically the node to address a query.
Thus, these nodes will only maintain a list of the nodes they should ask for several predefined queries.

% Limitar la información enviada también a otros nodos?
% Si tenemos 1024 nodos y cada gossiping es de 100KBs, vamos a tener 100MBs de gossipings :-S
% Aunque esto es justo lo que hay que discutir en el apartado de evaluación de clues...


\subsection{Selecting a White Page}
\label{sec:selection}
%   + Que queremos solucionar => subsección K (tipos de sistemas)
Depending on the setup, having a dedicated \emph{WP} (e.g., a high-end server) can be too expensive.
For example, in a domestic environment, it is not worth dedicating a full server to be a \emph{WP}.
However, when we have a large setup (e.g., a hotel), it is convenient to dedicate a few servers to decrease the load in small devices.
%   + Necesidad de que esos roles sean algo dinámico y el problema que ello presenta
%      (descubrimiento, pero sobre todo selección) => subsección M

We must be able to manage the complexity of a system composed by heterogeneous devices.
For example, making a small device a \emph{WP} may be inappropriate in highly populated environments.
On the contrary, having multiple dedicated servers implies a high management overhead which is unnecessary in simple environments.
Our architecture is flexible and any node can be a \emph{Provider}, a \emph{Consumer}, a \emph{WP} or any of there at the same time \citep{larrea2012specifying}.

\noindent\textbf{When does the \emph{WP} selection start?}
The selection process can start (1) when no \emph{WP} is available or (2) when the current \emph{WP} gets a worse score than other nodes.
In the first case, the first node to realize there is no \emph{WP} starts the selection process.
In the second case, the current \emph{WP} periodically checks if there is another node with a better score.
If this actually happens, it starts the process.
We limit the frequency for this checking to avoid the overhead associated with this change.

\noindent\textbf{How to select a WP?}
The node in charge of selecting a \emph{WP} retrieves information of available nodes using the discovery mechanism.
It ranks them, selects the top one, and informs it that it should become a \emph{WP}.
If a selected node rejects the new role, the selector chooses the next one on the ranking.

The node in charge of the selection sends its aggregated clue to the new \emph{WP}.
If this aggregated clue is fresher than the one in the \emph{WP}, the \emph{WP} will use this one.
Otherwise, the \emph{WP} will use its own.
In case there is no previous aggregated clue, the new \emph{WP} will start from scratch (version to -1).
Once the node becomes the new \emph{WP}, it notifies the version used to initialize to the discovery system.
Doing so, \emph{Providers} will realize if they must send their clue again because it is not included in the aggregation provided by the \emph{WP}.

Note that the discovery mechanism must provide the following information to the selection algorithm:
(1) memory of the device,
(2) storage capacity of the device,
(3) time since it joined the space, and % esto no debería cambiar a menudo para no generar mucho trafico
(4) its battery charge.
We use the algorithm detailed in Listing~\ref{list:selectionAlgorithm} to rank the nodes.

\begin{listing}
  \begin{algorithmic}[1]
  \State $nodes \gets filter_{threshold}(nodes, ``memory'', threshold_{memory})$
  \State $nodes \gets filter_{threshold}(nodes, ``storage'', total_{nodes} \times storage\_needed_{avg})$
  \If { $anyWith(nodes, battery_{infinite})$ }
    \State $nodes \gets filter(nodes, ``battery'', battery_{infinite})$
    \State $nodes \gets orderBy(nodes, ``battery'')$
  \Else
    \If { $anyGreaterThan(nodes, ``joined\_since'', joined_{threashold})$ }
      \State $nodes \gets filter_{score}(nodes, ``joined\_since'')$
    \EndIf
    \State $nodes \gets filter_{score}(candidates, ``battery'')$
    \State $nodes \gets orderBy(nodes, ``memory'')$
  \EndIf
  \State \Return $nodes$
  \end{algorithmic}
  \caption{White Page selection algorithm.}
  \label{list:selectionAlgorithm}
\end{listing}

Lines 1 and 2 of the algorithm apply filters based on thresholds.
In particular, the second threshold is variable and depends on the number of nodes (more nodes will generate more clues to store).
After that, we check the power availability of the devices.
We first select the nodes connected to the power grid (represented by $Battery_{infinite}$).
If there are devices connected to the grid, we select devices which are steady enough to apply a filter by $joined\_since$ field.
Filters in lines 8 and 10 choose nodes with z-scores higher than one for the specified fields.
If no node fits that filter, it returns the nodes with values higher or equal to the mean.

% AG: Intentando aclarar si un móvil puede ser WP o no (a un revisor no le había quedado claro)
Summarizing, the algorithm prioritizes energy autonomous nodes and within the nodes with battery limitations, it prefers steady ones.
% IG pues si quieres aclararlo, remata xD
In both cases, it finally selects a node with enough storage and memory (including mobile devices).

% AG: Problema a mencionar: y si no hay ningún \emph{WP} candidato? Broadcasting entre cacharros?

% \noindent\textbf{\emph{WP} conflict resolution}
Due to information out-of-date, two nodes may become \emph{WPs} at the same time.
Our solution would eventually correct this situation thanks to a conflict resolution algorithm.
When a \emph{WP A} detects another \emph{WP B} in the space, it has to check which one has a better score according to Listing~\ref{list:selectionAlgorithm}.
If \emph{B} has a better rank, \emph{A} simply resigns as \emph{WP} and notifies it to the discovery system.
Otherwise, it forces \emph{B} to resignate through a REST invocation.
Other nodes will be aware of these changes through the discovery mechanism.

% TODO revisar todas estas ideas felices que puse en su día sobre clues...
%   4. El nodo elegido como whitepage, comprueba que no hay otro ejerciendo de WP.
%        a) De ser así, comprueba si tiene un ranking mejor.
%            1) Si lo tiene se quita de WP.
%            2) Si no lo tiene
%               a) le coge clue agregado (porque es la más actualizada) % porq?
%               b) le avisa de que será el nuevo \emph{WP} haciendole un /whitepage/claim
%       b) Si no, comprueba que exista algún nodo con SAC=true en el espacio.
%	     1) De ser así, escoge uno de ellos al azar, y le pide la Clue agregada
%            2) Tan pronto como otros nodos con SAC=true detecten que hay un nuevo WP, le enviarán sus clues agregados para lograr que este tenga el más actualizado
%       c) Si no existen, irá recogiendo clues que le lleguen.
%       NOTA: Los providers envian sus clues si detectan que se ha pasado de una situación de no existir nadie con SAC=true en el espacio a tener WP.
%       IG: esto ultimo deja claro que podemos gestionar lo de multiples nodos sin problemas... eso va un poco en contra del assumption inicial, hay q unificar criterio
%       AG: No, de hecho evitamos que multiples nodos sean providers obligando a que se arreglen entre ellos.
%	IG: bueno, lo puedes enfocar, como que puede pasar que haya mas de un WP, pero que se corrige eventualmente.
%		digo por integrar lo del principio un poco mejor sin excepciones
%       AG: Compro.






\section{Shared Clues}
\label{sec:clues}
As we introduced before, \emph{WPs} host \emph{clues} about the information in the space.
Using a \emph{clue}, a \emph{Consumer} can find which node (or nodes) is the \emph{Provider} of a piece of information.
\emph{Providers} generate these clues by digesting the semantic information they store.

Thanks to these \emph{clues}, resource constrained devices do not have to process unnecessary requests.
We also limit the length of the \emph{clues} to reduce the bandwidth, the memory, and the storage overhead on these devices.

In this section, we describe in detail these \emph{clues}, the information they contain and their format.

% TODO IG habria que comprobar la consistencia de los titulos, todos los subs con mayuscula al inicio o no, etc.
\subsection{Content of a Clue}
To find out which is the most appropriate solution for AmI scenarios, we have to consider scenarios populated by mobile devices and sensors. % for IoT scenarios?
Mobile devices usually share data which rarely change described using a few ontologies (e.g., the user profile and his preferences).
On the other hand, sensors are constantly generating new instances of the same ontology (also called individuals).
In both cases, the data shared by each node is described according to one or few vocabularies or taxonomies.

At this point, it is important to define the \emph{TBox} and \emph{ABox} concepts following the definition of \citet{nardi2003introduction}.
\emph{TBox} contains knowledge describing general properties of concepts or terminology and
\emph{ABox} contains knowledge specific to the individuals of the domain of discourse.
An example of \emph{TBox} information is the device type or the elements it is made of
while \emph{ABox} can describe the mobile phone brand or the temperature sensed by a thermometer.

Each mobile device or sensor usually generates \emph{ABox} according to the same \emph{TBox}.
Using this information, we propose avoiding the use of URIs which represent \emph{ABox} information in general terms.
Specifically, we propose and evaluate three possible type of clues to share.
%The first two are inspired by early works on peer-to-peer semantic web $[CITAEDUTELLA,devadithya2007index]$.

\medskip

\noindent\textbf{Schema-based clue.}
The individuals (\emph{ABox}) are described according to different schemas or ontologies (\emph{TBox}).
Therefore, a coarse-grained step to find relevant nodes for a query is to ignore those which do not have \emph{ABox} for a specific \emph{TBox}.
To create schema-based clues, the \emph{Providers} have to extract the prefixes used in their graphs.
Although not all the prefixes used in a RDF graph can be related to specific schemas,
they can be used to know when the URIs of a template start by a prefix.
% Entiendo que puede no quedar claro, poner ejemplos ayudaría?
% IG: in the same what?

\medskip

\noindent\textbf{Predicate-based clue.}
The predicates relate subjects with other subjects or literals.
The predicates are defined in the \emph{TBox} (e.g., to state they relate concept A with concept B) and used in each triple of the \emph{ABox}.
In this approach, we propose extracting the set of predicates used in the graphs stored by each node.
Using this information, they can be simply matched with the predicate defined in the query template.
% AG: Posible pega, se ha explicado antes lo que es un "template"?
% IG: nope, de hecho en la anterior tb me ha surgido la duda de q era

\medskip

\noindent\textbf{Class-based clue.}
In the third approach, we propose sharing the classes of concepts (\textit{rdf:type}) shared by the nodes.
Using this information and the \emph{TBox}, each node can check if the information matching certain templates is susceptible to be stored in other nodes.
For this kind of clues, we assume that each node should have (or be able to obtain) the \emph{TBox} related to the queried template.
This is a reasonable assumption since ontologies describing \emph{TBox} are usually accessible on the URL described by its prefix.

Assume a template that defines a predicate \emph{p} and this predicate relates the concept \emph{A} with another concept according to the \emph{TBox}.
We will send this template to nodes storing instances of the class A assuming that these nodes are more likely to use this predicate in their graphs.

% IG: TODO esto de referirte a una ontologia del futuro... no me parece muy asines
% AG: Exactamente mismo caso que arriba, pero con ejemplos concretos.
%     La idea era que no quedase muy abstracto. Lo comento porque ahora puede quedar redundante.
%For instance, consider that the following template is queried: \textit{?s~ssn:observes~dbpedia:CO2}.
%According to the SSN ontology, which will be presented later on, only an instance of the class \textit{ssn:Sensor} \textit{observes} something.
%Therefore, the querying node will send the template to the nodes which have instances of the class \textit{ssn:Sensor}.


\subsection{Reasoning to expand clues} % Esto se relaciona principalmente con class-based, pero igual también puede comentarse para Predicate-based.
Through a reasoning process one can know unstated knowledge.
Using this knowledge, we can detect more relevant nodes.
For example, a node which has many instances of the class C, may also use the predicate \emph{p} if C is a subclass of A.
Thanks to the reasoning, we can discover that the node has knowledge of the type A and therefore, it may be relevant.
% AG: Comentado por lo mismo que arriba
%(in the example, instances of \textit{ssn:Sensor} class' subclasses such as \textit{Accelerometer}).
The drawback, is that reasoning consumes a lot of resources.
This limitation will be further analysed in Section~\ref{sec:clues_eval}.
% TODO Many authors solve this limitation by transforming the queries in the consumer side \cite{}.


\subsection{Use of ABox in clues}
\label{sec:aboxinclues}
% se podrían usar individuals que no cambian mucho
% location, individual en base al cual se generan todos los elementos.
As stated before, in general terms, we want to avoid the use of \emph{ABox} URIs (individuals) in our clues.
Thus, we fulfil two goals:
(1) we generate smaller clues and
(2) the clues will not change too frequently and therefore, we will require less communication to update clues.
However, in some cases, the use of \emph{ABox} content in the \emph{clues} may be beneficial.
For instance, assume a URI that refers to the specific location \emph{L}.
If we want to search for devices in location \emph{L}, we cannot deduce anything about it using the proposed \emph{clues}.

% AG: Atencion, a partir de aqui viene una propuesta que me acabo de sacar de la manga que habra que discutir.
%     El problema de esto es que no está en la evaluación y para el 15 de Octubre, lo veo ajustado de incluir.
For this reason, we need to consider sharing the \emph{N} most queried individuals in our clues.
To do that, the \emph{WP} needs to store a list with the statistics about the information collected by each \emph{Consumer}.
\emph{Consumers} can send this information together with the request to update a clue.
\emph{Providers} can obtain a list of the current most popular URIs before sending their updated clues to the \emph{WP}.
Using this list, \emph{Providers} can know if they have these URIs and include them in the clue to be sent to the \emph{WP}.
Note that this process would imply an extra request per provider before each update.

This simple approach implies sending not only \emph{TBox} but also \emph{ABox}.
The amount of extra information added to each clue will depend on the size of this list (\emph{N}).
The effectiveness of this method will depend on the number of queries using one of the \emph{N} URIs in their subjects or objects.
% Propuesta: uso de un porcentaje de las queries realizadas
% Problema: no vas a raspar a cada providers con 1000 comprobaciones tampoco!
% Solución: ¿merece la pena evaluar eso? ¿como?


\subsection{Format}
% Tras explicarlas, poner algunos fragmentos con las clues (JSON) para que se vea de que hablamos
We can use many formats to represent the content of a clue.
% pongo esto de EXI porque se que a la gente del mundillo de IoT se le hace el culo chupicola con sus estandares
One option is the ongoing Efficient XML Interchange (EXI) \footnote{\url{http://www.w3.org/XML/EXI}}. % o cite?
EXI is designed to efficiently interchange XML data and therefore, we could obtain better compression rates than with JSON. % aseveracion cierta o solo para XML?
However, we have chosen JSON for its simplicity and its wide adoption in the WWW.

% TODO no lo he acabado de pillar...
The schema-based clue is the easiest one to represent in JSON since it is formed by a set of URIs.
We can also represent the predicate-based and the class-based clues using a set of URIs.
However, the prefixes of those URIs are usually repeated and for this reason, we do not use plain URI transmission for these clues.
We show an example in Listing \ref{list:oneClue}.
It first defines the prefixes used and gives them a name and then, it specifies the URI endings for each prefix.

\begin{listing}
\begin{minted}[frame=single, framesep=3mm, linenos=true, xleftmargin=21pt, tabsize=4]{js}
{
  "s": [[
      "om-owl",
      "http://knoesis.wright.edu/ssw/ont/sensor-observation.owl#"
  ]],
  "p": {
    "om-owl": ["result", "procedure",
		"observedProperty", "samplingTime"]
}  }
\end{minted}
\caption{Representation of a predicate-based clue in JSON.}
\label{list:oneClue}
\end{listing}

\medskip

These are isolated clues sent from a \emph{Provider} to the \emph{WP}.
However, the \emph{WP} gathers all these clues in an aggregated clue which is sent to the \emph{Consumer}.
We show an example of an aggregated clue in Listing \ref{list:aggregatedClue}.
As one can see, the aggregated clue follows the format of an individual clue.
This is described by the numeric field \emph{i} which contains a common area that describes all the prefixes.
\emph{g} and \emph{v} form the version of this aggregated clue.
Finally, for each node, each prefix is related with the URI endings.

% TODO sacar un ejemplo real? poner esquematicamente?
\begin{listing}
\begin{minted}[frame=single, framesep=3mm, linenos=true, xleftmargin=21pt, tabsize=4]{js}
{
  "i": 1,
  "g": 2435467,
  "v": 556,
  "s": [
	["dc", "http://purl.org/dc/elements/1.1/"],
	["dul", "http://www.loa.istc.cnr.it/ontologies/DUL.owl#"],
	["ssn", "http://purl.oclc.org/NET/ssnx/ssn#"] ],
  "p": {
    "node1": {
      "ssn": ["observedBy", "observationResult"],
      "dul": ["isClassifiedBy"] },
    "node0": {
      "ssn": ["observes"],
      "dc": ["description"]
} } }
\end{minted}
\caption{Representation of an aggregated clue in JSON.}
\label{list:aggregatedClue}
\end{listing}

% Mencionar: Note that we can gossip sleeps periods to know when to query them...
% Comentar que se podrían gossipear los sleeps de los nodos embebidos para saber cuando interrogarles.
% AG: Esto es un aspecto que puede ser guay comentar, de cara a IoTificar la propuesta.
%     Pero pensandolo mejor, debería ir en la sección 3, a la hora d explicar el formato de las
%     clues.
\section{Experimental Environment}
\label{environment}
We have used simulation to study the performance of our solution and compare it against a flooding-based one.
Using a simulation, we can evaluate multiple scenarios and repeat experiments for different approaches under the same conditions.
The source code for the evaluation is publicly available\footnote{\url{http://gomezgoiri.net/files/code/gomezgoiri201Xenergy.html}}.

\subsection{Methodology}
Table~\ref{tab:configurationParameters} shows the main parameters of our simulator.
We vary these parameters to simulate a wide range of scenarios.
% Dado que los revisores parece que no hicieron cado de este aviso, mejor quitarlo.
% We omit topology considerations as we assume devices have IP connectivity.


\begin{table}
  \centering
    \begin{tabular}{l p{7cm}}
      \hline
      Name & Description \\
      \hline
      Network size & The number of nodes in a network. In the simulations conducted, all the nodes are \providers{}. \\
      Number of writes & Amount of writes performed during the simulation period. \\
      Number of queries & Amount of queries performed during the simulation period. \\
      Number of \consumers{} & Amount of nodes querying to other nodes in the \Space{}. \\
      Distribution strategy & Our solution or \ac{nb}. \\
      & In \ac{nb} nodes write locally and spread the queries to the rest of the nodes in the \Space{}. \\
      Drop interval & At the beginning of this interval a node abruptly leaves the network. \\
      & At the end, the node joins the network and a new one is chosen to leave it. \\
      % FUTURE WORK: añadir métrica de accuracy [ http://en.wikipedia.org/wiki/Recall_(information_retrieval) ]
      \hline
    \end{tabular}
    \caption {Configuration parameters.}
  \label{tab:configurationParameters}
\end{table}

% Since we wanted to simulate the Triple Space Computing paradigm over HTTP, the communication between the nodes was point to point and the data exchanged were RDF Triples.
% The node discovery process was ignored since it will show similar additional overhead for each strategy.
% It can be considered transversal to what was being measured.
% IG: version del parrafo anterior
% AG: Ok, pero esta es una de las cosas que en el uptade debería volar o cambiar. Sobre todo, porque no hablamos de TSC
%     en ningún otro lado. Puede ser más fácil aún, ya que como asumimos que queremos semántica en WoT, el HTTP está
%     justificado per se.
As we simulate \ac{http}, we assume a point to point communication between devices which exchange \ac{rdf} Triples.
We discuss how the discovery process affects the solution in Section~\ref{sec:dynamic}.
% TODO ENERO
In the rest of the sections we omit the node discovery process as it represents the same overhead for all strategies.

%To represent the data managed by each node, we first considered using LUBM\footnote{\url{http://swat.cse.lehigh.edu/projects/lubm/}}, a synthetic benchmark.
%Unfortunately, it creates instances from very few classes for each node, which makes all the nodes to have the same TBox.
%In our opinion, this does not faithfully represent Internet of Things scenarios with heterogeneous devices sharing disparate information.

%In our second attempt, we found a dataset which could represent this heterogeneity.
To represent the data managed by each node, we used data from diverse sensor network environments.
These data follow the \textit{Semantic Sensor Network Ontology} (SSN) \citeweb{ssn}.
% created by the W3C Semantic Sensor Network Incubator Group. to represent diverse sensor network environments.
SSN has been used in many projects and scenarios to describe semantically the data provided by heterogeneous sensors.
% TODO IG hasta aqui hemos llegado
Specifically, we used data from the following datasets:
AEMET metereological dataset \citeweb{aemet},
University of Luebeck Wisebed Sensor Readings \citeweb{luebeck},
\emph{Kno.e.sis} Linked Sensor Data \citeweb{knoesis}
and Bizkaisense \citeweb{bizkaisense}.
These datasets contain descriptions about the sensing stations and the data sensed by them during certain periods.
The analogy between stations which have different sensors and the \ac{iot} devices is reasonable.
The datasets have been adapted to provide just one measure of each sensor at each moment (to emulate the storage restrictions from embedded devices) and to use as many stations as nodes has the network (depending on the network size).

However, not only sensors but also personal devices (e.g. mobile phones) usually populate AmI environments.
To represent such circumstance, we added semantic data of people to represent their profiles \citeweb{morelab_people}.

% poner disponible el código
We use SimPy \citeweb{simpy} to simulate each scenario.
SimPy is a process-based discrete-event simulation language for Python.
To accurately simulate the time needed by each node to provide a response, we considered measures taken from
real embedded web servers running on a \textit{ConnectPort X2} IP gateway \citeweb{connectportx2} for \textit{Digi}'s \textit{XBee sensors} \citeweb{digixbeesensors} (\textit{XBee} from now on),
on a FoxG20 \citeweb{foxg20} and on a Samsung Galaxy Tab \citeweb{samsunggalaxytab}.
Besides, we also provide measures taken from a regular computer.
Table~\ref{tab:measures_embedded} shows the measures used for the parametrization.

% TODO TODO TODO poner o referenciar las características de los dispositivos cogidas de WoT 2012

\begin{table}
  \begin{center}
	\begin{tabular}{p{2.5cm} r r r r}
	  \hline
	  & \multicolumn{4}{c}{Devices} \\
	  \cline{2-5}
	  Concurrent & \multirow{2}{*}{XBee} & \multirow{2}{*}{FoxG20} & Regular  & Samsung \\
	  requests   &  ~    &   ~     & computer & Galaxy tab \\
	  \hline
	  1  &  77 (1)	&  17 ~(0)  &   5 ~(1)  &  223 (349) \\
	  5  & 392 (8)	&  97 (16) &   8 ~(4)  &  256 ~(76)  \\
	  10 & 775 (8)	& 174 (28) &  13 ~(8)  &  372 (171) \\
	  15 &  -   	& 282 (43) &  18 (13) &  497 (191) \\
% Tampoco creo que haga falta tanto detalle, ¿cuando se van a dar tantas peticiones concurrentes?
%	  20 &  -	    & 375 (30) &  23 (13) &  661 (444) \\
%	  25 &  -	    & 460 (30) &  30 (18) &  748 (288) \\
%	  30 &  -	    & 540 (35) &  38 (22) &  929 (805) \\
%	  35 &  -	    & 632 (29) &  38 (20) & 1029 (672) \\
	  \hline
	\end{tabular}
  \end{center}
  \caption{Mean of the measurements taken in different devices with the standard deviation in parenthesis (milliseconds).}
  \label{tab:measures_embedded}
\end{table}


\subsection{Performance Metrics}
To evaluate the fundamental properties of the strategies, we use the following metrics:
\begin{itemize}
  \item \textit{Precision}: the fraction of nodes which answered relevant results (those responses which were not \textit{not found} responses).
                            It measures the exactness of the results.
  \item \textit{Recall}: the fraction of relevant answers that are returned.
                         It measures the completeness of the results.% overload, failed search, etc.
  \item \textit{Size}: the size of each type of clue.
%  \item \textit{Throughput}: the average rate of successful message delivery over a communication channel.
%  \item \textit{Idle time}: the average time each node is in an idle state and therefore is consuming less energy.
%  \item \textit{Failed requests}: the amount of request which could not be answered due to the physical limitations of
% the nodes (i.e. both due to timeouts or connection rejections when the server is overloaded).
  \item \textit{Total requests}: the number of \acs{http} requests performed during a simulation.
  \item \textit{Response time}: the average time needed to obtain an \acs{http} response.
  \item \textit{Active time}: the total time spent by each node either querying other nodes or handling a query.
\end{itemize}
\section{Evaluation}
\label{sec:search_evaluation}

\subsection{Types of clues shared}
\label{sec:clues_eval}
As presented in Section~\ref{sec:clues}, the type of clue used will affect
\begin{enumerate*}[label=\itshape(\arabic*\upshape)]
  \item the \emph{precision} and \emph{recall} to find the nodes with the appropriate information; and % explained in section 4
  \item the amount of information to transfer over the network (both requests and responses) and nodes have to process.
\end{enumerate*}
Increasing \emph{precision} reduces the number of unsuccessful requests to handle and thus, it reduces the energy consumption. 
Similarly, sending more information over the network implies more processing time and more energy consumption.

\medskip

\noindent\textbf{Precision and recall.}
We evaluate the \emph{precision} and the \emph{recall} of the proposed algorithm in a network of 470 nodes issuing the query templates shown in Table~\ref{tab:evaluationTemplates}.
In average, the nodes manage instances belonging to 6.34 different classes (standard deviation, $SD=1.31$) among a total of 17 distinct classes in the \Space{}.
%When these classes are expanded with a reasoning process, the nodes store 20 different concepts among a total of 113 classes in the space ($SD=4$). % TODO actualizar
The distinct predicates managed by each node in average are 16.01 ($SD=1.53$) out of 68 different predicates in the \Space{}.


\newcommand{\tplone}{\emph{T1}}
\newcommand{\tpltwo}{\emph{T2}}
\newcommand{\tplthree}{\emph{T3}}
\newcommand{\tplfour}{\emph{T4}}
\newcommand{\tplfive}{\emph{T5}}


\InsertTab{tab:evaluationTemplates}{Templates used in the evaluation}{}{
  \begin{tabular}{ll}
    \hline
    Name & Template \\
    \hline
    \tplone{} & \texttt{?s~~rdf:type~~ssn-weather:RainfallObservation} \\
    \tpltwo{} & \texttt{?s~~wsg84:long~~?o} \\
    \tplthree{} & \texttt{?s~~ssn:observedProperty~~?o} \\
    \tplfour{} & \texttt{bizkaisense:ABANTO~~?p~~?o} \\
    \tplfive{} & \texttt{?s~~dc:identifier~~?o} \\
    %t9 & ?s~~~foaf:family_name~~~?o \\ % Quizá en futura version del paper con perfiles de usuarios, para meter mas variabilidad en contenidos ;-)
    \hline
  \end{tabular}
}{}
% Las consultas se podrían sacar también de bizkaisense o alguna app que hayamos hecho para darle mayor verosimilitud


% Explicar class based
In Figures~\ref{fig:recall_measures}~and~\ref{fig:precision_measures}, the class-based clue shows a good \emph{precision} and \emph{recall} for \tplone{} and \tpltwo{}.
\tplone{} asks exactly for the information this type of clues define (i.e., nodes having instances of a certain class).
\tpltwo{} evaluates which nodes have instances in the domain of the \emph{long} predicate (\emph{SpatialThing}).
Note that this works thanks to the \acs{rdfs} inference because some nodes in the \Space{} only write \emph{Point} instances (a subclass of \emph{SpatialThing}).
% TODO esto no se entiende muy bien sin explicación extra!
The domain of \tplthree{} and \tplfive{}'s predicates could not be inferred just using \acs{rdfs} inference. % (some properties of OWL are used => inverseof).
Even solving this limitation, we would expect a bad \emph{precision} since both predicates relate very general concepts.
In addition, when a class-based clue has no enough information to predict the nodes, it simply floods the query.
This is why the \emph{recall} of \tplfour{} is high.



\InsertFig{clues_recall}{fig:recall_measures}{\emph{Recall} for each type of clue used}{}{1}{}

\InsertFig{clues_precision}{fig:precision_measures}{\emph{Precision} for each type of clue used}{}{1}{}
% IG: TODO mencionar en el texto de referencia o en la caption algo como: The higher the better.

% Explicar predicate-based
We can see a bad prediction for \tplone{} and \tplfour{} for predicate-based clues.
\tplone{} defines a very common predicate and therefore, it cannot discriminate any node.
\tplfour{} suffers the same problem explained for the class-based clues.
We proposed a possible solution for this problem in Section~\ref{sec:aboxinclues}.

% Explicar prefix-based
Finally, prefix-based clue shows a slightly better \emph{precision} for \tplfour{}, since it can discriminate some nodes not using the \emph{bizkaisense} prefix.
On the other hand, it obtains marginally worse \emph{precision} than predicate-based clues for \tplthree{} and \tplfive{}.
% TODO reexplicar esto, no se entiende!
This worsening could be greater if few nodes using the prefixes \emph{ssn} and \emph{dc} used the predicates defined in both templates.

\medskip

\noindent\textbf{Verbosity.}
% Una frase para retomar lo anterior, y al grano.
The clues verbosity is also a critical aspect for resource constrained devices.
Figure~\ref{fig:clueSize} shows a higher variance for prefix-based clues' length and lower verbosity of class based clues.
This is because the nodes virtually have a different number of sensors.
In addition, the links to concepts of other ontologies vary within the datasets used in the parametrization.
In any case, the diagram shows a similar verbosity for all the clues for the semantic content considered in this evaluation. % TODO analizar si la media varía significativamente
% No es adecuado para cacharros pequeños: poner evaluación de WoT de inferencia


\InsertFig{clues_length}{fig:clueSize}{
  Length of the clues alternatives
}{
}{0.6}{}

% Añadir nuevas barras: Predicate+schema y predicate+schema+MostCommonsIndividuals
% Poner otra barra para saber cuanto contenido semántico guarda un nodo de media en el experimento?
% Comentar que cabe en MTU de ethernet y UDP en caso de querer enviarlo por CoAP


% TODO Añadir nuevo diagrama para el gossiping agregado y ver como crece a más elementos añadidos

\medskip

\noindent\textbf{Summary.}
Class-based clues are useful for templates asking for a specific type of content.
However, they still require inference to obtain a good \emph{precision}.
In \citet{gomez-goiri_restful_2012}, we tested the inference process on the devices and data used in this simulation.
We could not run any reasoner in the ConnectPort X2 Gateway.
Actually, we could only run \acs{rdfs} reasoners in more powerful embedded and mobile devices such as the FoxG20 and the Samsung Galaxy Tab.
In the FoxG20, it took 48.9 seconds the first load of all the ontologies used and 1.4 seconds to reason over each measurement written.
In a Samsung Galaxy Tab, it took 17.3 seconds and 0.2 the following measurement writings.
Considering these results, we can conclude that there is a clear need for efficient embedded reasoners.
Therefore, the class-based approach is promising but it is impossible to adopt in current embedded and mobile devices.

Between the predicate-based and prefix-based clues, we propose to use the predicate-based clues since they subsume much of the information provided by the prefix-based clues.
The rest of the prefixes are referred in the subjects or the objects.
They could be easily added to predicate-based clues on the prefixes field.
In addition to the use of predicate-based clues, we could implement the solution for the specific individual search proposed in Section~\ref{sec:aboxinclues}.

% Posible TODO
% GRAFICO 3: \emph{precision} y \emph{recall} comparando con y sin de cada uno

% GRAFICO 4: tamaño de gossiping individual y agregado comparando con y sin cada uno
%            (barra con barra superpuesta encima con cuanto más añade)
%            Según hay más nodos, cómo crece el gossiping a manejar?


% Discussion: ¿elegir un tipo u otro dependiendo del modo de operación?
% (i.e., si hay que perder precisión a costa de no intercambiar MBs...)
% AG: Interesante, pero yo no complicaría una sección ya de por si bastante liosa.


% TODOs importantes que me gustaria hacer para la siguiente version:
%   - Ver cómo crecen los clues agregados.
%   - Proponer una clue híbrida que mezcle a las anteriores.
%   - Evaluar de alguna forma la mejora propuesta para ABox.



\subsection{Network usage} % IG: network role
\label{sec:NetworkUsage}

% Donde ``explicar'' negative broadcasting y/o centralizado? En seccion 4?
% Explicar porque no se pone centralizado en la comparación
%    No es directamente comparable dado que depende directamente de otro factor distinto: frecuencia de escritura.

We conduct a simulation study to evaluate the benefits of our solution against a flooding-based approach (i.e., \ac{nb}).
In addition, to give a more exhaustive comparison, we implement and test query caching on top of \ac{nb}.
We simulate multiple nodes that join the same \Space{} as \providers{} and periodically write new information to the \Space{}.
During one hour, 1 or 100 \consumers{} perform 1000 queries in total using the templates described in Table~\ref{tab:evaluationTemplates}.

As expected, our solution scales much better than the one with \ac{nb} (Figure~\ref{fig:requestsByStrategies}).
However, adding caching to \ac{nb} works slightly better than our solution with just one node querying the \Space{}.
This is due to the limited amount of different query templates used in the simulation.
When we increase the number of \consumers{} in the \Space{}, the caching strategy behaves closer to the \ac{nb}.
In the same situation, our solution handles better an increase on the number of \consumers{} in the \Space{}.


\InsertFig{requests_by_strategies}{fig:requestsByStrategies}{
  Required requests for \acf{nb}, \ac{nb} with caching with 1 and 100 \consumers{} and our solution with 1 and 100 \consumers{}
}{
}{0.75}{}


In Figure~\ref{fig:requestsByRoles}, we take a closer look to the origin of the traffic of our approach in a \Space{} with 100 \consumers{}.
The communication between the \providers{} and the \ac{wp} is much more infrequent than the other communication types.
The reason is that writing into a node only results in a clue update when the structure of the managed information changes.
The first time the metadata about the node (sensor) is written, the second time the first measure and following writings, just add or replace a measure.
Therefore, the clue does not change after the second step.
This matches with the assumption made to share \emph{TBox} information in our clues.

The communication between \consumers{} and \ac{wp} is in between the other two communication patterns.
It is greater than the one from \providers{} to \ac{wp} because \consumers{} need to maintain an updated view of the \Space{}.
Recall that the update time depends on the query frequency of each \consumer{}.
The maximum and minimum updating frequency were set to 10 and 1 minute(s) respectively.

The communications between \consumers{} and \providers{} assumes most of the total communications.
This shows that the overhead added by the use of \ac{wp} on our solution is not significant and it is justified by the reduction of the total number of communications shown in Figure~\ref{fig:requestsByStrategies}.


\InsertFig{requests_by_roles}{fig:requestsByRoles}{Requests between roles in our solution in a \Space{} with 100 \consumers{}}{}{0.75}{}



\subsection{Energy consumption}
\label{sec:energyConsumption}
% Idea: Lo de arriba está muy bien, pero específicamente, cómo afecta a los cacharros?
Our solution tries to save energy by making \providers{} handle fewer requests from \consumers{}.
These savings contrast to the overhead added by the communication with the \ac{wp}.
However, our results demonstrate that this overhead is small in comparison to the total number of communications.

The energy consumption in mobile and embedded devices increases each time a device needs to process something or communicate with another node (see Figure~\ref{fig:energy_consumption}).
To analyse how communications impact their energy autonomy, we have to consider not only the number of communications but also their time length (see Table~\ref{tab:measuresEmbedded}).
For example, a mobile phone will consume less energy asking clues to a server than asking them to an embedded device as it has to wait less for the response.


\InsertFig{energy_consumption}{fig:energy_consumption}{Average power consumption for FoxG20 during different activity periods}{}{0.6}{}


The experiment consists of 300 nodes joined to a \Space{} running on 1 server, 30 galaxy tabs, 75 FoxG20 and 194 XBees.
We increase the number of devices as their price and capacity decrease.
Using this approach, we mimic a typical \Space{} where cheap devices are more common.

As shown in Figure~\ref{fig:activity_measures}, our solution reduces the activity of each device by more than 5 times compared to \acl{nb}.
The diagram on the right details the average activity for each type of device.

In our solution, we can check how the load moves from the embedded devices (XBee and FoxG20) to the server (which is indeed chosen as a \ac{wp}).
The exceptional activity registered by the Galaxy Tabs is caused by their extremely high response time.
% This implies that handling a request in the Galaxy Tab takes much longer than the average.
However, we plan to reduce this response time changing the \acs{http} library used in our Android implementation.


\InsertFig{activity_measures}{fig:activity_measures}{Activity time for each strategy}{
  The first part shows the average active time a node spends on each strategy.
  The second one shows the active time classified by the type of device each node has run on.
}{1}{}
% TODO añadir a la derecha los valores en NB?

%%%%%%%%%%%%%%%%%%%%%%%%%%%%%%%%%%% QUIZA para 2da VERSION %%%%%%%%%%%%%%%%%%%%%%%%%%%%%%%%%%%%%%%%%
%  + Vamos a medir y comparar distintas situaciones entre sí:
%   Nota: cuando hablo de servidor, movil o dispositivo embebido, ejemplifico capacidades.
%         dispositivo embebido puede almacenar menos cosas y hace consultas más concretas.
%
%      - Situación 1: Negative broadcasting (o todo dispositivos embebidos)
%      - Situación 2: 1 servidor (WP), 20 moviles, 120 dispositivos embebidos (proporción 1:20:120)
%      - Situación 3: 20 moviles, 120 dispositivos embebidos (proporción 1:6, entre móviles 5 con el cargador enchufado)
%      - Situación 4: todo dispositivos embebidos
%  + Eje Y: Tiempo medio en ejecución
%  + Eje X: A parte de las situaciones (ver esquema de \emph{recall}) algún otro aspecto que también afecte al consumo de energía:
%      - número de consultas?
%      - frecuencia de las mismas?
%      - nodos consultores?



\subsection{Performance in dynamic environments}
\label{sec:dynamic}
We evaluate the network usage of our solution in ordinary situations in sections~\ref{sec:NetworkUsage} and \ref{sec:energyConsumption}.
Nevertheless, we do not evaluate scenarios where the nodes frequently join and leave the \Space{}.
In such situation, the communication needed to manage the clues might be a burden.


\InsertFig{dynamism}{fig:dynamic_situations}{Effects of dynamic scenarios in our solution}{
  Note that the last interval in the x-axis represents a simulation with no drops.
}{1}{}


To assess the effect of dynamic networks on the performance of our solution, we used the scenario presented in Section~\ref{sec:energyConsumption},
Then, we simulate nodes joining and leaving the \Space{} at different intervals: 30 seconds, 1 minute, 5 minutes, 10 minutes, 20 minutes, 30 minutes, 45 minutes.
Particularly, for our solution, we tested the most harmful situation: the node leaving the \Space{} abruptly is always the \ac{wp}.
We also added an scenario with no drops as a baseline.
Note that we represent this scenario by configuring the drop-interval with a greater value than the simulation time.

In Figure~\ref{fig:dynamic_situations}, we see the results of these simulations.
These results show that even in such dynamic situations, our solution requires fewer communications than \acl{nb}.
In our solution, most of the communications are between \consumers{} and \providers{}.
To evaluate the overhead added by our solution, the graphic on the right hand side shows the communications involving \acp{wp}.

We can appreciate that the updates on \consumers{} are independent of the number of times the \ac{wp} changes.
We can also see a minimal change between the scenario where the \ac{wp} is always available and the one with 5 minutes drop-interval.
In this case, when the \ac{wp} drops, many \consumers{} have the latest version of the aggregated clue.
% Dado que se parte del que tenia el WP o del que le ha dado el consumer que le ha elegido. Ambos tienen más opciones de tener una versión actualizada.
This situation increases the chances of getting an updated version for the initialization of the new \ac{wp} (see Section~\ref{sec:selection}).
Thus, it reduces the number of messages from \providers{} to the new \ac{wp}.


\subsection{Effects on discovery mechanisms}
\label{sec:mdns}

We want to prove the feasibility of our solution using a common discovery mechanism.
To that end, we simulate the behaviour of the \ac{mdns} and \ac{dns-sd} \citeweb{dnssd2013} protocols.
Both protocols are based on the well known and widely accepted \ac{dns}.
\ac{dns} associates different pieces of information (i.e., records) with domain names in a distributed manner.

On the one hand, \ac{dns-sd} proposes a new use for \ac{dns}'s TXT records.
The TXT record was originally intended to associate an arbitrary human-readable text with a domain name.
\ac{dns-sd} proposes to use this type of record, to share key-value pairs in its data field.
We use these key-value pairs to share the information needed by the selection algorithm among the nodes.

On the other hand, \ac{mdns} defines how this and other records are shared through UDP multicast (or unicast in certain situations).
We ignore the cost of browsing the nodes to discover new nodes because it is the same for both strategies.
However, note that as explained above, our strategy does differ from \acl{nb} in the use of TXT records.

The nodes announce records during the start up or whenever they have a resource record with new data.
Therefore, each time a record is updated, we send a multicast message that increases the network traffic.
In our solution the TXT record may change 
\begin{enumerate*}[label=\itshape(\arabic*\upshape)]
  \item when a new \ac{wp} is selected or
  \item when we update the time elapsed since it joined the \Space{} and its battery charge level.
\end{enumerate*}
The last two parameters need to be updated to select an appropriate \ac{wp} but they do not need to change too frequently.

% Regarding the first case,  % no pillo la introduccion del first case
In the most static scenario the TXT record is written only once.
The more dynamic scenario from the previous section, on the contrary, updates that record 126 times after writing it for the first time.
This demonstrates that the overhead generated on the discovery system by our solution is minimal even in the worst-case scenario.
\section{Conclusion}
\label{conclusion}

In this chapter, we have presented a dynamic architecture to enhance the search of semantic contents in the \acl{wot}.
%  by any node part of the Web of Things.
In particular, this architecture chooses an intermediary according to its capabilities to support resource constrained devices.
Intermediaries can use different types of clues to summarize the information in the semantic \Space{}.
% managed by the nodes belonging to a space.
Thanks to this support, the devices can directly interrogate others to obtain fresh information while reducing their semantic overhead.

% Esto lo he cambiado porque Ipiña decía que habría que volver a aclarar porqué lo he comparado con flooding en la conclusión (por si alguien no se ha leído el paper ;-))
The main characteristic of our solution is the ability of the devices to share semantic data directly, no matter how simple they are. % o algo como guides our design
Under this assumption, a flooding-based strategy would obtain a high recall. % or a high fraction of relevant answers (por si no se entiende aún que es recall)
However, our evaluation shows that our solution requires less messages between devices than a flooding-based strategy (i.e. is far more precise).
% he puesto more para que no tenga la acepción peyorativa de "many" (no se requieren muchísimos más para que sea mejor)
Even if we use caching strategies to alleviate these effects, our solution performs better for scenarios with more data consumers and query types.
In addition, our approach reduces the workload of mobile and embedded devices which indirectly results into energy savings.

% Para trabajo futuro, se podría considerar: 
% query optimization y citar a alguna solución de distributed triple stores
% TODO poner algo más acorde con el related work actual!
For our future work, we will consider using query optimization techniques in the \providers{}.
Using these techniques, we could a) transform the queries before sending them to obtain more results
and b) refine the results.


% ----------------------------------------------------------------------


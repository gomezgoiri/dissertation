\section{Research methodology}

To perforn this thesis, I have followed the following phases:
\begin{enumerate*}[label=\itshape\alph*\upshape)]
  \item Awareness of the problem.
  \item Solution suggestion.
  \item Development and evaluation.
  \item Conclusion.
\end{enumerate*}


The awareness of the problem phase requires an intensive review of the state of the art in the areas of
space-based computing, \acl{ubicomp} and the \acl{sw}.
As a result, we need to clearly identify the limitations of the existing works,
and the areas where it was possible to make a scientific contribution.


In the second phase, we suggest a solution and identify its main parts.
For each of these parts, we compare the improvements it introduces with already existing solutions.
To do that, we studied the relevant approaches of similar application domains.
At the same time, the solution can be refined and adjusted based on the feedback given by experts on the field.
These experts are the supervisors, colleagues, partners in the research projects where I have been involved and the researchers from the external groups where I stayed (i.e. Lancaster University and the ETH).
When we iteratively reach to this phase after the third phase, feedback is also obtained from experts in workshops, conferences, journals.
Particularly, a lesson learn in this thesis is that sometimes a good review for a rejected work can be as valuable as face to face feedback obtained at the venue. % conference 


The third phase comprises planning, developing and evaluating the different parts which compose our solution.
The planning requires to clearly define how to evaluate that these contributions are achieved, identify the subphases of the development and evaluation and carefully schedule them.
Developing comprises the design and develop of the different modules.
To evaluation requires to prepare the experimental environment, develop the experiments and carry out them to precisely measure the indicators previously identified. % input
To conclude the third phase, we contrast the contributions with experts on the field through workshops, conferences, journals and stays.
As a result of the feedback obtained, the second phase can be iteratively resumed to refine the solution.


In the last phase, we analyze the results of our work to obtain conclusions and possible future research lines.

\section{Objectives}

To validate the hypothesis, the main objective of this dissertation is to
\emph{design a middleware which follows the \ac{tsc} paradigm according to the web principles and considering the energy and computation aspects}.

This objective has been crystallized in the following sub-objectives:

\begin{itemize} % por orden de importancia o por hilo de redacción? (ahora mismo por lo segundo)
  \item Design of a dual space model where participants can coordinate through the space in an uncoupling manner and enrich it with their own managed knowledge.
  \item Design of a \ac{tsc} interface for \ac{http} which is compatible with most of the \ac{rest} principles.
	This resource-oriented interface defines a minimal contract to access to the the semantic knowledge provided by the space. % no ``by each device'', eso será como una optimización para hacer menos consultas.
  \item Creation of a search-aware architecture which promotes the end-to-end communication between devices. % and balances their load by administrative tasks according to their capacities.
	This architecture tries to minimize the requests devices have to handle by enhancing their search mechanism.
	The extra tasks introduced by this enhancement are performed by nodes chosen according to their capacities.
  \item Comparison of different mechanisms to act on a physical environment: the space-based actuation patterns and the direct \ac{rest} service consumption.
  \item Alignment between space-based computing and direct \ac{rest} service consumption to seamlessly reuse third applications capabilities.
\end{itemize}
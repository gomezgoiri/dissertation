\section{Thesis outline}
\label{sec:Outline}

% TODO hacer un esquemita chulo con las distintas partes en colores e irlo poniendolo parcialmente en cada sección

The remainder of this dissertation is structured as follows:
Chapter~\ref{cha:background} contextualizes the thesis.
Chapter~\ref{cha:stateoftheart} first presents and compares some relevant works from the field of \acl{ubicomp}. % se entiende que en ambos casos es sobre ubicomp?
Then, it analyses the state of the art of existing semantic space-based solutions and scrutinizes their advantages and limitations.

Chapter~\ref{cha:tsc} proposes and analyses a model to adapt \ac{tsc} to the \ac{ubicomp}.
This model involves limited devices by enabling them to autonomously manage their own knowledge and later enrich the space with it.
Simultaneously, it conserves most of the space-based computing benefits through a classical space model.

Chapter~\ref{cha:searching} describes the core of this thesis.
It presents an energy-aware search architecture.
This architecture dynamically adapts to the needs of resource constrained devices.
The goal of this architecture is two-fold:
\begin{enumerate*}[label=\itshape\alph*\upshape)]
\item to promote the direct and fully distributed communication between devices;
and, at the same time,
\item to reduce network communications between them, particularly the ones directed to the less powerful devices.
\end{enumerate*}

Chapter~\ref{cha:actuate} explores two different mechanisms to change the physical environment using \ac{tsc}.
Ideally, these changes should be produced by writing into the space new semantic information. % a tasks-based approach
However, there are a range of actuators which already use \ac{rest} services whose actuation capacities could be reused.
To do that, we propose an alignment between both approaches. % a seamless alignment?


Each chapter evaluates our proposals, discusses the related work, and states our intermediate conclusions. % TODO comprobar que eso sea así al final

Finally, Chapter~\ref{cha:conclusions} concludes this thesis.
It remarks its advantages and limitations and discusses the achieved goals.
Besides, we explain the lessons learn and the future lines of research on this topic. % TODO comprobar que hay algo que se pueda interpretar como "lessons"
\section{Hypothesis}
\label{sec:Hypothesis}

% Checklist taken from http://www.uninova.pt/cam/teaching/SRMT/SRMTunit2.pdf

% FORMULATE HYPOTHESIS
% - A scientific hypothesis states the ‘predicted’ (educated guess) relationship amongst variables.
% - Serve to bring clarity, specificity and focus to a research problem
%   + But are not essential
%   + You can conduct valid research without constructing a hypothesis
%   + On the other hand you can construct as many hypothesis as appropriate
% - Stated in declarative form. Brief and up to the point. A possible format (formalized):
%   + “If ...... then .... (because ....) “
% - In the case of a PhD dissertation, one hypothesis after tested becomes a thesis being defended.
%   + One dissertation may include more than one thesis.
%   + Sometimes people refer to the dissertation as the “thesis”.

% CHARACTERISTICS OF AN HYPOTHESIS
% - Should be simple, specific and conceptually clear.
%   + Ambiguity would make verification almost impossible.
% - Should be capable of verification.
% - Should be related to the existing body of knowledge.
% - Should be operationalisable
%   + Expressed in terms that can be measured.
% - Often PhD dissertations fail to make explicit their hypothesis / thesis.

% Mis otros intentos: https://dev.morelab.deusto.es/trac/aigomez/wiki/sugerencias
This dissertation is guided by the following hypothesis:
% aimed: porque lo he conseguido se supone
% proof: porque se ha probado ya
% guided: porque en este punto, es como que avanzo que esto será así

\fbox {
  \parbox{\linewidth}{
The alignment of the \ac{tsc} paradigm with the web's principles and the consideration of its energy and computational impact,
leads to an independent, decoupled and interoperable communication between any type of device in \ac{ubicomp}.
  }
}
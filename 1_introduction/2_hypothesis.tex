\section{Hypothesis}
\label{sec:Hypothesis}

% Checklist taken from http://www.uninova.pt/cam/teaching/SRMT/SRMTunit2.pdf

% FORMULATE HYPOTHESIS
% - A scientific hypothesis states the ‘predicted’ (educated guess) relationship amongst variables.
% - Serve to bring clarity, specificity and focus to a research problem
%   + But are not essential
%   + You can conduct valid research without constructing a hypothesis
%   + On the other hand you can construct as many hypothesis as appropriate
% - Stated in declarative form. Brief and up to the point. A possible format (formalized):
%   + “If ...... then .... (because ....) “
% - In the case of a PhD dissertation, one hypothesis after tested becomes a thesis being defended.
%   + One dissertation may include more than one thesis.
%   + Sometimes people refer to the dissertation as the “thesis”.

% CHARACTERISTICS OF AN HYPOTHESIS
% - Should be simple, specific and conceptually clear.
%   + Ambiguity would make verification almost impossible.
% - Should be capable of verification.
% - Should be related to the existing body of knowledge.
% - Should be operationalisable
%   + Expressed in terms that can be measured.
% - Often PhD dissertations fail to make explicit their hypothesis / thesis.

This dissertation has the following hypothesis:
% aimed: porque lo he conseguido se supone
% proof: porque se ha probado ya
% guided: porque en este punto, es como que avanzo que esto será así

% Mis otros intentos: https://dev.morelab.deusto.es/trac/aigomez/wiki/sugerencias#Hipótesis
\noindent
\fbox {
  \parbox{\linewidth}{
The alignment of the \ac{tsc} paradigm with the web's principles together with the consideration of its energy and computational impact,
leads to \ac{ubicomp} environments where heterogeneous devices communicate autonomously in a decoupled and interoperable fashion.
  }
}

\bigskip

Note that the consequence of the hypothesis involves the following aspects:
\begin{description}
  \item[Heterogeneity]: Fully-fledged computers and resource constrained devices (e.g. mobile and embedded devices) must coexist in these environments.
  \item[Autonomy]: The devices must not depend on third nodes to consume or provide data.
                   However, they might be aided by other devices to complete some related tasks (e.g. search the appropriate nodes to request).
                   %In these cases, the self-organization of the nodes involved in such task must be promoted to avoid human-intervention.
  \item[Decoupling]: The communication must be driven by the data, from the user perspective devices do no directly refer to each other.
		    Additionally, the provider and the consumer should not coexist in time.
		    However, note that since this subaspect contradicts the autonomy principle, their selection might be left to the user choice.
  \item[Interoperability]: Devices must be able to exchange information with other systems and to use that information. % exchange: HTTP APIs & REST, use: semantics
\end{description}
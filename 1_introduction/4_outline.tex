\section{Thesis outline}
\label{sec:Outline}

% TODO hacer un esquemita chulo con las distintas partes en colores e irlo poniendolo parcialmente en cada sección

The remainder of this dissertation is structured as follows:
Chapter~\ref{cha:stateoftheart} first presents some relevant works in the field of \acl{ubicomp} regarding application integration and the \ac{sw}. % se entiende que en ambos casos es sobre ubicomp?
Then, it analyses the state of the art of the \acl{tsc} comparing the existing solutions and studying their advantages and limitations.

Chapter~\ref{cha:tsc} discusses the best strategies to adapt \ac{tsc} to the \ac{ubicomp} in a fully distributed way.
Besides, we explore the compatibility of \ac{tsc} with the \ac{wot} vision.

Chapter~\ref{cha:searching} describes the core of this thesis.
There we present an energy-aware search architecture.
This architecture dynamically adapts to the needs of resource constrained devices.
The goal of this architecture is two-fold:
1) to promote the direct and fully distributed communication between devices,
and at the same time,
2) to reduce the network communications between them, particularly the ones directed to the less powerful ones.

Chapter~\ref{cha:actuate} explores the different manners in which one can use \ac{tsc} to change the environment.
Ideally, this changes should be done by writing into the space new semantic information.
But due to the information dissemination strategy proposed in Chapter~\ref{cha:searching},
some new communication primitives are needed to inform the appropriate devices.
Particularly, we study two different ways of making these changes:
a tasks-based approach and a straightforward REST services consumption approach.

% Me sigue dando cosica no comentar nada de Otsopack.
% Chapter~\ref{cha:implementation} demonstrates the feasibility of some of the ideas expressed in the previous chapters.
% To that end, we have developed a middleware called \emph{Otsopack} which implements the \ac{tsc} for resource constrained devices.
% This middleware is assessed both in a quantitative and in a qualitative way. % measurements and scenarios

Each chapter evaluates our proposals, discusses the related work, and states our intermediate conclusions.

Finally, Chapter~\ref{cha:conclusions} concludes this thesis.
We discuss the achieved goals and remark the advantages and limitations of solution.
Besides, we explain the lessons learn and the future lines of research on this topic.
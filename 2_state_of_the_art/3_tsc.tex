\section{Triple Space Computing}
\label{sec:tsc_soa}

% aclarar que también es space-based computing

% de estado del arte de cursos
% de DAIS
% de IJWGS

% Tanasescu
% swarm
% decir que tiene mucho que ver con distributed triple stores
% analizarlos también??? vaya pereza! -> igual comentar diferencia y dejar el análisis para la sección tosca de IGWS

The most straightforward approach which mixes up Semantic Web with tuplespaces is Triple Space Computing (TSC) \cite{fensel_triple-space_2004}.
TSC uses similar primitives with RDF triples as exchanged units.


\InsertFig{venn-sec3}{fig:venn_tuplespaces_semantics}{
  Semantic Tuple Spaces.
}{
  Scope of this section.
}{0.6}{}


In recent years, both semantic tuplespaces or more generic tuplespaces works have been presented as a solution for the
problems brought forward by a wide range of application fields. Each solution presents a different strategy to
distribute the shared tuples over the space in the most convenient way.

In the centralized tuplespaces we can find solutions which mainly focus on offering access to a space stored in a unique
machine. This access is carried out by the clients using different methods which encapsulate the coordination primitives
as in Semantic Web Spaces \cite{nixon_towards_2007}. Remarkably, recently the Smart M3 middleware proposed an
architecture where so called Semantic Information Brokers (SIB) store the information and perform reasoning
process on behalf of the clients or Knowledge Processors (KP). Unfortunately, no scenario with more than a SIB has been
presented yet, making it de facto centralized.

TripCom\footnote{TripCom (IST-4-027324-STP, www.tripcom.org)} uses a hybrid solution where information is distributed on
an overlay network made up of different kernels using a distributed hash table. Each kernel stores some triples and
knows where the rest of the triples are located by using a hash function. The clients have to know one of these
kernels to address their queries through them. TripCom was conceived to store a huge amount of RDF triples taking
special care of the scalability issues.

Other tuplespaces solutions have used a negative broadcasting strategy
\cite{krummenacher_open_2009,murphy_transiently_2006,gomez-goiri_semantic_2011}, where the writings are
performed locally and the queries are broadcasted to all the reachable nodes which belong to a space. As can be seen,
this strategy ensures that all the possible answers will be received, but at a high resource consuming cost.



% TODO CONCLUSION
Although both semantic space-based and Triple Spaces implementations exist, none of them has been specifically designed to be run in devices with constrained capabilities apart from our solution.
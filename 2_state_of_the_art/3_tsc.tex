\section{Semantic \aclp{ts}}
\label{sec:soa_tsc}

% TODO
% Añadir  Citron (2005)? (eoa de Almeida)
% Tanasescu
% swarm
% decir que tiene mucho que ver con distributed triple stores
% analizarlos también??? vaya pereza! -> igual comentar diferencia y dejar el análisis para la sección tosca de IGWS

Semantic \aclp{ts} aim to join \aclp{ts} with the \acl{sw} to propose a more uncoupled solution.
Particularly, it benefits from the autonomies introduced by both \ac{ts} and the \ac{sw}:

\begin{itemize}
  \item Space uncoupling.
  \item Time uncoupling.
  \item Data-schema uncoupling. % explicar?
% 	enhances the interoperability of the
% 	applications built on top of TSC. This way, two applications using standard ontologies can interact among them
% 	automatically enriching one each other, as long as they use the same space and standard and linked ontologies.
\end{itemize}


In this section we provide a review of the existing Semantic \aclp{ts} solutions. % TODO citar nixon_tuplespace-based_2008 ??? % mencionar que la nuestra es más completa/nueva
% así no nos repetimos como el ajo
Instead of describing each of this solutions individually and then include a comparison,
the review is divided in sections which describe the main components of a Semantic \ac{ts}.


\InsertFig{venn-sec3}{fig:venn_tuplespaces_semantics}{
  Semantic \aclp{ts}.
}{
  Scope of this section.
}{0.6}{}


% Definir constantes para referirnos a los distintos middlewares analizados
\newcommand{\midtsc}{\acs{tscm} \citep{fensel_tsc_2007}}
\newcommand{\midsws}{\acs{sws} \citep{tolksdorf_coordination_2006}}
\newcommand{\midstuples}{\acs{stuples} \citep{khushraj_stuples:_2004}}
\newcommand{\midcspaces}{\acs{cspaces} \citep{martinrecuerda_towards_2005}}
\newcommand{\midtscpp}{\acs{tscpp} \citep{krummenacher_open_2009,blunder_distributed_2009}}
\newcommand{\midtripcom}{\acs{tripcom} \citep{XXX}}
\newcommand{\midsmartmt}{\acs{smartm3} \citep{honkola_smart-m3_2010}}
\newcommand{\midnardini}{\citet{nardini_semantic_2013}}


\subsection{Use of semantics}

The works analyzed present two strategies to semantically annotation of data: the use of \ac{sw} standards or the definition of its own language.
The \ac{sw} rely on the \ac{rdf} metadata data model. % también lo llaman language
As the majority of the semantic \ac{ts}s, we opt for using the widely accepted \ac{rdf} to ensure interoperability \emph{ab-initio}.
On the contrary, \ac{cspaces} \citep{martinrecuerda_towards_2005} and \citet{nardini_semantic_2013} define their own language-independent up to \ac{fol} notation.
However, these works have to face an inconvenience: most of the libraries and tools available are based on the \ac{sw} standards. % concretamente cuando quieren usar reasoners!
Therefore, they have to internally translate between both worlds in a not-straightforward way. % son cosas distintas!
% theoretical counterpart of OWL DL [7]. Since OWL is the W3C standard ontology description language for the Semantic Web,
% and the standard de-facto for semantic applications in general, we adopt OWL as the ontology language for the domain
% ontologies associated to ReSpecT semantic tuple centres in TuCSoN. While for the details of OWL we forward interested

% TODO explicar diferencias, citar o evitar verengenales que no comprendo y que sólo me complican la vida?
% http://www.cs.ox.ac.uk/ian.horrocks/Seminars/download/Horrocks_Ian_pt1.pdf
%     DL == Decidable fragments of First Order Logic
% http://en.wikipedia.org/wiki/Description_logic#First_order_logic
%     Many Description Logic models (DLs) are decidable fragments of first order logic (FOL).
%     Some DLs now include operations (for example, transitive closure of roles) that allow efficient inference but cannot be expressed in FOL.


On top of \ac{rdf}, the \ac{sw} defines new layers to increase the data expressivity.
First, \ac{rdfs}\footnote{\url{http://www.w3.org/TR/rdf-schema/}} describes a series of classes and properties to define vocabularies in \ac{rdf}.
On top of \ac{rdfs}, \ac{owl}\footnote{\url{http://www.w3.org/TR/owl-features/}} offers additional vocabulary for describing properties and classes.
% 3 niveles: Lite, DL (que se corresponde con DL) y Full
These properties and classes allow to reason over the knowledge both to validate the data inserted or to infer implicit (unstated) knowledge. % implicit y unstated son redundantes, pero para que se entienda mejor


Although reasoning may be desirable, its use is tangential to the model proposed in this thesis.
In any case, it is by not means not mandatory in our solution because:
\begin{itemize}
  \item The devices which may provide information to our \emph{federated space} are autonomous.
	They share a minimal contract with our middleware where we cannot define how they manage/provide their information.
  \item It is limited by the availability of reasoners, and nowadays there is no reasoner light enough to run in a reasonable time span in small devices.
        % TODO citar algo? quizá nuestro propio trabajo?
\end{itemize}
%For our solution, describing the data semantically is a first step towards semantically interpreting it.
%The Table~\ref{tab:comparisonSemantics} shows different features regarding the semantic model proposed by each solution.



\begin{table}[htbp]
\caption{Use of semantics in the analyzed solutions.}

% TODO En Languages decir si se basa en RDF o FOL, el hecho de que luego se use OWL para definir conocimiento, se sobreentiende
% TODO añadir reglas?
% TODO añadir Validity and consistency?
% TODO añadir semantic clustering

\begin{tabular}{ l p{3cm} p{3cm} p{3cm} }
\hline 
  & Languages  & Reasoning  & Semantic matching \tabularnewline
\hline 
 TSC & RDF  & -  & Graph pattern templates and N3QL resolution \tabularnewline
 SWS & RDF(s) extendable to OWL, SWRL  & Yes (to match and validate)  & Subsumption-based \tabularnewline
 sTuples & DAML+OIL  & Tuples with DAML+OIL field  & Subsumption-based \tabularnewline
 CSpaces & Language-independent, up to FOL  & Yes (used for query answering, rewriting and consistency checking)  & Based on query engines \tabularnewline
 tsc++ & RDF & No  & No \tabularnewline % comprobar %  lo de para query answering lo hice yo en su día
 TripCom & RDF  & Yes (query rewriting) & Yes \tabularnewline % TODO rehacer esta columna comprobandolo
 Smart-M3 & RDF  & ?  & G. patterns and complex \tabularnewline
 Nardini et al. & Language-independent, up to FOL & ?  & ? \tabularnewline
\hline 
\end{tabular}
\label{tab:comparisonSemantics}
\end{table}

% TODO añadir tabla con análisis de las autonomías?


\subsection{Tuple model}

% presentamos distintos modelos
On the one hand, several works embed semantic content in one of the fields of the tuple.
\ac{cspaces} \citep{martinrecuerda_towards_2005} defines seven-field tuple.
\ac{stuples}\citep{khushraj_stuples:_2004} extends the \emph{JavaSpace} \citep{freeman_javaspaces_1999} middleware adding a field with semantic content to the tuple object.
This content follows the DAML+OIL language (the \ac{owl} precursor).
\citet{nardini_semantic_2013} use the concept of \emph{semantic tuples} which are expressed in logic terms.


On the other hand, the most common tuple model is the three-field tuple which corresponds with a \ac{rdf} triple.
That is, a tuple with a field for the \emph{subject}, another for the \emph{predicate} and the third one for the \emph{object}.
% Sacado de la descripción de TSC que tenía antes:
% In a later work, \citet{krummenacher2006specification} proposed to use identifiers just for a set of triples (i.e. \ac{rdf} graphs).
% bundler cita el de "WWW: or what's wrong with the web" para explicar eso último
However, a triple by itself cannot express much information \citep{krummenacher2006specification}.
% también para reducir la complejidad del storage (TSC dixit)
% decir que al primero que se le ocurrió fue a TSC?
To solve this limitation, semantic \ac{ts} have usually adopted the concept of \ac{rdf} graphs too.
% En SWS usan algo muy parecido pero a lo que llaman subspace.
A \ac{rdf} graph is a set of \ac{rdf} triples identified by an \ac{uri}. % TSC distinguen entre NamedGraph (grafo con URI) y Graph (conjunto de grafos)
Although they can be accessed by their \ac{uri}, more interestingly, these middlewares also guarantee the associative access. % see the next section


\begin{savenotes}
  \begin{table}[htbp]
    \caption{Information units used by the different semantic \ac{ts} middlewares.}
    \centering
    \begin{tabular}{ l c c c }
      \hline 
	& Tuples with & \ac{rdf} triple- & \multirow{2}{*}{\ac{rdf} Graphs} \\
	& semantic field & like tuples & \\
      \hline 
      \midtsc{} & & $\checkmark$ & $\checkmark$ \\ % en una adición posterior!
      \midsws{} & & $\checkmark$ & $\checkmark$\footnote{\ac{sws}'s subspaces are conceptually equivalent to \ac{rdf} Graphs: an abstraction to work with a set of \ac{rdf} triples.} \\
      \midstuples{} & $\checkmark$ & & \\
      \midcspaces{} & $\checkmark$ & & \\ %7-field tuple
      \midtscpp{} & & $\checkmark$ & $\checkmark$ \\
      \midtripcom{} & & $\checkmark$ & $\checkmark$ \\
      \midsmartmt{} & & $\checkmark$ & $\checkmark$ \\
      \midnardini{} & $\checkmark$ & & \\
      \hline 
    \end{tabular}
    \label{tab:tuple_comparison}
  \end{table}
\end{savenotes}


\subsection{Query model}

Originally, in common \ac{ts}s tuples were selected using special tuples where wildcard values were allowed in the fields.
All of the \ac{ts} middleware which use \ac{rdf} triple as a tuple use this approach.
Beside, they provide access to the \ac{rdf} graphs by their \ac{uri}'s.

Most of these works also offer advanced query languages (e.g. SPARQL\footnote{\url{http://www.w3.org/TR/rdf-sparql-query/}}) as a more expressive way to match graphs.
This languages can be decomposed by simple triple patterns.
However, this requires a parser which may not be available for resource constrained platforms.
In this thesis we focus on queries based on graph patterns and left the adoption of more complex querying languages as a future work.

\ac{cspaces}, \ac{stuples} and \citeauthor{nardini_semantic_2013} use less standard querying approaches.
\ac{cspaces} \citep{martinrecuerda_towards_2005} and \citet{nardini_semantic_2013} offer a formal language to select appropriate tuples.
\ac{stuples}\citep{khushraj_stuples:_2004} extends JavaSpace's template by adding assertional axioms that can be used to match semantic tuples.

\begin{savenotes}
  \begin{table}[htbp]
    \caption{Querying units for semantic \ac{ts}s.}
    \centering
    \begin{tabular}{l c c c}
      \hline 
	& Graph  & Advanced query  & \multirow{2}{*}{Other} \\
	& patterns  & languages  & ~ \\
      \hline
      \midtsc{} & $\checkmark$ & & \\ % en una adición posterior!
      \midsws{} & $\checkmark$ & & \\
      \midstuples{} & & & $\checkmark$ \\
      \midcspaces{} & & & $\checkmark$ \\ %7-field tuple
      \midtscpp{} & $\checkmark$ & $\checkmark$\footnote{It completely depends on the underlying data store selected.} & \\ % but it can be SeRQL, SPARQL and N3QL
      \midtripcom{} & $\checkmark$ & $\checkmark$ & \\
      \midsmartmt{} & $\checkmark$ & $\checkmark$ & \\
      \midnardini{} & & & $\checkmark$ \\
      \hline
    \end{tabular}
    \label{tab:query_comparison}
  \end{table}
\end{savenotes}

% TODO añadir los lenguajes específicos en queries


\subsection{Space model}

The \emph{flat} model offers independent disjoint spaces.
\ac{stuples}, \citet{nardini_semantic_2013}, TSC, tsc++ and Smart-M3 use this model.
% sTuples:
%    Bundler dice que es "Centralized nested", pero no sé de dónde lo saca
%    Está basado en Javaspaces, que por lo que puedo ver es flat
%    En nixon2008 sólo dicen que es centralized
% decir que quiere decir: Tuple Centres Spread over the Network ?
Within them, \citet{nardini_semantic_2013} presents the most singular model.
It extends TuCSoN \cite{omicini_tucson:_1998}, which presents an evolution of the \ac{ts} called \emph{tuple centre}.
A \emph{tuple centre} can be adapted to the application needs through reactions to communication operations.
These reactions allow to trigger behaviors in response to any primitives or to define new ones.
% TuCSoN follows a flat model where different independent tuple centres can coexist at the same time. % o created?
% The agents can access to remote \emph{tuple centres} and migrate to them, but these interactions need to be programmed.
% Besides, the latter three works identify each space with an \ac{uri}.


% TODO Nosotros proponemos flat spaces enriquecidos con federated spaces
% Nested
More sophisticated models allow to create hierarchies of spaces.
Three examples are \ac{sws}, \ac{tripcom} and \ac{cspaces}.
% SWS -> Disjoinnt nested
% cosas relativas al despliegue
\ac{sws} \citep{tolksdorf_coordination_2006} propose two ways to partition the spaces: sub-spaces and contexts.
Sub-spaces are disjoint partitions of the main space.
Contexts enable to virtually divide the space into overlapping partitions.
These partitions are used to enable particular clients' views of the space.

% TripCom
\ac{tripcom} shares some similarities with \ac{sws}'s model.
It uses subspaces to form nested multiple spaces.
Doing so it restricts the communication to a part of the whole space leading to scalability and completeness. % según ellos
Besides, it offers a mechanism to overlap spaces called \emph{scopes}.
Using \emph{scopes} a client can create a temporary copy of some tuples.
However, any insertion and deletion would not apply to the whole Triple Space.

% CSpaces
% On the contrary of C04 where modifications need to be approved by all
% subscribers, the updates proposed by the members of a Shared CSpace are
% automatically included, and versioning mechanisms are in charge to track
% changes and provide rollback features if one of the members disagrees with
% the included updates.
% To join a Shared CSpaces and publish and retrieve data on it the new
% members should first complete a registration procedure in which one of the
% main tasks is to provide a semantic and alignment specification between the
% data that each new candidate want to share and the data that previous
% members have published beforehand.
\ac{cspaces} \citep{martinrecuerda_towards_2005} proposes two types of spaces: individual or shared.
An individual space belongs to a single process.
Two participants can agree on how to represent the knowledge to share their individual spaces forming a shared space.
Shared spaces can join to others forming a tree structure.
In a shared space the updates are versioned and can be revoked by any member.
However, neither the registration process needed for the agreement or the revocation process are detailed.
Furthermore, to the best of our knowledge, this conceptual exercise never went beyond a rather limited prototype.


\begin{table}[htbp]
  \caption{Space model used by the different works.}
  \centering
  \begin{tabular}{ l c c c }
    \hline 
    ~ & \multirow{2}{*}{Flat}  & Nested & Overlapping \\
    ~ & ~  & Disjoint  & views \\
    \hline 
    \midtsc{} & $\checkmark$ & & \\ % en una adición posterior!
    \midsws{} & & $\checkmark$ & \\
    \midstuples{} & $\checkmark$ & & \\
    \midcspaces{} & & $\checkmark$ & $\checkmark$ \\ %7-field tuple
    \midtscpp{} & $\checkmark$ & & \\
    \midtripcom{} & & $\checkmark$ & $\checkmark$ \\
    \midsmartmt{} & $\checkmark$ & & \\
    \midnardini{} & $\checkmark$ & & \\
    \hline 
  \end{tabular}
  \label{tab:space_comparison}
\end{table}



\subsection{Distribution} % o después de space model o integrado!
%	2. centralizado / distribuído
%	4. basado en clientes tontos

Participant nodes usually access to semantic spaces through client/server basis.
Tsc++ proposes an exception to the client/server access to the space.
It relies on Jxta P2P framework to propagates the queries using different strategies.
In tsc++ the spaces correspond to groups of nodes which locally manage their data.


In client/server spaces, the back-end of the server can be centralized in a single machine or distributed.
Centralized \aclp{ts} are much simpler and easier to implement.
Therefore, they usually offer more features than the distributed ones.
However, they also impose a single-point-of-failure.


Within the distributed approaches we can distinguish those works which replicate data and those which do not.
TSC belongs to the first group, and replicates all the triples in each deployed kernel.
In \ac{tripcom} each kernel stores one or more subspaces and can contact other kernels responsible for different spaces.
To do that, if the space \ac{url} is provided, it simply resolves this \ac{url} using \ac{dns} and contacts the other kernel.
Otherwise, the kernel uses three additional strategies:
\begin{itemize}
  \item Triple Provider.
	It uses shortcuts to know who answered a query in the past.
  \item Recommender.
	It uses shortcuts to know which kernel successfully routed a query in the past.
  \item Indexing - \ac{dht}.
	It creates indexes using a hash function over the subject, predicate, object and space \ac{url}.
	Then, it stores these indexes in a distributed database which relies in a structured \ac{p2p} system. % mencionar PGrid???
\end{itemize}
\ac{cspaces} uses a similar but vaguely described super-peer network \citep{martinrecuerda_application_2006}.


As discussed in Section~\ref{sec:soa_ts_ubicomp}, we also deal with distribution in one of the spaces of our hybrid model.
Specifically, we locally manage the content and we distribute the queries.
This resembles to the \emph{tsc++}'s strategy.
However, instead of using a \ac{p2p} framework to access to the contents,
we individually access to them using several \ac{http} requests. % añadir que son "parallel" o sólo liará?
In other words, each client may access to various server to obtain a result for a given primitive.


\begin{savenotes}
  \begin{table}[htbp]
    \caption{Distribution of the spaces.}
    \centering
    \begin{tabular}{ l c c p{4cm} }
    \hline 
    & C/S & Distributed & Distribution \\
    & access & space & strategy \\
    \hline 
    \midtsc{} & $\checkmark$ & $\checkmark$ & Replication \\ % Possitive broadcasting
    \midsws{} & $\checkmark$ & × & - \\
    \midstuples{} & $\checkmark$ & × & - \\
    \midcspaces{} & $\checkmark$ & $\checkmark$ & Not detailed \\ % además no implementado :-S
    \midtscpp{} & × & $\checkmark$  & Local writing, different query strategies \\ % Flooding, RW, etc.
    \midtripcom{} & $\checkmark$ & $\checkmark$ & Structured network, different strategies \\
    \midsmartmt{} & $\checkmark$ &  $\checkmark$ & Theoretical \\ % decidir si me convence que no haya sido implementado, TODO citar al tipo que dijo como hacer SIBs distribuidos
    \midnardini{} & $\checkmark$ & × & - \\
    \hline 
    \end{tabular}
    \label{tab:distribution_comparison}
  \end{table}
\end{savenotes}


\subsection{Discussion}

% Model: Triple Space
% Originalmente REST
% Luego ha ido derivando e implementando funcionalidades más complejas alejandose del diseño inicial
% Nosotros proponemos una vuelta a los origenes para recuperar una simplicidad de la que limited devices pueden beneficiarse


% Ver cómo TSC puede cumplir con todos los principios de diseño
% tabla de comparativa con REST
% no hay muchos que cumplan, nosotros: esto




\subsection{TSC}
% modelo propuesto
TSC \citep{fensel_tsc_2007} is the middleware which first implemented the ideas of Fensel et al. \citep{fensel_triple-space_2004}.
TSC explores a \acl{tsc} infrastructure based on the \ac{rest} architectural style.
Equivalently to \ac{rest}, TSC proposes a stateless infrastructure where the resources are identified by \acp{uri}.
In the first approximation, each tuple was considered a resource.
These tuples correspond with \ac{rdf} triples.
In a later work,  Krummenacher et al. \citep{krummenacher2006specification} proposed to use identifiers just for a set of triples (i.e. \ac{rdf} graphs).
% bundler cita el de "WWW: or what's wrong with the web" para explicar eso último
They argue that resources usually need to be modeled using more than just a triple.
Besides, the use of graphs reduces the complexity on storage.
To cope with the heterogeneity of data and scalability issues, they propose the use of independent spaces identified by their own \ac{uri}.
Using these disjoint spaces, they can restrict the number of participant in each one.


% cosas relativas al despliegue
TSC combines client/server and \ac{p2p} architectures using a hybrid super-peer one.
The super-peers of this architecture communicate with each other using a middleware which offers a virtual shared memory space of Java objects.
They present 3 different types of nodes:
\begin{itemize}
  \item The \emph{servers} are the responsible for storing and replicating the spaces.
  \item The \emph{heavy clients} are not always connected to the system, but they have the same responsibilities as the \emph{servers}.
	They can work off-line with their replicas.
  \item The \emph{light clients} can access to the spaces using proxies or an \ac{http} \acs{api}.
\end{itemize}


% arquitectura
TSC consist of three main components: a mediation engine, a data access layer and a coordination layer.
The mediation engine manages data heterogeneity by intermediating between nodes using different data schemas.
The data access layer ensures persistence of data, supports template matching and provides some reasoning over the knowledge stored.
The coordination layer replicates named graphs to all involved TS kernels and therefore has to be synchronized with the data access.
% http://tsc.deri.at/deliverables/D21.html
% http://tsc.deri.at/deliverables/D31/D31.html
To do that, they use a \ac{p2p} implementation of a virtual shared data space. % cita a CORSO?
% decir que ofrece transaccionalidad gracias a CORSO? transactionality
% It also offers a transactional context and a simple form to publish and subscribe to certain patterns.



\subsection{Semantic Web Spaces}

% modelo propuesto
Semantic Web Spaces (SWS) \citep{tolksdorf_coordination_2006} defines two different data coordination views: data view and information view.
In both views the tuples are represented with three fields which correspond with \ac{rdf} triples' subject, predicate and object.
The data view stores syntactically valid \ac{rdf} and it is accessed using Linda primitives.
The information view stores consistent and satisfiable data which are managed using new primitives.
This view takes into account the knowledge defined by ontologies to perform semantic matching over inferred triples.

% cosas relativas al despliegue
SWS are virtually partitioned using contexts, which provide a simple form of access control.
Clients can allocate contexts to control their view privately or share it with other clients. % This allow clustering ¿WTF?
The different active spaces and the tuples they contain are represented in a meta-space using a meta-model ontology.
This ontology also allows to express the hierarchical relation between the spaces and security policies.
% por lo que entiendo, está centralizadísimo

% arquitectura
The prototype contains three main components: publication, retrieval and security.
The publication component describes how the platform communicates with the clients or storage managers.
The retrieval component process queries and can reason to enhance results. % no se razona cada vez que se inserta una tupla
The security component restrict the access of client to spaces according to the policies defined.  % hablan de agentes
% dicen que experimentaron problemas de escalabilidad



\subsection{sTuples}

% modelo propuesto
STuples \citep{khushraj_stuples:_2004} was conceived by Nokia Research Center as a pervasive computing work.
It extends the JavaSpace middleware, a object-oriented \acl{ts}, using a semantic match over the semantic tuples.
To do that, a semantic tuple is an object with a field with the semantic description in the DAML+OIL format. % definir DAML+OIL o citarlo?


% cosas relativas al despliegue
The space is centralized and uses a framework called Vigil to create clients and services. % cita Vigil?
Vigil acts as a communication gateway with several protocols and it is directly integrated with sTuples.


% arquitectura
The JavaSpace extension is done in sTuples through two components: the Tuple Manager and the Tuple Matcher.
The manager adds, removes or states changes of tuples.
Every insertion requires the semantic data to be asserted.
The matcher implements the algorithm to match templates.
Besides, specialized agents reside on the space to offer added functionality to the users.
For instance, they implemented recommendations, subscriptions and task-executions in their prototype.



\subsection{Conceptual Spaces}

% modelo propuesto
Conceptual Spaces, or CSpaces, were born to study the applicability of semantic \aclp{ts} to different scenarios including \acl{ubicomp}.
Tuples in CSpaces contain seven different fields.
Some fields enable to express logic theories using different formal languages.
Other remarkable fields are used to define the version, specify the creator or to define to which subspaces the tuple belongs. % explicar los 7 campos?

% cosas relativas al despliegue
The spaces in CSpaces can be individual or shared.
An individual space belongs to a single process.
A shared one allows access to all the processes who have agreed on.

% arquitectura
%Es una IDEA y NO la habían IMPLEMENTADO (posteriormente hicieron un prototipo limitado para un paper de 2006).
CSpaces differentiates between the same three types of nodes as TSC.
They intended to implement individual spaces in heavy-clients and shared spaces in servers.
They first considered a centralized implementation for the shared-spaces, % lo dicen en la comparativa
and later on a P2P network of the nodes managing a shared-space. % paper de después
In any case, to the best of our knowledge, this conceptual exercise never went beyond a rather limited prototype.
% TODO confirmar 100%



\subsection{tsc++}

% modelo propuesto
Tsc++ \citep{krummenacher_open_2009,blunder_distributed_2009} is a new version of the TSC project.
Although the conceptual model is equivalent to TSC, its implementation completely changes.


% cosas relativas al despliegue
% Páginas 54 y 55
Tsc++ proposes to divide the spaces among the participants instead of replicating each space in the different kernels as TSC does.
Therefore, a space corresponds with a group a nodes on the \emph{Jxta} \ac{p2p} framework \footnote{https://jxta.dev.java.net/}. % jxta de otra forma?
These nodes write their information locally and discover information in other nodes using some special nodes.
This special nodes in Jxta are called \emph{rendezvous} (RDV).
Tsc++ proposes and evaluates two discovery algorithms: flooding and random walk (RDW).
\begin{itemize}
  \item The flooding algorithm sends the request to a RDV, which propagates it to all the known nodes.
	If this nodes are RDV, they may also propagate the request until a number of maximum hops is reached.
  \item The RDW algorithm chooses a new peer to send the message to after the request has been processed in the previous one.
	In Tsc++ the request is sent to a RDV which creates a list of peers and sends it to the first one in the list.
	Afterwards, the first node sends the request to a new node from that list.
	This step is repeated until a maximum number of hops is reached or a result is returned.
\end{itemize}
% De dónde leches saqué esto de NB?
%These nodes follow a simple flooding-based strategy to distribute their information called \emph{negative broadcasting}.
%In negative broadcasting each node writes the information locally and read querying to the rest of the nodes of the space.


% arquitectura
Tsc++ has two main components: the communication and the data access.
The data access layer relies on a semantic repository to perform queries and store the triples.
% mencionar que son Sesame \citep{broekstra_sesame:_2002} y Owlim \citep{kiryakov_owlimpragmatic_2005} ???
The communication layer maps the reading coordination primitives to group communication.
% No evaluan el mecanismo de subscripción porque dicen que se basa en DHT y que como es escalable no necesita más evaluación
% No me queda claro de qué forma usan DHT sobre los nodos RDV que son los que miran las subscripciones cómo están.
Besides, it offers a subscription mechanism independent to the content stored in the space which uses triples and templates as a basis.


% it does not make inference
% it does not allow expressive querying
% it has not been designed for devices with reduced computing capabilities, because tsc++ middleware focused on architecture and implementation in large scale and we focus in the short scale (local area networks with an intelligent environment).



\subsection{Triple Space Communication}

% modelo propuesto
Triple Space Communication\footnote{TripCom (IST-4-027324-STP, www.tripcom.org)} (TripCom) is a Triple Space solution which works with \ac{rdf} triples as resources.
The most remarkable aspects of the model proposed by TripCom are:
\begin{itemize}
  \item The use of subspaces to form nested multiple spaces.
	Using them it restricts the communication to a part of the whole space.
	This restriction leads to local scalability and completeness.
  \item A mechanism to overlap spaces called \emph{scopes}.
	Using \emph{scopes} a client can create a temporary copy of some tuples.
	However, any insertion and deletion would not apply to the whole Triple Space.
  \item A meta-space to describe the relations between spaces or to which graph a tuple belongs to.
  \item Different types of templates can be used to query the space, from simple wildcard-based templates to more expressive SPARQL queries.
  \item New primitives to read at triple or at graph level and to write multiple triples at once.
  \item A subscription mechanism.
\end{itemize}
% TODO No poner las antiguas??


% cosas relativas al despliegue
To access to a space a client needs to contact a TripCom instance, i.e. kernel.
% TODO Duda inglés, está bien hecho eso de "sujeto verbo objeto and verbo" o debería ser "sujeto verbo objeto and sujeto verbo objeto"???
% TODO Uniformizarlo a lo largo del texto
Each kernel stores one or more subspaces and can contact other kernels responsible for different spaces.
To do that, if the space \ac{url} is provided, it simply resolves this \ac{url} using \ac{dns} and contacts the other kernel.
Otherwise, the kernel uses 4 different strategies:
\begin{itemize}
  \item Triple Provider. It uses shortcuts to know who answered a query in the past.
  \item Recommender. It uses shortcuts to know which kernel successfully routed a query in the past.
  \item Indexing - \ac{dht}.
	It creates indexes using a hash function over the subject, predicate, object and space \ac{url}.
	Then, it stores these indexes in a distributed database which relies in a structured \ac{p2p} system. % mencionar PGrid???
\end{itemize}


% arquitectura
A TripCom kernel has several components which communicate with each other using a \acl{ts} solution.
The most important components regarding how the triples are distributed are:
\begin{itemize}
  \item Triple Store Adapter.
	It acts as a gateway with the selected repository.
  \item Query Processor. It descomposes query into subqueries.
	Then, to answer each subquery the kernel: a) queries the local data store or b) forwards it to other kernel.
  \item Distribution Manager.
	Forwards queries or write requests to the appropiate kernels.
\end{itemize}
Apart from these, there are also security, mediation, metadata and transaction managers.
On top of these components, TripCom offers three different \acp{api}:
Triple Space \acs{api} to map the primitives,
web service \acs{api}, which offers higher level service consumption \acs{api}
and management \acs{api} to configure the each kernel.



\subsection{Smart-M3}

% modelo propuesto
Smart-M3 \citep{honkola_smart-m3_2010} focuses on solving interoperability issues of different devices and protocols.
This open-source middleware supports three types of primitives: \acl{ts}-based primitives, subscriptions and a query over a whole space.
To that end, it  distinguish between two types of nodes: Semantic Information Brokers (SIB) and Knowledge Processors (KP).
The SIBs store the knowledge and perform reasoning process on behalf of the KPs.
A KP accesses to a SIB using two different types of interfaces (called KPI):
\begin{itemize}
  \item The low-level API uses triples as a base.
  \item The high-level API allows the developer to use ontological entities such as classes, relations and individuals.
\end{itemize}


% cosas relativas al despliegue
The space can be distributed over more than a SIB using a \emph{distributed deductive closure}.
 % TODO asegurarse 100% de esto que voy a decir!
However, to the best of our knowledge no evaluation with more than a SIB has been presented yet, making it de facto an centralized space.
For the communication between SIBs and KPs, Smart-M3 defines a protocol called \emph{Smart Access Protocol (SSAP)}. % decir que está basado en XML?
SSAP can be implemented on top of different communication mechanisms (e.g. SOA, XMPP, Bluetooth or TCP/IP) making the solution communication agnostic. % TODO acrónimos?


% arquitectura
% DUDA enfatizar que una capa usa a la de abajo?
A SIB is divided in five layers: transport, request handling, graph operations, triple operation and persistent store (see Figure~\ref{fig:smartm3_archi}).
The \emph{transport layer} has different processes for each communication mechanism supported.
Once the SSAP operations have been extracted, the \emph{request handling} creates a thread for each primitive.
The \emph{graph operations} layer works with primitives which use \acs{rdf} graphs.
In the \emph{triple operation}, expressive queries may need to be translated to query the store.
% decir que también soporta subscripción?

% Esto es para que el parrafo de arriba se entienda mejor
\InsertFig{smartm3_architecture}{fig:smartm3_archi}{
  Architecture of an Smart-M3 SIB.
}{
  Image extracted from \citep{honkola_smart-m3_2010}.
  % More info
}{0.6}{}

% De: http://www.diem.fi/files/KP_reference%20implementation.pdf
%  The Semantic Information Broker is the information repository of the Smart 
%  Environment. In theory, the Smart-M3 Smart Space can consist of one or more SIB 
%  entities. However, the SIB reference implementation does not currently support any 
%  kind of interaction between different SIBs, which would enable distribution of the 
%  Smart Space. On the IOP, the SIB is implemented as a NoTA SN.



\subsection{\citeauthor{nardini_semantic_2013}}

% modelo propuesto
\citet{nardini_semantic_2013} explore the use of Description Logics on top of an existing \emph{tuple centre} implementation called TuCSoN \cite{omicini_tucson:_1998}.
A tuple centre is a \ac{ts} where the behavior can be programmed as a result of the basic coordination primitives.
TuCSoN aims to provide \emph{tuple centres} that agents can use through Linda-type primitives.
This interaction can cause reactions which trigger behavior previously implemented in the \emph{tuple centre}.

% cosas relativas al despliegue
In TuCSoN each \emph{tuple centre} is deployed in a device's port and is identified by a name.
The communication with each centre is direct, not based in any underlying network. % no les mola P2P decian en algun lado
The agents can access to remote \emph{tuple centres} and migrate to them, but these interactions need to be programmed.

% arquitectura
According to \citeauthor{nardini_semantic_2013} each centre has a domain ontology attached.
These ontologies provide a semantic interpretation to the tuples stored in each centre.
They extend TuCSoN providing semantic tuples, templates, primitives, reactions and matching mechanism.
\begin{itemize}
  \item The semantic tuples express the logic terms.
  \item The semantic template defines a language based on logic.
  \item To discriminate common syntactic primitives from semantic ones they use a keyword.
  \item The semantic reactions redefine TuCSoN's events, guards and reactions enabling the use of first-order-logic.
	The events define when a reaction should be triggered.
	The reaction guard represent a set of conditions which need to be satisfied to execute a reaction.
  \item For the semantic matching mechanism they integrate TuCSoN with a reasoner (Pellet). % TODO citar pellet?
\end{itemize}



\subsection{Comparison}
% TODO CSpaces no hace más que liar el analisis, me planteo no incluirlo
%	Porque es muy bonito poner ideas felices, pero no ha sido implementado nunca!





\subsubsection{Additional features}

The Table~\ref{tab:compAdds} summarizes the additional mechanisms provided by the solutions analyzed.
Most of them offer subscription mechanisms which are particularly useful for distributed environments.
% TODO modificar esto que ahora no hay implementacion!
Although this dissertation focuses more in the querying mechanism, in the Chapter~\ref{cha:implementation} we will propose a subscription mechanism.
Other solutions offer , but we have discarded its usage due to the hard implementation for distributed systems running on dynamic environments.


% transacciones
% subscripciones
% etc.

\begin{table}[htbp]
\caption{Features offered by the analyzed solutions.}

\begin{tabular}{ l p{3cm} l p{4cm} }
\hline 
  & Subscriptions  & Transactions  & Other features \tabularnewline
\hline 
 TSC & Yes  & Yes  &  \tabularnewline
 SWS & No  & No  &  \tabularnewline
 sTuples & Yes  & No  &  \tabularnewline
 CSpaces & Yes  & No  & Multiple read and writes \tabularnewline % bussler dice que tiene algún tipo de transaccionalidad, yo no lo creo
 tsc++ & Yes  & No  &  \tabularnewline
 TripCom & Yes  & Yes  &  \tabularnewline
 Smart-M3 & Yes  & No  &  \tabularnewline
 Nardini et al. & No \footnote{But it has been implemented in TuCSoN before \cite{ricci_extending_2002}.} & No  &  \tabularnewline
\hline 
\end{tabular}
\label{tab:compAdds} 
\end{table}






%\subsubsection{Implementations}
%	5. Implementado?
%	6. Implementación disponible
%	Interoperability
%	Multiplatform
%	8. dependencias: librerias, etc. (para decir que lo nuestro se puede implementar en casi cualquier plataforma de forma fácil)



\subsection{Discussion}

% extraer algunas conclusiones de lo que falta
% tono general: están muy bien de características, pero a la hora de hacerlo distribuido falla
% hacen aguas en proponer un modelo que pueda ser usado más allá del client-servidor
% Suitability for Ubicomp

So far, semantic \aclp{ts} solutions have offered a sophisticated models with interesting additional features.
However, in most of the cases this is possible by delegating such tasks on a subset of more powerful and static nodes.
Therefore, the distribution of the space is restricted to some nodes when in the best case, or centralized in a unique node in the worst case.
In these cases, the rest of the nodes are head toward the use of the space as simple clients.


In this thesis we will explore the most decentralized case: the one in which each node manages its own information.
Doing so, a user can carry out his profile in the smart-phone or a sensor can provide the last measures on-demand.
% sólo tsc++ seguía esa premisa

% qué es necesario para cumplir esa visión y porque las alternativas vistas no se ajustan:
% dependencias mínimas, protocolos estándares, etc.


% TODO CONCLUSION
Although both semantic space-based and Triple Spaces implementations exist, none of them has been specifically designed to be run in devices with constrained capabilities apart from our solution.



% Lo que sería un puntazo:
%     el concepto de TSC nacio de REST
%     el interés por añadir features, hizo que la gente olvidase los ppios que guiaban la simplicidad de REST
%     creemos que recuperar esa simplicidad por encima de las features es importante => la clave para facilitar y en algunos casos posibilidar que los dispositivos constrained lo implementen
%     poner características de REST y quién las cumplía!
%     no queremos ser los más ortodoxos, pero si :-P
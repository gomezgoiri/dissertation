\subsection{The \acl{sw}}
\label{sec:soa_sw}

A problem of the initial view of the Web was that it was human-centered.
Regardless of whether the contents were machine processable, a human needed to interpret them to give them a meaning.
% mirar a ver cuando se empezó a usar \ac{sw} de verdad, porque en el artículo de 2001 dan a enterder que se llevaba un tiempo usando
In the early 2000s, \citet{berners-lee_semantic_2001} proposed the use of what they called the \acf{sw} to solve that problem.
The \ac{sw} was conceived as an extension of the ordinary web which would bring structure to its content.
As the ordinary web, the \ac{sw} benefits from the universality the hypertext provides by linking \emph{anything with anything}.
Besides, like the Internet, the \ac{sw} is intended to be as decentralized as possible.

The \acl{sw} and its key features are defined by the \emph{World Wide Web Consortium} \citep{semanticWeb-FAQ} as:
\begin{quote}
The vision of the \acl{sw} is to extend principles of the Web from documents to data.
Data should be accessed using the general Web architecture using, e.g., URI-s;
data should be \emph{related to one another} just as documents (or portions of documents) are already.
This also means creation of a common framework that allows data to be \emph{shared and reused} across application, enterprise, and community boundaries,
to be \emph{processed automatically} by tools as well as manually, including revealing possible \emph{new relationships} among pieces of data.
\end{quote}

Meaning is expressed in the \acl{sw} as sets of triples.
Each triple is composed by a subject, a predicate and an object like in a normal sentence (see Figure~\ref{fig:triples_example}).
Therefore, in  \citeauthor{berners-lee_semantic_2001} words, \emph{a document can assert that things (people, web pages or whatever) have properties (such as "is a sister of," "is the author of") with certain values (another person, another Web page)}.
A key difference with a normal sentence is that each concept is unambiguously defined by an URI.
These URIs form links between different triples as shown by the example of the Figure~\ref{fig:triples_example}.
% This set of triples can be expressed in multiple formats such as RDF, Turtle, Notation3 or NTriples. % TODO citas


\InsertFig{triples_example}{fig:triples_example}{
  Sample triples using different ontologies.
}{
  All the triples are represented graphically and some of them also textually.
  They describe academic and personal details about the author of this thesis.
  The knowledge is expressed using four different ontologies: FOAF, DC, SWRC and CiTE. % citar?
  Note that we use aliases, known as prefixes in most Semantic formats, to shorten some URIs and enhance the clarity of the Figure.
}{1}{}


A problem with the information described so far is that two different databases may use different URIs to express the same concept.
To overcome this the \ac{sw} offers collections of information called ontologies.
An ontology is a document which expresses the relations between terms commonly using a taxonomy and a set of rules to infer content.
The taxonomy defines the classes of a given domain and how they relate with each other.

Of course, different content providers could provide similar data described according to different ontologies.
Fortunately, ontologies can be easily mapped by providing equivalence relations within them.
In the same way, an ontology can be extended to adapt it to different application domains.
In any case, the reuse of the same models is beneficial and is therefore promoted through standardization.

% Remarkably, the use of the \ac{sw} has been promoted in the last years with the Linked Open Data (LOD) initiative.
% The LOD are datasets which follow a series of principles on how to open and publish data.
% The ultimate goal of the LOD is to publish linked terms using full semantics.
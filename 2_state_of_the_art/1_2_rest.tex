\section{\acl{rest}}
\label{sec:rest}

\acf{rest} is a network-based architectural style proposed by \citet{fielding_architectural_2000}.
% descripción de propiedades de REST: hipermedia distribuído
% que es hipertexto? http://roy.gbiv.com/untangled/2008/rest-apis-must-be-hypertext-driven#comment-718
It aims to cover certain properties, explained in Section~\ref{sec:network_properties}.
% Estilos de los que se deriva: ¿?
To achieve these properties, \ac{rest} establishes the following constraints from other network-based architectural styles:
% explicarlo o es demasiado obvio?
\begin{description}
 \item[\acf{restcs}.] Providing an \ac{api} to the clients, they are isolated from back-end implementation details.
		       % esto ayuda a: , scalability, evolvability (apps independientes pueden evolucionar mejor)
 \item[\acf{rests}.] The state is fully stored in the client and therefore each request has all the information needed to process it.
 \item[\acf{restcache}.] When added to the \ac{restcs} constraint, this style replicates content obtained from a server in the client.
 \item[\acf{restu}.] It is the key constraint which distinguishes \ac{rest} from other architectural styles.
                      This constraint is composed by the following ones:
    \begin{description}
	% explicados en sección 5.2 de Fielding, resumir?
	\item[\acf{restid}.] Resources are the conceptual targets of hypertext references.
			      Their identification offers a generic interface to access and change the values of a resource.
	\item[\acf{restrep}.] Representations are composed by a sequence of bytes and the metadata to describe those bytes.
	\item[\acf{restdesc}.] The client and server have to agree on standard methods and media types. % linkar al tipo de la Web esa
				% Explicación Fielding:
				% interaction is stateless between requests
				% standard methods and media types are used to indicate semantics and exchange information
				% responses explicitly indicate cacheability.
		                %\ac{http}'s content negotiation for instance, allows to reach an agreement on the media types.
	                        Beyond that point, each request or response should contain all the needed data to process it.
	                        Therefore, in \citeauthor{fielding_architectural_2000}'s words,
	                        the type should be registered,
	                        the registry should point to a specification and 
	                        the specification should explain how to process data according to its intent \citep{wahbe_self-descriptive_2010}. % no necesariamente tiene que ser estándar
	\item[\acf{resthateoas}.] This is a controversial constraint because most of the self-proclaimed ``\ac{rest}'' \acp{api} fail to follow it \citep{moore_hypermedia_2010,house_how_2012}.
	                           It states that no out-of-the-band information should guide the interaction with an \ac{api}.
	                           Instead, the hypertext should guide it.
	                           In other words, the client must know just an initial URL and the application's media types.
	                           From that point, it should select the alternatives proposed by the server to change to the next application state \citep{fielding_rest_2008}.
	                           % propuestas: respresentaciones (links) y la manipulación implicita de las mismas (CRUD)
    \end{description}
 \item[\acf{restl}.] Each layer provides services to the top layer. % e.g. TCP/IP
 \item[\acf{restcod}.] The client has a set of resources which does not know how to process.
                        This \emph{know-how} is downloaded from the server.
\end{description}



\subsection{\ac{rest} vs WS-* services}
\label{sec:protocols}

WS-* web services, also called ``Big Web services'', together with \acs{rest}ful architectures are probably the most common remote invocation substyles currently used in the Internet.
WS-* standards offer more features such as transactions, reliability or message-level security.
% TODO describir los estándares mejor
They syntactically describe the services' functionality and interfaces using \ac{wsdl} and the communication is carried out on top of \ac{http} in messages encapsulated using \ac{soap}.
% decir o poner algo de UDDI aquí?
However, they also require further architectural decisions on different layers of the WS-* stack.
These decisions make the people perceive it as more complex than the \ac{rest}ful style \citep{guinard_search_2011}.
% pensar bien cómo introducir el tema de la simplicidad: por diseño de REST, por las decisiones que hay que tomar en WS-*???

%The \emph{scalability} and the \emph{simplicity} stand out from these properties.


Besides the simplicity, another key difference between both approaches is how they use web standards. % (e.g. \ac{http}, \ac{uri} or \ac{mime})
The web implements \ac{rest}'s principles.
Therefore, a \ac{rest} solution can rely only on well-accepted standards used in the web (e.g. \acs{http}, \acs{uri} or \acs{mime}).
This acceptance also results in a massive availability of libraries in most of the computing platforms.
% This standards already have many libraries and tools available, which eases the adoption of this style by developers.


On the contrary, WS-* also uses some of these standards, but in ways they were not designed for.
For instance, it uses \ac{http} as a transport layer instead of as an application layer protocol.
This prevents the resulting architectures from achieving some of \ac{http}'s desirable properties such as scalability or visibility.


\bigskip


The use of both styles in resource-constrained devices is represented by
\ac{dpws} \citep{moritz_devices_2010}, based on WS-*, and the \ac{wot} \citep{guinard_internet_2011}. % o mejor poner su tesis?
% TODO reescribir esto para que no suene tan positivo y sobrado
\ac{dpws} defines a subset of WS-* to make it suitable for resource-constrained devices.
%\ac{dpws} is claimed to be used in industrial environments.
Its most remarkable features are: decentralized multicast-based discovery, secure message transmission, subscription and event notifications.
\citeauthor{moritz_devices_2010} came to the conclusion that \ac{dpws} can be restricted to be fully compatible with the \ac{rest}ful style while it still covers some missing features (e.g. eventing and discovery).
% no menciono que sean más pesados para cacharros porque no he encontrado un sitio en el que defiendan que \ac{dpws} lo es (si que hay alguno que comparan REST y SOA, pero no DPWS)


In this dissertation we focused on \ac{rest} because of its massive acceptance as a way to integrate resource-constrained devices. % TODO poner algún dato?!
We delve into the \ac{wot} in Section~\ref{sec:soa_ubicomp}. % explicación de porqué no aquí




This section introduces them both, but for an extensive analysis on their advantages and disadvantages please refer to \citet{pautasso_restful_2008}.



\subsection{Suitable Protocols for \ac{rest}}
\label{sec:protocols}
% hablar un poco de HTTP y CoAP y decir por qué no hablamos más a menudo de CoAP

% sólo por el correcto uso de ciertos protocolos se puede conseguir todo menos HATEOAS => citar el vídeo aquel
Although an \ac{api} does not adhere to the \ac{rest} style just because it uses certain protocols,
the correct use of some of them can help to achieve most of them \citep{moore_hypermedia_2010}.
Historically, \acf{http} has been considered a suitable protocol in that regard.
\ac{http} is a simple protocol whose adoption by computing platforms is massive.


However, in the last few years \acf{coap} has emerged as a specialized web transfer protocol for resource constrained devices. % has emerged, has arisen?
Some noteworthy features of \ac{coap} are
(1) the reduced message size,
(2) the use of UDP as a transport layer (with the possibility of using \emph{multicast} communication),
(3) similarity with \ac{http} (both to reuse its properties and to ease cross-protocol proxying), and
(4) a resource discovery mechanism. % TODO cita a CORE
% mencionar seguridad?


One could argue that to implement a lightweight \ac{tsc} solution, \ac{coap} should be used as a baseline.
However, we have chosen to work with \ac{http} for the following reasons:
\begin{itemize}
  \item Direct interoperation with other web-solutions.
        Most of the Internet-based \ac{api}s use \ac{http}.
        Directly using \ac{http} we can avoid proxies.
        Proxies may introduce latency in the response time degrading network performance. % siempre degradará la eficiencia algo, pero habría que verlo y testearlo, la verdad...
        % TODO, será cierto? o tardará menos en conjunto?
  \item \ac{rdf}-based media-formats can be rather verbosed.
	This contrasts with \ac{coap}'s message size limitations.
	Dealing with this limitations was not one main goals of the thesis.
	However, we have considered them at some points of the dissertation to avoid unrealistic assumptions.
  \item \ac{coap} is an ongoing standard.
        Therefore, its definition is currently changing in each draft version.
        A practical limitation of this is that at the moment there are few libraries and tools to work with \ac{coap}.
        This limits the range of platforms which could adopt any proposed solution.
\end{itemize}


Nevertheless, due to \ac{http}'s similarities with \ac{coap}, the future adoption of the latter should be relatively straightforward.

% Ideas no usadas:
% REST no dice nada, pero de facto el protocolo es HTTP
% durante el desarrollo de esta tesis en resource constrained ha surgido CoAP
%   muchas de las cosas aquí comentadas son iguales o mejor sobre CoAP
%   de todas formas, ha habido algo que ha guiado nuestro diseño: librerias existentes => afecta a plataformas
%         si, pero para eso está lo de layered approach! => motivos prácticos
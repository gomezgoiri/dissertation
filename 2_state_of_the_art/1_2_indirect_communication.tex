\subsection{\acl{ts}}
\label{sec:tuplespaces_eoa}

% Ojo cuidado porque según esto mi middleware va a dejar de ser time uncoupled! (mirar Figure 6.1 para mayor concrección)
% En esa misma sección aclara que algunas implementaciones no son time-uncoupled.
% por otro lado, asíncronismo != time uncoupling

\acf{ts} has its roots in the Linda parallel programming language \citep{gelernter_generative_1985}.
In this communication model different processes read and write pieces of information so-called tuples in a common space.
Tuples are composed by one or more typed data fields (e.g. $<"aitor",1984>$ or $<3,7,21.0>$).
The tuples are accessed associatively using a template.
A template provides either a value or type for different fields (e.g. $<String,1984>$ or $<Integer, Integer, Float>$).
The operations over the space are defined by different primitives.
Although the primitives may change from implementation to implementation, the most common ones allow to:

\begin{itemize}
  \item \emph{Access the tuples non-destructively}, using a primitive which is usually called \emph{read} or \emph{rd}.
	This primitive returns a tuple from the space which matches the given template without changing the space.
  \item \emph{Access the tuples destructively}, using a primitive which is usually called \emph{take} or \emph{in}.
	This primitive extracts a tuple which matches the given template from the space.
  \item \emph{Write a new tuple into the space}. This primitive is usually called \emph{write} or \emph{out}.
\end{itemize}


\InsertFig{venn-sec1_2}{fig:venn_tuplespaces_ubicomp}{
  \aclp{ts} solutions for Ubicomp.
}{
  Scope of this subsection.
}{0.6}{}

\bigskip

% Ignorar esta sección? Es realmente necesaria cuando ya he puesto la comparativa de TSC?
So far, \acl{ts} has been adapted to \acl{ubicomp} by different authors.

% TODO leer artículos que me faltan y corregir
The \emph{event heap} \citep{johanson_extending_2004} is a system used for a specific \ac{ubicomp} sub-domain: interactive workspaces.
In this scenario there are rooms with different devices deployed and where mobile devices can enter.
Each room has its own space where the devices exchanging tuples to cooperate.
%For example, a video can be presented in a display and through a remote controller, the user can place the tuple for pausing it.
%The display consumes these kind of tuples, so it pauses the video whenever somebody places that tuple on the space.
% This system is simple to implement but is limited since it centralizes the space in a single machine per room.
This work merely identifies the requirements of these environments and the properties which solve them.
Then, it discusses how these properties can be satisfied using \ac{ts} or some extensions.
Finally they compare its implementation both with other \acp{ts} implementations or other coordination infrastructures. % e.g. RMI, MOM, Pub/sub

Although the analysis presented by \citeauthor{johanson_extending_2004} identifies the key requirements for \ac{ts} in \ac{ubicomp},
the solution they propose is purely centralized.
We opted for a distributed solution at the expense of limiting some of the features.
% indicar cuales son o pasar?
% usar sus características a la hora de valorar nuestra solución TSC en futuras secciones?


\emph{L$^2$imbo} \citep{davies_l2imbo:_1998,friday_experiences_1999} replicates the tuples to avoid a single point of failure.
Each node joined to a space uses an IP multicast address to exchange messages with other nodes in that space.
Writing into a space involves sending a multicast message to inform to the rest of the nodes of the tuple written.
Reading operation usually requires local reading.
Destructive reading of the tuple is more complex as it requires a global withdrawal.
In \emph{L$^2$imbo} only the owner of a tuple can remove it from the space.
The ownership of a tuple initially belongs to its creator, but can be transferred. % TODO ver qué añadió Friday


 % meter mas citas de LIME que estas 2?
\emph{LIME} (Linda in a Mobile Environment) \citep{picco_lime:_1999} is a TS solution for mobile systems.
In LIME each mobile device has its own space where it generally writes its tuples.
This space is shared with other devices creating federated spaces, i.e. the aggregation of different shared spaces.
In this way, each mobile can access to tuples in other mobiles whenever they become available.
They also proposed a new writing primitive to insert tuples in remote spaces.
% ejemplo de service discovery?

This is difficult to implement and \citet{coulouris_distributed_2012} complain about the unrealistic assumptions they make to simplify the problem. % cita al libro
These assumptions are the uniform multicast connectivity between devices whose tuple spaces are aggregated and the serialized and ordered connections and disconnections.
In any case, LIME was adapted to several platforms to run both in embedded and mobile devices \citep{murphy_transiently_2006}.


In the \emph{TOTA} (Tuples On The Air) Project \citep{mamei_programming_2009} tuples are disseminated to different devices.
To that end, each tuple has 3 fields:
1) the content of the tuple,
2) a rule which defines how it should be propagated,
3) a rule to define its maintenance.
For instance, they consider a museum where a visitor writes a query tuple describing a piece of art he wants to see.
The propagation rule defines that it should be propagated to all nodes in the vicinity, increasing the distance by one each time.
The tuple is configured to be deleted after a time-to-live period using its maintenance rule.
When the it reaches the room where the piece of art is located, the piece of art will write a response tuple.
This response tuple jumps from a device to another until it reaches the device which queried for it.



\subsubsection{Discussion}

% comparación de todas ellas
% de entre los que ofrecen distribución: este no mola por esto, esto otro mola por esto otro, etc.

Within this solutions, \emph{The event heap} is not considered because of its centralized nature.
Both L2imbo, LIME and TOTA present different ways to disseminate the tuples.
The replication which L2imbo proposes may not be feasible for devices with constrained memory or storage capacity.
Particularly, in spaces populated by many devices like the Internet of Things too many tuples could be generated.
% need to configure the network

TOTA is useful for spatially connected adhoc environments, but it does not seem to suit well on networks where all the nodes can reach each other.
In this case, all the tuples could be replicated on the rest of the nodes showing the same limitations as L2imbo.% COMPROBAR LEYENDO CON ATENCION LO DE TOTA

Finally, despite of their unrealistic assumptions LIME proposes a model where each tuple is responsible of its own part of the space.
They share spaces whenever they become available.

% TODO tabla comparativa? distributed, ad-hoc, federated

\begin{table}%[position specifier]
  \centering
  \begin{tabular}{c|cccc}%p{5cm}}
      ~ & The event heap & L2imbo & LIME & TOTA \\
      \hline
      \hline
      Distributed & No & Yes & Yes & Yes \\
      Goal & x & Availability & Federation & Ad-hoc \\
      % Scalability?
      % Coger propiedades del libro de sistemas distribuidos?
  \end{tabular}
  \caption{Comparison of the most prominent \acl{ts} alternatives for \acl{ubicomp}.} % no columpiarme, o poner todas o ninguna?
  \label{tab:myfirsttable}
\end{table}
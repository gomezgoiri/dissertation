\subsection{\acl{ts}}
\label{sec:tuplespaces_eoa}

% Que un middleware TS sea time-uncoupled depende de la implementación
% por otro lado, asíncronismo != time uncoupling

\acf{ts} computing, also called space-based computing, offers an improvement over traditional distributed shared memory approaches.
% Colouris se refiere a ellos como distributed shared memory antes de presentar TS
Whereas the latters work at byte-level and accessing to memory addresses,
the \acl{ts} works with semi-structured data which is accessed in an associative manner.
In other words, in \ac{ts} the participants read data specifying patterns of interest.


\ac{ts} has its roots in the Linda parallel programming language \citep{gelernter_generative_1985}.
In this communication model different processes read and write pieces of information so-called tuples in a common space.
Tuples are composed by one or more typed data fields (e.g. $<"aitor",1984>$ or $<3,7,21.0>$).
The tuples are accessed associatively using a template.
A template provides either a value or type for different fields (e.g. $<String,1984>$ or $<Integer, Integer, Float>$).
The operations over the space are defined by different primitives.
Although the primitives may change from implementation to implementation, the most common ones allow to:

\begin{itemize}
  \item \emph{Access the tuples non-destructively}, using a primitive which is usually called \emph{read} or \emph{rd}.
	This primitive returns a tuple from the space which matches the given template without changing the space.
  \item \emph{Access the tuples destructively}, using a primitive which is usually called \emph{take} or \emph{in}.
	This primitive extracts a tuple which matches the given template from the space.
  \item \emph{Write a new tuple into the space}. This primitive is usually called \emph{write} or \emph{out}.
\end{itemize}
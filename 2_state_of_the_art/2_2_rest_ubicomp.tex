\subsection{The \acl{wot} and other \ac{rest} Solutions for \acs{ubicomp}}
% nueva sección para hablar de WoT (DPWS ya se ha presentado)

\InsertFig{venn-sec1_1}{fig:venn_ubicomp}{
  Non-semantic works for \ac{ubicomp} apart from \aclp{ts}.
}{
  This subsection focuses on the solutions using \ac{rest}.
}{0.6}{}


The \acl{wot} initiative encourages the use of \acs{rest}-based solutions embedding web servers in daily objects \citep{guinard_internet_2011}.
In this way, the objects can integrate with the \ac{www} as first-class citizens. % decir ``f-c cit of the Web" es reiterativo ya que WWW==web
This integration brings the following benefits:
\begin{itemize}
  \item The smart-things can be linked to enable its discovery by browsing. This involves using the tool most users are familiar with: the browser.
  \item They can be bookmarked or shared through social networks \citep{guinard_sharing_2010}.
  % explicar qué es un mashup?
  \item They can be integrated with other web applications through mash-ups \citep{guinard_towards_2009,ostermaier_webplug:_2010,pintus_anatomy_2011}.
  % TODO citar al canadiense tb? pintus, stirbu y a todo el mundo
  \item Mechanisms such as searching, caching, load-balancing and indexing can be used over the objects to achieve the scalability of the web. % cita al indio
\end{itemize}


% Mencionar un par de trabajos sobre móviles con servidores web embebidos, para que se vea que no es sólo \ac{wot}


\subsubsection{Discussion}

% TODO reescribir esto para hablar sólo de REST vs mi propuesta!

In Section~\ref{sec:remote_invocation} we have analyzed the current trends towards the remote invocation style.
Within them, we can see how \ac{wot} has gained much more attention both for its perceived simplicity and for its seamless integration with the web.
% o decir REST based solutions para no centrar mucho en IoT
Besides, the availability of tools and libraries for many mobile and embedded platforms ease its adoption.


However, as any remote invocation style it introduces coupling between senders and receivers.
% de johanson_extending_2004 he sacado esas características, no la particularización a WoT
These complicates the application changes both in the short-term and in the long-term \citep{johanson_extending_2004}.
In the short-term because nodes constantly join and leave the environment due to mobility or to failures.
In the long-term because the space is used to solve new problems and the obsolete devices are replaced with new technology.
Furthermore, the human interaction is eased by minimizing the configurations.
% quizá lo de johanson_extending_2004 se podría retomar en el capítulo cha:tsc
In Chapter~\ref{cha:tsc}, we deeply compare both \ac{rest} and \ac{ts} to see how they can benefit each other.
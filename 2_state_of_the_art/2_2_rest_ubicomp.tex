\subsection{The \acl{wot} and other \ac{rest} Solutions for \acs{ubicomp}}
% nueva sección para hablar de WoT (DPWS ya se ha presentado)

\InsertFig{venn-sec1_1}{fig:venn_ubicomp}{
  Non-semantic integration approaches for \ac{ubicomp} apart from \aclp{ts}.
}{
  Scope of this subsection.
}{0.6}{}


The \acl{wot} initiative encourages the use of \acs{rest}-based solutions embedding web servers in daily objects \citep{guinard_internet_2011}.
In this way, the objects can integrate with the \ac{www} as first-class citizens. % decir ``f-c cit of the Web" es reiterativo ya que WWW==web
This integration brings the following benefits:
\begin{itemize}
  \item The smart-things can be linked to enable its discovery by browsing. This involves using the tool most users are familiar with: the browser.
  \item They can be bookmarked or shared through social networks \citep{guinard_sharing_2010}.
  % explicar qué es un mashup?
  \item They can be integrated with other web applications through mash-ups \citep{guinard_towards_2009,ostermaier_webplug:_2010,pintus_anatomy_2011}.
  % TODO citar al canadiense tb? pintus, stirbu y a todo el mundo
  \item Mechanisms such as searching, caching, load-balancing and indexing can be used over the objects to achieve the scalability of the web. % cita al indio
\end{itemize}


% Mencionar un par de trabajos sobre móviles con servidores web embebidos, para que se vea que no es sólo \ac{wot}


\subsubsection{\acs{sw} by using Intermediaries}
\label{sec:sw_intermediaries}

% como los smart environments describen contexto usando web semántica

% uso concreto por parte de soluciones significativas: siempre centralizando el uso de semántica en cacharros grandes

% intro a que ahora se va a hablar de soluciones IoT que usen semántica

Adding semantics works well for devices with high computational capacity but may add too much overhead for most of the devices in the \ac{iot}.
To reduce this overhead in such devices, part of this computation is usually delegated to an intermediary.
Some noteworthy example is the one proposed by \citet{broring_semantic_2009}.
% buscar otros ejemplos de enjundia
% The Context Broker Architecture (CoBrA)[Che04] 
% Gu et al. (2007)
% AlarmNet (2008)


\InsertFig{venn-sec2}{fig:venn_semantics_ubicomp}{
  The \acl{sw} for \acl{ubicomp}.
}{
  Scope of this section.
}{0.6}{}


These intermediaries or \emph{Semantic Gateways} are in charge of managing the semantic annotation.
The devices send raw data (which can be compressed) to the intermediaries and the gateways annotate the content semantically.
Thus, the devices do not have to care about any semantic aspect and just collect the data as they did before.

These \emph{Semantic Gateways} reduce the load to embedded devices with limited resources by decreasing the number of requests they have to provide.
In addition, a centralized intermediary can gather all the information and thus, reduce the complexity of managing a distributed environment.

However, using intermediaries to store the semantic data of resource constrained devices also has some drawbacks.
On the one hand, centralization does not faithfully represent mobility situations were individuals carry their own semantic information in their personal devices.
In addition, the data obtained from an intermediary will always be less fresh than the one obtained where it is generated (i.e., sensors).
On the other hand, the servers are critical in centralized systems and therefore, their availability determines the operation of these solutions.
They also impose a burden on the maintenance which may be worthless in some simple scenarios.


% TODO meter alguna referencia a Cabilmonte sobre SPARQL en streaming
% Trata de consultar datos semánticos en streaming a una base de datos relacional.
% Por lo que entiendo, la comunicación sensor-DB se no es semántica.



% diferenciar de una forma menos descarada?

\subsubsection{\acs{sw} in Providers}
\label{sec:sw_providers}
% enumerar aquellas características de IoT???
% sacar algo del paper que habla de los retos de usar semántica en IoT

% pero ahora los cacharros cada vez son más potentes y no es difícil imaginar un mundo poblado por ellos blablah

% WoT y aquel que hacía cosas de móviles

% TODO generalizar a resource-constrained devices
Lately some solutions have arisen to semantically annotate data where it is generated. % in the provider
In the \acl{wot}, multiple solutions have considered using semantics to enrich the data definition in a machine processable manner.
Generally, these solutions embed the metadata in HTML using microdata, microformats or RDFa \citep{mayer_extensible_2011}.
These contents are returned by the Internet connected objects and are used to enhance the findability of the data by search engines.
% Ontologías no se pueden definir, tiene que haber consenso en su uso por parte de las máquinas de búsqueda.
% en vez de eso, queremos mejorar la busqueda por parte de los cacharros, no de los search engines
This approach does not focus on making the devices able to search and interact with others.
% end-to-end search
Instead, it considers them as mere human-oriented information providers which should be indexed by third party search services.
% aquí se puede enganchar también el discurso de Doulkeridis2007desent: coverage and scalability, freshness y monopoly

A more general way to represent semantic data is using RDF\footnote{\url{http://www.w3.org/RDF/}} based representations (i.e., full semantics).
SPITFIRE European project\footnote{\url{http://spitfire-project.eu}} represents the most remarkable effort on gathering full semantics and the \ac{wot}.
It focuses on fully integrating sensor data with the \ac{lod}. 
The \ac{lod} are datasets which follow a series of principles on how to open and publish data.
The goal of the \ac{lod} is to publish linked terms using full semantics.

SPITFIRE shares with our solution the vision of a world populated by devices acting as semantic data providers no matter how small they are \citep{hasemann_rdf_2012}.
Therefore, many of their efforts are complimentary to this work.
%This is done by defining an ontology for mapping other common ontologies and providing semi-automatic generation of semantic annotations from raw data.
%They also propose an abstraction to represent real-world entities (virtual sensors) using data provided by low-level sensors.
%Finally, and more specific to this work, they propose a searching model which predicts the current state of things by computing their periodic patterns in past states.
To search within these providers, \citet{pfisterer_spitfire:_2011} proposes a model which predicts the current state of things by computing their periodic patterns in past states.
Again, the goal of this method is to adapt search engines to the new fashion of data provided on the \emph{\acl{sw} of Things}.

Instead, we propose a model to enhance the search capability of web-connected things in a distributed fashion.
We aim to promote the interaction and collaboration between any devices in these environments.
Not only sensors, but also regular computers or mobile devices \citep{balandin_access_2011}.



\subsubsection{Discussion}

% TODO reescribir esto para hablar sólo de REST vs mi propuesta!

Section~\ref{sec:sw_intermediaries} presents the most common approaches to use \ac{sw} in \ac{ubicomp}: delegate on intermediaries the semantization of the raw data.
However, we do not intend to follow this path.
Instead, we propose to share semantic content directly from the embedded and mobile platforms as the solutions in Section~\ref{sec:sw_providers} do.
This completely distributed approach tries to simplify the maintenance burden and promote the access to the most updated data.


Regarding searching, the solutions on Section~\ref{sec:sw_providers} promote a less distributed search.
Instead, our model tries to enhance searching capability of web-connected things.
This improvement leads to environments where any devices directly interact and collaborate with each other.



In Section~\ref{sec:remote_invocation} we have analyzed the current trends towards the remote invocation style.
Within them, we can see how \ac{wot} has gained much more attention both for its perceived simplicity and for its seamless integration with the web.
% o decir REST based solutions para no centrar mucho en IoT
Besides, the availability of tools and libraries for many mobile and embedded platforms ease its adoption.


However, as any remote invocation style it introduces coupling between senders and receivers.
% de johanson_extending_2004 he sacado esas características, no la particularización a WoT
These complicates the application changes both in the short-term and in the long-term \citep{johanson_extending_2004}.
In the short-term because nodes constantly join and leave the environment due to mobility or to failures.
In the long-term because the space is used to solve new problems and the obsolete devices are replaced with new technology.
Furthermore, the human interaction is eased by minimizing the configurations.
% quizá lo de johanson_extending_2004 se podría retomar en el capítulo cha:tsc
In Chapter~\ref{cha:tsc}, we deeply compare both \ac{rest} and \ac{ts} to see how they can benefit each other.
\subsection{The \acl{wot} and other \ac{rest} Solutions for \acs{ubicomp}}

The \ac{ubicomp} idea is characterized by the presence and collaboration of a number of heterogeneous and often limited computing platforms.
% REST está en todos los lados, pero...
Dealing with this heterogeneity is probably the biggest benefit which \ac{rest} has brought to this field.
A representative example of how \ac{rest} has influenced \ac{ubicomp} is the \acf{wot}.
% \ac{wot} is probably the most prominent architectural solution used for smart-objects. % probably porque no tengo datos que lo defiendan


\InsertFig{venn-sec1_1}{fig:venn_ubicomp}{
  Non-semantic works for \ac{ubicomp} apart from \aclp{ts}.
}{
  This subsection focuses on the solutions using \ac{rest}.
}{0.6}{}


The \acl{wot} initiative encourages the use of \acs{rest}-based solutions embedding web servers in daily objects \citep{guinard_internet_2011,guinard_thesis_2011}.
In this way, the things can seamlessly integrate with the \ac{www} as first-class citizens \citep{gupta_network_2011}. % decir ``f-c cit of the Web" es reiterativo ya que WWW==web
This integration brings the following benefits:
\begin{itemize}
  \item The smart-things can be linked to enable its discovery by browsing.
	This involves using the tool most users are familiar with: the browser.
	% TODO The pervasive presence of web tools and libraries!!!
  \item They can be bookmarked or shared through social networks \citep{guinard_sharing_2010}.
  % explicar qué es un mashup?
  \item They can be integrated with other web applications through mash-ups \citep{guinard_towards_2009,ostermaier_webplug:_2010,pintus_anatomy_2011,blackstock_wotkit:_2012,stirbu_towards_2008}.
  \item Existing mechanisms such as searching, caching, load-balancing and indexing can be used over the objects to achieve the scalability of the web \citep{gupta_early_2010}.
\end{itemize}


% TODO TODO TODO Vestir un poco más esta sección con trabajos relevantes para mi tesis.
% TODO TODO TODO citar los trabajos de WoT que en su momento me molasen, porque si, bien por ellos :-P


\subsubsection{Discussion}


In the last decade, probably because of its perceived simplicity \citep{guinard_search_2011}, \ac{rest} has gained adopters in the \ac{iot}.
Particularly, the \ac{wot} has emerged as a simple mechanism to integrate smart objects with each other, but also with existing web applications.

However, as any remote invocation style \ac{rest} introduces coupling between senders and receivers.
% de johanson_extending_2004 he sacado esas características, no la particularización a WoT
These complicates the application changes both in the short-term and in the long-term \citep{johanson_extending_2004}.
In the short-term because nodes constantly join and leave the environment due to mobility or to failures.
In the long-term because the space is used to solve new problems and the obsolete devices are replaced with new technology.

In this dissertation, we propose a middleware to reduce this coupling.
This middleware ensures that, from the developer perspective, the communication is always driven by the data.
In other words, in our middleware the devices are unaware of others' presence (i.e. they are \emph{space uncoupled}).
\section{Introduction}
\label{sec:soa_intro}

% Otra forma de ver la interop:
%  La European Telecommunication Standards Institute (ETSI) define cuatro capas:
%      technical interop: a nivel de comunicación (p.e. de acuerdo en las 7 capas de OSI)
%      syntactic interop
%      semantic interop
%      organizational interop


% no habría que definir de nuevo y de manera formal ubicomp o vale con la intro?
The IEEE \citep{_ieee_1990} defines \emph{interoperability} as ``\emph{the ability of two or more systems or components to exchange information and to use the information that has been exchanged}''.
The heterogeneity of technologies present in \ac{ubicomp} environments makes this a key property to consider.
The definition clearly distinguishes between two requirements: % o goals o incremental requirements
(1) to exchange information; and
(2) to use that information. % to understand others data (i.e. \emph{interoperation}).


% Buscar una referencia mejor: http://en.wikipedia.org/wiki/Interoperability
Exchanging information in distributed systems comprehends the communication between two systems.
% lo de ab-initio practicamente sólo lo he visto en la wikipedia
For the lower communication levels, we opt for interoperability \emph{ab-initio} relying on standard and widely accepted communication protocols. % e.g. HTTP
For a higher-level (i.e. application layer), Section~\ref{sec:soa_integration} categorizes different integration approaches.
This dissertation aims to delve into the \emph{space-based computing} approach. % whose benefits etc. are described
However, \ac{rest} architectures' properties have made them massively accepted mechanisms to integrate applications.
Consequently, we also take into consideration the latter mechanism in our solution design.


% Sintáctica vs semántica
Regarding the second goal, it is usually divided in \emph{syntactic interoperability} and \emph{semantic interoperability}.
On the one hand, syntactic interoperability is associated with the format of the data (i.e. its syntax and encoding) \citep{van_der_veer_achieving_2006}. % e.g. high level: HTML, XML, etc.
On the other hand, semantic interoperability is concerned with ensuring that the exchanged information has a precise meaning.
Its ultimate goal is to make the information ``\emph{understandable by any other application that was not initially developed for this purpose}'' \citep{_european_2004}.
Section~\ref{sec:soa_sw} explains a well-accepted mechanisms to achieve the latter goal: the \acl{sw}. % TODO qué es la SW??? un mecanismo, conjunto de estándares, ¿?

\input{\pathchaptwo/1_1_integration}
\input{\pathchaptwo/1_2_semantic_web}
\subsection{Conclusions} % or Discussion???


In Section~\ref{sec:remote_invocation} we have analyzed the current trends towards the remote invocation style.
Within them, we can see how \ac{wot} has gained much more attention both for its perceived simplicity and for its seamless integration with the web.
% o decir REST based solutions para no centrar mucho en IoT
Besides, the availability of tools and libraries for many mobile and embedded platforms ease its adoption.


However, as any remote invocation style it introduces coupling between senders and receivers.
% de johanson_extending_2004 he sacado esas características, no la particularización a WoT
These complicates the application changes both in the short-term and in the long-term \citep{johanson_extending_2004}.
In the short-term because nodes constantly join and leave the environment due to mobility or to failures.
In the long-term because the space is used to solve new problems and the obsolete devices are replaced with new technology.
Furthermore, the human interaction is eased by minimizing the configurations.
% quizá lo de johanson_extending_2004 se podría retomar en el capítulo cha:tsc
In Chapter~\ref{cha:tsc}, we deeply compare both \ac{rest} and \ac{ts} to see how they can benefit each other.


\bigskip


% comparación de todas ellas
% de entre los que ofrecen distribución: este no mola por esto, esto otro mola por esto otro, etc.

In Section~\ref{sec:remote_invocation} we have studied the \ac{ts} solutions for \acl{ubicomp}.
Within these solutions, \emph{The event heap} is not considered because of its centralized nature.
The rest present different ways to disseminate the tuples. %  L2imbo, LIME and TOTA

The replication which L$^2$imbo proposes may not be feasible for devices with constrained memory or storage capacity.
Particularly, in spaces populated by many devices like the Internet of Things a large amount of tuples can be generated.
% need to configure the network

TOTA is useful for spatially connected \emph{ad hoc} environments, but it does not seem to suit well on networks where all the nodes can reach each other.
In this case, all the tuples could be replicated on the rest of the nodes showing the same limitations as L$^2$imbo.
% TODO COMPROBAR LEYENDO CON ATENCION LO DE TOTA!!!
Furthermore, \emph{ad hoc} environments are not the focus of our solution since we assume the connectivity between all the devices in a space.

% TODO no darle tan duro que si no a ver como luego me comparo con ellos :-)
Finally, despite of their unrealistic assumptions LIME proposes a model where each tuple is responsible of its own part of the space.
They share spaces whenever they become available.


\begin{table}%[position specifier]
  \centering
  \begin{tabular}{ c | c c c c }%p{5cm}}
      ~ & The event heap & L2imbo & LIME & TOTA \\
      \hline
      Distributed & No & Yes & Yes & Yes \\
      Objective & - & Availability & Federation & Ad-hoc \\
      % Scalability?
      % Coger propiedades del libro de sistemas distribuidos?
  \end{tabular}
  \caption{Comparison of the most prominent \acl{ts} alternatives for \acl{ubicomp}.} % no columpiarme, o poner todas o ninguna?
  \label{tab:ubicomp_ts_comparison}
\end{table}


\medskip

In our model, we argue that in an \ac{ubicomp} environment each node should manage their own semantic information.
By delegating responsibility 
we naturally represent mobility scenarios where users carry their own profiles in their mobiles to new spaces 
and 
we acknowledge the fact that embedded devices directly control their own actuators and sensors.
Therefore, we propose to directly access to the source of the data avoiding the use of intermediaries whenever it is possible.
In this aspect, our solution has more in common with LIME.
Two major differences with LIME are the use of semantics and the natural integration with the web. % i.e. use of REST/HTTP
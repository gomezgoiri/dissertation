\section{Introduction}
\label{sec:soa_intro}

% Otra forma de ver la interop:
%  La European Telecommunication Standards Institute (ETSI) define cuatro capas:
%      technical interop: a nivel de comunicación (p.e. de acuerdo en las 7 capas de OSI)
%      syntactic interop
%      semantic interop
%      organizational interop


% no habría que definir de nuevo y de manera formal ubicomp o vale con la intro?
The IEEE \citep{_ieee_1990} defines \emph{interoperability} as ``\emph{the ability of two or more systems or components to exchange information and to use the information that has been exchanged}''.
The heterogeneity of technologies present in \ac{ubicomp} environments makes this a key property to consider.
The definition clearly distinguishes between two requirements: % o goals o incremental requirements
\textbf{(1) to exchange} information; and
\textbf{(2) to use} that information. % to understand others data (i.e. \emph{interoperation}).


% Buscar una referencia mejor: http://en.wikipedia.org/wiki/Interoperability
Exchanging information in distributed systems comprehends the communication between two systems.
% lo de ab-initio practicamente sólo lo he visto en la wikipedia
For the lower communication levels, we rely on standard and widely accepted communication protocols (i.e. interoperability \emph{ab-initio}). % e.g. HTTP
For higher-levels (i.e. application layer), this dissertation delves into the \emph{space-based computing} and \ac{rest} architectures.
Section~\ref{sec:soa_integration} categorizes both approaches together with other integration approaches.
Then, we individually present both integration styles in sections \ref{sec:rest} and \ref{sec:tuplespaces_eoa}.


% Sintáctica vs semántica
Regarding the second goal, it is usually divided in \emph{syntactic interoperability} and \emph{semantic interoperability}.
On the one hand, syntactic interoperability is associated with the format of the data (i.e. its syntax and encoding) \citep{van_der_veer_achieving_2006}. % e.g. high level: HTML, XML, etc.
On the other hand, semantic interoperability is concerned with ensuring that the exchanged information has a precise meaning.
Its ultimate goal is to make the information ``\emph{understandable by any other application that was not initially developed for this purpose}'' \citep{_european_2004}.
Section~\ref{sec:soa_sw} explains well-accepted mechanisms to achieve the latter goal: the \acl{sw}. % TODO qué es la SW??? un mecanismo, conjunto de estándares, ¿?

\subsection{Application Integration}
\label{sec:soa_integration}

% Definiciones de middleware en el libro de Coulouris:
%
% (pag 17):
% Middleware • The term middleware applies to a software layer that provides a
% programming abstraction as well as masking the heterogeneity of the underlying
% networks, hardware, operating systems and programming languages.
%
% (en otra parte):
% The task of middleware is to provide a higher-level
% programming abstraction for the development of distributed systems and, through
% layering, to abstract over heterogeneity in the underlying infrastructure to promote
% interoperability and portability.


The integration of two applications is driven by how they communicate.
To ease that communication the applications use middlewares.
A middleware is a software layer which provides a higher level of abstraction and masks the underlying heterogeneity.
\citet{coulouris_distributed_2012} define two communication styles on the upper layer of a middleware: % figura 4.1 modificada para incluir elementos
the remote invocation and the indirect communication (see Figure~\ref{fig:middleware_layers}).

\InsertFig{middleware_layers}{fig:middleware_layers}{Middleware layers}{Middleware layers according to \citet{coulouris_distributed_2012} classification.}{1}{}
% According to Coulouris et al. HTTP is An example of a request-reply protocol!


The \emph{remote invocation} involves the most common two-way exchange between senders and receivers in distributed systems.
Within the \emph{remote invocation} style, RESTful style \citep{fielding_architectural_2000} and the well known set of WS-* \citep{alonso_web_2010} standards probably represent the most popular substyles.
In Section~\ref{sec:remote_invocation}, we analyze their application to the \ac{ubicomp} field, particularly focusing on the \ac{iot}.


\medskip


The \emph{indirect communication} style comprehends all the techniques with no direct coupling between the sender and the receiver.
% Otra forma:
% In contrast, the \emph{indirect communication} comprehends decoupled communications between senders and receivers.
The group communication, publish-subscribe systems, message queues or shared memory approaches are examples of indirect communication.
These paradigms are characterized by two key properties \citep{gelernter_generative_1985,coulouris_distributed_2012}
\footnote{
  We use the terminology of the  \aclp{ts}' seminal paper \citep{gelernter_generative_1985}, but please note that:
  % nomenclature, naming or terminology?
  \begin{itemize}
    %\item Gelernter et al. stated that \emph{distributed sharing} was just a consequence of the these properties.
    \item The \emph{space uncoupling} property is referred as \emph{reference autonomy} by some authors \citep{fensel_triple-space_2004}.
	  % El primero fue un tal Angerer en el 2002, pero en un artículo en alemán.
	  % En el 2003 hay un artículo suyo en Internet, pero no sé si mencionarlo.
    \item These authors mention a third property confusingly called \emph{space autonomy} (or \emph{location autonomy}).
	  According to \citet{fensel_triple-space_2004} this autonomy is achieved because:
	  \begin{quote}
	    The processes can run in completely different computational environments as long as both can access the same space.
	  \end{quote}
  \end{itemize}
}
:

\begin{itemize}
 \item \emph{Space uncoupling}, which is achieved when the sender does not need to know the receiver or receivers and vice versa.
 \item \emph{Time uncoupling}, which happens when senders and receivers do not need to exist in the same time\footnote{
	  Although some authors \citep{fensel_triple-space_2004,krummenacher_www_2005} explain this property just in terms of communication asynchrony,
	  % mencionar a otros? o no porque simplemente siguen lo dicho por Fensel?
	  % a Bundler no lo cito, porque era una master thesis "sólo"
	  % en este caso cito a krummenacher porque en esa publicación lo define directamente como el no uso de comunicación sincrona
	  \citet{coulouris_distributed_2012} make a clear distinction between them.
	  In their words, a communication is asynchronous when \emph{a sender sends a message and then continues without blocking},
	  whereas time uncoupling adds an extra dimension: \emph{the sender and the receiver can have independent existences}.
	  }.
 
\end{itemize}

\acl{ts} computing, also called space-based computing, offers an improvement of the shared memory approach.
Whereas the shared memory works at byte-level and accessing to memory addresses,
the \acl{ts} works with semi-structured data which is accessed in an associative manner.
In other words, in \ac{ts} the participants read data specifying patterns of interest.
In Section~\ref{sec:tuplespaces_eoa}, we study the most relevant \acl{ts} solutions proposed for the \acl{ubicomp}.


\subsubsection{Remote invocation}
\label{sec:remote_invocation}

% TODO definir en algun lado que es REST y que es WS-*???

% De Guinard 2011 rest-vs-ws
% 
% WS-* These services declare their functionality and interfaces in a Web Ser-
% vices Description Language (WSDL) file. Client requests and service response
% objects are encapsulated using the Simple Object Access Protocol (SOAP) and
% transmitted over the network, usually using the HTTP protocol. Further WS-
% * standards define concepts such as addressing, security, discovery or service
% composition. Although WS-* was initially created to achieve interoperability of
% enterprise applications, work has been done to adapt it to the needs of resource-
% constrained devices [10, 13]. Furthermore, lighter forms of WS-* services, such
% as the Devices Profile for Web Services (DPWS)1, were proposed [6].
% 
% REST At the core of a RESTful architecture [3] lie resources that are uniquely
% identified through Uniform Resource Identifiers (URIs). The Web is an imple-
% mentation of RESTful principles – it uses URLs to identify resources and HTTP
% as their service interface. Resources can have several representation formats (e.g.,
% HTML, JSON2 ) negotiated at run time using HTTP content negotiation. In a
% typical REST request, the client discovers the URL of a service it wants to call by
% browsing or crawling its HTML representation. The client then sends an HTTP
% call to this URL with a given verb (GET, POST, PUT, etc.), a number of options
% (e.g., accepted format), and a payload in the negotiated format (e.g., XML or
% JSON). Several recent research projects implement RESTful Web services for
% smart things [2] within what has become to be known as the Web of Things [5].


% De Pautasso 2008
% 
% REpresentational State Transfer (REST) was originally introduc-
% ed as an architectural style for building large-scale distributed hy-
% permedia systems. This architectural style is a rather abstract entity,
% whose principles have been used to explain the excellent scalability
% of the HTTP 1.0 protocol and have also constrained the design of
% its following version, HTTP 1.1. Thus, the term REST very often
% is used in conjunction with HTTP.


As stated before, the RESTful style and the set of WS-* standards are probably the most common remote invocation substyles currently used in the Internet.
Their advantages and disadvantages were extensively discussed by \citet{pautasso_restful_2008}.

WS-* standards syntactically describe the services' functionality and interfaces using \ac{wsdl}.
The communication between providers and consumers is usually transmitted over the \ac{http} in messages encapsulated using \ac{soap}.
WS-* web services, also called ``Big Web services'', offer more features such as transactions, reliability or message-level security.
However, they also require further architectural decisions on different layers of the WS-* stack.
These decisions make the people perceive it as more complex than the RESTful style \citep{guinard_search_2011}.


Probably thanks to that perception, \ac{rest} has gained momentum during the last decade. %  simpler alternative for services
% la explicación de los principios los he sacado sobre todo de pautasso_restful_2008, pero he intentado no plagiar las frases.
The \acl{rest} is defined by four principles \citep{fielding_architectural_2000}:
\begin{enumerate}
  \item The resources are identified with \acp{uri}.
  \item They are manipulated through a uniform interface.
	  This interface offers creation, reading, updating and deletion operations.
  \item The resources are decoupled from their representation so the content can be accessed in different formats.
  \item The interactions are stateful and hypertext-driven.
	This means that the state transitions are guided by the actions identified in the hypermedia by the server.
	% De Wikipedia: This means that clients make state transitions only through actions that are dynamically identified within hypermedia by the server.
\end{enumerate}

These principles are implemented by the Web,
so \ac{rest}  is often used in conjunction with well-known web standards such as \ac{http}, \ac{uri} or \ac{mime}.
This standards already have many libraries and tools available, which eases the adoption of this style by developers.
Besides, REST naturally integrates with the \ac{www} because it uses the Web as an application architecture instead of as a transport layer as WS-* services do.
% ojo a la frase: "se integra con la web porque usa la web como app architecture"

\bigskip

The use of both substyles in resource-constrained devices is represented by
\ac{dpws} \citep{moritz_devices_2010}, based on WS-*,
and the \ac{wot} \citep{guinard_internet_2011} \footnote{Due to its importance for this disseration, \ac{wot} is analyzed in Section~\ref{sec:soa_ubicomp}.}. % o mejor poner su tesis?


% TODO reescribir esto para que no suene tan positivo y sobrado
\ac{dpws} defines a subset of WS-* to make it suitable for resource-constrained devices.
%\ac{dpws} is claimed to be used in industrial environments.
Its most remarkable features are: decentralized multicast-based discovery, secure message transmission, subscription and event notifications.
\citet{moritz_devices_2010} compared \ac{dpws} with the \ac{rest} approach coming to the conclusion that \ac{dpws} can be restricted to be fully compatible with the RESTful style.
Even in that case, \ac{dpws} will still cover some missing features of the RESTful approach such as eventing and discovery \citep{moritz_devices_2010}.
% no menciono que sean más pesados para cacharros porque no he encontrado un sitio en el que defiendan que \ac{dpws} lo es (si que hay alguno que comparan REST y SOA, pero no DPWS)
\subsubsection{\acl{ts}}
\label{sec:tuplespaces_eoa}

% Ojo cuidado porque según esto mi middleware va a dejar de ser time uncoupled! (mirar Figure 6.1 para mayor concrección)
% En esa misma sección aclara que algunas implementaciones no son time-uncoupled.
% por otro lado, asíncronismo != time uncoupling

\acf{ts} has its roots in the Linda parallel programming language \citep{gelernter_generative_1985}.
In this communication model different processes read and write pieces of information so-called tuples in a common space.
Tuples are composed by one or more typed data fields (e.g. $<"aitor",1984>$ or $<3,7,21.0>$).
The tuples are accessed associatively using a template.
A template provides either a value or type for different fields (e.g. $<String,1984>$ or $<Integer, Integer, Float>$).
The operations over the space are defined by different primitives.
Although the primitives may change from implementation to implementation, the most common ones allow to:

\begin{itemize}
  \item \emph{Access the tuples non-destructively}, using a primitive which is usually called \emph{read} or \emph{rd}.
	This primitive returns a tuple from the space which matches the given template without changing the space.
  \item \emph{Access the tuples destructively}, using a primitive which is usually called \emph{take} or \emph{in}.
	This primitive extracts a tuple which matches the given template from the space.
  \item \emph{Write a new tuple into the space}. This primitive is usually called \emph{write} or \emph{out}.
\end{itemize}


\InsertFig{venn-sec1_2}{fig:venn_tuplespaces_ubicomp}{
  \aclp{ts} solutions for Ubicomp.
}{
  Scope of this subsection.
}{0.6}{}

\bigskip

% Ignorar esta sección? Es realmente necesaria cuando ya he puesto la comparativa de TSC?
So far, \acl{ts} has been adapted to \acl{ubicomp} by different authors.

% TODO leer artículos que me faltan y corregir
The \emph{event heap} \citep{johanson_extending_2004} is a system used for a specific \ac{ubicomp} sub-domain: interactive workspaces.
In this scenario there are rooms with different devices deployed and where mobile devices can enter.
Each room has its own space where the devices exchanging tuples to cooperate.
%For example, a video can be presented in a display and through a remote controller, the user can place the tuple for pausing it.
%The display consumes these kind of tuples, so it pauses the video whenever somebody places that tuple on the space.
% This system is simple to implement but is limited since it centralizes the space in a single machine per room.
This work merely identifies the requirements of these environments and the properties which solve them.
Then, it discusses how these properties can be satisfied using \ac{ts} or some extensions.
Finally they compare its implementation both with other \acp{ts} implementations or other coordination infrastructures. % e.g. RMI, MOM, Pub/sub

Although the analysis presented by \citeauthor{johanson_extending_2004} identifies the key requirements for \ac{ts} in \ac{ubicomp},
the solution they propose is purely centralized.
We opted for a distributed solution at the expense of limiting some of the features.
% indicar cuales son o pasar?
% usar sus características a la hora de valorar nuestra solución TSC en futuras secciones?


\emph{L$^2$imbo} \citep{davies_l2imbo:_1998,friday_experiences_1999} replicates the tuples to avoid a single point of failure.
Each node joined to a space uses an IP multicast address to exchange messages with other nodes in that space.
Writing into a space involves sending a multicast message to inform to the rest of the nodes of the tuple written.
Reading operation usually requires local reading.
Destructive reading of the tuple is more complex as it requires a global withdrawal.
In \emph{L$^2$imbo} only the owner of a tuple can remove it from the space.
The ownership of a tuple initially belongs to its creator, but can be transferred. % TODO ver qué añadió Friday


 % meter mas citas de LIME que estas 2?
\emph{LIME} (Linda in a Mobile Environment) \citep{picco_lime:_1999} is a TS solution for mobile systems.
In LIME each mobile device has its own space where it generally writes its tuples.
This space is shared with other devices creating federated spaces, i.e. the aggregation of different shared spaces.
In this way, each mobile can access to tuples in other mobiles whenever they become available.
They also proposed a new writing primitive to insert tuples in remote spaces.
% ejemplo de service discovery?

This is difficult to implement and \citet{coulouris_distributed_2012} complain about the unrealistic assumptions they make to simplify the problem. % cita al libro
These assumptions are the uniform multicast connectivity between devices whose tuple spaces are aggregated and the serialized and ordered connections and disconnections.
In any case, LIME was adapted to several platforms to run both in embedded and mobile devices \citep{murphy_transiently_2006}.


In the \emph{TOTA} (Tuples On The Air) Project \citep{mamei_programming_2009} tuples are disseminated to different devices.
To that end, each tuple has 3 fields:
1) the content of the tuple,
2) a rule which defines how it should be propagated,
3) a rule to define its maintenance.
For instance, they consider a museum where a visitor writes a query tuple describing a piece of art he wants to see.
The propagation rule defines that it should be propagated to all nodes in the vicinity, increasing the distance by one each time.
The tuple is configured to be deleted after a time-to-live period using its maintenance rule.
When the it reaches the room where the piece of art is located, the piece of art will write a response tuple.
This response tuple jumps from a device to another until it reaches the device which queried for it.


% TODO Analizar otras como TOTAM o CRIME
\subsection{\acl{rest}}
\label{sec:rest}

\acf{rest} is a network-based architectural style proposed by \citet{fielding_architectural_2000}.
% descripción de propiedades de REST: hipermedia distribuído
% que es hipertexto? http://roy.gbiv.com/untangled/2008/rest-apis-must-be-hypertext-driven#comment-718
It aims to cover certain properties, explained in Section~\ref{sec:network_properties}. % Mencionar que scalability y simplicity destacan sobre el resto?
% Estilos de los que se deriva: ¿?
To achieve these properties, \ac{rest} establishes the following constraints from other network-based architectural styles:
% explicarlo o es demasiado obvio?
\begin{description}
 \item[\acf{restcs}.] Providing an \ac{api} to the clients, they are isolated from back-end implementation details.
		       % esto ayuda a: , scalability, evolvability (apps independientes pueden evolucionar mejor)
 \item[\acf{rests}.] The state is fully stored in the client and therefore each request has all the information needed to process it.
 \item[\acf{restcache}.] When added to the \ac{restcs} constraint, this style replicates content obtained from a server in the client.
 \item[\acf{restu}.] It is the key constraint which distinguishes \ac{rest} from other architectural styles.
                      This constraint is composed by the following ones:
    \begin{description}
	% explicados en sección 5.2 de Fielding, resumir?
	\item[\acf{restid}.] Resources are the conceptual targets of hypertext references.
			      Their identification offers a generic interface to access and change the values of a resource.
	\item[\acf{restrep}.] Representations are composed by a sequence of bytes and the metadata to describe those bytes.
	\item[\acf{restdesc}.] The client and server have to agree on standard methods and media types. % linkar al tipo de la Web esa
				% Explicación Fielding:
				% interaction is stateless between requests
				% standard methods and media types are used to indicate semantics and exchange information
				% responses explicitly indicate cacheability.
		                %\ac{http}'s content negotiation for instance, allows to reach an agreement on the media types.
	                        Beyond that point, each request or response should contain all the needed data to process it.
	                        Therefore, in \citeauthor{fielding_architectural_2000}'s words,
	                        the type should be registered,
	                        the registry should point to a specification and 
	                        the specification should explain how to process data according to its intent \citep{wahbe_self-descriptive_2010}. % no necesariamente tiene que ser estándar
	\item[\acf{resthateoas}.] This is a controversial constraint because most of the self-proclaimed ``\ac{rest}'' \acp{api} fail to follow it \citep{moore_hypermedia_2010,house_how_2012}.
	                           It states that no out-of-the-band information should guide the interaction with an \ac{api}.
	                           Instead, the hypertext should guide it.
	                           In other words, the client must know just an initial URL and the application's media types.
	                           From that point, it should select the alternatives proposed by the server to change to the next application state \citep{fielding_rest_2008}.
	                           % propuestas: respresentaciones (links) y la manipulación implicita de las mismas (CRUD)
    \end{description}
 \item[\acf{restl}.] Each layer provides services to the top layer. % e.g. TCP/IP
 \item[\acf{restcod}.] It is the only optional \emph{constraint} of \ac{rest}.
		       It occurs when the client downloads from the server the \emph{know-how} needed to process the set of resources it already has.
\end{description}


% TODO TODO TODO
% Explicar aquí cosas de nomenclatura:
%     REST o hypermedia API,
%     REST-like (sin HATEOAS)
%     Resource Oriented Architecture o ROA como termino que engloba a ambos
% Hablar de orthodoxly and heterodoxly!


\subsubsection{\ac{rest} vs WS-* services}
\label{sec:protocols}

% TODO mirar a Guinard como se refiere a WS-*.
%   ¿Sólo como WS-*? ¿Siempre como WS-* standards? ¿Alguna vez como WS-* web services?
% TODO citar esto mejor!
WS-*, also called ``Big Web services'', together with \acs{rest}ful architectures are probably the most common remote invocation substyles currently used in the Internet.
They syntactically describe the services' functionality and interfaces using \ac{wsdl}.
The communication is carried out on top of \ac{http} in messages encapsulated using \ac{soap}.
WS-* standards offer more features than the \ac{rest} such as transactions, reliability or message-level security.
% TODO describir los estándares mejor
% decir o poner algo de UDDI aquí?


WS-* has been adapted to the needs of resource-constrained devices in the \ac{dpws} specification\footnote{\url{http://docs.oasis-open.org/ws-dd/ns/dpws/2009/01}}. % TODO poner como cita mejor?
The specification defines a minimal set of implementation constraints.
\ac{dpws}'s most remarkable features are: decentralized multicast-based discovery, secure message transmission, subscription and event notifications \citep{moritz_devices_2010}.


% la web implementa los principios de REST y WS-* no
Note that although the web implements \ac{rest}'s principles and WS-* can contradict them
\footnote{Interestingly, \citet{moritz_devices_2010} came to the conclusion that \ac{dpws} can be restricted to be fully compatible with the \ac{rest}ful style and still cover some missing features (e.g. eventing and discovery).},
WS-* stands for \emph{web services}. % citas a What's wrong with the web y demás papers que explicaban esto
% a WS-* se le llama web  porque usa estándares de esta
The reason for this is that it uses web standards. % (e.g. \ac{http}, \ac{uri} or \ac{mime}) <-- o los que fuesen
% esto es porque WS-* usa protocolos con propositos distintos a los que fueron diseñados
However, it uses some of them in  ways they were not designed for.
% ejemplo de HTTP
The most paradoxical case is how it uses \ac{http} as a transport layer instead of as an application layer protocol.
% por eso no consigue las propiedades de REST: simplicidad y escalabilidad
This prevents the resulting architectures from achieving some of \ac{http}'s desirable properties such as scalability or visibility.
% se podría decir que las sacrifica en pro de features adicionales
In other words, it sacrifices some of the web's properties on behalf of the additional features it provides.


% posiblemente esa simplicidad se ve afectada negativamente porque los usuarios necesitan decidir
One of the sacrificed properties is the simplicity.
The complexity negatively affects to
(1) the ability of limited computing platforms to adopt the WS-* stack; and % explicar esto más? => cuanto menos requieras, más fácil que alguien haya implementado HTTP y eso
(2) the developers.
% y por eso o no, WS-* está perdiendo peso respecto a REST
Specifically, developers need to take further architectural decisions on different layers of the WS-* stack.
These decisions make the people perceive it as more complex than the \ac{rest}ful style \citep{guinard_search_2011}.
% This acceptance also results in a massive availability of libraries in most of the computing platforms.


% Nosotros elegimos el segundo por sus propiedades, pero también por su relevancia actual.
However, independently of further considerations
\footnote{For an extensive analysis on the advantages and disadvantages of \ac{rest} and WS-*, we refer the reader to \citep{pautasso_restful_2008} and \citep{guinard_search_2011}.}
, there is a practical motivation behind the rationale of focusing this dissertation on the first one: its massive acceptance. % TODO TODO TODO mirar en tesis de Guinard a ver si puedo sacar una molona sobre eso
% Al final, si queremos facilitar que lo nuestro se integre con otros elementos, que mejor que usar su mecanismo de integración.
In the end, if the middleware presented in the dissertation aims to reuse other applications' data, what is better than seamlessly integrating with most of them?
% Es difícil predecir cual ganará o cual será más popular, pero la web lleva ahí desde hace 20 años.
Obviously, this acceptance is subject to changes in the future, but the life of the web backs \ac{rest} as a long-term choice.
% Cita de Fielding:
Or in \citeauthor{fielding_architectural_2000}'s words,
\quote{REST is intended for long-lived network-based applications that span multiple organizations} \cite[comment 21]{fielding_rest_2008}.



\subsubsection{Suitable Protocols for \ac{rest}}
\label{sec:protocols}
% hablar un poco de HTTP y CoAP y decir por qué no hablamos más a menudo de CoAP

% sólo por el correcto uso de ciertos protocolos se puede conseguir todo menos HATEOAS => citar el vídeo aquel
Although an \ac{api} does not adhere to the \ac{rest} style just because it uses certain protocols,
the correct use of some of them can help to achieve most of its principles \citep{moore_hypermedia_2010}.
Historically, \acf{http} has been considered a suitable protocol in that regard.
\ac{http} is a simple protocol whose adoption by computing platforms is massive.


However, in the last few years \acf{coap} has emerged as a specialized web transfer protocol for resource constrained devices. % has emerged, has arisen?
Some noteworthy features of \ac{coap} are
(1) the reduced message size,
(2) the use of UDP as a transport layer (with the possibility of using \emph{multicast} communication),
(3) similarity with \ac{http} (both to reuse its properties and to ease cross-protocol proxying), and
(4) a resource discovery mechanism. % TODO cita a CORE
% mencionar seguridad?


One could argue that to implement a lightweight \ac{tsc} solution, \ac{coap} should be used as a baseline.
However, we have chosen to work with \ac{http} for the following reasons:
\begin{itemize}
  \item Direct interoperation with other web-solutions.
        Most of the Internet-based \ac{api}s use \ac{http}.
        Directly using \ac{http} we can avoid proxies.
        Proxies may introduce latency in the response time degrading network performance. % siempre degradará la eficiencia algo, pero habría que verlo y testearlo, la verdad...
        % TODO, será cierto? o tardará menos en conjunto?
  \item \ac{rdf}-based media-formats can be rather verbosed.
	This contrasts with \ac{coap}'s message size limitations.
	Dealing with this limitations was not one main goals of the thesis.
	However, we have considered them at some points of the dissertation to avoid unrealistic assumptions.
  \item \ac{coap} is an ongoing standard.
        Therefore, its definition is currently changing in each draft version.
        A practical limitation of this is that at the moment there are few libraries and tools to work with \ac{coap}.
        This limits the range of platforms which could adopt any proposed solution.
\end{itemize}


Nevertheless, due to \ac{http}'s similarities with \ac{coap}, the future adoption of the latter should be relatively straightforward.

% Ideas no usadas:
% REST no dice nada, pero de facto el protocolo es HTTP
% durante el desarrollo de esta tesis en resource constrained ha surgido CoAP
%   muchas de las cosas aquí comentadas son iguales o mejor sobre CoAP
%   de todas formas, ha habido algo que ha guiado nuestro diseño: librerias existentes => afecta a plataformas
%         si, pero para eso está lo de layered approach! => motivos prácticos
\subsection{\acl{ts}}
\label{sec:tuplespaces_eoa}

% Ojo cuidado porque según esto mi middleware va a dejar de ser time uncoupled! (mirar Figure 6.1 para mayor concrección)
% En esa misma sección aclara que algunas implementaciones no son time-uncoupled.
% por otro lado, asíncronismo != time uncoupling

\acf{ts} computing, also called space-based computing, offers an improvement over traditional distributed shared memory approaches.
% Colouris se refiere a ellos como distributed shared memory antes de presentar TS
Whereas they work at byte-level and accessing to memory addresses,
the \acl{ts} works with semi-structured data which is accessed in an associative manner.
In other words, in \ac{ts} the participants read data specifying patterns of interest.


\ac{ts} has its roots in the Linda parallel programming language \citep{gelernter_generative_1985}.
In this communication model different processes read and write pieces of information so-called tuples in a common space.
Tuples are composed by one or more typed data fields (e.g. $<"aitor",1984>$ or $<3,7,21.0>$).
The tuples are accessed associatively using a template.
A template provides either a value or type for different fields (e.g. $<String,1984>$ or $<Integer, Integer, Float>$).
The operations over the space are defined by different primitives.
Although the primitives may change from implementation to implementation, the most common ones allow to:

\begin{itemize}
  \item \emph{Access the tuples non-destructively}, using a primitive which is usually called \emph{read} or \emph{rd}.
	This primitive returns a tuple from the space which matches the given template without changing the space.
  \item \emph{Access the tuples destructively}, using a primitive which is usually called \emph{take} or \emph{in}.
	This primitive extracts a tuple which matches the given template from the space.
  \item \emph{Write a new tuple into the space}. This primitive is usually called \emph{write} or \emph{out}.
\end{itemize}
\subsection{The \acl{sw}}
\label{sec:soa_sw}

A problem of the initial view of the Web was that it was human-centered.
Regardless of whether the contents were machine processable, a human needed to interpret them to give them a meaning.
% mirar a ver cuando se empezó a usar \ac{sw} de verdad, porque en el artículo de 2001 dan a enterder que se llevaba un tiempo usando
In the early 2000s, \citet{berners-lee_semantic_2001} proposed the use of what they called the \acf{sw} to solve that problem.
The \ac{sw} was conceived as an extension of the ordinary web which would bring structure to its content.
As the ordinary web, the \ac{sw} benefits from the universality the hypertext provides by linking \emph{anything with anything}.
Besides, like the Internet, the \ac{sw} is intended to be as decentralized as possible.

The \acl{sw} and its key features are defined by the \emph{World Wide Web Consortium} \citep{semanticWeb-FAQ} as:
\begin{quote}
The vision of the \acl{sw} is to extend principles of the Web from documents to data.
Data should be accessed using the general Web architecture using, e.g., URI-s;
data should be \emph{related to one another} just as documents (or portions of documents) are already.
This also means creation of a common framework that allows data to be \emph{shared and reused} across application, enterprise, and community boundaries,
to be \emph{processed automatically} by tools as well as manually, including revealing possible \emph{new relationships} among pieces of data.
\end{quote}

Meaning is expressed in the \acl{sw} as sets of triples.
Each triple is composed by a subject, a predicate and an object like in a normal sentence (see Figure~\ref{fig:triples_example}).
Therefore, in  \citeauthor{berners-lee_semantic_2001} words, \emph{a document can assert that things (people, web pages or whatever) have properties (such as "is a sister of," "is the author of") with certain values (another person, another Web page)}.
A key difference with a normal sentence is that each concept is unambiguously defined by an URI.
These URIs form links between different triples as shown by the example of the Figure~\ref{fig:triples_example}.
% This set of triples can be expressed in multiple formats such as RDF, Turtle, Notation3 or NTriples. % TODO citas


\InsertFig{triples_example}{fig:triples_example}{
  Sample triples using different ontologies.
}{
  All the triples are represented graphically and some of them also textually.
  They describe academic and personal details about the author of this thesis.
  The knowledge is expressed using four different ontologies: FOAF, DC, SWRC and CiTE. % citar?
  Note that we use aliases, known as prefixes in most Semantic formats, to shorten some URIs and enhance the clarity of the Figure.
}{1}{}


A problem with the information described so far is that two different databases may use different URIs to express the same concept.
To overcome this the \ac{sw} offers collections of information called ontologies.
An ontology is a document which expresses the relations between terms commonly using a taxonomy and a set of rules to infer content.
The taxonomy defines the classes of a given domain and how they relate with each other.

Of course, different content providers could provide similar data described according to different ontologies.
Fortunately, ontologies can be easily mapped by providing equivalence relations within them.
In the same way, an ontology can be extended to adapt it to different application domains.
In any case, the reuse of the same models is beneficial and is therefore promoted through standardization.

% Remarkably, the use of the \ac{sw} has been promoted in the last years with the Linked Open Data (LOD) initiative.
% The LOD are datasets which follow a series of principles on how to open and publish data.
% The ultimate goal of the LOD is to publish linked terms using full semantics.
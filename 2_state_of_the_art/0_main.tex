% ----------------------------------------------------------------------

\begin{savequote}[50mm]
I was born not knowing and have had only a little time to change that here and there.
\qauthor{Richard P. Feynman}
\end{savequote}


\chapter{State of the Art}
\label{cha:stateoftheart}
\newcommand{\pathchaptwo}{2_state_of_the_art}

% the code below specifies where the figures are stored
\ifpdf
    \graphicspath{{\pathchaptwo/figures/PNG/}{\pathchaptwo/figures/PDF/}{\pathchaptwo/figures/}}
\else
    \graphicspath{{\pathchaptwo/figures/EPS/}{\pathchaptwo/figures/}}
\fi


%------------------------------------------------------------------------- 

% TODO IG yo creo que deberias orientarlo a un state of the art, ahora suena a related work. Tambien creo que se deberia introducir un poco mas, del estilo de: este capitulo presenta el related work para dar X y Y.
% TODO uniformizar labels!
This chapter presents the related work.
First, Section~\ref{sec:soa_intro} introduces, categorizes, and describes the related research topics. % o llamarle research areas, important concepts?
% nombrarlos? \ac{rest}, \ac{ts} and the \ac{sw}
% "this survey is not intended to be exhaustive (nor exhausting)"
Then, Section~\ref{sec:soa_ubicomp} explores their impact on the \ac{ubicomp} field. % rather than present an exhaustive (and exhausting) survey on these topics.
To this end, we identify the relevant works remarking their interesting characteristics and how they relate to this dissertation.
Finally, Section~\ref{sec:soa_tsc} analyzes and compares other semantic \ac{ts} middlewares independently of their application domain.


\InsertFig{venn-summary}{fig:venn_ubicomp_tuple_semantic}{
  Background areas of the thesis, which focuses on their intersection.
}{
  This thesis focuses on the intersection of the three areas represented in the Figure.
  Therefore, this chapter will be driven by their analysis.
}{0.6}{}


% TODO valorar meter en otra sección descripción de IoT y UbiComp y sus características?
\section{Introduction}
\label{sec:soa_intro}

% Otra forma de ver la interop:
%  La European Telecommunication Standards Institute (ETSI) define cuatro capas:
%      technical interop: a nivel de comunicación (p.e. de acuerdo en las 7 capas de OSI)
%      syntactic interop
%      semantic interop
%      organizational interop


% no habría que definir de nuevo y de manera formal ubicomp o vale con la intro?
The IEEE \citep{_ieee_1990} defines \emph{interoperability} as ``\emph{the ability of two or more systems or components to exchange information and to use the information that has been exchanged}''.
The heterogeneity of technologies present in \ac{ubicomp} environments makes this a key property to consider.
The definition clearly distinguishes between two requirements: % o goals o incremental requirements
(1) to exchange information; and
(2) to use that information. % to understand others data (i.e. \emph{interoperation}).


% Buscar una referencia mejor: http://en.wikipedia.org/wiki/Interoperability
Exchanging information in distributed systems comprehends the communication between two systems.
% lo de ab-initio practicamente sólo lo he visto en la wikipedia
For the lower communication levels, we opt for interoperability \emph{ab-initio} relying on standard and widely accepted communication protocols. % e.g. HTTP
For a higher-level (i.e. application layer), Section~\ref{sec:soa_integration} categorizes different integration approaches.
This dissertation aims to delve into the \emph{space-based computing} approach. % whose benefits etc. are described
However, \ac{rest} architectures' properties have made them massively accepted mechanisms to integrate applications.
Consequently, we also take into consideration the latter mechanism in our solution design.


% Sintáctica vs semántica
Regarding the second goal, it is usually divided in \emph{syntactic interoperability} and \emph{semantic interoperability}.
On the one hand, syntactic interoperability is associated with the format of the data (i.e. its syntax and encoding) \citep{van_der_veer_achieving_2006}. % e.g. high level: HTML, XML, etc.
On the other hand, semantic interoperability is concerned with ensuring that the exchanged information has a precise meaning.
Its ultimate goal is to make the information ``\emph{understandable by any other application that was not initially developed for this purpose}'' \citep{_european_2004}.
Section~\ref{sec:soa_sw} explains a well-accepted mechanisms to achieve the latter goal: the \acl{sw}. % TODO qué es la SW??? un mecanismo, conjunto de estándares, ¿?

\input{\pathchaptwo/1_1_integration}
\input{\pathchaptwo/1_2_semantic_web}
\subsection{Conclusions} % or Discussion???


In Section~\ref{sec:remote_invocation} we have analyzed the current trends towards the remote invocation style.
Within them, we can see how \ac{wot} has gained much more attention both for its perceived simplicity and for its seamless integration with the web.
% o decir REST based solutions para no centrar mucho en IoT
Besides, the availability of tools and libraries for many mobile and embedded platforms ease its adoption.


However, as any remote invocation style it introduces coupling between senders and receivers.
% de johanson_extending_2004 he sacado esas características, no la particularización a WoT
These complicates the application changes both in the short-term and in the long-term \citep{johanson_extending_2004}.
In the short-term because nodes constantly join and leave the environment due to mobility or to failures.
In the long-term because the space is used to solve new problems and the obsolete devices are replaced with new technology.
Furthermore, the human interaction is eased by minimizing the configurations.
% quizá lo de johanson_extending_2004 se podría retomar en el capítulo cha:tsc
In Chapter~\ref{cha:tsc}, we deeply compare both \ac{rest} and \ac{ts} to see how they can benefit each other.


\bigskip


% comparación de todas ellas
% de entre los que ofrecen distribución: este no mola por esto, esto otro mola por esto otro, etc.

In Section~\ref{sec:remote_invocation} we have studied the \ac{ts} solutions for \acl{ubicomp}.
Within these solutions, \emph{The event heap} is not considered because of its centralized nature.
The rest present different ways to disseminate the tuples. %  L2imbo, LIME and TOTA

The replication which L$^2$imbo proposes may not be feasible for devices with constrained memory or storage capacity.
Particularly, in spaces populated by many devices like the Internet of Things a large amount of tuples can be generated.
% need to configure the network

TOTA is useful for spatially connected \emph{ad hoc} environments, but it does not seem to suit well on networks where all the nodes can reach each other.
In this case, all the tuples could be replicated on the rest of the nodes showing the same limitations as L$^2$imbo.
% TODO COMPROBAR LEYENDO CON ATENCION LO DE TOTA!!!
Furthermore, \emph{ad hoc} environments are not the focus of our solution since we assume the connectivity between all the devices in a space.

% TODO no darle tan duro que si no a ver como luego me comparo con ellos :-)
Finally, despite of their unrealistic assumptions LIME proposes a model where each tuple is responsible of its own part of the space.
They share spaces whenever they become available.


\begin{table}%[position specifier]
  \centering
  \begin{tabular}{ c | c c c c }%p{5cm}}
      ~ & The event heap & L2imbo & LIME & TOTA \\
      \hline
      Distributed & No & Yes & Yes & Yes \\
      Objective & - & Availability & Federation & Ad-hoc \\
      % Scalability?
      % Coger propiedades del libro de sistemas distribuidos?
  \end{tabular}
  \caption{Comparison of the most prominent \acl{ts} alternatives for \acl{ubicomp}.} % no columpiarme, o poner todas o ninguna?
  \label{tab:ubicomp_ts_comparison}
\end{table}


\medskip

In our model, we argue that in an \ac{ubicomp} environment each node should manage their own semantic information.
By delegating responsibility 
we naturally represent mobility scenarios where users carry their own profiles in their mobiles to new spaces 
and 
we acknowledge the fact that embedded devices directly control their own actuators and sensors.
Therefore, we propose to directly access to the source of the data avoiding the use of intermediaries whenever it is possible.
In this aspect, our solution has more in common with LIME.
Two major differences with LIME are the use of semantics and the natural integration with the web. % i.e. use of REST/HTTP
\section{Related \acl{ubicomp} Works}
\label{sec:soa_ubicomp}

The previous section presented the main research topics related to this dissertation.
% The middleware presented in this thesis adopts some relevant characteristics from them.
From their intersection with the \ac{ubicomp} field emerges relevant related work.
Consequently, this section describes these works and presents their differences and similarities with our work.
% hablar de comparacion de ventajas en inconvenientes? hablar de contribuciones a dichas areas?



% TODO definir en algún puto lado a qué nos referimos con Ubicomp
%      para mostrar porque no incluimos otros trabajos


% Ignorar esta sección? Es realmente necesaria cuando ya he puesto la comparativa de TSC?
\subsection{\acl{ts} for \ac{ubicomp}}

So far, \acl{ts} has been adapted to \acl{ubicomp} by different authors.


\InsertFig{venn-sec1_2}{fig:venn_tuplespaces_ubicomp}{
  \aclp{ts} solutions for Ubicomp.
}{
  Scope of this subsection.
}{0.6}{}


% TODO leer artículos que me faltan y corregir
The \emph{event heap} \citep{johanson_extending_2004} is a system used for a specific \ac{ubicomp} sub-domain: interactive workspaces.
In this scenario there are rooms with different devices deployed and where mobile devices can enter.
Each room has its own space where the devices exchanging tuples to cooperate.
%For example, a video can be presented in a display and through a remote controller, the user can place the tuple for pausing it.
%The display consumes these kind of tuples, so it pauses the video whenever somebody places that tuple on the space.
% This system is simple to implement but is limited since it centralizes the space in a single machine per room.
This work merely identifies the requirements of these environments and the properties which solve them.
Then, it discusses how these properties can be satisfied using \ac{ts} or some extensions.
Finally they compare its implementation both with other \acp{ts} implementations or other coordination infrastructures. % e.g. RMI, MOM, Pub/sub


\emph{L$^2$imbo} \citep{davies_l2imbo:_1998,friday_experiences_1999} replicates the tuples to avoid a single point of failure.
Each node joined to a space uses an IP multicast address to exchange messages with other nodes in that space.
Writing into a space involves sending a multicast message to inform to the rest of the nodes of the tuple written.
Reading operation usually requires local reading.
Destructive reading of the tuple is more complex as it requires a global withdrawal.
In \emph{L$^2$imbo} only the owner of a tuple can remove it from the space.
The ownership of a tuple initially belongs to its creator, but can be transferred. % TODO ver qué añadió Friday


% TODO meter mas citas de LIME que estas 2?
\emph{LIME} (Linda in a Mobile Environment) \citep{picco_lime:_1999} is a \ac{ts} solution for mobile systems.
In LIME each mobile device has its own space where it generally writes its tuples.
This space is shared with other devices creating federated spaces, i.e. the aggregation of different shared spaces.
In this way, each mobile can access to tuples in other mobiles whenever they become available.
They also proposed a new writing primitive to insert tuples in remote spaces.
% ejemplo de service discovery?

This is difficult to implement and \citet{coulouris_distributed_2012} complain about the unrealistic assumptions they make to simplify the problem. % cita al libro
These assumptions are the uniform multicast connectivity between devices whose tuple spaces are aggregated and the serialized and ordered connections and disconnections.
In any case, LIME was adapted to several platforms to run both in embedded and mobile devices \citep{murphy_transiently_2006}.


In the \emph{TOTA} (Tuples On The Air) Project \citep{mamei_programming_2009} tuples are disseminated to different devices.
To that end, each tuple has 3 fields:
1) the content of the tuple,
2) a rule which defines how it should be propagated,
3) a rule to define its maintenance.
For instance, they consider a museum where a visitor writes a query tuple describing a piece of art he wants to see.
The propagation rule defines that it should be propagated to all nodes in the vicinity, increasing the distance by one each time.
The tuple is configured to be deleted after a time-to-live period using its maintenance rule.
When the it reaches the room where the piece of art is located, the piece of art will write a response tuple.
This response tuple jumps from a device to another until it reaches the device which queried for it.


% TODO Añadir tablita a modo de resumen?
% \begin{table}
%   \centering
%   \begin{tabular}{ c | c c c c }%p{5cm}}
%       ~ & The event heap & L2imbo & LIME & TOTA \\
%       \hline
%       Distributed & No & Yes & Yes & Yes \\
%       Objective & - & Availability & Federation & Ad-hoc \\
%       % Scalability?
%       % Coger propiedades del libro de sistemas distribuidos?
%   \end{tabular}
%   \caption{Comparison of the most prominent \acl{ts} alternatives for \acl{ubicomp}.} % no columpiarme, o poner todas o ninguna?
%   \label{tab:ubicomp_ts_comparison}
% \end{table}


% TODO Analizar otras como TOTAM o CRIME?


\subsubsection{Discussion} % TODO el índice quedará raro

% vemos el espacio necesario para:
The space can be used to
%    consultar que pasa en un entorno
(1) coordinate with other devices by writing and extracting content and
%    escribir y extraer para coordinarse con otros
(2) check what happens in the environment by reading.
% sin embargo, hay veces donde esa escritura y extraccion es mejor delegarla a los responsables de ella
The solutions presented integrate both uses in the same space, which can be distributed or not.
However, we argue that these uses face different needs and, therefore, should be treated separately.


% Teorema de CAP
The first usage demands consistency to avoid the unexpected consequences of two devices extracting the same tuple.
For this usage, we propose a typical coordination space which will be accessed through a \emph{resource oriented} \ac{http} \ac{api}.
% the space itself: el systema es distribuído (CS), otra cosa es que el espacio lo sea
Whether the space itself is distributed and how, is out of the scope of this dissertation and left up to the implementor.
%However, following the CAP theorem, we believe that the availability is more necessary than the partition tolerance.
%Otherwise the nodes using the space, will not be able to coordinate. % disponibilidad del espacio o de los nodos que pertenecen al mismo?


% 2do problema:
%    queriamos hacer a los dispositivos todo lo dependientes que fuese posible del espacio
For the second usage type, we pursued to integrate as many data sources as possible.
This integration demands to acknowledge their independence and limit the degree of collaboration between them.
%The rationale behind this decision is that we believe that some content may be better managed by the device which creates them.
% ejemplo
%For instance, measures of a sensor or a user profile in a smartphone.
Therefore, for the second space we propose a distributed space which resembles more to \emph{LIME}'s \emph{federated spaces} than to \emph{TOTA} or \emph{L$^2$imbo}.


In other words, each device locally manages its own part of the space and only collaborates with others to share content.
This means that writing does not imply replication or migration of the \emph{tuples}.
The main drawback of this decision is that the availability or the data depends only on its manager's availability.
In other words, we loose \emph{time autonomy} in this second space.
However, this space reflects the current state of the environment and reduces the dependency of two devices in third ones.
Besides, for those cases where \emph{time autonomy} or \emph{data availability} are important, the first space can be used. % TS para ubicomp
\subsection{The \acl{wot} and other \ac{rest} Solutions for \acs{ubicomp}}
% nueva sección para hablar de WoT (DPWS ya se ha presentado)

\InsertFig{venn-sec1_1}{fig:venn_ubicomp}{
  Non-semantic integration approaches for \ac{ubicomp} apart from \aclp{ts}.
}{
  Scope of this subsection.
}{0.6}{}


The \acl{wot} initiative encourages the use of \acs{rest}-based solutions embedding web servers in daily objects \citep{guinard_internet_2011}.
In this way, the objects can integrate with the \ac{www} as first-class citizens. % decir ``f-c cit of the Web" es reiterativo ya que WWW==web
This integration brings the following benefits:
\begin{itemize}
  \item The smart-things can be linked to enable its discovery by browsing. This involves using the tool most users are familiar with: the browser.
  \item They can be bookmarked or shared through social networks \citep{guinard_sharing_2010}.
  % explicar qué es un mashup?
  \item They can be integrated with other web applications through mash-ups \citep{guinard_towards_2009,ostermaier_webplug:_2010,pintus_anatomy_2011}.
  % TODO citar al canadiense tb? pintus, stirbu y a todo el mundo
  \item Mechanisms such as searching, caching, load-balancing and indexing can be used over the objects to achieve the scalability of the web. % cita al indio
\end{itemize}


% Mencionar un par de trabajos sobre móviles con servidores web embebidos, para que se vea que no es sólo \ac{wot}


\subsubsection{\acs{sw} by using Intermediaries}
\label{sec:sw_intermediaries}

% como los smart environments describen contexto usando web semántica

% uso concreto por parte de soluciones significativas: siempre centralizando el uso de semántica en cacharros grandes

% intro a que ahora se va a hablar de soluciones IoT que usen semántica

Adding semantics works well for devices with high computational capacity but may add too much overhead for most of the devices in the \ac{iot}.
To reduce this overhead in such devices, part of this computation is usually delegated to an intermediary.
Some noteworthy example is the one proposed by \citet{broring_semantic_2009}.
% buscar otros ejemplos de enjundia
% The Context Broker Architecture (CoBrA)[Che04] 
% Gu et al. (2007)
% AlarmNet (2008)


\InsertFig{venn-sec2}{fig:venn_semantics_ubicomp}{
  The \acl{sw} for \acl{ubicomp}.
}{
  Scope of this section.
}{0.6}{}


These intermediaries or \emph{Semantic Gateways} are in charge of managing the semantic annotation.
The devices send raw data (which can be compressed) to the intermediaries and the gateways annotate the content semantically.
Thus, the devices do not have to care about any semantic aspect and just collect the data as they did before.

These \emph{Semantic Gateways} reduce the load to embedded devices with limited resources by decreasing the number of requests they have to provide.
In addition, a centralized intermediary can gather all the information and thus, reduce the complexity of managing a distributed environment.

However, using intermediaries to store the semantic data of resource constrained devices also has some drawbacks.
On the one hand, centralization does not faithfully represent mobility situations were individuals carry their own semantic information in their personal devices.
In addition, the data obtained from an intermediary will always be less fresh than the one obtained where it is generated (i.e., sensors).
On the other hand, the servers are critical in centralized systems and therefore, their availability determines the operation of these solutions.
They also impose a burden on the maintenance which may be worthless in some simple scenarios.


% TODO meter alguna referencia a Cabilmonte sobre SPARQL en streaming
% Trata de consultar datos semánticos en streaming a una base de datos relacional.
% Por lo que entiendo, la comunicación sensor-DB se no es semántica.



% diferenciar de una forma menos descarada?

\subsubsection{\acs{sw} in Providers}
\label{sec:sw_providers}
% enumerar aquellas características de IoT???
% sacar algo del paper que habla de los retos de usar semántica en IoT

% pero ahora los cacharros cada vez son más potentes y no es difícil imaginar un mundo poblado por ellos blablah

% WoT y aquel que hacía cosas de móviles

% TODO generalizar a resource-constrained devices
Lately some solutions have arisen to semantically annotate data where it is generated. % in the provider
In the \acl{wot}, multiple solutions have considered using semantics to enrich the data definition in a machine processable manner.
Generally, these solutions embed the metadata in HTML using microdata, microformats or RDFa \citep{mayer_extensible_2011}.
These contents are returned by the Internet connected objects and are used to enhance the findability of the data by search engines.
% Ontologías no se pueden definir, tiene que haber consenso en su uso por parte de las máquinas de búsqueda.
% en vez de eso, queremos mejorar la busqueda por parte de los cacharros, no de los search engines
This approach does not focus on making the devices able to search and interact with others.
% end-to-end search
Instead, it considers them as mere human-oriented information providers which should be indexed by third party search services.
% aquí se puede enganchar también el discurso de Doulkeridis2007desent: coverage and scalability, freshness y monopoly

A more general way to represent semantic data is using RDF\footnote{\url{http://www.w3.org/RDF/}} based representations (i.e., full semantics).
SPITFIRE European project\footnote{\url{http://spitfire-project.eu}} represents the most remarkable effort on gathering full semantics and the \ac{wot}.
It focuses on fully integrating sensor data with the \ac{lod}. 
The \ac{lod} are datasets which follow a series of principles on how to open and publish data.
The goal of the \ac{lod} is to publish linked terms using full semantics.

SPITFIRE shares with our solution the vision of a world populated by devices acting as semantic data providers no matter how small they are \citep{hasemann_rdf_2012}.
Therefore, many of their efforts are complimentary to this work.
%This is done by defining an ontology for mapping other common ontologies and providing semi-automatic generation of semantic annotations from raw data.
%They also propose an abstraction to represent real-world entities (virtual sensors) using data provided by low-level sensors.
%Finally, and more specific to this work, they propose a searching model which predicts the current state of things by computing their periodic patterns in past states.
To search within these providers, \citet{pfisterer_spitfire:_2011} proposes a model which predicts the current state of things by computing their periodic patterns in past states.
Again, the goal of this method is to adapt search engines to the new fashion of data provided on the \emph{\acl{sw} of Things}.

Instead, we propose a model to enhance the search capability of web-connected things in a distributed fashion.
We aim to promote the interaction and collaboration between any devices in these environments.
Not only sensors, but also regular computers or mobile devices \citep{balandin_access_2011}.



\subsubsection{Discussion}

% TODO reescribir esto para hablar sólo de REST vs mi propuesta!

Section~\ref{sec:sw_intermediaries} presents the most common approaches to use \ac{sw} in \ac{ubicomp}: delegate on intermediaries the semantization of the raw data.
However, we do not intend to follow this path.
Instead, we propose to share semantic content directly from the embedded and mobile platforms as the solutions in Section~\ref{sec:sw_providers} do.
This completely distributed approach tries to simplify the maintenance burden and promote the access to the most updated data.


Regarding searching, the solutions on Section~\ref{sec:sw_providers} promote a less distributed search.
Instead, our model tries to enhance searching capability of web-connected things.
This improvement leads to environments where any devices directly interact and collaborate with each other.



In Section~\ref{sec:remote_invocation} we have analyzed the current trends towards the remote invocation style.
Within them, we can see how \ac{wot} has gained much more attention both for its perceived simplicity and for its seamless integration with the web.
% o decir REST based solutions para no centrar mucho en IoT
Besides, the availability of tools and libraries for many mobile and embedded platforms ease its adoption.


However, as any remote invocation style it introduces coupling between senders and receivers.
% de johanson_extending_2004 he sacado esas características, no la particularización a WoT
These complicates the application changes both in the short-term and in the long-term \citep{johanson_extending_2004}.
In the short-term because nodes constantly join and leave the environment due to mobility or to failures.
In the long-term because the space is used to solve new problems and the obsolete devices are replaced with new technology.
Furthermore, the human interaction is eased by minimizing the configurations.
% quizá lo de johanson_extending_2004 se podría retomar en el capítulo cha:tsc
In Chapter~\ref{cha:tsc}, we deeply compare both \ac{rest} and \ac{ts} to see how they can benefit each other. % WoT (REST + ubicomp)
% TODO
%  + reescribirlo como una misma cosa?
%  + añadirle una 

\subsection{\acs{sw} by using Intermediaries}
\label{sec:sw_intermediaries}

% como los smart environments describen contexto usando web semántica

% uso concreto por parte de soluciones significativas: siempre centralizando el uso de semántica en cacharros grandes

% intro a que ahora se va a hablar de soluciones IoT que usen semántica

Adding semantics works well for devices with high computational capacity but may add too much overhead for most of the devices in the \ac{iot}.
To reduce this overhead in such devices, part of this computation is usually delegated to an intermediary.
Some noteworthy example is the one proposed by \citet{broring_semantic_2009}.
% buscar otros ejemplos de enjundia
% The Context Broker Architecture (CoBrA)[Che04] 
% Gu et al. (2007)
% AlarmNet (2008)


\InsertFig{venn-sec2}{fig:venn_semantics_ubicomp}{
  The \acl{sw} for \acl{ubicomp}.
}{
  Scope of this section.
}{0.6}{}


These intermediaries or \emph{Semantic Gateways} are in charge of managing the semantic annotation.
The devices send raw data (which can be compressed) to the intermediaries and the gateways annotate the content semantically.
Thus, the devices do not have to care about any semantic aspect and just collect the data as they did before.

These \emph{Semantic Gateways} reduce the load to embedded devices with limited resources by decreasing the number of requests they have to provide.
In addition, a centralized intermediary can gather all the information and thus, reduce the complexity of managing a distributed environment.

However, using intermediaries to store the semantic data of resource constrained devices also has some drawbacks.
On the one hand, centralization does not faithfully represent mobility situations were individuals carry their own semantic information in their personal devices.
In addition, the data obtained from an intermediary will always be less fresh than the one obtained where it is generated (i.e., sensors).
On the other hand, the servers are critical in centralized systems and therefore, their availability determines the operation of these solutions.
They also impose a burden on the maintenance which may be worthless in some simple scenarios.


% diferenciar de una forma menos descarada?

\subsection{\acs{sw} in Providers}
\label{sec:sw_providers}
% enumerar aquellas características de IoT???
% sacar algo del paper que habla de los retos de usar semántica en IoT

% pero ahora los cacharros cada vez son más potentes y no es difícil imaginar un mundo poblado por ellos blablah

% WoT y aquel que hacía cosas de móviles

% TODO generalizar a resource-constrained devices
Lately some solutions have arisen to semantically annotate data where it is generated. % in the provider
In the \acl{wot}, multiple solutions have considered using semantics to enrich the data definition in a machine processable manner.
Generally, these solutions embed the metadata in HTML using microdata, microformats or RDFa \citep{mayer_extensible_2011}.
These contents are returned by the Internet connected objects and are used to enhance the findability of the data by search engines.
% Ontologías no se pueden definir, tiene que haber consenso en su uso por parte de las máquinas de búsqueda.
% en vez de eso, queremos mejorar la busqueda por parte de los cacharros, no de los search engines
This approach does not focus on making the devices able to search and interact with others.
% end-to-end search
Instead, it considers them as mere human-oriented information providers which should be indexed by third party search services.
% aquí se puede enganchar también el discurso de Doulkeridis2007desent: coverage and scalability, freshness y monopoly

A more general way to represent semantic data is using RDF\footnote{\url{http://www.w3.org/RDF/}} based representations (i.e., full semantics).
SPITFIRE European project\footnote{\url{http://spitfire-project.eu}} represents the most remarkable effort on gathering full semantics and the \ac{wot}.
It focuses on fully integrating sensor data with the \ac{lod}. 
The \ac{lod} are datasets which follow a series of principles on how to open and publish data.
The goal of the \ac{lod} is to publish linked terms using full semantics.

SPITFIRE shares with our solution the vision of a world populated by devices acting as semantic data providers no matter how small they are \citep{hasemann_rdf_2012}.
Therefore, many of their efforts are complimentary to this work.
%This is done by defining an ontology for mapping other common ontologies and providing semi-automatic generation of semantic annotations from raw data.
%They also propose an abstraction to represent real-world entities (virtual sensors) using data provided by low-level sensors.
%Finally, and more specific to this work, they propose a searching model which predicts the current state of things by computing their periodic patterns in past states.
To search within these providers, \citet{pfisterer_spitfire:_2011} proposes a model which predicts the current state of things by computing their periodic patterns in past states.
Again, the goal of this method is to adapt search engines to the new fashion of data provided on the \emph{\acl{sw} of Things}.

Instead, we propose a model to enhance the search capability of web-connected things in a distributed fashion.
We aim to promote the interaction and collaboration between any devices in these environments.
Not only sensors, but also regular computers or mobile devices \citep{balandin_access_2011}. % WoT (REST + ubicomp)
\section{Triple Space Computing}
\label{sec:tsc_soa}

% aclarar que también es space-based computing





% ----------------------------------------------------------------------


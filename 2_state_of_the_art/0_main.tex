% ----------------------------------------------------------------------

\begin{savequote}[50mm]
I was born not knowing and have had only a little time to change that here and there.
\qauthor{Richard P. Feynman}
\end{savequote}


\chapter{State of the Art}
\label{cha:stateoftheart}
\newcommand{\pathchaptwo}{2_state_of_the_art}

% the code below specifies where the figures are stored
\ifpdf
    \graphicspath{{\pathchaptwo/figures/PNG/}{\pathchaptwo/figures/PDF/}{\pathchaptwo/figures/}}
\else
    \graphicspath{{\pathchaptwo/figures/EPS/}{\pathchaptwo/figures/}}
\fi


%------------------------------------------------------------------------- 

% TODO IG yo creo que deberias orientarlo a un state of the art, ahora suena a related work. Tambien creo que se deberia introducir un poco mas, del estilo de: este capitulo presenta el related work para dar X y Y.
% TODO uniformizar labels!
This chapter presents the related work.
First, Section~\ref{sec:soa_intro} introduces, categorizes, and describes the related research topics. % o llamarle research areas, important concepts?
% nombrarlos? \ac{rest}, \ac{ts} and the \ac{sw}
% "this survey is not intended to be exhaustive (nor exhausting)"
Then, Section~\ref{sec:soa_ubicomp} explores their impact on the \ac{ubicomp} field. % rather than present an exhaustive (and exhausting) survey on these topics.
To this end, we identify the relevant works remarking their interesting characteristics and how they relate to this dissertation.
Finally, Section~\ref{sec:soa_tsc} analyzes and compares other semantic \ac{ts} middlewares independently of their application domain.


\InsertFig{venn-summary}{fig:venn_ubicomp_tuple_semantic}{
  Background areas of the thesis, which focuses on their intersection.
}{
  This thesis focuses on the intersection of the three areas represented in the Figure.
  Therefore, this chapter will be driven by their analysis.
}{0.6}{}


% TODO valorar meter en otra sección descripción de IoT y UbiComp y sus características?
\section{Introduction}
\label{sec:soa_intro}

% Otra forma de ver la interop:
%  La European Telecommunication Standards Institute (ETSI) define cuatro capas:
%      technical interop: a nivel de comunicación (p.e. de acuerdo en las 7 capas de OSI)
%      syntactic interop
%      semantic interop
%      organizational interop


% no habría que definir de nuevo y de manera formal ubicomp o vale con la intro?
The IEEE \citep{_ieee_1990} defines \emph{interoperability} as ``\emph{the ability of two or more systems or components to exchange information and to use the information that has been exchanged}''.
The heterogeneity of technologies present in \ac{ubicomp} environments makes this a key property to consider.
The definition clearly distinguishes between two requirements: % o goals o incremental requirements
(1) to exchange information; and
(2) to use that information. % to understand others data (i.e. \emph{interoperation}).


% Buscar una referencia mejor: http://en.wikipedia.org/wiki/Interoperability
Exchanging information in distributed systems comprehends the communication between two systems.
% lo de ab-initio practicamente sólo lo he visto en la wikipedia
For the lower communication levels, we opt for interoperability \emph{ab-initio} relying on standard and widely accepted communication protocols. % e.g. HTTP
For a higher-level (i.e. application layer), Section~\ref{sec:soa_integration} categorizes different integration approaches.
This dissertation aims to delve into the \emph{space-based computing} approach. % whose benefits etc. are described
However, \ac{rest} architectures' properties have made them massively accepted mechanisms to integrate applications.
Consequently, we also take into consideration the latter mechanism in our solution design.


% Sintáctica vs semántica
Regarding the second goal, it is usually divided in \emph{syntactic interoperability} and \emph{semantic interoperability}.
On the one hand, syntactic interoperability is associated with the format of the data (i.e. its syntax and encoding) \citep{van_der_veer_achieving_2006}. % e.g. high level: HTML, XML, etc.
On the other hand, semantic interoperability is concerned with ensuring that the exchanged information has a precise meaning.
Its ultimate goal is to make the information ``\emph{understandable by any other application that was not initially developed for this purpose}'' \citep{_european_2004}.
Section~\ref{sec:soa_sw} explains well-accepted mechanisms to achieve the latter goal: the \acl{sw}. % TODO qué es la SW??? un mecanismo, conjunto de estándares, ¿?

\subsection{Application Integration}
\label{sec:soa_integration}

% Definiciones de middleware en el libro de Coulouris:
%
% (pag 17):
% Middleware • The term middleware applies to a software layer that provides a
% programming abstraction as well as masking the heterogeneity of the underlying
% networks, hardware, operating systems and programming languages.
%
% (en otra parte):
% The task of middleware is to provide a higher-level
% programming abstraction for the development of distributed systems and, through
% layering, to abstract over heterogeneity in the underlying infrastructure to promote
% interoperability and portability.


The integration of two applications is driven by how they communicate.
To ease that communication the applications use middlewares.
A middleware is a software layer which provides a higher level of abstraction and masks the underlying heterogeneity.
\citet{coulouris_distributed_2012} define two communication styles on the upper layer of a middleware: % figura 4.1 modificada para incluir elementos
the remote invocation and the indirect communication (see Figure~\ref{fig:middleware_layers}).

\InsertFig{middleware_layers}{fig:middleware_layers}{Middleware layers}{Middleware layers according to \citet{coulouris_distributed_2012} classification.}{1}{}
% According to Coulouris et al. HTTP is An example of a request-reply protocol!


The \emph{remote invocation} involves the most common two-way exchange between senders and receivers in distributed systems.
Within the \emph{remote invocation} style, RESTful style \citep{fielding_architectural_2000} and the well known set of WS-* \citep{alonso_web_2010} standards probably represent the most popular substyles.
In Section~\ref{sec:remote_invocation}, we analyze their application to the \ac{ubicomp} field, particularly focusing on the \ac{iot}.


\medskip


The \emph{indirect communication} style comprehends all the techniques with no direct coupling between the sender and the receiver.
% Otra forma:
% In contrast, the \emph{indirect communication} comprehends decoupled communications between senders and receivers.
The group communication, publish-subscribe systems, message queues or shared memory approaches are examples of indirect communication.
These paradigms are characterized by two key properties \citep{gelernter_generative_1985,coulouris_distributed_2012}
\footnote{
  We use the terminology of the  \aclp{ts}' seminal paper \citep{gelernter_generative_1985}, but please note that:
  % nomenclature, naming or terminology?
  \begin{itemize}
    %\item Gelernter et al. stated that \emph{distributed sharing} was just a consequence of the these properties.
    \item The \emph{space uncoupling} property is referred as \emph{reference autonomy} by some authors \citep{fensel_triple-space_2004}.
	  % El primero fue un tal Angerer en el 2002, pero en un artículo en alemán.
	  % En el 2003 hay un artículo suyo en Internet, pero no sé si mencionarlo.
    \item These authors mention a third property confusingly called \emph{space autonomy} (or \emph{location autonomy}).
	  According to \citet{fensel_triple-space_2004} this autonomy is achieved because:
	  \begin{quote}
	    The processes can run in completely different computational environments as long as both can access the same space.
	  \end{quote}
  \end{itemize}
}
:

\begin{itemize}
 \item \emph{Space uncoupling}, which is achieved when the sender does not need to know the receiver or receivers and vice versa.
 \item \emph{Time uncoupling}, which happens when senders and receivers do not need to exist in the same time\footnote{
	  Although some authors \citep{fensel_triple-space_2004,krummenacher_www_2005} explain this property just in terms of communication asynchrony,
	  % mencionar a otros? o no porque simplemente siguen lo dicho por Fensel?
	  % a Bundler no lo cito, porque era una master thesis "sólo"
	  % en este caso cito a krummenacher porque en esa publicación lo define directamente como el no uso de comunicación sincrona
	  \citet{coulouris_distributed_2012} make a clear distinction between them.
	  In their words, a communication is asynchronous when \emph{a sender sends a message and then continues without blocking},
	  whereas time uncoupling adds an extra dimension: \emph{the sender and the receiver can have independent existences}.
	  }.
 
\end{itemize}

\acl{ts} computing, also called space-based computing, offers an improvement of the shared memory approach.
Whereas the shared memory works at byte-level and accessing to memory addresses,
the \acl{ts} works with semi-structured data which is accessed in an associative manner.
In other words, in \ac{ts} the participants read data specifying patterns of interest.
In Section~\ref{sec:tuplespaces_eoa}, we study the most relevant \acl{ts} solutions proposed for the \acl{ubicomp}.


\subsubsection{Remote invocation}
\label{sec:remote_invocation}

% TODO definir en algun lado que es REST y que es WS-*???

% De Guinard 2011 rest-vs-ws
% 
% WS-* These services declare their functionality and interfaces in a Web Ser-
% vices Description Language (WSDL) file. Client requests and service response
% objects are encapsulated using the Simple Object Access Protocol (SOAP) and
% transmitted over the network, usually using the HTTP protocol. Further WS-
% * standards define concepts such as addressing, security, discovery or service
% composition. Although WS-* was initially created to achieve interoperability of
% enterprise applications, work has been done to adapt it to the needs of resource-
% constrained devices [10, 13]. Furthermore, lighter forms of WS-* services, such
% as the Devices Profile for Web Services (DPWS)1, were proposed [6].
% 
% REST At the core of a RESTful architecture [3] lie resources that are uniquely
% identified through Uniform Resource Identifiers (URIs). The Web is an imple-
% mentation of RESTful principles – it uses URLs to identify resources and HTTP
% as their service interface. Resources can have several representation formats (e.g.,
% HTML, JSON2 ) negotiated at run time using HTTP content negotiation. In a
% typical REST request, the client discovers the URL of a service it wants to call by
% browsing or crawling its HTML representation. The client then sends an HTTP
% call to this URL with a given verb (GET, POST, PUT, etc.), a number of options
% (e.g., accepted format), and a payload in the negotiated format (e.g., XML or
% JSON). Several recent research projects implement RESTful Web services for
% smart things [2] within what has become to be known as the Web of Things [5].


% De Pautasso 2008
% 
% REpresentational State Transfer (REST) was originally introduc-
% ed as an architectural style for building large-scale distributed hy-
% permedia systems. This architectural style is a rather abstract entity,
% whose principles have been used to explain the excellent scalability
% of the HTTP 1.0 protocol and have also constrained the design of
% its following version, HTTP 1.1. Thus, the term REST very often
% is used in conjunction with HTTP.


As stated before, the RESTful style and the set of WS-* standards are probably the most common remote invocation substyles currently used in the Internet.
Their advantages and disadvantages were extensively discussed by \citet{pautasso_restful_2008}.

WS-* standards syntactically describe the services' functionality and interfaces using \ac{wsdl}.
The communication between providers and consumers is usually transmitted over the \ac{http} in messages encapsulated using \ac{soap}.
WS-* web services, also called ``Big Web services'', offer more features such as transactions, reliability or message-level security.
However, they also require further architectural decisions on different layers of the WS-* stack.
These decisions make the people perceive it as more complex than the RESTful style \citep{guinard_search_2011}.


Probably thanks to that perception, \ac{rest} has gained momentum during the last decade. %  simpler alternative for services
% la explicación de los principios los he sacado sobre todo de pautasso_restful_2008, pero he intentado no plagiar las frases.
The \acl{rest} is defined by four principles \citep{fielding_architectural_2000}:
\begin{enumerate}
  \item The resources are identified with \acp{uri}.
  \item They are manipulated through a uniform interface.
	  This interface offers creation, reading, updating and deletion operations.
  \item The resources are decoupled from their representation so the content can be accessed in different formats.
  \item The interactions are stateful and hypertext-driven.
	This means that the state transitions are guided by the actions identified in the hypermedia by the server.
	% De Wikipedia: This means that clients make state transitions only through actions that are dynamically identified within hypermedia by the server.
\end{enumerate}

These principles are implemented by the Web,
so \ac{rest}  is often used in conjunction with well-known web standards such as \ac{http}, \ac{uri} or \ac{mime}.
This standards already have many libraries and tools available, which eases the adoption of this style by developers.
Besides, REST naturally integrates with the \ac{www} because it uses the Web as an application architecture instead of as a transport layer as WS-* services do.
% ojo a la frase: "se integra con la web porque usa la web como app architecture"

\bigskip

The use of both substyles in resource-constrained devices is represented by
\ac{dpws} \citep{moritz_devices_2010}, based on WS-*,
and the \ac{wot} \citep{guinard_internet_2011} \footnote{Due to its importance for this disseration, \ac{wot} is analyzed in Section~\ref{sec:soa_ubicomp}.}. % o mejor poner su tesis?


% TODO reescribir esto para que no suene tan positivo y sobrado
\ac{dpws} defines a subset of WS-* to make it suitable for resource-constrained devices.
%\ac{dpws} is claimed to be used in industrial environments.
Its most remarkable features are: decentralized multicast-based discovery, secure message transmission, subscription and event notifications.
\citet{moritz_devices_2010} compared \ac{dpws} with the \ac{rest} approach coming to the conclusion that \ac{dpws} can be restricted to be fully compatible with the RESTful style.
Even in that case, \ac{dpws} will still cover some missing features of the RESTful approach such as eventing and discovery \citep{moritz_devices_2010}.
% no menciono que sean más pesados para cacharros porque no he encontrado un sitio en el que defiendan que \ac{dpws} lo es (si que hay alguno que comparan REST y SOA, pero no DPWS)
\subsubsection{\acl{ts}}
\label{sec:tuplespaces_eoa}

% Ojo cuidado porque según esto mi middleware va a dejar de ser time uncoupled! (mirar Figure 6.1 para mayor concrección)
% En esa misma sección aclara que algunas implementaciones no son time-uncoupled.
% por otro lado, asíncronismo != time uncoupling

\acf{ts} has its roots in the Linda parallel programming language \citep{gelernter_generative_1985}.
In this communication model different processes read and write pieces of information so-called tuples in a common space.
Tuples are composed by one or more typed data fields (e.g. $<"aitor",1984>$ or $<3,7,21.0>$).
The tuples are accessed associatively using a template.
A template provides either a value or type for different fields (e.g. $<String,1984>$ or $<Integer, Integer, Float>$).
The operations over the space are defined by different primitives.
Although the primitives may change from implementation to implementation, the most common ones allow to:

\begin{itemize}
  \item \emph{Access the tuples non-destructively}, using a primitive which is usually called \emph{read} or \emph{rd}.
	This primitive returns a tuple from the space which matches the given template without changing the space.
  \item \emph{Access the tuples destructively}, using a primitive which is usually called \emph{take} or \emph{in}.
	This primitive extracts a tuple which matches the given template from the space.
  \item \emph{Write a new tuple into the space}. This primitive is usually called \emph{write} or \emph{out}.
\end{itemize}


\InsertFig{venn-sec1_2}{fig:venn_tuplespaces_ubicomp}{
  \aclp{ts} solutions for Ubicomp.
}{
  Scope of this subsection.
}{0.6}{}

\bigskip

% Ignorar esta sección? Es realmente necesaria cuando ya he puesto la comparativa de TSC?
So far, \acl{ts} has been adapted to \acl{ubicomp} by different authors.

% TODO leer artículos que me faltan y corregir
The \emph{event heap} \citep{johanson_extending_2004} is a system used for a specific \ac{ubicomp} sub-domain: interactive workspaces.
In this scenario there are rooms with different devices deployed and where mobile devices can enter.
Each room has its own space where the devices exchanging tuples to cooperate.
%For example, a video can be presented in a display and through a remote controller, the user can place the tuple for pausing it.
%The display consumes these kind of tuples, so it pauses the video whenever somebody places that tuple on the space.
% This system is simple to implement but is limited since it centralizes the space in a single machine per room.
This work merely identifies the requirements of these environments and the properties which solve them.
Then, it discusses how these properties can be satisfied using \ac{ts} or some extensions.
Finally they compare its implementation both with other \acp{ts} implementations or other coordination infrastructures. % e.g. RMI, MOM, Pub/sub

Although the analysis presented by \citeauthor{johanson_extending_2004} identifies the key requirements for \ac{ts} in \ac{ubicomp},
the solution they propose is purely centralized.
We opted for a distributed solution at the expense of limiting some of the features.
% indicar cuales son o pasar?
% usar sus características a la hora de valorar nuestra solución TSC en futuras secciones?


\emph{L$^2$imbo} \citep{davies_l2imbo:_1998,friday_experiences_1999} replicates the tuples to avoid a single point of failure.
Each node joined to a space uses an IP multicast address to exchange messages with other nodes in that space.
Writing into a space involves sending a multicast message to inform to the rest of the nodes of the tuple written.
Reading operation usually requires local reading.
Destructive reading of the tuple is more complex as it requires a global withdrawal.
In \emph{L$^2$imbo} only the owner of a tuple can remove it from the space.
The ownership of a tuple initially belongs to its creator, but can be transferred. % TODO ver qué añadió Friday


 % meter mas citas de LIME que estas 2?
\emph{LIME} (Linda in a Mobile Environment) \citep{picco_lime:_1999} is a TS solution for mobile systems.
In LIME each mobile device has its own space where it generally writes its tuples.
This space is shared with other devices creating federated spaces, i.e. the aggregation of different shared spaces.
In this way, each mobile can access to tuples in other mobiles whenever they become available.
They also proposed a new writing primitive to insert tuples in remote spaces.
% ejemplo de service discovery?

This is difficult to implement and \citet{coulouris_distributed_2012} complain about the unrealistic assumptions they make to simplify the problem. % cita al libro
These assumptions are the uniform multicast connectivity between devices whose tuple spaces are aggregated and the serialized and ordered connections and disconnections.
In any case, LIME was adapted to several platforms to run both in embedded and mobile devices \citep{murphy_transiently_2006}.


In the \emph{TOTA} (Tuples On The Air) Project \citep{mamei_programming_2009} tuples are disseminated to different devices.
To that end, each tuple has 3 fields:
1) the content of the tuple,
2) a rule which defines how it should be propagated,
3) a rule to define its maintenance.
For instance, they consider a museum where a visitor writes a query tuple describing a piece of art he wants to see.
The propagation rule defines that it should be propagated to all nodes in the vicinity, increasing the distance by one each time.
The tuple is configured to be deleted after a time-to-live period using its maintenance rule.
When the it reaches the room where the piece of art is located, the piece of art will write a response tuple.
This response tuple jumps from a device to another until it reaches the device which queried for it.


% TODO Analizar otras como TOTAM o CRIME
\subsection{The \acl{sw}}
\label{sec:soa_sw}

A problem of the initial view of the Web was that it was human-centered.
Regardless of whether the contents were machine processable, a human needed to interpret them to give them a meaning.
% mirar a ver cuando se empezó a usar \ac{sw} de verdad, porque en el artículo de 2001 dan a enterder que se llevaba un tiempo usando
In the early 2000s, \citet{berners-lee_semantic_2001} proposed the use of what they called the \acf{sw} to solve that problem.
The \ac{sw} was conceived as an extension of the ordinary web which would bring structure to its content.
As the ordinary web, the \ac{sw} benefits from the universality the hypertext provides by linking \emph{anything with anything}.
Besides, like the Internet, the \ac{sw} is intended to be as decentralized as possible.

The \acl{sw} and its key features are defined by the \emph{World Wide Web Consortium} \citep{semanticWeb-FAQ} as:
\begin{quote}
The vision of the \acl{sw} is to extend principles of the Web from documents to data.
Data should be accessed using the general Web architecture using, e.g., URI-s;
data should be \emph{related to one another} just as documents (or portions of documents) are already.
This also means creation of a common framework that allows data to be \emph{shared and reused} across application, enterprise, and community boundaries,
to be \emph{processed automatically} by tools as well as manually, including revealing possible \emph{new relationships} among pieces of data.
\end{quote}

Meaning is expressed in the \acl{sw} as sets of triples.
Each triple is composed by a subject, a predicate and an object like in a normal sentence (see Figure~\ref{fig:triples_example}).
Therefore, in  \citeauthor{berners-lee_semantic_2001} words, \emph{a document can assert that things (people, web pages or whatever) have properties (such as "is a sister of," "is the author of") with certain values (another person, another Web page)}.
A key difference with a normal sentence is that each concept is unambiguously defined by an URI.
These URIs form links between different triples as shown by the example of the Figure~\ref{fig:triples_example}.
% This set of triples can be expressed in multiple formats such as RDF, Turtle, Notation3 or NTriples. % TODO citas


\InsertFig{triples_example}{fig:triples_example}{
  Sample triples using different ontologies.
}{
  All the triples are represented graphically and some of them also textually.
  They describe academic and personal details about the author of this thesis.
  The knowledge is expressed using four different ontologies: FOAF, DC, SWRC and CiTE. % citar?
  Note that we use aliases, known as prefixes in most Semantic formats, to shorten some URIs and enhance the clarity of the Figure.
}{1}{}


A problem with the information described so far is that two different databases may use different URIs to express the same concept.
To overcome this the \ac{sw} offers collections of information called ontologies.
An ontology is a document which expresses the relations between terms commonly using a taxonomy and a set of rules to infer content.
The taxonomy defines the classes of a given domain and how they relate with each other.

Of course, different content providers could provide similar data described according to different ontologies.
Fortunately, ontologies can be easily mapped by providing equivalence relations within them.
In the same way, an ontology can be extended to adapt it to different application domains.
In any case, the reuse of the same models is beneficial and is therefore promoted through standardization.

% Remarkably, the use of the \ac{sw} has been promoted in the last years with the Linked Open Data (LOD) initiative.
% The LOD are datasets which follow a series of principles on how to open and publish data.
% The ultimate goal of the LOD is to publish linked terms using full semantics.
\subsection{Conclusions} % or Discussion???


In Section~\ref{sec:remote_invocation} we have analyzed the current trends towards the remote invocation style.
Within them, we can see how \ac{wot} has gained much more attention both for its perceived simplicity and for its seamless integration with the web.
% o decir REST based solutions para no centrar mucho en IoT
Besides, the availability of tools and libraries for many mobile and embedded platforms ease its adoption.


However, as any remote invocation style it introduces coupling between senders and receivers.
% de johanson_extending_2004 he sacado esas características, no la particularización a WoT
These complicates the application changes both in the short-term and in the long-term \citep{johanson_extending_2004}.
In the short-term because nodes constantly join and leave the environment due to mobility or to failures.
In the long-term because the space is used to solve new problems and the obsolete devices are replaced with new technology.
Furthermore, the human interaction is eased by minimizing the configurations.
% quizá lo de johanson_extending_2004 se podría retomar en el capítulo cha:tsc
In Chapter~\ref{cha:tsc}, we deeply compare both \ac{rest} and \ac{ts} to see how they can benefit each other.


\bigskip


% comparación de todas ellas
% de entre los que ofrecen distribución: este no mola por esto, esto otro mola por esto otro, etc.

In Section~\ref{sec:remote_invocation} we have studied the \ac{ts} solutions for \acl{ubicomp}.
Within these solutions, \emph{The event heap} is not considered because of its centralized nature.
The rest present different ways to disseminate the tuples. %  L2imbo, LIME and TOTA

The replication which L$^2$imbo proposes may not be feasible for devices with constrained memory or storage capacity.
Particularly, in spaces populated by many devices like the Internet of Things a large amount of tuples can be generated.
% need to configure the network

TOTA is useful for spatially connected \emph{ad hoc} environments, but it does not seem to suit well on networks where all the nodes can reach each other.
In this case, all the tuples could be replicated on the rest of the nodes showing the same limitations as L$^2$imbo.
% TODO COMPROBAR LEYENDO CON ATENCION LO DE TOTA!!!
Furthermore, \emph{ad hoc} environments are not the focus of our solution since we assume the connectivity between all the devices in a space.

% TODO no darle tan duro que si no a ver como luego me comparo con ellos :-)
Finally, despite of their unrealistic assumptions LIME proposes a model where each tuple is responsible of its own part of the space.
They share spaces whenever they become available.


\begin{table}%[position specifier]
  \centering
  \begin{tabular}{c|cccc}%p{5cm}}
      ~ & The event heap & L2imbo & LIME & TOTA \\
      \hline
      \hline
      Distributed & No & Yes & Yes & Yes \\
      Goal & - & Availability & Federation & Ad-hoc \\
      % Scalability?
      % Coger propiedades del libro de sistemas distribuidos?
  \end{tabular}
  \caption{Comparison of the most prominent \acl{ts} alternatives for \acl{ubicomp}.} % no columpiarme, o poner todas o ninguna?
  \label{tab:ubicomp_ts_comparison}
\end{table}


\medskip

In our model, we argue that in an \ac{ubicomp} environment each node should manage their own semantic information.
By delegating responsibility 
we naturally represent mobility scenarios where users carry their own profiles in their mobiles to new spaces 
and 
we acknowledge the fact that embedded devices directly control their own actuators and sensors.
Therefore, we propose to directly access to the source of the data avoiding the use of intermediaries whenever it is possible.
In this aspect, our solution has more in common with LIME.
Two major differences with LIME are the use of semantics and the natural integration with the web. % i.e. use of REST/HTTP
\section{\acl{ubicomp}}
\label{sec:soa_ubicomp}

The previous section presented the main research topics related to this dissertation.
% The middleware presented in this thesis adopts some relevant characteristics from them.
From their intersection with the \ac{ubicomp} field emerges relevant related work.
Consequently, this section describes these works and presents their differences and similarities with our work.
% hablar de comparacion de ventajas en inconvenientes? hablar de contribuciones a dichas areas?

% presentar secciones
% we will scrutinize how other authors have use the \ac{sw}.


\subsection{The \acl{wot} and other \ac{rest} Solutions for \acs{ubicomp}}
% nueva sección para hablar de WoT (DPWS ya se ha presentado)

\InsertFig{venn-sec1_1}{fig:venn_ubicomp}{
  Non-semantic integration approaches for \ac{ubicomp} apart from \aclp{ts}.
}{
  Scope of this subsection.
}{0.6}{}


The \acl{wot} initiative encourages the use of \acs{rest}-based solutions embedding web servers in daily objects \citep{guinard_internet_2011}.
In this way, the objects can integrate with the \ac{www} as first-class citizens. % decir ``f-c cit of the Web" es reiterativo ya que WWW==web
This integration brings the following benefits:
\begin{itemize}
  \item The smart-things can be linked to enable its discovery by browsing. This involves using the tool most users are familiar with: the browser.
  \item They can be bookmarked or shared through social networks \citep{guinard_sharing_2010}.
  % explicar qué es un mashup?
  \item They can be integrated with other web applications through mash-ups \citep{guinard_towards_2009,ostermaier_webplug:_2010,pintus_anatomy_2011}.
  % TODO citar al canadiense tb? pintus, stirbu y a todo el mundo
  \item Mechanisms such as searching, caching, load-balancing and indexing can be used over the objects to achieve the scalability of the web. % cita al indio
\end{itemize}


% lo he comentado antes, no sé si no me estoy repitiendo más que el ajo...
But besides the functional differences, one of the major advantages of REST compared with the WS-* is that is perceived as simpler.
\citet{guinard_search_2011} proved that developers consider it easier to learn and find it more suitable for programming smart things.
This makes \ac{wot} solutions more popular for resource constrained devices than the WS-* based ones.

% Mencionar un par de trabajos sobre móviles con servidores web embebidos, para que se vea que no es sólo \ac{wot} % WoT (REST + ubicomp)
% Ignorar esta sección? Es realmente necesaria cuando ya he puesto la comparativa de TSC?
\subsection{\acl{ts} for \ac{ubicomp}}

So far, \acl{ts} has been adapted to \acl{ubicomp} by different authors.


\InsertFig{venn-sec1_2}{fig:venn_tuplespaces_ubicomp}{
  \aclp{ts} solutions for Ubicomp.
}{
  Scope of this subsection.
}{0.6}{}


% TODO leer artículos que me faltan y corregir
The \emph{event heap} \citep{johanson_extending_2004} is a system used for a specific \ac{ubicomp} sub-domain: interactive workspaces.
In this scenario there are rooms with different devices deployed and where mobile devices can enter.
Each room has its own space where the devices exchanging tuples to cooperate.
%For example, a video can be presented in a display and through a remote controller, the user can place the tuple for pausing it.
%The display consumes these kind of tuples, so it pauses the video whenever somebody places that tuple on the space.
% This system is simple to implement but is limited since it centralizes the space in a single machine per room.
This work merely identifies the requirements of these environments and the properties which solve them.
Then, it discusses how these properties can be satisfied using \ac{ts} or some extensions.
Finally they compare its implementation both with other \acp{ts} implementations or other coordination infrastructures. % e.g. RMI, MOM, Pub/sub

Although the analysis presented by \citeauthor{johanson_extending_2004} identifies the key requirements for \ac{ts} in \ac{ubicomp},
the solution they propose is purely centralized.
We opted for a distributed solution at the expense of limiting some of the features.
% indicar cuales son o pasar?
% usar sus características a la hora de valorar nuestra solución TSC en futuras secciones?


\emph{L$^2$imbo} \citep{davies_l2imbo:_1998,friday_experiences_1999} replicates the tuples to avoid a single point of failure.
Each node joined to a space uses an IP multicast address to exchange messages with other nodes in that space.
Writing into a space involves sending a multicast message to inform to the rest of the nodes of the tuple written.
Reading operation usually requires local reading.
Destructive reading of the tuple is more complex as it requires a global withdrawal.
In \emph{L$^2$imbo} only the owner of a tuple can remove it from the space.
The ownership of a tuple initially belongs to its creator, but can be transferred. % TODO ver qué añadió Friday


 % meter mas citas de LIME que estas 2?
\emph{LIME} (Linda in a Mobile Environment) \citep{picco_lime:_1999} is a TS solution for mobile systems.
In LIME each mobile device has its own space where it generally writes its tuples.
This space is shared with other devices creating federated spaces, i.e. the aggregation of different shared spaces.
In this way, each mobile can access to tuples in other mobiles whenever they become available.
They also proposed a new writing primitive to insert tuples in remote spaces.
% ejemplo de service discovery?

This is difficult to implement and \citet{coulouris_distributed_2012} complain about the unrealistic assumptions they make to simplify the problem. % cita al libro
These assumptions are the uniform multicast connectivity between devices whose tuple spaces are aggregated and the serialized and ordered connections and disconnections.
In any case, LIME was adapted to several platforms to run both in embedded and mobile devices \citep{murphy_transiently_2006}.


In the \emph{TOTA} (Tuples On The Air) Project \citep{mamei_programming_2009} tuples are disseminated to different devices.
To that end, each tuple has 3 fields:
1) the content of the tuple,
2) a rule which defines how it should be propagated,
3) a rule to define its maintenance.
For instance, they consider a museum where a visitor writes a query tuple describing a piece of art he wants to see.
The propagation rule defines that it should be propagated to all nodes in the vicinity, increasing the distance by one each time.
The tuple is configured to be deleted after a time-to-live period using its maintenance rule.
When the it reaches the room where the piece of art is located, the piece of art will write a response tuple.
This response tuple jumps from a device to another until it reaches the device which queried for it.


% TODO Analizar otras como TOTAM o CRIME % TS para ubicomp
% TODO
%  + reescribirlo como una misma cosa?
%  + añadirle una 

\subsection{\acs{sw} by using Intermediaries}
\label{sec:sw_intermediaries}

% como los smart environments describen contexto usando web semántica

% uso concreto por parte de soluciones significativas: siempre centralizando el uso de semántica en cacharros grandes

% intro a que ahora se va a hablar de soluciones IoT que usen semántica

Adding semantics works well for devices with high computational capacity but may add too much overhead for most of the devices in the \ac{iot}.
To reduce this overhead in such devices, part of this computation is usually delegated to an intermediary.
Some noteworthy example is the one proposed by \citet{broring_semantic_2009}.
% buscar otros ejemplos de enjundia
% The Context Broker Architecture (CoBrA)[Che04] 
% Gu et al. (2007)
% AlarmNet (2008)


\InsertFig{venn-sec2}{fig:venn_semantics_ubicomp}{
  The \acl{sw} for \acl{ubicomp}.
}{
  Scope of this section.
}{0.6}{}


These intermediaries or \emph{Semantic Gateways} are in charge of managing the semantic annotation.
The devices send raw data (which can be compressed) to the intermediaries and the gateways annotate the content semantically.
Thus, the devices do not have to care about any semantic aspect and just collect the data as they did before.

These \emph{Semantic Gateways} reduce the load to embedded devices with limited resources by decreasing the number of requests they have to provide.
In addition, a centralized intermediary can gather all the information and thus, reduce the complexity of managing a distributed environment.

However, using intermediaries to store the semantic data of resource constrained devices also has some drawbacks.
On the one hand, centralization does not faithfully represent mobility situations were individuals carry their own semantic information in their personal devices.
In addition, the data obtained from an intermediary will always be less fresh than the one obtained where it is generated (i.e., sensors).
On the other hand, the servers are critical in centralized systems and therefore, their availability determines the operation of these solutions.
They also impose a burden on the maintenance which may be worthless in some simple scenarios.


% TODO meter alguna referencia a Cabilmonte sobre SPARQL en streaming
% Trata de consultar datos semánticos en streaming a una base de datos relacional.
% Por lo que entiendo, la comunicación sensor-DB se no es semántica.



% diferenciar de una forma menos descarada?

\subsection{\acs{sw} in Providers}
\label{sec:sw_providers}
% enumerar aquellas características de IoT???
% sacar algo del paper que habla de los retos de usar semántica en IoT

% pero ahora los cacharros cada vez son más potentes y no es difícil imaginar un mundo poblado por ellos blablah

% WoT y aquel que hacía cosas de móviles

% TODO generalizar a resource-constrained devices
Lately some solutions have arisen to semantically annotate data where it is generated. % in the provider
In the \acl{wot}, multiple solutions have considered using semantics to enrich the data definition in a machine processable manner.
Generally, these solutions embed the metadata in HTML using microdata, microformats or RDFa \citep{mayer_extensible_2011}.
These contents are returned by the Internet connected objects and are used to enhance the findability of the data by search engines.
% Ontologías no se pueden definir, tiene que haber consenso en su uso por parte de las máquinas de búsqueda.
% en vez de eso, queremos mejorar la busqueda por parte de los cacharros, no de los search engines
This approach does not focus on making the devices able to search and interact with others.
% end-to-end search
Instead, it considers them as mere human-oriented information providers which should be indexed by third party search services.
% aquí se puede enganchar también el discurso de Doulkeridis2007desent: coverage and scalability, freshness y monopoly

A more general way to represent semantic data is using RDF\footnote{\url{http://www.w3.org/RDF/}} based representations (i.e., full semantics).
SPITFIRE European project\footnote{\url{http://spitfire-project.eu}} represents the most remarkable effort on gathering full semantics and the \ac{wot}.
It focuses on fully integrating sensor data with the \ac{lod}. 
The \ac{lod} are datasets which follow a series of principles on how to open and publish data.
The goal of the \ac{lod} is to publish linked terms using full semantics.

SPITFIRE shares with our solution the vision of a world populated by devices acting as semantic data providers no matter how small they are \citep{hasemann_rdf_2012}.
Therefore, many of their efforts are complimentary to this work.
%This is done by defining an ontology for mapping other common ontologies and providing semi-automatic generation of semantic annotations from raw data.
%They also propose an abstraction to represent real-world entities (virtual sensors) using data provided by low-level sensors.
%Finally, and more specific to this work, they propose a searching model which predicts the current state of things by computing their periodic patterns in past states.
To search within these providers, \citet{pfisterer_spitfire:_2011} proposes a model which predicts the current state of things by computing their periodic patterns in past states.
Again, the goal of this method is to adapt search engines to the new fashion of data provided on the \emph{\acl{sw} of Things}.

Instead, we propose a model to enhance the search capability of web-connected things in a distributed fashion.
We aim to promote the interaction and collaboration between any devices in these environments.
Not only sensors, but also regular computers or mobile devices \citep{balandin_access_2011}. % semántica para ubicomp
%\subsection{Conclusions} % or Discussion???

Section~\ref{sec:sw_intermediaries} presents the most common approaches to use \ac{sw} in \ac{ubicomp}: delegate on intermediaries the semantization of the raw data.
However, we do not intend to follow this path.
Instead, we propose to share semantic content directly from the embedded and mobile platforms as the solutions in Section~\ref{sec:sw_providers} do.
This completely distributed approach tries to simplify the maintenance burden and promote the access to the most updated data.


Regarding searching, the solutions on Section~\ref{sec:sw_providers} promote a less distributed search.
Instead, our model tries to enhance searching capability of web-connected things.
This improvement leads to environments where any devices directly interact and collaborate with each other.
\section{Semantic \aclp{ts}}
\label{sec:tsc_soa}

% Añadir  Citron (2005)? (eoa de Almeida)
% Tanasescu
% swarm
% decir que tiene mucho que ver con distributed triple stores
% analizarlos también??? vaya pereza! -> igual comentar diferencia y dejar el análisis para la sección tosca de IGWS

Semantic \aclp{ts} aim to join \aclp{ts} with the \acl{sw} to propose a more uncoupled solution.
Particularly, it benefits from the autonomies introduced by both \ac{ts} and the \ac{sw}:

\begin{itemize}
  \item Space uncoupling.
  \item Time uncoupling.
  \item Data-schema uncoupling. % explicar?
% 	enhances the interoperability of the
% 	applications built on top of TSC. This way, two applications using standard ontologies can interact among them
% 	automatically enriching one each other, as long as they use the same space and standard and linked ontologies.
\end{itemize}


In this section we provide a review of the existing Semantic \aclp{ts} solutions. % citar nixon_tuplespace-based_2008 ???
To summarize, we present a comparison of them at the end of the section, analyzing their limitations.


\InsertFig{venn-sec3}{fig:venn_tuplespaces_semantics}{
  Semantic \aclp{ts}.
}{
  Scope of this section.
}{0.6}{}


\subsection{TSC}
% modelo propuesto
TSC \citep{fensel_tsc_2007} is the middleware which first implemented the ideas of Fensel et al. \citep{fensel_triple-space_2004}.
TSC explores a \acl{tsc} infrastructure based on the \ac{rest} architectural style.
Equivalently to \ac{rest}, TSC proposes a stateless infrastructure where the resources are identified by \acp{uri}.
In the first approximation, each tuple was considered a resource.
These tuples correspond with \ac{rdf} triples.
In a later work,  Krummenacher et al. \citep{krummenacher2006specification} proposed to use identifiers just for a set of triples (i.e. \ac{rdf} graphs).
% bundler cita el de "WWW: or what's wrong with the web" para explicar eso último
They argue that resources usually need to be modeled using more than just a triple.
Besides, the use of graphs reduces the complexity on storage.
To cope with the heterogeneity of data and scalability issues, they propose the use of independent spaces identified by their own \ac{uri}.
Using these disjoint spaces, they can restrict the number of participant in each one.


% cosas relativas al despliegue
TSC combines client/server and \ac{p2p} architectures using a hybrid super-peer one.
The super-peers of this architecture communicate with each other using a middleware which offers a virtual shared memory space of Java objects.
They present 3 different types of nodes:
\begin{itemize}
  \item The \emph{servers} are the responsible for storing and replicating the spaces.
  \item The \emph{heavy clients} are not always connected to the system, but they have the same responsibilities as the \emph{servers}.
	They can work off-line with their replicas.
  \item The \emph{light clients} can access to the spaces using proxies or an \ac{http} \acs{api}.
\end{itemize}


% arquitectura
TSC consist of three main components: a mediation engine, a data access layer and a coordination layer.
The mediation engine manages data heterogeneity by intermediating between nodes using different data schemas.
The data access layer ensures persistence of data, supports template matching and provides some reasoning over the knowledge stored.
The coordination layer replicates named graphs to all involved TS kernels and therefore has to be synchronized with the data access.
% http://tsc.deri.at/deliverables/D21.html
% http://tsc.deri.at/deliverables/D31/D31.html
To do that, they use a \ac{p2p} implementation of a virtual shared data space. % cita a CORSO?
% decir que ofrece transaccionalidad gracias a CORSO? transactionality
% It also offers a transactional context and a simple form to publish and subscribe to certain patterns.



\subsection{Semantic Web Spaces}

% modelo propuesto
Semantic Web Spaces (SWS) \citep{tolksdorf_coordination_2006} defines two different data coordination views: data view and information view.
In both views the tuples are represented with three fields which correspond with \ac{rdf} triples' subject, predicate and object.
The data view stores syntactically valid \ac{rdf} and it is accessed using Linda primitives.
The information view stores consistent and satisfiable data which are managed using new primitives.
This view takes into account the knowledge defined by ontologies to perform semantic matching over inferred triples.

% cosas relativas al despliegue
SWS are virtually partitioned using contexts, which provide a simple form of access control.
Clients can allocate contexts to control their view privately or share it with other clients. % This allow clustering ¿WTF?
The different active spaces and the tuples they contain are represented in a meta-space using a meta-model ontology.
This ontology also allows to express the hierarchical relation between the spaces and security policies.
% por lo que entiendo, está centralizadísimo

% arquitectura
The prototype contains three main components: publication, retrieval and security.
The publication component describes how the platform communicates with the clients or storage managers.
The retrieval component process queries and can reason to enhance results. % no se razona cada vez que se inserta una tupla
The security component restrict the access of client to spaces according to the policies defined.  % hablan de agentes
% dicen que experimentaron problemas de escalabilidad



\subsection{sTuples}

% modelo propuesto
STuples \citep{khushraj_stuples:_2004} was conceived by Nokia Research Center as a pervasive computing work.
It extends the JavaSpace middleware, a object-oriented \acl{ts}, using a semantic match over the semantic tuples.
To do that, a semantic tuple is an object with a field with the semantic description in the DAML+OIL format. % definir DAML+OIL o citarlo?


% cosas relativas al despliegue
The space is centralized and uses a framework called Vigil to create clients and services. % cita Vigil?
Vigil acts as a communication gateway with several protocols and it is directly integrated with sTuples.


% arquitectura
The JavaSpace extension is done in sTuples through two components: the Tuple Manager and the Tuple Matcher.
The manager adds, removes or states changes of tuples.
Every insertion requires the semantic data to be asserted.
The matcher implements the algorithm to match templates.
Besides, specialized agents reside on the space to offer added functionality to the users.
For instance, they implemented recommendations, subscriptions and task-executions in their prototype.



\subsection{Conceptual Spaces}

% modelo propuesto
Conceptual Spaces, or CSpaces, were born to study the applicability of semantic \aclp{ts} to different scenarios including \acl{ubicomp}.
Tuples in CSpaces contain seven different fields.
Some fields enable to express logic theories using different formal languages.
Other remarkable fields are used to define the version, specify the creator or to define to which subspaces the tuple belongs. % explicar los 7 campos?

% cosas relativas al despliegue
The spaces in CSpaces can be individual or shared.
An individual space belongs to a single process.
A shared one allows access to all the processes who have agreed on.

% arquitectura
%Es una IDEA y NO la habían IMPLEMENTADO (posteriormente hicieron un prototipo limitado para un paper de 2006).
CSpaces differentiates between the same three types of nodes as TSC.
They intended to implement individual spaces in heavy-clients and shared spaces in servers.
They first considered a centralized implementation for the shared-spaces, % lo dicen en la comparativa
and later on a P2P network of the nodes managing a shared-space. % paper de después
In any case, to the best of our knowledge, this conceptual exercise never went beyond a rather limited prototype.
% TODO confirmar 100%



\subsection{tsc++}

% modelo propuesto
Tsc++ \citep{krummenacher_open_2009,blunder_distributed_2009} is a new version of the TSC project.
Therefore, although the model is equivalent to TSC, its implementation completely changes.


% cosas relativas al despliegue
% Páginas 54 y 55
Instead of replicating each space in the different kernels as TSC does, tsc++ proposes to divide the spaces among the participants.
Therefore, a space corresponds with a group a nodes on the \emph{Jxta} \ac{p2p} framework \footnote{https://jxta.dev.java.net/}. % jxta de otra forma?
These nodes write their information locally and discover information in other nodes using some special nodes.
This special nodes in Jxta are called \emph{rendezvous} (RDV).
Tsc++ proposes and evaluates two discovery algorithms: flooding and random walk (RDW).
\begin{itemize}
  \item The flooding algorithm sends the request to a RDV, which propagates it to all the known nodes.
	If this nodes are RDV, they may propagate the request until a number of maximum hops is reached.
  \item The RDW algorithm chooses one peer to send the message to after the request has been processed in the current one.
	In Tsc++ the request is sent to a RDV which creates a list of all the nodes and sends it to the first one in the list.
	At each step a new node from that list will be chosen.
	Besides, the \emph{walk} ends when a result is returned or when a maximum number of hops is reached if no response was found.
\end{itemize}
% De dónde leches saqué esto de NB?
%These nodes follow a simple flooding-based strategy to distribute their information called \emph{negative broadcasting}.
%In negative broadcasting each node writes the information locally and read querying to the rest of the nodes of the space.


% arquitectura
Tsc++ has 3 main components: the communication and the data access.
The data access layer relies on a semantic repository to perform queries and store the triples.
% mencionar que son Sesame \citep{broekstra_sesame:_2002} y Owlim \citep{kiryakov_owlimpragmatic_2005} ???
The communication layer maps the reading coordination primitives to group communication.
% No evaluan el mecanismo de subscripción porque dicen que se basa en DHT y que como es escalable no necesita más evaluación
% No me queda claro de qué forma usan DHT sobre los nodos RDV que son los que miran las subscripciones cómo están.
Besides, it offers a subscription mechanism independent to the content stored in the space which uses triples and templates as a basis.


% it does not make inference
% it does not allow expressive querying
% it has not been designed for devices with reduced computing capabilities, because tsc++ middleware focused on architecture and implementation in large scale and we focus in the short scale (local area networks with an intelligent environment).



\subsection{Triple Space Communication (TripCom)}

% modelo propuesto
TripCom\footnote{TripCom (IST-4-027324-STP, www.tripcom.org)} is a Triple Space solution which works with \ac{rdf} triples as resources.
The most remarkable aspects of the model proposed by TripCom are:
\begin{itemize}
  \item The use of subspaces to form form nested multiple spaces.
	Using them it restricts the communication to a part of the whole space.
	This restriction leads to local scalability and completeness.
  \item A mechanism to overlap spaces called \emph{scopes}.
	Using the scopes a client can have a temporary copy of some tuples.
	However, any insertion and deletion would not apply to the whole Triple Space.
  \item A meta-space to describe the relations between spaces or to which graph a tuple belongs.
  \item Different types of templates can be used to query the space, from simple wildcard-based templates to more expressive SPARQL queries.
  \item New primitives: read at triple or a graph level and writing multiple triples at once.
  \item A subscription mechanism.
\end{itemize}


% cosas relativas al despliegue
To access to a space a client needs to contact a TripCom instance, i.e. kernel.
% TODO Duda inglés, está bien hecho eso de "sujeto verbo objeto and verbo" o debería ser "sujeto verbo objeto and sujeto verbo objeto"???
% TODO Uniformizarlo a lo largo del texto
Each kernel stores one or more subspaces and can contact other kernels responsible for different spaces.
To do that, if the space \ac{url} is provided, it simply resolves this \ac{url} using \ac{dns} to contact the other kernel.
Otherwise, the kernel uses 4 different strategies:
\begin{itemize}
  \item Triple Provider. It uses shortcuts to know who answered a query in the past.
  \item Recommender. It uses shortcuts to know which kernel successfully routed a query in the past.
  \item Indexing - \ac{dht}.
	It creates indexes using a hash function over the subject, predicate, object and space \ac{url}.
	Then, it stores these indexes in a distributed database which relies in a structured \ac{p2p} system. % mencionar PGrid???
\end{itemize}


% arquitectura
A TripCom kernel has several components which communicate with each other using a \acl{ts} solution.
The most important components regarding how the triples are distributed are:
\begin{itemize}
  \item Triple Store Adapter.
	It acts as a gateway with the selected repository.
  \item Query Processor. It descomposes query into subqueries.
	Then, to answer each subquery the kernel: a) queries the local data store or b) forwards it to other kernel.
  \item Distribution Manager.
	Forwards queries or write requests to the appropiate kernels.
\end{itemize}
Apart from these, there are also security, mediation, metadata and transaction managers.
On top of these components, TripCom offers three different \acp{api}:
Triple Space \acs{api} to map the primitives,
web service \acs{api}, which offers higher level service consumption \acs{api}
and management \acs{api} to configure the each kernel.



\subsection{Smart-M3}

% modelo propuesto
Smart-M3 \citep{honkola_smart-m3_2010} is an open-source distributed middleware which uses \ac{rdf} graphs.
It supports three types of primitives: \acl{ts}-based primitives, subscriptions and a query over a whole space.
Smart-M3 defines two types of nodes: Semantic Information Brokers (SIB) and Knowledge Processors (KP).
The SIBs store the knowledge and perform reasoning process on behalf of the KPs.
A KP access to a SIB using two different types of interfaces (called KPI):
\begin{itemize}
  \item The low-level API uses triples as a base.
  \item The high-level API allows the developer to use ontological entities such as classes, relations and individuals.
\end{itemize}


% cosas relativas al despliegue
The space can be distributed over more than a SIB using a \emph{distributed deductive closure}.
 % TODO asegurarse 100% de esto que voy a decir!
However, to the best of our knowledge no evaluation with more than a SIB has been presented yet, making it de facto an centralized space.
For the communication between SIBs and KPs, Smart-M3 defines a protocol called \emph{Smart Access Protocol (SSAP)}. % decir que está basado en XML?
SSAP can be implemented on top of different communication mechanisms (e.g. SOA, XMPP, Bluetooth or TCP/IP) making the solution communication agnostic. % TODO acrónimos?


% arquitectura
% TODO repasar cómo está escrito, porque con esos nombres queda un poco confuso
A SIB is divided in 5 layers: transport, request handling, graph operations, triple operation and persistent store.
The transport layer has different processes for each communication mechanism supported.
Once the SSAP operations have been extracted, the request handling creates a thread for each primitive.
The \emph{graph operations} layer works with primitives which use \acs{rdf} graphs and it uses the underlying \emph{triple operations}. % ¿?
In the \emph{triple operation}, expressive queries may need to be translated to query the store.
% decir que también soporta subscripción?

% dibujo?

% De: http://www.diem.fi/files/KP_reference%20implementation.pdf
%  The Semantic Information Broker is the information repository of the Smart 
%  Environment. In theory, the Smart-M3 Smart Space can consist of one or more SIB 
%  entities. However, the SIB reference implementation does not currently support any 
%  kind of interaction between different SIBs, which would enable distribution of the 
%  Smart Space. On the IOP, the SIB is implemented as a NoTA SN.



\subsection{Nardini et al.}

% modelo propuesto
\citet{nardini_semantic_2013} explores the use of Description Logics on top of an existing \emph{tuple centre} implementation called TuCSoN. % TODO CITA TuCSoN???
A tuple centre is a tuple space where the behavior can be programmed as a result of the basic coordination primitives.
TuCSoN aims to provide \emph{tuple centres} that agents can use through Linda-type primitives.
This interaction can cause reactions which trigger behavior previously implemented in the \emph{tuple centre}.

% cosas relativas al despliegue
In TuCSoN each \emph{tuple centre} is deployed in device's port and it is identified by a name.
The communication with each centre is direct, not based in any underlying network. % no les mola P2P decian en algun lado
The agents can access to remote \emph{tuple centres} and migrate to them, but these interactions need to be programmed.

% arquitectura
Nardini et al. defines a domain ontology attached to each centre to provide semantic interpretation to the tuples stored.
They extend TuCSoN providing semantic tuples, templates, primitives, reactions and matching mechanism.
\begin{itemize}
  \item The semantic tuples express the logic terms.
  \item The semantic template defines a language based on logic.
  \item To discriminate common syntactic primitives from semantic ones they use a keyword.
  \item The semantic reactions redefine events, guards and reactions enabling the use of first-order-logic.
	The events define when a reaction should be triggered.
	The reaction guard represent a set of conditions which need to be satisfied to execute a reaction.
  \item For the semantic matching mechanism they integrate TuCSoN with a reasoner (Pellet). % citar pellet?
\end{itemize}



\subsection{Comparison}
% TODO CSpaces no hace más que liar el analisis, me planteo no incluirlo
%		Porque es muy bonito poner ideas felices, pero no ha sido implementado nunca!


\subsubsection{\acl{ts} model}

A \acl{ts} model is mainly characterized by the information hold (i.e. \emph{tuple}),
how it is hold (i.e. the \emph{space} model)
and how a node can query the space (i.e. the type of \emph{query}).
The Table~\ref{tab:comparisonTS} summarizes the model used in each of the semantic \aclp{ts}.

The \emph{flat} model is the simplest manner of disjoining spaces.
Identifying this spaces with URIs, one can link them in the knowledge similarly to what other tuple models do (e.g. Nardini et al or CSpaces). % comprobar
More sophisticated models propose virtual view of spaces (i.e. scopes) or hierarchies between spaces.
Nevertheless, we argue that this models are not easy to implement in a distributed way. % extenderlo despues?

The most common tuple model is the three-field tuple with \emph{subject}, \emph{predicate} and \emph{object}.
Other solutions propose their own tuple format.
We believe that for a developer used to work with the Semantic Web the use of \ac{rdf} triples is straightforward.
Grouping these triples in \ac{rdf} graphs also provides a useful abstraction for resources.
% TODO aclarar bien porqu TSC se centra en grafos y tsc++ no!

Regarding the queries, all the solutions offer graph patterns-based queries.
These queries can be easily processed by any device, whereas complex queries need parsers.
These parsers are not always available for resource constrained platforms.
Therefore, in this thesis we focus on queries based on graph patterns.
The adoption of more complex querying languages such as SPARQL can be done as a future work. % cita SPARQL?




\begin{table}[htbp]
\caption{Tuple Space model comparison for the analyzed solutions.}

\begin{tabular}{ l p{3cm} p{3cm} p{3cm} }
\hline 
  & Space  & Tuples  & Queries \tabularnewline
\hline 
 TSC & Flat  & Named graphs  & Graph patterns \tabularnewline
 SWS & Disjoint nested  & Triples  & G. patterns and complex \tabularnewline
 sTuples & Centralized nested  & Tuples with DAML+OIL field  & DL-based \tabularnewline
 CSpaces & Interconnected  & 7-field tuple  & Formal lenguage \tabularnewline
 tsc++ & Flat  & Triples  & Queries \tabularnewline
 TripCom & Nested  & Triples  & G. patterns and complex \tabularnewline
 Smart-M3 & Flat  & Triples  & G. patterns and complex \tabularnewline
 Nardini et al. & Tuple centres managed in different nodes & Tuples based on formal lenguage  & Formal lenguage \tabularnewline
\hline 
\end{tabular}
\label{tab:comparisonTS}
\end{table}



\subsubsection{Additional features}

The Table~\ref{tab:compAdds} summarizes the additional mechanisms provided by the solutions analyzed.
Most of them offer subscription mechanisms which are particularly useful for distributed environments.
% TODO modificar esto que ahora no hay implementacion!
Although this dissertation focuses more in the querying mechanism, in the Chapter~\ref{cha:implementation} we will propose a subscription mechanism.
Other solutions offer , but we have discarded its usage due to the hard implementation for distributed systems running on dynamic environments.


% transacciones
% subscripciones
% etc.

\begin{table}[htbp]
\caption{Features offered by the analyzed solutions.}

\begin{tabular}{ l p{3cm} l p{4cm} }
\hline 
  & Subscriptions  & Transactions  & Other features \tabularnewline
\hline 
 TSC & Yes  & Yes  &  \tabularnewline
 SWS & No  & No  &  \tabularnewline
 sTuples & Yes  & No  &  \tabularnewline
 CSpaces & Yes  & No  & Multiple read and writes \tabularnewline % bussler dice que tiene algún tipo de transaccionalidad, yo no lo creo
 tsc++ & Yes  & No  &  \tabularnewline
 TripCom & Yes  & Yes  &  \tabularnewline
 Smart-M3 & Yes  & No  &  \tabularnewline
 Nardini et al. & No \footnote{But it has been implemented in TuCSoN before \cite{ricci_extending_2002}.} & No  &  \tabularnewline
\hline 
\end{tabular}
\label{tab:compAdds} 
\end{table}





\subsubsection{Use of semantics}
% TODO evitar verengenales que no comprendo y que sólo me complican la vida

The Table~\ref{tab:comparisonSemantics} shows different features regarding the semantic model proposed by each solution.
First we can see how most of the solutions use the standard Semantic Web language \ac{rdf} to express knowledge.
The use of \ac{rdf} does not impose or limits the use of more advanced languages such as \ac{rdfs} or \ac{owl}.

CSpaces and Nardini et al. propose the use of a language-independent \ac{fol}.
However, since most advanced reasoners only implement a subset of \ac{fol}, any implementation is limited by them. % también estas
% theoretical counterpart of OWL DL [7]. Since OWL is the W3C standard ontology description language for the Semantic Web,
% and the standard de-facto for semantic applications in general, we adopt OWL as the ontology language for the domain
% ontologies associated to ReSpecT semantic tuple centres in TuCSoN. While for the details of OWL we forward interested

Reasoning is used both to validate the data inserted into the space or to query over inferred triples.
We do not validate the data inserted on the space to reduce the insertion time.
The possibility of querying over the inferred knowledge is limited by the availability of reasoners.
As will be shown in the next sections, nowadays there is no reasoner light enough to run in a reasonable time span in small devices.
Therefore, reasoning is not mandatory in our solution.
In any case, the use of reasoners over local content is tangential to the model proposed in this thesis.

Semantic matching is usually derived from the reasoning process.
Although it is a desirable characteristic, it is not imposed by any mean.
For our solution, describing the data semantically is a first step towards semantically interpreting it.




\begin{table}[htbp]
\caption{Use of semantics in the analyzed solutions.}

% TODO En Languages decir si se basa en RDF o FOL, el hecho de que luego se use OWL para definir conocimiento, se sobreentiende
% TODO añadir reglas?
% TODO añadir Validity and consistency?
% TODO añadir semantic clustering

\begin{tabular}{ l p{3cm} p{3cm} p{3cm} }
\hline 
  & Languages  & Reasoning  & Semantic matching \tabularnewline
\hline 
 TSC & RDF  & -  & Graph pattern templates and N3QL resolution \tabularnewline
 SWS & RDF(s) extendable to OWL, SWRL  & Yes (to match and validate)  & Subsumption-based \tabularnewline
 sTuples & DAML+OIL  & Tuples with DAML+OIL field  & Subsumption-based \tabularnewline
 CSpaces & Language-independent, up to FOL  & Yes (used for query answering, rewriting and consistency checking)  & Based on query engines \tabularnewline
 tsc++ & RDF & No  & No \tabularnewline % comprobar %  lo de para query answering lo hice yo en su día
 TripCom & RDF  & Yes (query rewriting) & Yes \tabularnewline % TODO rehacer esta columna comprobandolo
 Smart-M3 & RDF  & ?  & G. patterns and complex \tabularnewline
 Nardini et al. & Language-independent, up to FOL & ?  & ? \tabularnewline
\hline 
\end{tabular}
\label{tab:comparisonSemantics}
\end{table}

% TODO añadir tabla con análisis de las autonomías?



\subsubsection{Distribution}
%	2. centralizado / distribuído
%	4. basado en clientes tontos

Centralized \aclp{ts} are much simpler and easier to implement.
Therefore, they usually offer more features than the distributed ones.
However, they also impose a single-point-of-failure.

This thesis aims to solve that by proposing a distributed model.
Within the distributed \aclp{ts}, a wide range of distribution strategies have been detected.
The Table~\ref{tab:comparisonDistribution} summarizes the space distribution strategy adopted by each solution.

TSC uses flooding to replicate the knowledge in all the nodes.
Tsc++ propagates the queries in flooding-based approaches.
TripCom is built on top of a structured-network which imposes where the content is stored.
The suitability of these models is subject of analysis in the Chapter~XX.

% TODO reescribir desde aquí

%In this thesis, we aim to implement a completely distributed Semantic \acl{ts}.
%In that space all the nodes should be able to participate trying to avoid any distinction between clients and servers.
%Ubicomp is be characterized for being populated by very dynamic nodes.
%This nodes usually face severe computation and energy autonomy restrictions.



% TODO y así te evitas engorrosas explicaciones aquí
%We argue that the replication is not suitable for devices with severe storage, memory and computation restrictions.
%Flooding-based queries does not scalable for spaces with many nodes, .
%Finally structured networks does not comply on

\begin{savenotes}
  \begin{table}[htbp]
    \caption{Distribution of the spaces.}
    \centering
    \begin{tabular}{ l c c p{4cm} }
    \hline 
    & C/S & Distributed & Distribution \\
    & access & space & strategy \\
    \hline 
    \midtsc{} & $\checkmark$ & $\checkmark$ & Replication \\ % Possitive broadcasting
    \midsws{} & $\checkmark$ & × & - \\
    \midstuples{} & $\checkmark$ & × & - \\
    \midcspaces{} & $\checkmark$ & $\checkmark$ & Not detailed \\ % además no implementado :-S
    \midtscpp{} & × & $\checkmark$  & Local writing, different query strategies \\ % Flooding, RW, etc.
    \midtripcom{} & $\checkmark$ & $\checkmark$ & Structured network, different strategies \\
    \midsmartmt{} & $\checkmark$ &  $\checkmark$ & Theoretical \\ % decidir si me convence que no haya sido implementado, TODO citar al tipo que dijo como hacer SIBs distribuidos
    \midnardini{} & $\checkmark$ & × & - \\
    \hline 
    \end{tabular}
    \label{tab:distribution_comparison}
  \end{table}
\end{savenotes}



\subsubsection{Implementations}
%	5. Implementado?
%	6. Implementación disponible
%	Interoperability
%	Multiplatform
%	8. dependencias: librerias, etc. (para decir que lo nuestro se puede implementar en casi cualquier plataforma de forma fácil)



\subsection{Conclusion}

% extraer algunas conclusiones de lo que falta
% tono general: están muy bien de características, pero a la hora de hacerlo distribuido falla
% hacen aguas en proponer un modelo que pueda ser usado más allá del client-servidor
% Suitability for Ubicomp

So far, semantic \aclp{ts} solutions have offered a sophisticated models with interesting additional features.
However, in most of the cases this is possible by delegating such tasks on a subset of more powerful and static nodes.
Therefore, the distribution of the space is restricted to some nodes when in the best case, or centralized in a unique node in the worst case.
In these cases, the rest of the nodes are head toward the use of the space as simple clients.


In this thesis we will explore the most decentralized case: the one in which each node manages its own information.
Doing so, a user can carry out his profile in the smart-phone or a sensor can provide the last measures on-demand.
% sólo tsc++ seguía esa premisa

% qué es necesario para cumplir esa visión y porque las alternativas vistas no se ajustan:
% dependencias mínimas, protocolos estándares, etc.


% TODO CONCLUSION
Although both semantic space-based and Triple Spaces implementations exist, none of them has been specifically designed to be run in devices with constrained capabilities apart from our solution.




% ----------------------------------------------------------------------


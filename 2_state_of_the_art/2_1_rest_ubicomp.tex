\subsection{The \acl{wot} and other \ac{rest} Solutions for \acs{ubicomp}}
% nueva sección para hablar de WoT (DPWS ya se ha presentado)

\InsertFig{venn-sec1_1}{fig:venn_ubicomp}{
  Non-semantic integration approaches for \ac{ubicomp} apart from \aclp{ts}.
}{
  Scope of this subsection.
}{0.6}{}


The \acl{wot} initiative encourages the use of \acs{rest}-based solutions embedding web servers in daily objects \citep{guinard_internet_2011}.
In this way, the objects can integrate with the \ac{www} as first-class citizens. % decir ``f-c cit of the Web" es reiterativo ya que WWW==web
This integration brings the following benefits:
\begin{itemize}
  \item The smart-things can be linked to enable its discovery by browsing. This involves using the tool most users are familiar with: the browser.
  \item They can be bookmarked or shared through social networks \citep{guinard_sharing_2010}.
  % explicar qué es un mashup?
  \item They can be integrated with other web applications through mash-ups \citep{guinard_towards_2009,ostermaier_webplug:_2010,pintus_anatomy_2011}.
  % TODO citar al canadiense tb? pintus, stirbu y a todo el mundo
  \item Mechanisms such as searching, caching, load-balancing and indexing can be used over the objects to achieve the scalability of the web. % cita al indio
\end{itemize}


% lo he comentado antes, no sé si no me estoy repitiendo más que el ajo...
But besides the functional differences, one of the major advantages of REST compared with the WS-* is that is perceived as simpler.
\citet{guinard_search_2011} proved that developers consider it easier to learn and find it more suitable for programming smart things.
This makes \ac{wot} solutions more popular for resource constrained devices than the WS-* based ones.

% Mencionar un par de trabajos sobre móviles con servidores web embebidos, para que se vea que no es sólo \ac{wot}
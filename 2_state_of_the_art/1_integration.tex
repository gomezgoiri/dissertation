\section{Integrating Things}
\label{sec:integration}

The way in which the applications integrate is driven by how the participants communicate.
Coulouris et al. \cite{coulouris_distributed_2012} define two communication styles on the upper layer of a middleware (see Figure~\ref{}): % figura 4.1 modificada para incluir elementos
the remote invocation and the indirect communication.
% TODO figura middleware


The \emph{remote invocation} involves the most common two-way exchange between senders and receivers in distributed systems.
In contrast, the \emph{indirect communication} comprehends decoupled communications between senders and receivers.
Specifically, they are space uncoupled, because the sender does not need to know the receiver;
and time uncoupled, because senders and receivers do not need to exists in the same time.


Within the \emph{remote invocation} style, RESTful style \cite{fielding_architectural_2000} and the well known set of WS-* \cite{} standards probably represent the most popular substyles.
% TODO poner esta referencia en WS-* cuando el ACM deje de banearme
% http://dl.acm.org/citation.cfm?id=1965308
In Section~\ref{sec:remote_invocation}, we analyze their application to the Ubicomp field, particularly focusing on the IoT.


Tuple Spaces, is a type of indirect communication where the entities write data in a persistent space and read specifying patterns of interest.
In Section~\ref{sec:indirect_communication}, we study the most relevant Tuple Space solutions proposed for the ubiquitous computing.



\subsection{Remote invocation}
\label{sec:remote_invocation}

As stated before, RESTful style and the set of WS-* standards are probably the most common remote invocation substyles used currently in the Internet.
Their advantages and disadvantages were extensively discussed by Pautasso et al. \cite{pautasso_restful_2008}.
WS-* web services, also called ``Big Web services'', offer more features such as transactions, reliability or message-level security.
However, they also require further architectural decisions on different layers of the WS-* stack.
These decisions make the people perceive it as more complex than the RESTful style. % CITE


The RESTful style has gained momentum during the last decade as a simpler alternative to services.
REpresentational State Transfer (REST) was designed to build large-scale distributed hypermedia systems,
so it is often used in conjunction with well-known web standards such as HTTP, URI or MIME.
This standards already have many libraries and tools available, which eases the adoption of this style by developers.
Besides, REST naturally integrates with the World Wide Web because it uses the Web as an application architecture instead of as a transport layer as WS-* services do.


\medskip


The use of both substyles in resource-constrained devices is represented by
\textit{Device Profile for Web Services} (DPWS) \cite{moritz_devices_2010}, based on WS-*,
and the Web of Things (WoT) \cite{guinard_internet_2011}. % Mejor poner su tesis?


DPWS defines a subset of WS-* to make it suitable for resource-constrained devices.
%DPWS is claimed to be used in industrial environments.
Its most remarkable features are: decentralized multicast-based discovery, secure message transmission, subscription and event notifications.
Moritz et al. compared DPWS with the REST approach coming to the conclusion that DPWS can be restricted to be fully compatible with the RESTful style.
Even in that case, DPWS will still cover some missing features of the RESTful approach such as eventing and discovery \cite{moritz_devices_2010}.
% no menciono que sean más pesados para cacharros porque no he encontrado un sitio en el que defiendan que DPWS lo es


The Web of Things initiative encourages the use of REST-based solutions embedding web servers in daily objects \cite{guinard_internet_2011}.
In this way, the objects can integrate with the WWW as first-class citizens of the Web.
This integration brings the following benefits:
\begin{itemize}
  \item The smart-things can be linked to enable its discovery by browsing. This involves using the tool most users are familiar with: the browser.
  \item They can be bookmarked or shared through social networks \cite{guinard_sharing_2010}.
  \item They can be integrated with other web applications using through mashups \cite{ostermaier_webplug:_2010}. % cita a pintus, stirbu y a todo el mundo
  \item Mechanisms such as searching, caching, load-balancing and indexing can be used over the objects to achieve the scalability of the web. % cita al indio
\end{itemize}


But besides the functional differences, one of the major advantages of REST compared with the WS-* is that is perceived as simpler.
Guinard et al. proved that developers consider it easier to learn and find it more suitable for programming smart things \cite{}.
This makes WoT solutions more popular for resource constrained devices than the WS-* based ones.

% WoT necesita de Web Semántica, se analizará más tarde




\subsection{Indirect Communication}
\label{sec:indirect_communication}


% tuple spaces: 19.3 Interoperation del libro de Gordon Blair tiene un estado del arte majo
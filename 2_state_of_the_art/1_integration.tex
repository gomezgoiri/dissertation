\section{Integrating Things}
\label{sec:integration}

The way in which the applications integrate is driven by how the participants communicate.
Coulouris et al. \cite{coulouris_distributed_2012} define two communication styles on the upper layer of a middleware (see Figure~\ref{}): % figura 4.1 modificada para incluir elementos
the remote invocation and the indirect communication.

The \emph{remote invocation} involves the most common two-way exchange between senders and receivers in a distributed system.
On contrast, the \emph{indirect communication} comprehends decoupled communications between senders and receivers.
Specifically, they are space uncoupled, because the sender does not need to know the receiver;
and time uncoupled, because senders and receivers do not need to exists in the same time.

Within the \emph{remote invocation} style, RESTful style and the well known set of WS-* standards probably represent the most popular substyles.
In Section~\ref{sec:remote_invocation}, we analyze their application to the Ubicomp field, particularly focusing on the IoT.

Tuple Spaces, is a type of indirect communication where the entities write data in a persistent space and read specifying patterns of interest.
In Section~\ref{sec:indirect_communication}, we study the most relevant Tuple Space solutions proposed for the ubiquitous computing.


\subsection{Remote invocation}
\label{sec:remote_invocation}

As stated before RESTful style and the set of WS-* standards are probably the most common remote invocation substyles used in the Internet today.
The use of both substyles in resource-constrained devices is represented by \textit{Device Profile for Web Services} (DPWS) \cite{moritz_devices_2010}, based on SOAP,
and the Web of Things (WoT) \cite{guinard_internet_2011}. % Mejor poner su tesis?

% decir que mola mas REST porque es menos rollo para los cacharros y para los desarrolladores.
% una tendencia que se confirma a nivel global en internet (coger alguna referencia de Guinard)

In the RESTful style basic CRUD operations are defined over the information units identified by an URL, which is completely compliant with the standard HTTP features.
In other words, using the HTTP layer as an application layer and not just as a transport layer (as WS-*does), avoids the definition of extra protocols keeping the applications simpler.
Similarly, WoT stresses the benefits of defining REST interfaces which expose the capabilities of the different objects to enhance their integration with traditional web applications in a more web-developer friendly manner
\cite{dguinard-rest-vs-ws}.

Even though WoT exposes the access to each resource through the URL which identifies it, we believe that these
resources can be accessed through a common API in an associative way instead of always referring to them by their
identifier. Specifically, the resources used in our API are the core concepts of Triple Space Computing paradigm, i.e.
spaces, graphs and templates.

% WoT necesita de Web Semántica, se analizará más tarde


\subsection{Indirect Communication}
\label{sec:indirect_communication}


% tuple spaces: 19.3 Interoperation del libro de Gordon Blair tiene un estado del arte majo
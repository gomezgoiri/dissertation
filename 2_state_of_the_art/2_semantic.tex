\section{Looking for Application-level Interoperability}
\label{sec:interoperability}
% centrarse en el ejemplo de interoperabilidad d nivel de aplicaciones

A common problem of most of the solutions used in Ubicomp is the use of application domain dependent data formats.
% learn about the data-schema, learn about the syntax and the meaning??? mirar cómo lo explican en otros lados
These formats require developers of new applications to learn about the syntax and the meaning of that data in order to be able to reuse that information.
This is the case of the solutions scrutinized in the previous section.

\medskip

% TODO mejor que listar "lo que un developer" debería hacer, listarlo mejor como requisitos para que 2 aplicaciones YA existentes puedan usar datos las una de la otra.
On the one hand, a developer using a Tuple Space middleware could try to reuse the data from another application.
Let us imagine an \emph{application A} where some devices (masters) write tasks of the same type in a space.
For the same application, some other devices (workers) take each tuple representing the task, perform the task, and write the result in the same space.
Providing that a new \emph{application B} needs the same type of task to be done, a clever can try to reuse the work done by the workers.
To take advantage of these workers already implemented and deployed, he needs to:
\begin{itemize}
 \item Use the same space as the \emph{application A}.
 \item Write the tasks in the same format as the \emph{application A}.
	For that, he should be able to figure out the kind of tuple a worker consumes.
 \item Read the results by using the proper template.
	Therefore, he should know how the workers represent the result in the \emph{application A}.
\end{itemize}

While the first step may be trivial, the second and third steps require the developer to carefully study the \emph{application A}.
Specifically, he will need to know the number and the type of fields of the tuples which represent a task and a result.
This process is inherently manual, and therefore it is a stumbling block to achieve application-level interoperability.

\medskip

On the other hand, most of the solutions in the WoT use common web media formats such as HyperText Markup Language (HTML) or JavaScript Object Notation (JSON). % link
HTML is oriented for humans while JSON is a machine processable format.
Imagine a developer wants to regulate the brightness of a room by using some existing and already deployed WoT-enabled devices.
A device provides temperature through a REST \emph{service A} and other device has a REST \emph{service B} to switch on or switch off the lights.
Then, the developer will need to know:
\begin{itemize}
 \item The URL of the service A which provides the temperature.
 \item The URL of the service B which controls the lights of the room.
 \item The media-types that the service A returns.
 \item The media-types that the service B accepts.
 \item Once he has chosen a media-type, the syntax of the result provided by the service A.
 \item Once he has chosen a media-type, the syntax of the piece of data the service B is able to interpret.
\end{itemize}

The first two steps require the developer to browse and identify the key services.
This identification needs of a human interpretation unless the services provide some semantics in their description. % RESTdesc
The remaining steps require the developer to study each service to create and send or receive and interpret each piece of data.
Furthermore, imagine that someone replaces the temperature sensor with a new one used in a third application.
Let us assume that the service which will now return the temperature uses the same URL and returns the same media-type as the previous one.
Even in that case, if the second device provides the temperature using a different syntax, the light regulation application will not work anymore.

\medskip

% no me acaba de convencer este parrafo introductorio
In conclusion, the systems analyzed use information not meaningful for other applications or domains.
One solution to that problem is to use specialized systems which convert and reinterpret the data from one domain to the other.
Other one is to promote the use of standardized models to express the data.
In the next section we will analyze how the Semantic Web tackles these problems.


\subsection{The Semantic Web}

% explicar web semántica en plan facilon
% mencionar LOD
% poner ejemplo de interoperabilidad
% hablar de aquello de que la interoperabilidad sólo no sirve de nada



\subsection{The Semantic Web in the Ubiquitous computing}

% como los smart environments describen contexto usando web semántica

% uso concreto por parte de soluciones significativas: siempre centralizando el uso de semántica en cacharros grandes

% intro a que ahora se va a hablar de soluciones IoT que usen semántica

% Semántica en IoT: SmartM3 => guardar esta para la sección de TSC!
Adding semantics works well for devices with high computational capacity but may add too much overhead for most of the devices in the IoT.
To reduce this overhead in such devices, part of this computation is usually delegated to an intermediary.
Some noteworthy examples are Smart-M3 \citep{honkola_smart-m3_2010} and the one proposed by \citet{broring_semantic_2009}.
%In Smart-M3 These intermediaries are called knowledge processors.

These intermediaries or \emph{Semantic Gateways} are in charge of managing the semantic annotation.
The devices send raw data (which can be compressed) to the intermediaries and the gateways annotate the content semantically.
Thus, the devices do not have to care about any semantic aspect and just collect the data as they did before.

These \emph{Semantic Gateways} reduce the load to embedded devices with limited resources by decreasing the number of requests they have to provide.
In addition, a centralized intermediary can gather all the information and thus, reduce the complexity of managing a distributed environment.

However, as previously explained, using intermediaries to store the semantic data of resource constrained devices also has some drawbacks.
On the one hand, centralization does not faithfully represent mobility situations were individuals carry their own semantic information in their personal devices.
In addition, the data obtained from an intermediary will always be less fresh than the one obtained where it is generated (i.e., sensors).
On the other hand, the servers are critical in centralized systems and therefore, their availability determines the operation of these solutions.
They also impose a burden on the maintenance which may be worthless in some simple scenarios.


\subsection{The Semantic Web in resource constrained devices}
% enumerar aquellas características de IoT???
% sacar algo del paper que habla de los retos de usar semántica en IoT

% pero ahora los cacharros cada vez son más potentes y no es difícil imaginar un mundo poblado por ellos blablah

% WoT y aquel que hacía cosas de móviles

% TODO generalizar a resource-constrained devices
In the Web of Things, multiple solutions have considered using semantics to enrich the data definition in a machine processable manner.
Generally, these solutions embed the metadata in HTML using microdata, microformats or RDFa \citep{mayer_extensible_2011}.
These contents are returned by the Internet connected objects and are used to enhance the findability of the data by search engines.
% Ontologías no se pueden definir, tiene que haber consenso en su uso por parte de las máquinas de búsqueda.
% en vez de eso, queremos mejorar la busqueda por parte de los cacharros, no de los search engines
This approach does not focus on making the devices able to search and interact with others.
% end-to-end search
Instead, it considers them as mere human-oriented information providers which should be indexed by third party search services.
% aquí se puede enganchar también el discurso de Doulkeridis2007desent: coverage and scalability, freshness y monopoly

A more general way to represent semantic data is using RDF based representations (i.e., full semantics).
SPITFIRE European project\footnote{\url{http://spitfire-project.eu}} represents the most remarkable effort on gathering full semantics and the WoT.
It focuses on fully integrating sensor data with the Linked Open Data (LOD). 
The LOD are datasets which follow a series of principles on how to open and publish data.
The goal of the LOD is to publish linked terms using full semantics.

SPITFIRE shares with our solution the vision of a world populated by devices acting as semantic data providers no matter how small they are \citep{hasemann_rdf_2012}.
Therefore, many of their efforts are complimentary to this work.
%This is done by defining an ontology for mapping other common ontologies and providing semi-automatic generation of semantic annotations from raw data.
%They also propose an abstraction to represent real-world entities (virtual sensors) using data provided by low-level sensors.
%Finally, and more specific to this work, they propose a searching model which predicts the current state of things by computing their periodic patterns in past states.
To search within these providers, \citet{pfisterer_spitfire:_2011} proposes a model which predicts the current state of things by computing their periodic patterns in past states.
Again, the goal of this method is to adapt search engines to the new fashion of data provided on the \emph{Semantic Web of Things}.

Instead, we propose a model to enhance the search capability of web-connected things in a distributed fashion.
We aim to promote the interaction and collaboration between any devices in these environments.
Not only sensors, but also regular computers or mobile devices \citep{balandin_access_2011}.
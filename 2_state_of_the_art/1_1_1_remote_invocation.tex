\subsubsection{Remote invocation}
\label{sec:remote_invocation}

% TODO definir en algun lado que es REST y que es WS-*???

As stated before, \acs{rest}ful architectures and the set of WS-* standards are probably the most common remote invocation substyles currently used in the Internet.
This section introduces them both, but for an extensive analysis on their advantages and disadvantages please refer to \citet{pautasso_restful_2008}.

\bigskip

On the one hand, \ac{rest} architectures are those which follow the design principles described in Section~\ref{sec:rest}.
Each of these principles contribute to different networking properties (see Section~\ref{sec:network_properties}).
The \emph{scalability} and the \emph{simplicity} stand out from these properties.
% cita de fielding sobre simplicidad y sobre good at long term-design
% already have many libraries and tools available, which eases the adoption of this style by developers.


On the other hand, WS-* standards syntactically describe the services' functionality and interfaces using \ac{wsdl}.
The communication between providers and consumers is usually transmitted over the \ac{http} in messages encapsulated using \ac{soap}.
WS-* web services, also called ``Big Web services'', offer more features such as transactions, reliability or message-level security.
However, they also require further architectural decisions on different layers of the WS-* stack.
These decisions make the people perceive it as more complex than the \ac{rest}ful style \citep{guinard_search_2011}.
% pensar bien cómo introducir el tema de la simplicidad: por diseño de REST, por las decisiones que hay que tomar en WS-*???


\bigskip

Besides the simplicity, another key difference between both approaches is how they use web standards. % (e.g. \ac{http}, \ac{uri} or \ac{mime})
The web implements \ac{rest}'s principles.
Therefore, a \ac{rest} solution can rely only on well-accepted standards used in the web (e.g. \acs{http}, \acs{uri} or \acs{mime}).
This acceptance also results in a massive availability of libraries in most of the computing platforms.
% This standards already have many libraries and tools available, which eases the adoption of this style by developers.


On the contrary, WS-* also uses some of these standards, but in ways they were not designed for.
For instance, it uses \ac{http} as a transport layer instead of as an application layer protocol.
This prevents the resulting architectures from achieving some of \ac{http}'s desirable properties such as scalability or visibility.


\bigskip


The use of both styles in resource-constrained devices is represented by
\ac{dpws} \citep{moritz_devices_2010}, based on WS-*, and the \ac{wot} \citep{guinard_internet_2011}. % o mejor poner su tesis?
% TODO reescribir esto para que no suene tan positivo y sobrado
\ac{dpws} defines a subset of WS-* to make it suitable for resource-constrained devices.
%\ac{dpws} is claimed to be used in industrial environments.
Its most remarkable features are: decentralized multicast-based discovery, secure message transmission, subscription and event notifications.
\citeauthor{moritz_devices_2010} came to the conclusion that \ac{dpws} can be restricted to be fully compatible with the \ac{rest}ful style while it still covers some missing features (e.g. eventing and discovery).
% no menciono que sean más pesados para cacharros porque no he encontrado un sitio en el que defiendan que \ac{dpws} lo es (si que hay alguno que comparan REST y SOA, pero no DPWS)


In this dissertation we focused on \ac{rest} because of its massive acceptance as a way to integrate resource-constrained devices. % TODO poner algún dato?!
We delve into the \ac{wot} in Section~\ref{sec:soa_ubicomp}. % explicación de porqué no aquí
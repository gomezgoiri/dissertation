\section{Contributions}

% De una charla con Buján recuerdo que dijo que para O. Corcho...
%   + Contribuciones científicas: lo que nadie había hecho antes (innovación)
%   + Contribuciones técnicas: implementar o aplicar en otro campo un aspecto innovado por otro


This section presents a summary of the contributions explained in this dissertation.
\begin{itemize}
  \item Chapter~\ref{cha:stateoftheart} presents an in-depth state-of-the-art review.
    \begin{itemize}
      \item It presents the research fields to which the dissertation contributes.
      \item It portrays these related research fields by presenting some relevant works.
      \item It introduces equivalent semantic space-based computing middleware.
	    Later, it analyzes their characteristics and how they do not address the problem tacked in this dissertation.
    \end{itemize}
  
  \item Chapter~\ref{cha:tsc} depicts a two-space model which (1) preserves \ac{tsc}'s decoupling properties and (2) respects \ac{ubicomp}'s distributed nature.
    % designs a middleware
    \begin{itemize}
      \item It scrutinizes the \ac{rest} style and its compatibility with \ac{tsc}.
      \item It describes the two-space model. % compromise between
            The first space is a commonplace where all the participants can write and read.
            The second space is a \emph{read-only} virtual \Space{} formed by contents managed by the different participants.
            It is accessed by any node as a whole using the same primitives as the first one.
      \item It designs the middleware by presenting the adopted primitives and how the spaces can be accessed through a \ac{http} \acp{api}.
	    %Doing so, we retain some of \ac{rest}'s properties and enable interoperability with other elements.
      \item It analyzes the properties of the designed middleware from different perspectives.
	    Particularly, it scrutinizes which beneficial properties from \ac{tsc} or \ac{rest} are retained.
	    Besides, it remarks the middleware's suitability for resource constrained devices.
    \end{itemize}
  
  \item Chapter~\ref{cha:searching} presents a searching architecture for semantic \ac{wot} solutions.
        This architecture can be used in this dissertation's middleware, but its applicability is not limited to any specific solution.
    \begin{itemize}
      \item It balances the load between devices depending on their computing and energy capacities.
      \item It structures the nodes in different dynamic roles with different responsibilities. % roles: WP, Provider y Consumer, dynamic: ppalmente WP
      \item It enables end-to-end \ac{http} consumption without the need of an intermediary to search. % sé que esto es tricky, pero no lo necesita necesariamente
      \item It presents and assesses different types of information summaries the nodes can use to better address their requests.
      \item It evaluates the proposal with a simulated but yet realistic environment.
      % TODO considerar poner:
      %       This architecture tries to minimize the requests devices have to handle by enhancing their search mechanism.
      %       The extra-tasks introduced by this enhancement are performed by nodes chosen according to their capacities.
    \end{itemize}
    
  \item Chapter~\ref{cha:actuate} analyzes and compares two mechanisms to actuate over the space. % uno direct y otro indirect
    \begin{itemize}
      \item It explains how to use \ac{tsc} to coordinate nodes willing to actuate over the physical environment and actuators.
      \item It depicts the need of a subscription system and its key requirements.
      \item It presents how to adopt a proof-based actuation mechanism in the middleware.
      \item It compares the requirements and benefits of both approaches.
    \end{itemize}
  
  % contribuciones técnicas: implementaciones varias
  \item Besides, as a result of the theoretical work made on this dissertation, we have done the following technical contributions:
    \begin{itemize}
      \item Otsopack\footnote{\url{https://github.com/gomezgoiri/otsopack}}: a \ac{tsc} middleware which works over \ac{http} and implements most of the ideas presented in the dissertation.
            This middleware is publicly available for different computing platforms and has been used in several research projects.
      % Esto deberiamos incluirlo aunque sea brevemente en algun lugar...
      \item Exploration of the use of semantic content in several embedded platforms.
	    We have tested several platforms, semantic frameworks and reasoners to verify the feasibility of the ideas and assumptions presented in the dissertation.
      \item A fully parameterizable simulation framework to evaluate different communication strategies. % TODO open-sourcearlo y poner link!
      % TODO open-sourcear WoT 2013 y comentarlo? No se si merece la pena
    \end{itemize}
\end{itemize}


% el tema de haber trabajado con otros
Finally, parallel to this thesis, but deeply influenced by it, we have collaborated with other entities to:
\begin{itemize}
  \item Analyze the security aspects which need to be addressed in this environments to present a solution. % que puede ser adaptada a esto
  \item Applicate the Otsopack middleware in several scenarios:
    \begin{itemize}
      \item Homes: automating them according to the user's preferences. % y lo del podometro?
                                                                        % en los escenarios de distintos papers
                                                                        % lo que se hizo en TALIS+Engine en Alicante o no se donde
      \item Hospitals and residences: tracking the evolution of patients with cognitive impairments.  % TODO traducir centro de día mejor que hospital
      \item Supermarkets: aiding to buy and easing mobility through robots.
      \item Hotels: intelligently adaptating it to the customers' needs.
    \end{itemize}
\end{itemize}
\section{Future Work}

% Limitations => Future Work
% ¿Poner lista de la compra o sólo las cosas que se destilen de las limitaciones del trabajo?

Previous chapters of this dissertation have pointed out several limitations and open issues.
These limitations and issues open the door to interesting future work which is detailed below.


\subsection{Closing the Gap with Third \acs{wot} Applications}

One of the middleware's goals is to achieve interoperability with third applications.
In other words, to reuse their data transparently to the user from our middleware and vice-versa.
This goal can be seen as a two-fold strategy: ease the access to the data and make them reusable.

\begin{description}
  \item[Access to the data.] It must be done by means of an \ac{api}.
			    In this thesis we have defended the need of using \ac{http} for that purpose.
			    An adaptation to the upcoming \ac{coap} would ease its adoption by new limited platforms. % "base de usuarios"
			    However, the most challenging aspect is not to replicate it in other suitable protocols.
			    
			    The most challenging aspect is to automatize the consumption of heterogeneous third \acp{api}.
			    According to the \ac{rest} principles, the \ac{resthateoas} constraint can help automated the clients to use any \ac{api} independently of its specific shape. % shape or implementation
			    Instead, this dissertation proposes a common \ac{api} that needs to be implemented in any device willing to share content. % TODO data api vs domain api => mirar el capitulo 3 y explicar esto
			    
			    In any case, an agent would need knowledge about the domain explored to reuse any \ac{api}.
			    And to reuse many \acp{api} without reimplementing the agent, the data types and states of a domain should be standardized. % TODO poner cita del blog de fielding sobre la discussion esa
			    Therefore, any step in that direction would require the agreement of the community around the domain (e.g. the \ac{wot} community).
			    
			    The counter-pad of third \acp{api} automatic consumption would be to ensure that our \ac{tsc} \ac{api} complies with the \ac{resthateoas} constraint.
			    Some steps have been done to make the \ac{api} human discoverable through \ac{html} representations.
			    However, more effort needs to be done to ease its automatic consumption regardless of its specific shape. % shape or implementation? % which would depend on the version
			    % se pueden usar varias estrategias en paralelo
			    
			    Finally, the \ac{resthateoas} constraint may lead to more network usage, which is particularly harmful for devices with energy autonomy limitations.
			    Therefore, this trade-off between the interoperability and the network efficiency should be also assessed.
			    
  \item[Reusing the data.] The semantic knowledge contributes to that goal.
			  However, as \citet{barnaghi_semantics_2012} point out, the use of semantic descriptions alone does not ensure interoperability.
			  Additionally, they suggest the need of:
			  (1) standardizing ontologies and using them (or at least when using own ontologies, referencing an upper ontology); and
			  (2) interpreting the annotations effectively. % efectivamente: con razonadores que funcionen bien
			  While the first one requires the consensus of the community, the second demands for efficient reasoners for mobile and embedded platforms.
			  This efficiency is particularly important since it directly influences their energy consumption.
			  % TODO Mencionar que estamos trabajando en ello?
			  % further assessment of existing ones, creating news for platforms
			  % pero decir que lo nuestro es un primer paso en paralelo a otros que puedan solucionar los otros problemas
			  % semantica y cacharros pequeños, sigue siendo algo difícil. Necesidad de razonadores ligeros.
			  
\end{description}


\subsection{Increasing Search Expressibility and Efficiency}

% Searching: SPARQL y demás
% 1ero: mayor eficiencia
% reescritura puede mejorar interop
We did not consider using query languages such as SPARQL because of the difficulties most constrained platforms may find in parsing them.
With this decision, we design a uniform \ac{api} for all the types of devices.
However, for the future work, one can adopt SPARQL thanks to an optional \ac{api}.
In that case, all the nodes should be aware that some of them are only able to interpret triple pattern templates.

The use of SPARQL in some nodes would benefit the developers and improve the middleware performance. % searching
The users of the middleware would be able to made more expressive queries.
The performance could be enhanced by introducing optimization techniques from the distributed databases field \cite{schwarte_fedx_optimization_2011}. % see references from Section~\ref{background}) or reference them here?


\subsection{Coordination Space's Distribution, Replication or Migration} % no se me ocurre un título más sexy

The current design demands the coordination space to be accessible through a \ac{http} \ac{api}.
For the sake of clarity, at some points of the dissertation we have assumed that this space is centralized in a unique machine.
However, the access through a \ac{http} \ac{api} does not restrict that.
Some alternatives that can be worth to explore are: % mencionar cloud para algo?
\begin{itemize}
  \item Replication of the space to increase the system performance (e.g. load balancing).
  \item Distribution of the content over different machines to enhance its scalability. % TODO decir que ya ha sido tratado por más soluciones? (e.g. TripCom)
  \item Migration from a machine to another to avoid a single point of failure.
	%No node is essential so the system will still work.
        This strategy will allow, for example, to select the space holder dynamically easing the middleware manageability and deployment.
        This would be similar to the \ac{wp} role of our searching architecture. % mencionar Chapter~\ref{cha:searching}?
\end{itemize}

% aquella idea que tenían los de Lancaster de que el middleware cambia de forma dependiendo del entorno
Besides, monitoring the middleware usage, one alternative or another could be dynamically chosen depending on the specific needs of each situation.


\subsection{Integration of the Proof-based Actuation Mechanism in the Space}

Chapter~\ref{cha:actuate} presented an alignment between a proof-based mechanism and actuation through the space.
However, its implementation and evaluation was left as future work.
To enable the usage of the proof-based mechanism, tasks from the latter must be translated into goals.
This translation must be generalizable to automatize it.
However, it is not clear whether it will always be possible this direct translation for complex goals and tasks.

% Idea del fallido WoT2013:
% Let us imagine that at the end of the previous section more than a path is found for the same desired goal.
% For instance, the light of a room can be increased to a certain level by turning on a lamp or by raising the blinds.
% In that case, different behaviors could be applied to select the preferred path.
% For example, one will want to save energy in the long term by raising the blinds.
% On the other hand, another application will decide to reach the light level no matter of the outside conditions.
% 
% Furthermore, these behaviors might be reused among the applications. % además, seguimos con la idea de reutilizar
% For that, we envision to share the behaviors in the space itself. % se infiere que semánticamente anotados
% The behaviors could be defined by the facility administrators or the developers.
% After that, the developers will be able to consume the behaviors through another middleware abstraction.
% Besides, it will monitor the state to check if the state of the space is effectively changed or other actions need to be taken.
Another interesting issue is to discern which path to follow to achieve a goal when two or more paths are available.
We believe that additional high-level rules or policies can be defined (e.g. to select the path which consumed less energy).
Storing them into the space may enable reusing this \emph{behaviors} by third applications' developers.

Finally, apart from reusing third \ac{wot} applications' actuation capacities in \ac{tsc}, the opposite question could also be explored:
How to reuse actuation mechanisms of nodes using \ac{ts} patterns from \ac{wot} solutions?


\subsection{Security for the Middleware}

% si no, es un poco fiesta y cualquiera podría poner datos falsos
Two key aspect for \ac{ubicomp} deployments are
(1) to authenticate the parts involved in the communication; and
(2) to ensure the privacy of the message being transmitted.
% Ya hemos hecho algo para dispositivos limitados y móviles
Although both problems have been deeply discussed in the literature,
the nodes in an \ac{ubicomp} environment face severe computing and energy restrictions.
To solve that, we have already propose a solution appropriate for sensor networks and mobile devices.

The next step is to evaluate it in more demanding scenarios and the final goal is to integrate it in our \ac{tsc} middleware.



%%% A partir de aquí, no son aspectos técnicos concretos


\subsection{Measuring Ease of Usage}

A subtle goal of any middleware is to ease the development of applications by encapsulating complex tasks (i.e. searching on a distributed space).
% Medir de forma objetiva aspectos del ser humano como la sencillez de uso o de implementación
Some objective indicators can be extracted from the code by implementing the same application with and without the middleware (i.e. number of lines).
However, from the human perspective these indicators do not necessarily express how easy, comfortable or pleasant was to use the middleware.
% facilitar modelado?? => el hecho de que sea difícil anotar, puede echar para atrás a desarrolladores: eso es problema de mi app o de quien?
Although the primitives are rather simple, the configuration or just the semantic annotation can complicate the learning process of any developer.
Future work could measure the impact the middleware has on them.
As a result, some improvements could be proposed (e.g. to create a version for a specific domain which masks all the semantic issues).


% Al final, a modo de colofón
\subsection{Further Evaluation on Real Deployments}

% muchas de nuestras decisiones en la implementación del middleware fueron guiadas por la realidad
Rather than designing the best theoretical solution, in this dissertation we have tried to present a realizable solution.
For that, we have implemented most of the ideas in the \emph{Otsopack} middleware and we have promote its use in different research projects.
This has helped us to identify the key problems to solve.

Besides, to efficiently evaluate our solution under very different and extreme circumstances, in some occasions we have simulated the scenarios. % con muchos cacharros
In this simulations we have tried to present always realistic assumptions.
% de simulaciones => a la realidad
However, there is always a gap between what is perceived as realistic and the reality.

To bridge both worlds, the next natural step would be assessing some aspects of our middleware in a real world environment: % real world es académico???
\begin{itemize}
	% Big ecosystem of interconnected objects and devices
  \item The search architecture.
        For that, we should deploy an scenario with a number of heterogeneous devices running a task in a long term.
        % viendo cuales son las necesidades reales de quienes lo usan
	%    para justificar los parametros que se usaron
        Then, we could measure their network usage and energy consumption to either tune up some parameters of the architecture or redesign some aspects.
  % relacionado con el punto de interoperabilidad: reusar distintas applicaciones existentes y ver que realmente se cumple
  %    una forma de asegurarse, es haciendo que distintos usuarios las creen
  \item The interoperability.
	New unconsidered problems could arise trying to reuse data from existing \ac{wot} applications.
	Each manual intervention needed to make our middleware work together with these applications would be a sign of an interoperability problem.
\end{itemize}
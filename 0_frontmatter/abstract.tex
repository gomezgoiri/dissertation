
% Thesis Abstract -----------------------------------------------------


%\begin{abstractslong}    %uncommenting this line, gives a different abstract heading


\begin{abstracts}        %this creates the heading for the abstract page
% Ubicomp, mobile e IoT
\acf{ubicomp} envisions environments where devices interact among themselves to work seamlessly together on behalf of humans.
In recent years, the emergence of the \acf{iot} concept, which opts for connecting everyday objects to the Internet, and the Mobile Computing paradigm have contributed to strengthening \ac{ubicomp}.
For this reason, \ac{ubicomp} environments are not necessarily populated by powerful computers.
On the contrary, resource constrained devices (e.g., embedded and mobile devices) are the main actors in these environments.
% Retos de Ubicomp
Thus, it is important for the environment to deal with their heterogeneity, unreliability, and replaceability.


% Tuplespaces y semántica
In order to cope with heterogeneity, the Semantic Web has proposed several standards and models to clearly define the terms so that they can be reused across applications boundaries.
Regarding unreliability and replaceability, space-based computing (or \aclp{ts}) promotes the uncoupled coordination of the devices.
Solutions based on semantic tuple spaces combine these three beneficial aspects resulting from bridging the Semantic Web and \aclp{ts} domains for \ac{ubicomp}.

% Reto que supone adaptar el paradigma para cumplir los objetivos de la hipótesis
Most of these semantic tuple spaces consider embedded and mobile devices as mere clients in a space managed by more powerful devices.
Such delegation helps to reduce the workload of devices with computing and energy limitations.
However,  this delegation moves the data away from where it is physically generated.
This creates a conflict between providing updated data and generating unnecessary network traffic for unused information.
In addition, this delegation makes constrained devices intrinsically dependent on other devices when it might not always be necessary.
This dissertation explores how these constrained devices can act as fully fledged semantic knowledge providers to create a more decentralized space.

% Aspectos que se analizan en la tesis
In conclusion, this dissertation presents a novel adaptation of semantic tuple space which considers the energy and computational impact on the devices.
Specifically, this dissertation proposes the following contributions:
\begin{itemize}
  \item A space model which considers the principles which have made the web flourish in the last decades, together with the uncoupling properties of space-based computing.
  \item An energy-aware search mechanism for autonomous constrained devices.
  \item An alignment of two approaches to act on the physical environment, namely a space-based indirect actuation and a web-based direct actuation.
\end{itemize}

\end{abstracts}



\begin{resumen}        %this creates the heading for the abstract page
La computación ubicua (\acs{ubicomp}) concibe entornos donde los dispositivos interactúan entre sí para trabajar en beneficio de los seres humanos, pero de forma imperceptible para los mismos.
En los últimos años, la emergencia del Internet de las Cosas (\acs{iot}), que aboga por conectar objetos cotidianos a Internet, y la computación móvil han contribuido a fortalecer la idea de \ac{ubicomp}.
Es por ello que los entornos ubicuos no están necesariamente poblados por computadoras potentes.
Por contra, los actores principales de dichos entornos suelen ser en su mayor parte dispositivos con recursos limitados (p.e. dispositivos móviles y embebidos).
Es por ello que para estos entornos es de vital importancia enfrentarse a la heterogeneidad, falta de fiabilidad y facilidad de reemplazo de dichos dispositivos.


Para hacer frente a la heterogeneidad, la web semántica propone diversos estándares y modelos para proveer un significado preciso a los términos para que puedan reusarse más allá de las fronteras marcadas por las aplicaciones.
En lo relativo a la falta de fiabilidad y facilidad de reemplazo, la computación basada en espacios (o espacios de tuplas o \aclp{ts}) promueve una coordinación desacoplada de los dispositivos.
Las soluciones basadas en espacios de tuplas semánticos, que unen los dominios de la web semántica y los \aclp{ts}, combinan estos tres aspectos beneficiosos para \ac{ubicomp}.


Muchos de estos espacios semánticos de tuplas consideran a los dispositivos móviles y embebidos como meros clientes de un espacio gestionado por dispositivos más potentes.
Dicha delegación ayuda a reducir la carga de trabajo de los dispositivos con limitaciones computacionales y energéticas.
Sin embargo, al mismo tiempo distancia los datos de donde fueron generados.
Esto crea un conflicto entre proveer datos actualizados y generar tráfico de red innecesario para información no usada.
Además, esta delegación hace a los dispositivos limitados intrínsecamente dependientes de otros cuando no siempre es necesario.
Esta tesis explora cómo esos dispositivos limitados pueden actuar como auténticos proveedores de conocimiento semántico para crear un espacio más descentralizado.


En conclusión, esta tesis describe una adaptación novedosa de los espacios semánticos de tuplas que considera el impacto computacional y energético en los dispositivos.
Específicamente, esta tesis presenta las siguientes contribuciones:
\begin{itemize}
  \item Un modelo de espacio que considera los principios que han hecho florecer a la web en las últimas décadas, junto con las propiedades de desacoplamiento de la computación basada en espacios.
  \item Un mecanismo de búsqueda energéticamente eficiente para dispositivos limitados autónomos.
  \item Un alineamiento entre dos formas de actuar en un entorno físico: actuación indirecta basada en espacios y actuación directa basada en la web.
\end{itemize}

\end{resumen}



\begin{laburpena}        %this creates the heading for the abstract page

Konputazio ubikuoak (\ac{ubicomp}) gizakientzat era hautemanezinean lan egiten duten gailuen arteko interakzioa sustatzen duten inguruneak proposatzen ditu.
Azken urteotan, \ac{ubicomp} indartu egin da, Gauzen Interneten (\ac{iot}) eta konputazio mugikorraren gorakada dela eta.
Hori dela eta, \ac{ubicomp} inguruneetako eragile nagusiak ez dira ahalmen handiko ordenagailuak, baliabide gutxikoak baizik (adibidez, kapsulatutako gailuak eta mugikorrak).
Hortaz, garrantzitsua da ingurune horietan beraien heterogeneotasuna, fidagarritasun eza eta ordezkagarritasuna kontutan hartzea.


Heterogeneotasunari aurre egiteko, web semantikoak hainbat estandar eta eredu proposatu ditu terminoei esanahi zehatza emateko eta jatorrizko aplikazioen mugetatik kanpo erabili ahal izateko.
Bestalde, fidagarritasun faltari eta ordezkatzeko erraztasunari aurre egiteko, espazioetan oinarritutako konputazioak (edo \acl{ts} delakoak) gailuen arteko koordinazio desakoplatua sustatzen du.
Tupla espazio semantikoetan oinarritutako soluzioek, hau da, web semantikoaren eta \acl{ts} delakoen domeinuak lotzen dituztenek, UbiComp-erako hiru ezaugarri onuragarri horiek bateratzen dituzte.


Kapsulatutako gailuak eta mugikorrak ahalmen handiko ordenagailuek kudeaturiko lekuen bezerotzat hartzen ditu Tupla espazio semantiko askok.
Delegazio horrek konputazio ahalmen eta autonomia  mugatuko gailuen lan karga murrizten laguntzen du.
Zoritxarrez, datuak fisikoki sortzen diren lekutik aldentzen ditu delegazio horrek.
Hori dela eta, datu eguneratuak eskaintzearen eta erabiltzen ez den informaziorako beharrezkoa ez den trafikoa sortzearen arteko gatazka sortzen da.
Gainera, delegazio horrek gailu mugatuak berez beste gailu batzuen menpe uzten ditu, beti beharrezkoa ez bada ere.
Tesi honek ikertzen du nola erabili gailu mugatuak ezagutza semantikoaren hornitzaile gisa, espazio deszentralizatuagoa lortzeko.


Ondorioz, tesi honek tupla espazio semantikoen egokitzapen  berri bat aurkezten du.
Egokitzapen horrek gailuetako energia eta konputazio eragina kontuan hartzen ditu.
Zehazki, tesiak honako ekarpenak proposatzen ditu:
\begin{itemize}
  \item Azken hamarkadetan weba arrakastatsu bihurtu duten oinarriak kontuan hartzen dituen espazio eredua eta, era berean, espazioan oinarritutako konputazioaren propietate desakoplatuak mantentzen dituena. 
  \item Gailu autonomo eta mugatuentzako energiaren aldetik eraginkorra den bilaketa mekanismoa.
  \item Ingurune fisikoa aldatzeko bi teknikaren errenkada: espazioetan oinarritutako zeharkako jarduera eta webean oinarritutako jarduera zuzena.
\end{itemize}

\end{laburpena}

%\end{abstractlongs}


% ---------------------------------------------------------------------- 


% Thesis Abstract -----------------------------------------------------


%\begin{abstractslong}    %uncommenting this line, gives a different abstract heading


\begin{abstracts}        %this creates the heading for the abstract page
% Ubicomp, mobile e IoT
Ubiquitous Computing envisions environments where devices interact among them to seamlessly work on behalf of the humans.
In recent years, the emergence of the Internet of Things (IoT) concept, which opts for connecting everyday objects to the Internet, and the Mobile Computing paradigm have contributed to strengthen UbiComp in our days.
For this reason, UbiComp environments are not necessarily populated by powerful computers.
On the contrary, mainly resource constrained devices (e.g., embedded and mobile devices) are the main actors of these environments.
% Retos de Ubicomp
Thus, it is important for the environment to deal with their heterogeneity, unreliability, and replaceability.


% Tuplespaces y semántica
To face heterogeneity, the Semantic Web has proposed several standards and models to provide a precise meaning to the terms so that they can be reused across applications boundaries.
With regards to unreliability and replaceability, space-based computing (or Tuple Spaces) promotes the uncoupled coordination of the devices.
Solutions based on semantic tuple spaces combine these three beneficial aspects, resulting from bridging the Semantic Web and Tuple Spaces domains, for UbiComp.

% Reto que supone adaptar el paradigma para cumplir los objetivos de la hipótesis
Most of these semantic tuple spaces consider embedded and mobile devices as mere clients of a space managed by more powerful devices.
Such delegation helps to reduce the workload of devices with computing and energy limitations.
However,  this delegation moves the data away from where it is physically generated.
This creates a conflict between providing updated data and generating unnecessary network traffic for unused information.
Besides, this delegation makes limited devices intrinsically dependent on each other when it might not always be necessary.
This thesis explores how these limited devices can act as fully fledged semantic knowledge providers to create a more decentralized space.

% Aspectos que se analizan en la tesis
In conclusion, this dissertation presents a novel adaptation of semantic tuple space which considers the energy and computational impact on the devices.
Specifically, this dissertation proposes the following contributions:
\begin{itemize}
  \item A space model which considers the principles which made the web flourish in the last decades, together with the uncoupling properties of the space-based computing.
  \item An energy-aware search mechanism for autonomous limited devices.
  \item An alignment of two approaches to act on the physical environment, namely a space-based indirect actuation and a web-based direct actuation.
\end{itemize}

\end{abstracts}



\begin{resumen}        %this creates the heading for the abstract page
% Pon tu resumen aquí.

En los ultimos años el aumento de dispositivos móviles y embebidos conectados a Internet ha sido notable.
En el futúro, se preveé que esta tendencia no haga más que aumentar.
Por lo tanto, la forma de integrar eficazmente estos dispositivos y las aplicaciones construidas sobre ellos resulta vital.


El paradigma de \aclp{ts} constituye una forma indirecta y altamente desacoplada de comunicación entre los dispositivos.
En este paradigma los procesos se coordinan escribiendo y leyendo en un espacio común de forma asociativa.


%Otra piedra angular de esta tesis es el uso de 
Por otro lado, la web semántica permite describir el conocimiento compartido en dichos espacios.
De esta forma, se busca reaprovechar el conocimiento de forma que se promueve interoperabilidad a nivel de aplicación.
De la conjunción entre la \acf{ws} y \acf{ts} nació \ac{tsc}.


Tradicionalmente el espacio de \ac{tsc} se ha centralizado en uno o un subconjunto de los dispositivos/participantes.
No obstante, en esta tesis exploramos la completa distribución del espacio entre todos los dispositivos, incluidos móviles y dispositivos embebidos.
De esta forma se accede a los datos allí donde se generan, en los dispositivos directamente conectados a los sensores y actuadores.
Además, se ajusta de forma natural a escenarios de movilidad donde los dispositivos transportan sus datos de un espacio a otro.
Pero esta distribución conlleva no pocos retos.
En esta tesis contemplamos ambas la búsqueda en dichos espacios y la actuación a través de la escritura en los mismos.


Para la búsqueda dentro de los espacios de conocimiento, proponemos una arquitectura que permita tener noción de lo que otros nodos comparten.
De esta forma se puede dirigir la busqueda a aquellos dispositivos que tienen datos relevantes para una consulta dada.
Así, se pretende perseguir un menor nivel de actividad en dispositivos con limitaciones de autonomía energética.
Para la compartición de dichos datos, se elige dinámicamente un dispositivo acorde a sus capacidades.
En cualquier caso, este nodo no intermedia entre los consumidores y los proveedores a la hora de realizar una consulta, de forma que se promueve la comunicación directa entre los dispositivos.


Para actuar sobre el entorno a través de la escritura en el espacio proponemos y comparamos dos alternativas.
La primera consiste en escribir tareas en el espacio que otros dispositivos leen y resuelven.
Los resultados para dichas tareas son escritos de nuevo en el espacio y leídos por el dispositivo de origen.
Para la correcta coordinación entre el nodo que realiza la tarea y el que la pide, se necesita un mecanismo de suscripción.


La segunda aproximación trata de razonar sobre la descripción semántica de servicios REST de otros dispositivos.
Con ellos, consigue saber de qué forma los debe invocar para conseguir el estado final deseado.


\bigskip


Además, en esta disertación exploramos la compatibilidad de \ac{tsc} con la web de las cosas (\acs{wot}).
La web de las cosas busca integrar los dispositivos de forma natural con la web.
Concretamente, propone embeber servidores web en los dispositivos para que estos faciliten sus recursos siguiendo el estilo RESTful.
Esto tiene la ventaja de que se integra de forma natural con la web, un sistema que se ha mostrado eficaz en terminos de escalabilidad y con el que la mayoría de usuarios está habituado.
En esta tesis defendemos que permitiendo que los dispositivos faciliten sus datos acorde con el estilo REST, podemos lograr compatibilidad con la inminente web de las cosas.


Por otro lado, el uso de REST sobre HTTP, facilita la adopción de nuestro middleware en nuevas plataformas.
La mayoría de plataformas móviles o embebidas ofrecen hoy en día multitud de herramientas y librerías HTTP.
Finalmente, cabe destacar el hecho de que estas ideas han sido implementadas en un middleware open-source disponible para multiples plataforma denominado \emph{Otsopack}.


\end{resumen}



\begin{laburpena}        %this creates the heading for the abstract page
% Jarri zure laburpena hemen.


Vivamus ullamcorper erat in sapien dignissim pellentesque.


\end{laburpena}

%\end{abstractlongs}


% ---------------------------------------------------------------------- 

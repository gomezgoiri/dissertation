
% Thesis Abstract -----------------------------------------------------


%\begin{abstractslong}    %uncommenting this line, gives a different abstract heading


\begin{abstracts}        %this creates the heading for the abstract page
% Ubicomp, mobile e IoT
\acf{ubicomp} envisions environments where devices interact among them to seamlessly work on behalf of the humans.
% TODO TODO TODO In recent years,... in our days.
In recent years, the emergence of the \acf{iot} concept, which opts for connecting everyday objects to the Internet, and the Mobile Computing paradigm have contributed to strengthen UbiComp in our days.
For this reason, UbiComp environments are not necessarily populated by powerful computers.
On the contrary, mainly resource constrained devices (e.g., embedded and mobile devices) are the main actors of these environments.
% Retos de Ubicomp
Thus, it is important for the environment to deal with their heterogeneity, unreliability, and replaceability.


% Tuplespaces y semántica
To face heterogeneity, the Semantic Web has proposed several standards and models to provide a precise meaning to the terms so that they can be reused across applications boundaries.
Regarding unreliability and replaceability, space-based computing (or Tuple Spaces) promotes the uncoupled coordination of the devices.
Solutions based on semantic tuple spaces combine these three beneficial aspects, resulting from bridging the Semantic Web and Tuple Spaces domains, for UbiComp.

% Reto que supone adaptar el paradigma para cumplir los objetivos de la hipótesis
Most of these semantic tuple spaces consider embedded and mobile devices as mere clients of a space managed by more powerful devices.
Such delegation helps to reduce the workload of devices with computing and energy limitations.
However,  this delegation moves the data away from where it is physically generated.
This creates a conflict between providing updated data and generating unnecessary network traffic for unused information.
% TODO TODO TODO each other no!
Besides, this delegation makes constrained devices intrinsically dependent on each other when it might not always be necessary.
This dissertation explores how these constrained devices can act as fully fledged semantic knowledge providers to create a more decentralized space.

% Aspectos que se analizan en la tesis
In conclusion, this dissertation presents a novel adaptation of semantic tuple space which considers the energy and computational impact on the devices.
Specifically, this dissertation proposes the following contributions:
\begin{itemize}
  \item A space model which considers the principles which made the web flourish in the last decades, together with the uncoupling properties of the space-based computing.
  \item An energy-aware search mechanism for autonomous constrained devices.
  \item An alignment of two approaches to act on the physical environment, namely a space-based indirect actuation and a web-based direct actuation.
\end{itemize}

\end{abstracts}



\begin{resumen}        %this creates the heading for the abstract page
La computación ubicua (\acs{ubicomp}) vislumbra entornos donde los dispositivos interactuan entre sí para trabajar de forma imperceptible para los seres humanos.
En los últimos años, la emergencia del Internet de las Cosas (\acs{iot}), que aboga por conectar objetos cotidianos a Internet, y la computación móvil han contribuído a fortalecer \ac{ubicomp}.
Por ello, los entornos ubicuos no se componen necesariamente de computadoras potentes.
Por el contrario, los actores principales de dichos entornos suelen ser en su mayor parte dispositivos con recursos limitados (p.e. dispositivos móviles y embebidos).
Es por ello que es importante para estos entornos enfrentarse a la heterogeneidad, falta de fiabilidad y facilidad de reemplazo de estos dispositivos.


Para enfrentarse a la heterogeneidad, la web semántica propone diversos estándares y modelos para proveer un significado precido a los términos para que puedan reusarse más allá de las fronteras marcadas por las aplicaciones.
Relativo a la falta de fiabilidad y facilidad de reemplazo, la computación basada en espacios (o \acl{ts}) promueve una coordinación desacoplada de los dispositivos.
Las soluciones basadas en espacios de tuplas semánticos combinan estos tres aspectos beneficiosos para \ac{ubicomp} uniendo los dominios de la web semántica y los \acl{ts}.


Muchos de estos espacios de tuplas semánticos consideran a los dispositivos móviles y embebidos como meros clientes de un espacio gestionado por dispositivos más potentes.
Dicha delegación ayuda a reducir la carga de trabajo de los dispositivos con limitaciones computacionales y energéticas.
Sin embargo, esta delegación distancia los datos de donde son generados.
Esto crea un conflicto entre proveer datos actualizados y generar tráfico de red innecesario para información no usada.
Además, esta delegación hace a los dispositivos limitados intrinsicamente dependiente de otros cuando no siempre es necesario.
Esta tesis explora cómo esos dispositivos limitados pueden actuar como autenticos proveedores de conocimiento semántico para crear un espacio más descentralizado.


En conclusión, esta tesis presenta una adaptación novedora de los espacios de tuplas semánticos que considera el impacto computacional y energético en los dispositivos.
Especificamente, esta disertación propone las siguientes contribuciones:
\begin{itemize}
  \item Un modelo de espacio que considera los principios que han hecho florecer a la web en las últimas décadas, junto con las propiedades de desacoplamiento de la computación basada en espacios.
  \item Un mecanismo de búsqueda energéticamente eficiente para dispositivos limitados autónomos.
  \item Un alineamiento entre dos formas de actuar en un entorno físico: actuación indirecta basada en espacios y actuación directa basada en la web.
\end{itemize}

\end{resumen}



\begin{laburpena}        %this creates the heading for the abstract page
% Jarri zure laburpena hemen.


Vivamus ullamcorper erat in sapien dignissim pellentesque.


\end{laburpena}

%\end{abstractlongs}


% ---------------------------------------------------------------------- 

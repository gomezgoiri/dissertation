% this file is called up by thesis.tex
% content in this file will be fed into the main document

% Glossary entries are defined with the command \nomenclature{1}{2}
% 1 = Entry name, e.g. abbreviation; 2 = Explanation
% You can place all explanations in this separate file or declare them in the middle of the text. Either way they will be collected in the glossary.

% required to print nomenclature name to page header
\markboth{\MakeUppercase{\nomname}}{\MakeUppercase{\nomname}}

% ----------------------- contents from here ------------------------
%

%
%

% Usando ACRONYMS, puedo definir palabras rollo Tuple Spaces para ser uniforme a lo largo del texto.
% Así evito acabar llamandolas tspaces, tuplespaces, o tuple spaces.

%% acronyms
%% http://staff.science.uva.nl/~polko/HOWTO/LATEX/acronym.html
%\acrodef{label}[acronym]{written out form}
\acrodef{ts}[TS]{Tuple Space}
\acrodef{tsc}[TSC]{Triple Space Computing}
\acrodef{ami}[AmI]{Ambient Intelligence}
\acrodef{ubicomp}[UbiComp]{ubiquitous computing}
\acrodef{wot}[WoT]{Web of Things}
\acrodef{iot}[IoT]{Internet of Things}



\nomenclature{AmI}{Ambient Intelligence}
\nomenclature{IP}{Internet Protocol}
\nomenclature{WoT}{Web of Things}
\nomenclature{TS}{Tuple Space}
\nomenclature{TSC}{Triple Space Computing}
\nomenclature{UbiComp}{Ubiquitous Computing}
\nomenclature{HTML}{HyperText Markup Language}
\nomenclature{JSON}{JavaScript Object Notation}



% si quiero gustarme: http://www.latex-community.org/know-how/263-glossaries-nomenclature-lists-of-symbols-and-acronyms
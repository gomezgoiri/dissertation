
% Usando ACRONYMS, puedo definir palabras rollo Tuple Spaces para ser uniforme a lo largo del texto.
% Así evito acabar llamandolas tspaces, tuplespaces, o tuple spaces.
% Comprobar sobre texto escrito que uso variantes de \ac siempre
% Faltan por comprobar los capítulos 4, 5 y 6

%% acronyms
%% http://staff.science.uva.nl/~polko/HOWTO/LATEX/acronym.html
%\acrodef{label}[acronym]{written out form}

\chapter*{Acronyms}
\addcontentsline{toc}{chapter}{Acronyms}

\begin{acronym}
  \acro{ami}[AmI]{Ambient Intelligence}
  \acro{api}[API]{application programming interface}
  \acro{coap}[CoAP]{Constrained Application Protocol}
  \acro{dht}[DHT]{Distributed hash table}
  \acro{dns}[DNS]{Domain Name System}
  \acro{dns-sd}[DNS-SD]{DNS service discovery}
  \acro{ws}[SW]{Web Semántica}
  \acro{dpws}[DPWS]{Device Profile for Web Services}
  \acro{fol}[FOL]{first-order logic}
  \acro{html}[HTML]{Hypertext Markup Language}
  \acro{http}[HTTP]{Hypertext Transfer Protocol}
  \acro{iot}[IoT]{Internet of Things}
  \acro{lod}[LOD]{Linked Open Data}
  \acro{mdns}[mDNS]{Multicast DNS}
  \acro{mime}[MIME]{Multipurpose Internet Mail Extensions}
  \acro{n3}[N3]{Notation 3}
  \acro{nb}[NB]{negative broadcasting}
  \acro{owl}[OWL]{Web Ontology Language}
  \acro{p2p}[P2P]{peer-to-peer}
  \acro{rdf}[RDF]{Resource Description Framework}
  \acro{rdfs}[RDFS]{RDF Schema}
  \acro{rest}[REST]{REpresentational State Transfer} % OJO aquí, creo que uso REST y RESTful indistintamente a lo largo del texto => comprobarlo!
    % Stylos de los que hereda REST
    \acro{restcs}[CS]{Client-server}
    \acro{rests}[S]{Stateless}
    \acro{restcache}[\$]{Cache}
    \acro{restu}[U]{Uniform interface}
    \acro{restid}[ID]{Identification of resources}
    \acro{restrep}[REP]{Manipulation of resources through representations}
    \acro{restdesc}[DESC]{Self-descriptive messages}
    \acro{resthateoas}[HATEOAS]{Hypermedia as the engine of application state}
    \acro{restl}[L]{Layered system}
    \acro{restcod}[COD]{Code on Demand} % en los títulos lo pone sin, el en texto con guines: code-on-demand
  \acro{sparql}[SPARQL]{SPARQL Protocol and RDF Query Language}
  \acro{soap}[SOAP]{Simple Object Access Protocol}
  \acro{sw}[SW]{Semantic Web}
  \acro{ts}[TS]{Tuple Space}
  \acro{tsc}[TSC]{Triple Space Computing}
    % Semantic TS middlewares
    \acro{sws}[SWS]{Semantic Web Spaces}
    \acro{cspaces}[CSpaces]{Conceptual Spaces}
    \acro{stuples}[STuples]{Semantic Tuple Spaces}
    %\acro{tscpp}[tsc++]{tsc++}
    \acro{tripcom}[TripCom]{Triple Space Communication}
    \acro{tscm}[TSC]{Triple Space Computing}
    \acro{tscpp}[tsc++]{tsc++}
    \acro{smartm3}[smartm3]{Smart-M3}
  \acro{ubicomp}[UbiComp]{ubiquitous computing}
  \acro{upnp}[UPnP]{Universal Plug and Play}
  \acro{uri}[URI]{Uniform Resource Identifier}
  \acro{url}[URL]{Uniform Resource Locator}
  \acro{wot}[WoT]{Web of Things}
  \acro{wp}[WP]{White Page}
  \acro{wsdl}[WSDL]{Web Services Description Language}
  \acro{www}[WWW]{World Wide Web} % sinonimo de web
\end{acronym}

% si decido gustarme: http://www.latex-community.org/know-how/263-glossaries-nomenclature-lists-of-symbols-and-acronyms
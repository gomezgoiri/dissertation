
% Usando ACRONYMS, puedo definir palabras rollo Tuple Spaces para ser uniforme a lo largo del texto.
% Así evito acabar llamandolas tspaces, tuplespaces, o tuple spaces.
% Comprobar sobre texto escrito que uso variantes de \ac siempre
% Faltan por comprobar los capítulos 4, 5 y 6

%% acronyms
%% http://staff.science.uva.nl/~polko/HOWTO/LATEX/acronym.html
%\acrodef{label}[acronym]{written out form}

\chapter*{Acronyms}
\addcontentsline{toc}{chapter}{Acronyms}

\begin{acronym}
  \acro{ts}[TS]{Tuple Space}
  \acro{tsc}[TSC]{Triple Space Computing}
  \acro{ami}[AmI]{Ambient Intelligence}
  \acro{ubicomp}[UbiComp]{ubiquitous computing}
  \acro{wot}[WoT]{Web of Things}
  \acro{iot}[IoT]{Internet of Things}
  \acro{sw}[SW]{Semantic Web}
  \acro{www}[WWW]{World Wide Web} % sinonimo de web
  \acro{http}[HTTP]{Hypertext Transfer Protocol}
  \acro{uri}[URI]{Uniform Resource Identifier}
  \acro{mime}[MIME]{Multipurpose Internet Mail Extensions}
  \acro{wsdl}[WSDL]{Web Services Description Language}
  \acro{soap}[SOAP]{Simple Object Access Protocol}
  \acro{rest}[REST]{REpresentational State Transfer} % OJO aquí, creo que uso REST y RESTful indistintamente a lo largo del texto => comprobarlo!
  \acro{dpws}[DPWS]{Device Profile for Web Services}
\end{acronym}

% si decido gustarme: http://www.latex-community.org/know-how/263-glossaries-nomenclature-lists-of-symbols-and-acronyms
\section{Evaluation}
\label{sec:middleware_qualitative_evaluation}

With the model presented in this chapter, we aimed to retain the desirable properties of both \ac{rest} and \ac{tsc}. % model, architecture, solution, ¿?
% analísis más largo en la siguiente sección
% 1. cómo sé que esto es mejor que usar REST y TSC por separado? Sinergia?
% 2. que aporta unirlo bajo un mismo middleware?
% 3. tiene beneficios?
However, some questions arise from this integration:
\begin{enumerate*}[label=\itshape(\arabic*\upshape)]
  \item does in fact retain all the properties of \ac{rest} and \ac{tsc} separately? Otherwise, which properties are affected?
  \item which other benefits offers compared with the usage of separate \ac{tsc} and \ac{rest} middleware?
\end{enumerate*}

% Fielding distingue entre:
%      + "distributed-system": hace que el usuario vea el sistema como un todo, sin saber cuando se realizan llamadas en remoto y cuando no
%      + "networking-system": no necesariamente oculta las particularidades de la red <- el se centra en estos


To answer these questions, we analyse the presented solution from different point-of-views:
\begin{enumerate*}[label=\itshape(\arabic*\upshape)]
  \item its coordination properties (Section~\ref{sec:coordination_properties}),
  \item its network-level properties (Section~\ref{sec:network_properties}),
  \item how the latter work together to contribute to the challenges of an \ac{ubicomp} environment (Section~\ref{sec:middleware_properties}).
\end{enumerate*}


% Cosas rescatables del paper original de WoT 2011 que no son tocadas en este capítulo:
%     Uniform API vs. specific usage of each API !!!
%         esto sería como domain model vs data model
%     Scalability: se ha hablado y ahora no es tan mala como antes ;-) Entre cacharros seguiría dando asco
%     Semantics: Microformats vs Full
%     Comprensión por parte de usuario: sencillo misma interfaz, no tienes que andar descubriendo nuevas APIs y viendo cómo funcionan, qué formato devuelven, etc.
%     Comprensión por parte de desarrollador: API sencillica siempre


\subsection{Coordination properties}
\label{sec:coordination_properties}

% TODO añadir la autonomía que se refería a la semántica?
As we have already discussed in Section~\ref{sec:soa_integration}, indirect communication middleware can have two key properties:
\begin{itemize}
  \item The sender does not need to know the receiver or receivers and vice versa (i.e. \emph{space uncoupling}).
  \item Senders and receivers do not need to exists in the same time to communicate with each other (i.e. \emph{time uncoupling}).
\end{itemize}


The primitives used in our space-based computing implementation force it to be \emph{space uncoupled}.
On the contrary, it is not always \emph{time uncoupled}.
If two nodes write and read from the \coordspace{}, they are \emph{time uncoupled};
but if a node accesses to others' content (i.e. \selfgraphs{}), they are not.
In other words, \emph{time uncoupling} is not achieved by the extension of the model presented in Section~\ref{sec:federated_space} (i.e. in the \emph{outer space}).
As a future work, this could be alleviated by a caching mechanism implemented in the \coordinator{} or \coordinators{}. % TODO hablar con propiedad de esos elementos => nombrarlos antes
% see \ref{tab:middleware_coordinationprop}

\InsertTab{tab:middleware_coordinationprop}{Uncoupling levels achieved by the different parts of the middleware presented}{}{
  % Tablita de cómo hereda propiedades de los estilos anteriores?
  % (y si quieren más información, que miren en la tesis de Fielding)
  \begin{tabular}{lcc}
    \hline
    ~ & Space &	Time \\
    ~ & uncoupling & uncoupling \\
    \hline
    % O mejor ponerlo con Yes y No?
    Coordination space & $\checkmark$ & $\checkmark$ \\
    Outer space & $\checkmark$ & × \\
    \hline
  \end{tabular}
}{htbp}


\subsection{Networking properties} % Architectural Properties for Network-based Styles
\label{sec:network_properties}
% Cómo afecta la app propuesta => ir actualizando al final de otras secciones???

Section~\ref{sec:soa_tsc_discussion} described how \ac{tsc} does not intrinsically contradict any of the \ac{rest} principles.
However, the adaptation presented in this thesis does not completely adhere to the \ac{rest} style.
This section presents how these divergences affect to \ac{rest}'s properties.
% decir que el objetivo final es que pudiera ser REST


In \citeauthor{fielding_architectural_2000}'s words, the relevant properties which describe a network-based system are the following ones \footnote{We refer to the reader to \citet{fielding_architectural_2000} for the complete thorough analysis.}:
% verificar que no se parece demasiado a sus definiciones o sino poner cursiva.
\begin{description}
  \item[Performance] is divided into network performance, user-perceived performance and efficiency.
    \begin{description}
      \item[Network performance] is affected by the number of interactions and the granularity of data elements.
      \item[User-perceived performance] refers to the impact a user in front of an application perceives. % UP Performance es una feature que no aporta mucho a nuestro caso
      \item[Efficiency] is achieved by minimizing the use of the network.
    \end{description}
  \item[Scalability] measures how an architecture supports a big amount of components and interactions between them.
  \item[Simplicity] is achieved through the separation of concerns for the components and the generality of architectural elements.
  \item[Modifiability] indicates how easily gradual changes can be introduced in the system.
                       These changes contribute to form different implementations which should coexist.
                       % Dynamic modifiability avoids the whole system to restart.
    \begin{description}
      \item[Evolvability] refers to the degree in which a component can be implemented without negatively impacting on others.
      \item[Extensibility] measures the ability to add functionality to the system.
      \item[Customization] is the ability to adjust the behaviour of an architectural element temporarily.
      \item[Configurability] is related with the extensibility and the reusability.
      \item[Reusability] is the ability to reuse components, connectors or data elements without modifying other applications.
    \end{description}
  \item[Visibility] is the ability of a component of monitoring or mediating in the interaction between other two components.
  \item[Portability] is the ability of working in different environments.
  \item[Reliability] is the degree in which an architecture depends on the failures of the system or components, connectors or partially incorrect data.
\end{description}

% Requisitos de ealy web, propiedades deseables
%   Low entry-barrier
%   Extensibilidad
%   Distributed Hypermedia
%   Internet-scale

The Table~\ref{tab:network_properties} summarizes how different architectural styles achieve these properties.
Particularly, it shows the styles from which \ac{rest} derives (see Section~\ref{sec:rest}).
The ultimate goal of the solution explained in Section~\ref{sec:hybrid_solution} is twofold:
\begin{enumerate}
  \item Provide a \ac{rest} access to a semantic space.
       This would ease its integration with the rest of the web.
  \item Enrich that space with the knowledge provided by other \ac{rest} \acp{api}.
\end{enumerate}
Therefore, ideally, its networking properties would be the sames as the \ac{rest} style.


% TODO Comprobar que no se salga por el borde derecho!
\InsertTab{tab:network_properties}{Properties of different architectural styles for network-based applications as defined by \citet{fielding_architectural_2000}}{
  Note that the original table has been slightly adapted.
  These adaptations are remarked inside the table. % using several footnotes.
}{
  % Tablita de cómo hereda propiedades de los estilos anteriores?
  % (y si quieren más información, que miren en la tesis de Fielding)
  \footnotesize
  \begin{tabular}{lccccccccccccc}
    Style &
    \rotatebox{90}{Net Perform} &
    \rotatebox{90}{UP Perform} &
    \rotatebox{90}{Efficiency} &
    \rotatebox{90}{Scalability} &
    \rotatebox{90}{Simplicity} &
    \rotatebox{90}{Evolvability} &
    \rotatebox{90}{Extensibility} &
    \rotatebox{90}{Customiz.} &
    \rotatebox{90}{Configur.} &
    \rotatebox{90}{Reusability} &
    \rotatebox{90}{Visibility} &
    \rotatebox{90}{Portability} &
    \rotatebox{90}{Reliability} \\
    \hline
    CS & ~ & ~ & ~ & $+$ & $+$ & $+$ & ~ & ~ & ~ & ~ & ~ & ~ & ~ \\
    S\footnote{\emph{S} represents the difference between \emph{CSS} and \emph{CS} in \citep{fielding_architectural_2000}.}
      & $-$ & ~ & ~ & $+$ & ~ & ~ & ~ & ~ & ~ & ~ & $+$ & ~ & $+$ \\ % = CSS - CS
    \$ & ~ & $+$ & $+$ & $+$ & $+$ & ~ & ~ & ~ & ~ & ~ & ~ & ~ & ~ \\
    \hline
    Early web\footnote{Corresponds to the \emph{C\$SS} style in \citep{fielding_architectural_2000}.}
      & $-$ & $+$ & $+$ & $++$ & $+$ & $+$ & ~ & ~ & ~ & ~ & $+$ & ~ & $+$ \\ % = C$SS
    L & ~ & $-$ & ~ & $+$ & ~ & $+$ & ~ & ~ & ~ & $+$ & ~ & $+$ & ~ \\ % = LS
    COD & ~ & $+$ & $+$ & $+$ & $\pm$ & ~ & $+$ & ~ & $+$ & ~ & $-$ & ~ & ~ \\
    \hline
    LCODC\$SS & $-$ & $++$ & $++$ & $+4+$ & $+\pm+$ & $++$ & $+$ & ~ & $+$ & $+$ & $\pm$ & $+$ & $+$ \\
    U\footnote{Although it is not explicitly included in the original table, \emph{U} has been derived from \citeauthor{fielding_architectural_2000}'s description.}
      & ~ & ~ & ~ & ~ & $+$ & ~ & ~ & ~ & ~ & $+$ & $+$ & ~ & ~ \\ % Uniform Interface (simple, visible, reusable) && en el texto dice que degrada efficiency
    \hline
    REST\footnote{Derived from the addition of \emph{U} to \emph{LCODC\$SS}.} % esto se podría intuír por la línea, pero quién sabe..
      & $-$ & $++$ & $++$ & $+4+$ & $+\pm++$ & $++$ & $+$ & ~ & $+$ & $++$ & $+\pm$ & $+$ & $+$ \\ % = LCODC$SS + U
%	TSC (own) & × & × & × & × & × & × & × & × & × & × & × & × & ×\\ % mejor describirlo sólo con palabras que meterme en berenjenales
    \hline
  \end{tabular}
}{htbp}



\subsubsection{\acs{rest} or \acs{rest}-like \ac{api}?}

Defining a 100\% \ac{rest} compliant \ac{api} may be challenging.
Indeed, as explained in Section~\ref{sec:rest}, most of the self-proclaimed \ac{rest}ful \acp{api} are not \citep{house_how_2012}.
Most of them fail to achieve the \ac{resthateoas} constraint.


% semantic-based \ac{api} is not easy. => referencias a los que lo han intentado, a que no hay hipertexto, a que te puede interesar inferir con bastantes
Regarding the \ac{sw}, some recent efforts have tried to move closer to the \emph{hypermedia} constraint \citep{steiner_fulfilling_2011,kjernsmo_necessity_2012}.
However, these worlds remain quite isolated.
In fact, \ac{sw} \acp{api} usually present another main divergence with the \ac{rest} style: the use of query endpoints.
% i.e. no devuelven listas de enlaces, sino todo a lo bruto
% esto quiere decir que no se devuelve todo y ala, tu procesa esa burrada de datos, más bien se filtran
These endpoints intend to solve some inefficiency issues which \ac{rest} shows when working with a big amount of data. % TODO mencionar \ac{lod} en algún punto de aquí?
To that end, they allow expressive query languages and offer results where the boundaries of the different resources often blurs \citep{wilde_restful_2009}.
%The most paradigmatic case may be to decide to which graph does a inferred content belongs. % ejemplo de lo anterior

% TODO TODO TODO poner algo de data model vs domain model??


% Aterrizar esto a IoT y a nuestro caso:
%        La eficiencia es importante => para evitar hacer 500 llamadas a un cliente
%        La reusabilidad es importante también => para potencialmente adaptarse a cualquier WoT semántico
%           en el futúro => más trabajo en este área
%       Solución de compromiso:
%           nivel de madurez X de Richardson. Suponemos que alguien debe serguir nuestro API mínimo en los objetos WoT

In the \ac{ubicomp} both the efficiency on the communications and the reusability of the \ac{api} are important:
\begin{itemize}
  \item \emph{Efficiency}: mobile and embedded devices have restricted energy autonomy.
                    This autonomy is severely affected by network communications. % TODO cite
                    Particularly, the access to the data they provide by means of hypertext may result in many \ac{http} requests. % aunque eso se puede adaptar por lo visto
                    %  Por que sería muy costoso hacer rollo araña?!
		    %     lo que ganas de reusabilidad, lo aumentas en complejidad en el cliente
                    %     interpretar código del servidor no es tampoco super-sencillo (xHTML!)
  \item \emph{Reusability}: due to the heterogeneity of the devices, assuming they all share a common and unevolvable \ac{api} may not be realistic.
			    Normally, an environment will be populated by different \acp{api} from many \ac{wot} solutions.
			    In that situation, allowing a client to autonomously learn how to use them would be ideal.
\end{itemize}


We opt for promoting efficiency at expense of reusability.
In other words, we use two \ac{rest}-like \acp{api} whose use is known out-of-band by the clients (i.e. they are not guided by the \emph{hypermedia}). % presented in Section~\ref{} and the \emph{OSAPI}
However, we provide a human-oriented version of our \ac{api} to ease its use by developers which does comply with \ac{resthateoas}.% pero aún así el API de TSC seguirá siendo HATEOAS para los usuarios, no para las máquinas
The evolution of the \acp{api} to fully \ac{rest}ful machine-oriented ones is left as future work. % as the \ac{sw} and \ac{hateoas} evolve

\bigskip

In conclusion, the properties achieved by the network communication style selected corresponds with \emph{LCODC\$SS} plus simplicity and visibility (see Table~\ref{tab:network_properties}). % hacer una nueva fila con esto
% TODO comprobar que se explica eso tal cual en la tesis de Fielding: HATEOAS => sólo afecta a reusabilidad

% Reusar algo de esto?:
%     Discovery: HATEOAS puede ser complejo para cacharros peques (parseo),
%                proponemos discovery basada en API uniforme y a nivel de cacharros con algún mecanismo en el que no entramos.
%          One of the main drawbacks in \ac{wot} is the lack of a discovery mechanism for new objects and the data they provide.
%          Even when this data can be linked in each object response (using HATEOAS) and microformats are sometimes included to ease the search-ability of these objects by search engines, it is difficult for an object which may change of location and context to be referred.
%          Thus, \ac{wot} may have a tendency to create isolated islands of data.
%          Several workarounds have been proposed to overcome this limitation, such as using a central repository\footnote{http://www.pachube.com/}, a framework which uses federated repositories responsible for different administrative domains \citep{stirbu_towards_2008} or making each connected sensor announce itself to let an intermediary know its presence \citep{kamilaris_smart_2010}.



\subsection{Properties for \acs{ubicomp}}
\label{sec:middleware_properties}

A middleware is a software layer which provides a higher level of abstraction and masks the underlying heterogeneity.
The middleware presented in this thesis is oriented for \ac{ubicomp} environments and the devices which populate them (particularly mobile and embedded devices).
Whereas the devices part of the \ac{iot} are a subset of the ones present in \ac{ubicomp}, we observed that the challenges they have to face are the same ones.
For instance, smartphones are not part of the \ac{iot}, but face similar energy and computational limitations as the embedded devices. % posiblemente no todas las propiedades!

% ¿Qué propiedades deseables para IoT añadiríamos?
Therefore, we will consider the challenges identified by the \emph{Internet-of-Things Architecture} European project \citep{walewski_project_2011} to analyse our solution.
\citeauthor{walewski_project_2011} state that the \ac{iot} must overcome the following challenges:
interoperability, scalability, manageability, mobility, security and privacy and reliability.
Furthermore, we also consider energetic and computational constraints which devices in \ac{ubicomp} have to face. % ponerlo de forma que sea menos pegote?


% TODO cómo las distintas propiedades vistas antes se combinan para aportar a estas
Table~\ref{tab:middleware_netprop}, shows how the properties explained in the previous sections directly affect these challenges.


% TODO Comprobar que no se salga por el borde derecho!

% Tablita de cómo hereda propiedades de los estilos anteriores?
% (y si quieren más información, que miren en la tesis de Fielding)
\begin{savenotes} % to use footnotes inside
  \begin{table}[htbp]
    \caption{Direct relations between network and coordination properties and challenges for lightweight middleware.}
    \begin{center}
      \begin{tabular}{lcccccccp{0.3cm}ccp{0.3cm}c}
        ~ & \multicolumn{7}{c}{\multirow{2}{*}{Network prop.}}& ~ & \multicolumn{2}{c}{Coord.} & ~ & ~\\
        ~ & \multicolumn{7}{c}{~} & ~ & \multicolumn{2}{c}{prop.} & ~ & ~\\[0.3cm]
        \cline{2-8}\cline{10-11} % column 8 only exists to create that separation
	~ &
	\rotatebox{90}{Performance} &
	\rotatebox{90}{Scalability} &
	\rotatebox{90}{Simplicity} &
	\rotatebox{90}{Modifiability} &
	\rotatebox{90}{Visibility} &
	\rotatebox{90}{Portability} &
	\rotatebox{90}{Reliability} &
	~ &
	\rotatebox{90}{Space uncoupl.} &
	\rotatebox{90}{Time uncoupl.} &
	~ &
	\rotatebox{90}{Semantic Web} \\ % TODO cómo hacer para que esto tenga sentido y esté a la misma altura
	\hline
	Interoperability & ~ & ~ & ~ & × & ~ & × & ~ & ~ & ~ & ~ & ~ & × \\ % ×
	Scalability & ~ & × & ~ & ~ & ~ & ~ & ~ & ~ & ~ & ~ & ~ & ~\\
	Manageability & ~ & ~ & ~ & ~ & ~ & ~ & ~ & ~ & ~ & ~ & ~ & ~\\
	Mobility & ~ & ~ & ~ & ~ & ~ & ~ & ~ & ~ & × & × & ~ & ~\\
	Security &  ~ & ~ & ~ & ~ & ~ & ~ & ~ & ~ & ~ & ~ & ~ & ~\\
	Reliability & ~ & ~ & ~ & ~ & ~ & ~ & × & ~ & ~ & × & ~ & ~\\
	Restricted energy & \multirow{2}{*}{×} & ~ & ~ & ~ & ~ & ~ & ~ & ~ & ~ & ~ & ~ & \multirow{2}{*}{×}\\ % también simplicity
	~~\& comp. & ~ & ~ & ~ & ~ & ~ & ~ & ~ & ~ & ~ & ~ & ~ & ~ \\
	\hline
      \end{tabular}
    \end{center}
    \label{tab:middleware_netprop}
  \end{table}
\end{savenotes}


% TODO ver si tiene sentido hablar de estas cosas
% explicar cómo HTTP trataba de evitar complejidades
% Propiedades que jodes de HTTP de cara al usuario: simplicidad...
% explicar como al crear una capa por encima, complicas todo un poco
%       sencillez
%       user-based library vs a lo otro
% drawbacks frente a HTTP: sencillez, etc.
%A middleware is a software layer which provides a higher level of abstraction and masks the underlying heterogeneity.
%\ac{http} is a middleware indeed, but sacrifices masking exceptions on behalf of the simplicity. % TODO checkear con la tesis de Fielding
% hablar de user-based library vs a lo otro


\subsubsection{Interoperability}

% 2 cosas contribuyen a esta propiedad:
%     Uso de HTTP
%     Semántica => ponerlo como una autonomía (encaja bien en la tablita) o no (en realidad se podría usar tanto sobre TSC como sobre HTTP)?

Interoperability is the key to deal with a wide range of heterogeneous technologies.
The use of both the \textbf{network style} and the \textbf{semantic data} contribute to this challenge.


\paragraph{Semantics}

Using semantics we promote the reuse of data describing them in a more rich and abstract way.
The semantic web is composed by a series of standardized technologies which allow to formally describe the models (i.e. the concepts and how they relate to each other).
Two key mechanisms of the Semantic Web in this aspect are the inference of new data and the mapping of equivalent models.
% hablar de cómo contribuye o no a su portabilidad??
% Besides, the Semantic Web is built on top of well-defined and vastly supported tools.

% Drawbacks
% TODO referencia al paper de qué impide conseguir challenges semánticos


\paragraph{Networking}

For the communication between nodes, we rely on the widely supported \ac{http} protocol. % usar la palabra ubiquitous o pervasive???
Communication with devices using other protocols must be done by means of specialized gateways. % out-of-the-scope
Therefore, adhering to a unique protocol does not help dealing with various communication technologies.

% opción muy común hoy en día para definir APIs
However, a wide range of current applications use \ac{http} to define their network accessible \ac{api}'s. % through so-called REST APIs
% posiblemente indirectly due to its simplicity => pervasiveness and massive adoption
This makes it a \emph{de facto} requirement to interoperate with these applications. % te da acceso a más apps % interop ab-initio
% ejemplo de ello: mashups
% también en resource constrained devices: el éxito de WoT es sólo una prueba de ello
In fact, this success has also its counterpart on resource constrained platforms.
The clearer sign for this tendency is the upcoming \ac{wot} initiative.


At a finer grained level, there are several properties which help different \ac{http} implementations to interoperate: % (of interoperability)
\begin{description}
  % \item[Simplicity] => a nivel de red no, pero un middleware simple contribuye a una mayor portabilidad % PROPIEDAD INDIRECTA
  \item[Modifiability] is the ease for introducing gradual changes in the system.
                       Particularly dynamic modifiability avoids restarting the entire system when introducing a change.
                       This property allows different implementations to coexist
                       reducing issues on the communication between two nodes.
                       
		      % mejor no meterme en fregados de justificar cada subpunto:
		      % is important due to the \emph{evolvability} and the \emph{reusability}.
		      % Evolvability allows two nodes to have components implemented at different degrees without negatively impacting each other.
		      % contribuyen a eso (sacado de Fielding):
		      %   + CS: The separation of concerns also allows the two types of components to evolve independently, provided that the interface doesn’t change.
		      %   + Layered systems reduce coupling across multiple layers by hiding the inner layers from all except the adjacent outer layer, thus improving evolvability and reusability.
		      % extensibility no: lo introduce COD, así que echa calculos de si me importa esa característica de HTTP
		      % visibility => qué tiene que ver lo fácil que es monitorizarlo con esto?
  
  \item[Portability:] \ac{http}'s portability is backed by the plethora of \ac{http} libraries and frameworks available for most of the computing platforms.
                    % Cachis, no he podido usar una palabra tan molona como: pervasiveness
                    % Justificación: permite interoperar entre distintas plataformas?
                    Consequently, a huge number of platforms can interoperate through \ac{http}.
\end{description}

\bigskip

Besides, in our \ac{http} \ac{api} the following issues remain unsolved:
% estos 2 últimos son como redefinir interop
\begin{itemize} % posiblemente incluir lo que se explica aquí en otro punto
  \item \emph{Using third party applications' data}:
  \begin{enumerate}
    \item Section~\ref{sec:network_properties} details how a pure \ac{rest} \ac{api} contributes to a higher decoupling between clients and servers.
	  In \ac{rest} architectures clients can autonomously adapt to use different \acp{api} or to different \ac{api} versions. % i.e. evolvable
	  Of course, this comes at the cost of more complex clients.
	  These clients have the additional responsibility of discerning the next state transition from the representations obtained. % cacharros parseando HTML, interpretandolo, etc.
	  
	  Our middleware is not hypermedia-driven.
	  As a consequence, it requires third party applications to implement a uniform \ac{api} (i.e. \osapi{}) on top of them to allow our middleware consume their data.
	  However, the \ac{api} is resource-oriented and simple to implement.
	  
    \item Another obstacle for achieving a real application-level interoperability in our middleware is the use of distinct isolated spaces.
	  In other words, two applications using two different spaces would not share their content.
	  Although this could be avoided by defining a default space, as explained in Section~\ref{sec:soa_tsc_space}, their use is justified in terms of scalability.
  \end{enumerate}
  
  \item \emph{Making our data reusable by third party applications}:
	
	We encourage sharing data in resource constrained devices through an \ac{http} \ac{api}.
	Using this \ac{api} any other application able to use \ac{http} and to manage semantics can reuse data from our middleware.
	However, since this \ac{api} is not hypermedia-driven either, 
	third party applications must know how to use it beforehand. % con todo lo que ello conlleva, no quiero volver a aburrir al lector contandolo otra vez
	% In other words, out-of-band information drives the \acs{http} interaction with this \ac{api}.
\end{itemize}


\subsubsection{Scalability}
% ¿Cómo afecta la escalabilidad a la escalabilidad? => Directamente :-S

%  + Web es escalable por diseño
The architecture of the web is scalable by design.
Indeed, the large number of components and interactions between them that coexist in the web are a good proof this.
%The server off-loads work on the clients and frees resources quickly thanks to its statelessness. % Las causas
%This contributes to the coexistence of a huge number of components and interactions between them.
These components share an even larger amount of data which leads to scalability issues on the search process \citep{krummenacher_scalability_2008}. % porque tienes que preprocesar todo el contenido a buscar
In fact, considering that in \ac{ubicomp} environments the data is constantly generated or updated, this challenge can be harder.
The search process is analysed in depth in the next chapter. % TODO IG algo mas que decir aqui? Queda un poco colgado esto, algun conector o algo con la frase anterior

% However, in our model we limit the search scope using flat spaces (see Section~\ref{sec:soa_tsc_space}). % i.e. groups of devices which share information

% Para simplificar y no adelantar discusiones:
% Besides, the search process presented in the next chapter relies in each client to autonomously search for the appropriate servers to request.
% As a consequence, each device manages a piece of information for each server in the space.
% Regarding the \coordspace{}, this dissertation simply ignores how it deals with the scalability issues (e.g. by distributing the space). % también le afecta lo del flat space

%  Idea más compleja que se ha dejado de lado tras dar vueltas:
%  + Buscar un punto de entrada inicial concreto no es tan sencillo/escalable: hay que hacer scrapping de todos los datos.
%  + Problema añadido de la web semántica:
%    -> se basa en enlaces, pero seguir una URI no necesariamente lleva al documento que tiene información sobre ella
%    -> más de un dispositivo pueden hacer referencia a una web (DERREFERENCEABLE URI)
%  + Hacer una visión del espacio, hace que no accedas por una dirección (URI), sino por un template o lo que sea.
%  + La URI de una tripleta no te permite localizarla => necesario mecanismo de búsqueda.


\subsubsection{Manageability}

Manageability helps to cope with a big number of devices allowing them to have an autonomous behaviour. % specially in environments where they cannot be controlled through centrality
It encompasses self-management, self-configuration, self-healing, self-optimization, and self-protection.
This dissertation proposes a self-managed architecture which eases searching self-managed graphs.


\subsubsection{Mobility}

The ability to face mobility situations can be improved through the decoupling brought by space-based computing.
First, mobility situations may impede two nodes coexist at the same time.
In this case, \emph{time uncoupling} enables their communication.

Second, the provider of a piece of information may vary over the time.
The primitives used care about the data and not about the specific provider of these data (i.e. the nodes are \emph{space uncoupled}).
This avoids any reconfiguration of the data consumer when the provider is replaced.
In fact, a content can be also delivered by more than a provider.


\subsubsection{Security and privacy}
% En una versión nueva hablan de Proof

This property is considered out-of-the-scope of this dissertation.
However, we are currently working on a lightweight security solution \citep{naranjo_lightweight_2012}. % TODO definir mejor esto
For our future work, we are considering to integrate it with the presented middleware.


\subsubsection{Reliability}
% definición de reliability modificada yendo a la fuente de iot-a.eu

Reliability in \ac{ubicomp} is crucial to handle connectivity losses in various \emph{ad hoc}-like ways.
% ver cómo se relaciona con REST
Particularly, \ac{http} handles it through caching.
% Honestidad: This feature has not been properly addressed in our middleware.


\subsubsection{Limited Computation}

% depende de lo limitado que sea el cacharro, puede ser tan así como XXX
Mobile and embedded devices impose computation restrictions.
These restrictions result in unacceptable response times from the application perspective.
Due to \ac{http}'s simplicity, computation complexity is found in semantic processing.
For example, some really limited platforms may found difficult parsing long files.
However, we made the following assumptions:
\begin{itemize}
  \item Data providers will be able to process wildcard-based templates and generate appropriate responses from their self-managed graphs.
  \item Data consumers are able to process the responses they receive at least in one semantic format. % por ejemplo :-P
\end{itemize}


Beyond that point, our middleware will not impose anything.
This means that a developer using the middleware cannot assume:
\begin{itemize}
  \item The inference process on data providers.
        Reasoning over the content to provide unstated knowledge must be desirable.
        Still, most of the mobile and embedded device are not able to process big amount of data. % TODO citar WoT2012
        Besides, reasoners are not available or optimized for many limited platforms.
  \item The inference process on data consumers.
        Another alternative or complement for inferring data on the providers is to do it in the consumers after receiving the responses.
        Again, it is a desirable property, but our middleware does not require it.
        % también out of the scope: razonamiento distribuído
  \item General use of more complex query languages.
        Their expressibility benefits the network performance since they can restrict the results returned.
        However, parsing them is not trivial and consequently parsing libraries are unavailable for most limited platforms.
        Therefore, a provider cannot assume that providers know how to process something beyond the wildcard based templates.
        As an alternative, providers can decompose these queries in simpler wildcard based templates and locally process the results afterwards using the complex queries as filters.
        % This is out of the scope.
	% Querying Distributedly over the Space => optamos por no meternos en jardines y up to the middleware implementer to decide each ;-)
	% más en el capítulo 4
\end{itemize}


% por eso, el mecanismo de búsqueda se hace en los cacharros más potentes
The searching mechanism explained in the Chapter~\ref{cha:searching} takes into account computing limitations.



\subsubsection{Limited Energy}

% 2 cosas contribuyen: computación y networking => TODO mostrar diagrama de FoxG20
Both computing data and networking operations directly affect energy consumption.
As an example, Figure~\ref{fig:energy_consumption_tsc_section} shows the energy consumption on an embedded platform in an inactivity period, doing networking operations and computing semantic data. % más detalles en la sección correspondiente
There are some crucial characteristics which negatively affect energy autonomy of resource constrained devices:
\begin{itemize}
  \item Too complex computation tasks.
	 % They result in excessive networking throughput in some occasions.
        As explained in the previous section, semantic processing is the clearest example of that.
        % como se ha explicado: hemos limitado
        We have limited its impact on resource constrained platforms through the searching mechanism.
  \item Regarding \ac{http}, its statelessness contributes to a bad \textbf{network performance}.
        Besides, the verboseness of most of the semantic formats also contributes negatively to that aspect.
\end{itemize}


\InsertFig{\pathchapfive/figures/PDF/energy_consumption.pdf}{fig:energy_consumption_tsc_section}{
  Energy consumption for an embedded platform during different activity periods (miliwatts).
}{
}{0.6}{}


On the other hand, the reduction of the network usage contributes to energy savings (i.e. improves \emph{network efficiency}):
% efficiency: minimizar el uso de la red
\begin{itemize}
  \item \ac{http}'s network \emph{efficiency} can be improved with \emph{code on demand} and \emph{caching}.
  % por eso, el mecanismo de búsqueda se hace en los cacharros más potentes
  \item The searching mechanism improves \emph{recall} and \emph{precision} (see Chapter~\ref{cha:searching}).
\end{itemize}



% También se podría intentar meter algo del teorema CAP (Consistencia, Availability, Partition Tolerance) de Brewer
%    No lo he visto claro y además esto ya es suficientemente extenso.


% TODO TODO TODO aquí o como anexo, meter lo de WoT 2012 para demostrar que se probó en distintas plataformas
%\subsection{Feasibility in platforms}
% ----------------------------------------------------------------------

\begin{savequote}[50mm]
The original idea of the web was that it should be a collaborative space where you can communicate through sharing information.
\qauthor{Tim Berners-Lee}
\end{savequote}


\newcommand{\codigo}[1]{``\texttt{#1}''}
\newcommand{\primquery}{\emph{query}}
\newcommand{\primread}{\emph{read}}
\newcommand{\primtake}{\emph{take}}
\newcommand{\primwrite}{\emph{write}}

% Model proposal
\newcommand{\coordspace}{\emph{coordination space}}
\newcommand{\outerspace}{\emph{outer space}}
\newcommand{\coordinator}{\emph{coordinator}}
\newcommand{\coordinators}{\emph{coordinators}}
\newcommand{\asteroids}{\emph{asteroids}}
\newcommand{\asteroid}{\emph{asteroid}}
\newcommand{\selfgraphs}{\emph{self-managed graphs}}
\newcommand{\osapi}{\emph{OSAPI}}


\chapter{Triple Spaces for Constrained Devices}
\label{cha:tsc}

% the code below specifies where the figures are stored
\ifpdf
    \graphicspath{{\pathchapfour/figures/PNG/}{\pathchapfour/figures/PDF/}{\pathchapfour/figures/}}
\else
    \graphicspath{{\pathchapfour/figures/EPS/}{\pathchapfour/figures/}}
\fi



% ----------------------------------------------------------------------



% TODO TODO TODO me he dado cuenta que si quiero interoperar, la clave sería meter en algún lado algo sobre LOD y SPARQL Update 1.1!!!
% Igual que quiero interoperar con Semántic WoT, debería interoperar con LOD !!!



In recent years, the \acf{iot} has become a reality due to the increasing number of everyday objects equipped with computing and networking capabilities.
The use of these objects together with the rising of mobile computing greatly contributes to the \ac{ubicomp} vision.
%In the beginning the community put more effort on those devices' connectivity issues, materialized in the spread of technologies such as Zigbee\footnote{http://www.zigbee.org}, 6LoWPAN\footnote{http://datatracker.ietf.org/wg/6lowpan/charter/} or Bluetooth\footnote{http://www.bluetooth.com}.
%However, nowadays it is focusing on the architectural solutions used by the applications built around these objects.
%A classification of these solutions has been already presented in the previous chapter. % Coulouris et al.
This vision can benefit from \acf{tsc} paradigm's decoupling properties. % mencionarlas? citar a alguien que diga que TSC puede molar para eso?
However, other properties are derived from how \ac{tsc} is materialized. % materialized puede sonar mal, pero implemented suena muy umpa-lumpa


This chapter focuses on this materialization, i.e. the design of our own space model.
To that end, Section~\ref{sec:guiding_principles} discusses the principles which guide our proposal.
Section~\ref{sec:hybrid_solution} proposes a mix model which encompasses
\begin{enumerate*}[label=\itshape(\alph*\upshape)]
  \item an independent space which can be used for coordination purposes, and
  \item an enriched view of this space which also shows the information provided by autonomous devices.
\end{enumerate*}
Section~\ref{sec:align_tsc_http} describes a \ac{rest}-like \ac{api} to access the first space.
Section~\ref{sec:federated_space} details the additional extensions needed to support the enriched view.
Finally, Section~\ref{sec:middleware_qualitative_evaluation} evaluates the properties achieved with this model.


\section{Guiding Principles} % o problems
\label{sec:guiding_principles}

% cómo y porqué hacer un TSC distribuído
In Section~\ref{sec:soa_tsc_discussion} we showed how \ac{tsc} can comply with all the constraints of the \ac{rest} style.
One of these constraints is that the access to a space is done in a client-server basis.
The easiest way to achieve this constraint is by centralizing all the content of each space in the server itself.


% Pero un entorno Ubicomp es eminentemente distribuído
However, data in a \ac{ubicomp} environment is not generated at a single point. %  intelligent environment
%   La información se genera en fuentes separadas, ¿cómo accedo a los últimos datos?
In fact, the plethora of sensors where data are generated can be enormous. % are => data es plural
Besides, each of these sensors may generate data continuously.
This creates a trade off between efficiency and freshness difficult to overcome:
\begin{itemize}
  \item The more frequently each sensor sends contents to the server, the more inefficient is the solution in terms of network usage.
  \item The less frequent this communication is, the less updated is the data in the server.
        This leads to spaces which misrepresent the environments.
\end{itemize}



%   ademas no practico:
%        las fuentes del conocimiento son las que mejor saben como gestionar una informacion (crearla, acualizarla o eliminarla)
%        la información se transporta en dispositivos
%        reusar datos en distintas apps
Consequently, a sensible solution is to let these sensors manage their own information.
This strategy is not only useful to ensure the access to up-to-date data, but also for the following causes: % beneficios o razones?
% This strategy == delegating responsibilities between the devices
\begin{itemize}
  \item The devices directly connected to the sensors and actuators will know how to represent these contents better than others. % ejemplo en grados centigrados o celsius?
	For example, the unit of a temperature measure.
  \item They know when to create, update or delete data. % ejemplo if a server receives 2 contradictory measures... algo
        Furthermore, they can opt for creating data on demand.
  \item Data may be reused by other spaces or even other applications. % Interop!
        These applications would not be required to use a space-based approach.
        Therefore, they will not depend on the correct functioning of our whole system. % del servidor central
  \item Carrying the information can be useful in certain mobility situations.
        For example, let us imagine a person which stores her user profile in her personal smartphone.
        She could share it with different spaces or applications as it moves along the city.
\end{itemize}


% Aún así, que los dispositivos se coordinen a través de una entidad central es conveniente.
%       Modo: caching (para reliability) o modo ver qué hay en el espacio ahora mismo (visión en tiempo real)
%       -->> INDEPENDENCIA DE LOCALIZACIÓN!!!
Still, most of the \ac{iot} devices or mobile phones are unreliable: they can join and leave at any moment.
As a result, distributing the shared space among unreliable nodes comes with a number of drawbacks:
\begin{itemize}
  \item Devices rely on the data written and read from the space to coordinate themselves.
	Therefore, the access to these pieces of information must be guaranteed regardless of dynamic devices' availability. % dynamic == unreliable
        % i.e. que no desaparezca una tarea en el espacio
  % Sencillo de implementar: notificaciones. En temas de razonamiento no he entrado.
  \item A blocking mechanism is important in space-based computing. 
        For example, a worker node may block until some tasks become available in the space.
        A way to implement it in a distributed space is by means of a notification system.
        However, the efficient implementation of this system using unreliable devices is challenging. % no sé si esto último no sonará demasiado a mofa...
\end{itemize}
Overcoming the previous difficulties, usually implies a high network traffic.
This traffic negatively influences the energy consumption of nodes whose energy autonomy might be limited.

\bigskip

In conclusion, our design must face two apparently contradictory principles:
\begin{itemize}
  \item To consider data from independent and limited data providers.
  \item To rely on the providers of the data which enables the coordination of the space participants.
\end{itemize}
\section{An Hybrid Solution} % o "A Halfway Solution"
\label{sec:hybrid_solution}

In this thesis, we propose to conserve the desirable characteristics of both \ac{rest} and \ac{tsc} by mixing them.

On the one hand, we propose a uniform access to the space by the \acs{http} \ac{api} described previously.
The space can then be distributed using any existing approaches. % TODO cite
However, for the sake of clarity, we will assume that each space is managed by a unique server.

On the other hand, each space will be enriched by the data provided by various autonomous providers.
These providers might be any kind of mobile or embedded device, no matter how limited they are.
This \emph{enrichment} can be materialized:
\begin{enumerate}[(a)]
  \item Considering additional data in a reasoning process triggered when a primitive is called.
  \item Including additional data as a response.
        For example, a graph read from an embedded device could be returned as a response to a primitive.
\end{enumerate}
In this thesis, we have focused on the latter alternative.


To do that, we propose an extension of the already presented \ac{tsc} model by means of new concepts, new primitives and new behavior.
Figure~\ref{fig:new_model} presents the key elements of this new model.


\InsertFig{new_model}{fig:new_model}{
  Key concepts of the new \ac{tsc} model presented.
}{
  The coordination space, is where the graphs can be written, read and taken by any participant.
  The coordination space is hold by a device called \emph{coordinator}.
  The current view of all the \emph{self-managed graphs} in the space forms the \emph{outer space}.
  Therefore, the \emph{outer space} is hold by many devices (also called \emph{asteroids}).
  The \ac{tsc} \ac{http} \ac{api} corresponds with the generic \ac{tsc} to access to the primitives presented in Section~\ref{sec:align_tsc_http}.
  The \emph{OSAPI} is the \ac{api} which must be implemented by any node willing to share \emph{self-managed graphs}.
}{1}{} % o halfway solution?
\section{Aligning \acs{tsc} with \acs{http}}
\label{sec:align_tsc_http}

% Otros trabajos citables parecidos a esto:
%      posiblemente el primero que expuso explicitamente semejanzas entre HTTP y primitivas de TSC: Riemer 2006 (comprobar y citar!)
%      Otros papers de TSC y la WWW, que ya han sido citados en la sección 3_tsc

This section presents our materialization of a \ac{tsc} \acs{api} over \ac{http}.
This materialization gives a practical overview of \ac{tsc}'s \acs{rest}fulness.


\subsection{\acs{tsc} resources}

Three key concepts are important to understand the resources in our proposal: agents share information in a common \textbf{space}.
A space is identified by an \acs{uri}.
Therefore, all the operations in \ac{tsc} are performed against a particular space.
%By default, all applications connect to a common standard space, but they can optionally choose to connect to a particular private space.
Within a space, the information is stored in sets of \textbf{triples} called \textbf{graphs}.
Each graph can also be identified by an \acs{uri}.
% The \acs{rdf} \textbf{triples} are the underlying concept of all the \ac{sw} languages. % a estas alturas esto está explicadísimo
Each triple is composed by a subject (which is a \acs{uri} or a \emph{blank node}), a predicate (a \acs{uri}) and an object (which can be a \acs{uri}, a \emph{blank node} or a literal), as shown in the Figure~\ref{fig:triples_example}.
% TODO cuidado con los blank nodes!!! Los tenemos en cuenta? De ser así, cómo???

As detailed later, \ac{tsc}'s primitives add or remove graphs.
% este es un tema "delicado" desde el punto de vista de TSC, así que mejor si lo tratamos a parte:
% as well as to query for graphs or for sets of triples retrieved from different graphs.
To perform these operations, which enable the selection of a subset of the semantic content hold in a given space, a \textbf{template} is required. % these operations: add or remove
The wildcard templates used are special triples with optional wildcard subject, predicate and/or object.
For example, the template \texttt{?s foaf:knows gomezgoiri:aitor} could be employed to select instances which represent people who know Aitor (see Figure~\ref{fig:tsc_resources}).

% TODO Discussión profunda de SPARQL y REST???

% "Cartoon Cloud" by egyninja is marked as "public domain": http://openclipart.org/detail/25319/cartoon-cloud-by-egyninja
\InsertFig{tsc_resources}{fig:tsc_resources}{Schematic view of a space with four graphs and sample triples for one of these graphs}{
The figure uses prefixes, i.e., aliases for the beginning of the \acp{uri}, for the sake of clarity.
}{1}{}


Note that this thesis does not consider using more sophisticated query languages like SPARQL \citeweb{sparql2008}. % sophisticated o expressive?
The rationale behind this decision is that we wanted to avoid the complexity introduced by these languages in our \ac{api}.
While the advanced query languages need to be interpreted by parsers not available in many embedded platforms,
wildcard templates are straightforward to implement and to process.
This simplicity eases the adoption of the \ac{api} by as many platforms as possible.
However, since the wildcard templates are the base for the advanced query languages, our \ac{api} could be extended to allow their use.
In any case, this extension is left as a future work.



\subsection{Adopted \acs{tsc} primitives}
\label{sec:primitives}

\acs{tsc} derives some primitives originally defined in the Linda language \citep{gelernter_generative_1985} to access the semantic information hold in each graph.

\begin{itemize}
 \item The \textbf{write} primitive allows writing a graph into a given space (identified by its \acs{uri}).
	The set of triples received by this primitive will be stored together in the same graph, returning the \acs{uri} which identifies that graph.
	The graph \acs{uri} can be used to access directly to that graph later on, or to create new triples and relate contents.
       % Esto no encaja necesariamente con el concepto de "Browsable graph": http://www.w3.org/DesignIssues/LinkedData
  \begin{lstlisting}
    write(space_URI, triples): URI              [1]
  \end{lstlisting}


  \item The \textbf{read} returns a graph belonging to a given space which contains at least a triple matching the given template or has the given \acs{uri} as its identifier.
	If more than one graph fulfils one of these conditions, just one of them is returned (nondeterministically).
	It should be remarked that it has been designed as a non blocking operation.
  \begin{lstlisting}
    read(space_URI, graph_URI): triples         [2]
    read(space_URI, template): triples          [3]
  \end{lstlisting}
  
  
% why?
% TODO READ al ser no-deterministica puede afectar a caching???
  \item The \textbf{take} primitive behaves like a destructive \textbf{read}, deleting the graph returned from the space.
  \begin{lstlisting}
    take(space_URI, graph_URI): triples         [4]
    take(space_URI, template): triples          [5]
  \end{lstlisting}

%  \item Space management primitives.
%	A node can join or leave a space using \linebreak \texttt{joinSpace(space\_URI)} or \texttt{leaveSpace(space\_URI)}.
\end{itemize}



\subsection{\acs{http} \acs{api} for \acs{tsc}}
\label{sec:httpapi}

As both \ac{tsc} and \ac{http} are \ac{rest} compliant, their similarities are evident.
In the same way \ac{tsc} has the already explained primitives, \ac{http} has verbs to get, create, update or remove resources (GET, PUT, POST, DELETE).
%The main purpose of \acs{tsc} primitives is not to access data but to coordinate the nodes accessing to that data.
Consequently, the translation between these two worlds is straightforward.

\begin{sloppypar}
According to the \ac{rest} principles the interaction with an API must be hypertext-driven \citep{fielding_rest_2008}.
To ease its usage by developers, we provide a human-oriented \acs{html} representation of the \ac{api} which is completely \emph{browseable}.
% un poco repetido de la parte de HATEOAS
Regarding the machine-oriented representation of the \ac{api}, \citet{verborgh_functional_2012} and \citet{kjernsmo_necessity_2012} have proposed solutions to enable hypermedia driven semantic \acp{api}. % mencionar que son experimentales?
Independently of the mechanism adopted, in this section we propose a optional \ac{api} which stresses \ac{tsc}'s compliance with \ac{http}.
\end{sloppypar}

The list of spaces a node is joined to are available under \textit{/spaces}.
Each space is identified by an \acs{uri} (e.g., \url{http://space1}).
If this \acs{uri} is also a \acs{url}, $/spaces/\{space\_uri\}$ (summarized by \emph{sp} from now on) can simply redirect to it.
All the resources of that space, both real (i.e., graphs) or virtual (i.e., query) are listed under $\{sp\}/$.
Each graph is available on ${\{sp\}/graph/\{graph\_uri\}}$.
If we make an \acs{http} DELETE to that resource, under \ac{tsc}'s perspective, we take that graph from the space.
The rest of the mappings are shown in the Table~\ref{tab:tscAPI}.


\InsertTab{tab:tscAPI}{\acs{http} mapping for the primitives detailed in the Section~\ref{sec:primitives}}{
  \textit{sp} is a space \acs{uri},
  \textit{g} is a graph \acs{uri}, \textit{s}, \textit{p} and \textit{o-uri} are subject, predicate and object \acsp{uri} or wildcards (represented with an as \textit{*}).
  When the template's object is a literal, it can be expressed specifying its value (\textit{o-val}) and its type (\textit{o-type}).
}{
  \begin{tabular}{llc}
      \hline
      \acs{http} request & \acs{url} & Returns \\
      \hline
      POST & \{sp\}/graphs/ & [1] \\
      GET & \{sp\}/graphs/\{g\} & [2] \\
      GET & \{sp\}/graphs/wildcards/\{s\}/\{p\}/\{o-uri\} & [3] \\
      & \{sp\}/graphs/wildcards/\{s\}/\{p\}/\{o-type\}/\{o-val\} & \\
      DELETE & \{sp\}/graphs/\{g\} & [4] \\
      DELETE & \{sp\}/graphs/wildcards/\{s\}/\{p\}/\{o-uri\} & [5] \\
      & \{sp\}/graphs/wildcards/\{s\}/\{p\}/\{o-type\}/\{o-val\} & \\
      \hline
  \end{tabular}
}


\subsubsection{Status codes}
\label{sec:status_codes}
The \acs{api} should be compliant with the standardized \acs{http} status codes \citeweb{http2008status}.
These codes are sent back in the response as part of the header.
For instance, the \ac{tsc} middleware returns the \emph{404 error} when no significant result can be found for a primitive.
This adoption, apart from enhancing the compatibility with other web applications, can enable the modular adoption of the \ac{api}.
% Esto tiene más sentido para el outer space que para este API
For example, if a space does not offer a wildcard based \textit{read}, it can simply return a \emph{501 Not Implemented}.
The participants would then be aware of the problem and use an alternative primitive to obtain the data needed.
%Instead, they will interpret these cases as empty responses.
This modularity becomes crucial to ease the partial adoption on new platforms.


\subsubsection{Content negotiation}
Another key aspect of the \acs{http} protocol that our \acs{api} should take advantage of is the \textit{content negotiation}.
This mechanism allows to specify the desired representation for a content on the client side and to express what representation is sent as a response from the data provider side.
For that purpose, the client adds an \textit{Accept} field to the \acs{http} header with a weighted list of media types it knows how to interpret.
Then, the server will answer with the best possible format it understands, specifying the \textit{Content-type} in the response.

The benefits of using this mechanism in the \ac{tsc} middleware presented are two-fold.
Firstly, it enhances the browsability of the primitives with human understandable \acs{http} responses. % no sé si esto es de Content Negotiation per se
Secondly, it allows different semantic representations (e.g., RDF/XML \citeweb{rdfxml2004}, N-Triples \citeweb{ntriples2004} or \acs{n3} \citeweb{n32011}).
The latter characteristic becomes crucial since not all the nodes may understand all the formats (e.g., a mobile phone may not have a RDF/XML parser).
%This is true even if all the languages use the same basic concepts (i.e., \acs{rdf} Triples).
In these cases, the compatibility of both sides can be ensured through a conversion carried out in the server side.
Furthermore, expressing the preference for a semantic format can be useful too in other cases.
For example, to obtain the less verbose answer.

\section{Federated Space}
\label{sec:federated_space}

The \osapi{} extends the \ac{api} previously presented with the primitives and concepts explained in this section.


% grafos autogestionados (no takeables por otros) => write en local
\subsection{Self-managed graphs}

These graphs are shared with other participants, but can only be managed by the devices called \asteroids{}. % write and taken locally
In other words, \selfgraphs{} enrich the space but cannot be externally written or removed. % No device can remove them apart from their creator.
% poner ejemplo de por qué no tiene sentido que eliminen a través del espacio
Therefore, they are \emph{second-class graphs} which provide information about the environment but cannot be used for coordination purposes.


Each \asteroid{} makes these graphs accessible to others through \ac{http}.
% objetivo: que se pueda acceder a estos grafos incluso si no son parte de nuestra app => interop
The final goal is to potentially allow to reuse the data provided by any existing \ac{rest}ful service. % app level interop
Therefore, the \ac{api} should be or trend to be \ac{rest}ful.
% para ello es importante usar un RESTful approach, como sabemos que es complejo ofrecemos otra opción en el siguiente capítulo
However, we leave the \emph{hypermedia} \ac{api} as a future work.
Instead, we require a mandatory \osapi{} to be implemented in each \asteroid{} to guarantee access to the \selfgraphs{}.
% lo sé, no está justificado 0:-)


\subsection{New primitives}

To make the most of the information in a space, we propose a new primitive to query all over the semantic information stored.
% esto puede ser relativo, al final devolver una lista completa de recursos (o parte de sus atributos) no deja de ser REST según los libros de hypermedia design...
The \ac{rest}fulness of this primitive could be argued since it does not operate at resource level (i.e. returning graphs), but mixing several resources (i.e. triples from different graphs).
However, we believe that it is useful to have an endpoint for the queries which involve many graphs.
\citet{kjernsmo_necessity_2012} discusses this topic in depth. % TODO Mirar otras citas???

This new primitive is defined as follows:
\begin{itemize}
  \item The \textbf{query} primitive aims to see the space as a whole, returning all the triples matching the given template.
  
  \begin{lstlisting}
    query(space_URI,template): triples          [6]
  \end{lstlisting}
\end{itemize}


The Table~\ref{tab:queryAPI} extends Table~\ref{tab:tscAPI} to include this new primitive.
% TODO otra clave: debería no tener porqué pasar por el servidor

\begin{table} %http://en.wikibooks.org/wiki/LaTeX/Floats,_Figures_and_Captions#Wide_figures_in_two_column_documents
  \centering
  \caption {
    \acs{http} mapping for the \emph{query} primitive.
    \textit{sp} is a space \acs{uri}, \textit{s}, \textit{p} and \textit{o-uri} are subject, predicate and object \acsp{uri} or wildcards (represented with an as \textit{*}).
    When the template's object is a literal, it can be expressed specifying its value (\textit{o-val}) and its type (\textit{o-type}).
    \medskip
  }
  \begin{tabular}{c|l|c}
      \acs{http} request & \acs{url} & Returns \\
      \hline
      GET & \{sp\}/query/wildcards/\{s\}/\{p\}/\{o-uri\} &  [6] \\
      & \{sp\}/query/wildcards/\{s\}/\{p\}/\{o-type\}/\{o-val\} & \\
  \end{tabular}
  \label{tab:queryAPI}
\end{table}


A key point of the \ac{api} is that the \asteroids{} might not even follow the \ac{tsc} paradigm.
For instance, the \osapi{} can encapsulate data provided by a third middleware.
However, a primitive to ease that management can be a convenient for the developers which do not need a more customized behavior.
With that in mind, we propose another writing primitive.
This primitive only has local effects and therefore has no HTTP equivalent:
\begin{itemize}
  \item The \textbf{write\_self} primitive writes a \emph{self-managed graph} and returns an \ac{uri} which identifies it.
  
  % tiene sentido definir el space_URI???
  % mejor definir su propio espacio?
  % tiene sentido especificar su URI? write_self(graph_URI, triples): URI
  \begin{lstlisting}
    write_self(space_URI, triples): URI
  \end{lstlisting}
  
  \item The \textbf{read\_self} and \textbf{take\_self} primitives only affect to \selfgraphs{}.
  
  \begin{lstlisting}
    read_self(space_URI, graph_URI): triples
    read_self(space_URI, template): triples
    take_self(space_URI, graph_URI): triples
    take_self(space_URI, template): triples
  \end{lstlisting}
  
\end{itemize}



\subsection{New behaviors}

% explicar cómo se escribe y lee en el espacio
% solución enfoque híbrido:
%      cambios en actuadores => directamente a través de HTTP o indirectamente a través de tasks escritas en espacio
%                               o mejor: podrían esas tasks ser directamente esos servicios???
%      grafos que sí => write al "servidor HTTP"
%      query => en todos los dispositivos (capítulo 4) - Porque a veces es necesario a través de todos los nodos
%      take + read => sobre los grafos takeables (o inferencia con toda la info del espacio, cómo prefieras)
In this section we will try to clarify how the different behaviors coexist:

\begin{description}
 \item[Writing.]
      The most basic writing primitive allows a client to write a graph into the space hold by the server.
      However, we also presented the \emph{write\_self} primitive.
      \emph{Write\_self} writes into the local device an externally untakeable graph (i.e. \selfgraphs{}).
 \item[Reading.]
      \emph{Query} performs a traversal query which aggregates all the graphs of the space.
      \emph{Read} and \emph{take} work at resource level.
      \emph{Read\_self} and \emph{take\_self} are their equivalents for \selfgraphs{}.
      Finally, \emph{read} primitive's results will be enriched with \selfgraphs{} from the \outerspace{}. % mediante redirect o lo que sea
\end{description}
\section{Evaluation}
\label{sec:search_evaluation}

\subsection{Types of clues shared}
\label{sec:clues_eval}
As presented in Section~\ref{sec:clues}, the type of clue used will affect
\begin{enumerate*}[label=\itshape(\arabic*\upshape)]
  \item the \emph{precision} and \emph{recall} to find the nodes with the appropriate information; and % explained in section 4
  \item the amount of information to transfer over the network (both requests and responses) and nodes have to process.
\end{enumerate*}
Increasing \emph{precision} reduces the number of unsuccessful requests to handle and thus, it reduces the energy consumption. 
Similarly, sending more information over the network implies more processing time and more energy consumption.

\medskip

\noindent\textbf{Precision and recall.}
We evaluate the \emph{precision} and the \emph{recall} of the proposed algorithm in a network of 470 nodes issuing the query templates shown in Table~\ref{tab:evaluationTemplates}.
In average, the nodes manage instances belonging to 6.34 different classes (standard deviation, $SD=1.31$) among a total of 17 distinct classes in the \Space{}.
%When these classes are expanded with a reasoning process, the nodes store 20 different concepts among a total of 113 classes in the space ($SD=4$). % TODO actualizar
The distinct predicates managed by each node in average are 16.01 ($SD=1.53$) out of 68 different predicates in the \Space{}.


\newcommand{\tplone}{\emph{T1}}
\newcommand{\tpltwo}{\emph{T2}}
\newcommand{\tplthree}{\emph{T3}}
\newcommand{\tplfour}{\emph{T4}}
\newcommand{\tplfive}{\emph{T5}}


\InsertTab{tab:evaluationTemplates}{Templates used in the evaluation}{}{
  \begin{tabular}{ll}
    \hline
    Name & Template \\
    \hline
    \tplone{} & \texttt{?s~~rdf:type~~ssn-weather:RainfallObservation} \\
    \tpltwo{} & \texttt{?s~~wsg84:long~~?o} \\
    \tplthree{} & \texttt{?s~~ssn:observedProperty~~?o} \\
    \tplfour{} & \texttt{bizkaisense:ABANTO~~?p~~?o} \\
    \tplfive{} & \texttt{?s~~dc:identifier~~?o} \\
    %t9 & ?s~~~foaf:family_name~~~?o \\ % Quizá en futura version del paper con perfiles de usuarios, para meter mas variabilidad en contenidos ;-)
    \hline
  \end{tabular}
}{}
% Las consultas se podrían sacar también de bizkaisense o alguna app que hayamos hecho para darle mayor verosimilitud


% Explicar class based
In Figures~\ref{fig:recall_measures}~and~\ref{fig:precision_measures}, the class-based clue shows a good \emph{precision} and \emph{recall} for \tplone{} and \tpltwo{}.
\tplone{} asks exactly for the information this type of clues define (i.e., nodes having instances of a certain class).
\tpltwo{} evaluates which nodes have instances in the domain of the \emph{long} predicate (\emph{SpatialThing}).
Note that this works thanks to the \acs{rdfs} inference because some nodes in the \Space{} only write \emph{Point} instances (a subclass of \emph{SpatialThing}).
% TODO esto no se entiende muy bien sin explicación extra!
The domain of \tplthree{} and \tplfive{}'s predicates could not be inferred just using \acs{rdfs} inference. % (some properties of OWL are used => inverseof).
Even solving this limitation, we would expect a bad \emph{precision} since both predicates relate very general concepts.
In addition, when a class-based clue has no enough information to predict the nodes, it simply floods the query.
This is why the \emph{recall} of \tplfour{} is high.



\InsertFig{clues_recall}{fig:recall_measures}{\emph{Recall} for each type of clue used}{}{1}{}

\InsertFig{clues_precision}{fig:precision_measures}{\emph{Precision} for each type of clue used}{}{1}{}
% IG: TODO mencionar en el texto de referencia o en la caption algo como: The higher the better.

% Explicar predicate-based
We can see a bad prediction for \tplone{} and \tplfour{} for predicate-based clues.
\tplone{} defines a very common predicate and therefore, it cannot discriminate any node.
\tplfour{} suffers the same problem explained for the class-based clues.
We proposed a possible solution for this problem in Section~\ref{sec:aboxinclues}.

% Explicar prefix-based
Finally, prefix-based clue shows a slightly better \emph{precision} for \tplfour{}, since it can discriminate some nodes not using the \emph{bizkaisense} prefix.
On the other hand, it obtains marginally worse \emph{precision} than predicate-based clues for \tplthree{} and \tplfive{}.
% TODO reexplicar esto, no se entiende!
This worsening could be greater if few nodes using the prefixes \emph{ssn} and \emph{dc} used the predicates defined in both templates.

\medskip

\noindent\textbf{Verbosity.}
% Una frase para retomar lo anterior, y al grano.
The clues verbosity is also a critical aspect for resource constrained devices.
Figure~\ref{fig:clueSize} shows a higher variance for prefix-based clues' length and lower verbosity of class based clues.
This is because the nodes virtually have a different number of sensors.
In addition, the links to concepts of other ontologies vary within the datasets used in the parametrization.
In any case, the diagram shows a similar verbosity for all the clues for the semantic content considered in this evaluation. % TODO analizar si la media varía significativamente
% No es adecuado para cacharros pequeños: poner evaluación de WoT de inferencia


\InsertFig{clues_length}{fig:clueSize}{
  Length of the clues alternatives
}{
}{0.6}{}

% Añadir nuevas barras: Predicate+schema y predicate+schema+MostCommonsIndividuals
% Poner otra barra para saber cuanto contenido semántico guarda un nodo de media en el experimento?
% Comentar que cabe en MTU de ethernet y UDP en caso de querer enviarlo por CoAP


% TODO Añadir nuevo diagrama para el gossiping agregado y ver como crece a más elementos añadidos

\medskip

\noindent\textbf{Summary.}
Class-based clues are useful for templates asking for a specific type of content.
However, they still require inference to obtain a good \emph{precision}.
In \citet{gomez-goiri_restful_2012}, we tested the inference process on the devices and data used in this simulation.
We could not run any reasoner in the ConnectPort X2 Gateway.
Actually, we could only run \acs{rdfs} reasoners in more powerful embedded and mobile devices such as the FoxG20 and the Samsung Galaxy Tab.
In the FoxG20, it took 48.9 seconds the first load of all the ontologies used and 1.4 seconds to reason over each measurement written.
In a Samsung Galaxy Tab, it took 17.3 seconds and 0.2 the following measurement writings.
Considering these results, we can conclude that there is a clear need for efficient embedded reasoners.
Therefore, the class-based approach is promising but it is impossible to adopt in current embedded and mobile devices.

Between the predicate-based and prefix-based clues, we propose to use the predicate-based clues since they subsume much of the information provided by the prefix-based clues.
The rest of the prefixes are referred in the subjects or the objects.
They could be easily added to predicate-based clues on the prefixes field.
In addition to the use of predicate-based clues, we could implement the solution for the specific individual search proposed in Section~\ref{sec:aboxinclues}.

% Posible TODO
% GRAFICO 3: \emph{precision} y \emph{recall} comparando con y sin de cada uno

% GRAFICO 4: tamaño de gossiping individual y agregado comparando con y sin cada uno
%            (barra con barra superpuesta encima con cuanto más añade)
%            Según hay más nodos, cómo crece el gossiping a manejar?


% Discussion: ¿elegir un tipo u otro dependiendo del modo de operación?
% (i.e., si hay que perder precisión a costa de no intercambiar MBs...)
% AG: Interesante, pero yo no complicaría una sección ya de por si bastante liosa.


% TODOs importantes que me gustaria hacer para la siguiente version:
%   - Ver cómo crecen los clues agregados.
%   - Proponer una clue híbrida que mezcle a las anteriores.
%   - Evaluar de alguna forma la mejora propuesta para ABox.



\subsection{Network usage} % IG: network role
\label{sec:NetworkUsage}

% Donde ``explicar'' negative broadcasting y/o centralizado? En seccion 4?
% Explicar porque no se pone centralizado en la comparación
%    No es directamente comparable dado que depende directamente de otro factor distinto: frecuencia de escritura.

We conduct a simulation study to evaluate the benefits of our solution against a flooding-based approach (i.e., \ac{nb}).
In addition, to give a more exhaustive comparison, we implement and test query caching on top of \ac{nb}.
We simulate multiple nodes that join the same \Space{} as \providers{} and periodically write new information to the \Space{}.
During one hour, 1 or 100 \consumers{} perform 1000 queries in total using the templates described in Table~\ref{tab:evaluationTemplates}.

As expected, our solution scales much better than the one with \ac{nb} (Figure~\ref{fig:requestsByStrategies}).
However, adding caching to \ac{nb} works slightly better than our solution with just one node querying the \Space{}.
This is due to the limited amount of different query templates used in the simulation.
When we increase the number of \consumers{} in the \Space{}, the caching strategy behaves closer to the \ac{nb}.
In the same situation, our solution handles better an increase on the number of \consumers{} in the \Space{}.


\InsertFig{requests_by_strategies}{fig:requestsByStrategies}{
  Required requests for \acf{nb}, \ac{nb} with caching with 1 and 100 \consumers{} and our solution with 1 and 100 \consumers{}
}{
}{0.75}{}


In Figure~\ref{fig:requestsByRoles}, we take a closer look to the origin of the traffic of our approach in a \Space{} with 100 \consumers{}.
The communication between the \providers{} and the \ac{wp} is much more infrequent than the other communication types.
The reason is that writing into a node only results in a clue update when the structure of the managed information changes.
The first time the metadata about the node (sensor) is written, the second time the first measure and following writings, just add or replace a measure.
Therefore, the clue does not change after the second step.
This matches with the assumption made to share \emph{TBox} information in our clues.

The communication between \consumers{} and \ac{wp} is in between the other two communication patterns.
It is greater than the one from \providers{} to \ac{wp} because \consumers{} need to maintain an updated view of the \Space{}.
Recall that the update time depends on the query frequency of each \consumer{}.
The maximum and minimum updating frequency were set to 10 and 1 minute(s) respectively.

The communications between \consumers{} and \providers{} assumes most of the total communications.
This shows that the overhead added by the use of \ac{wp} on our solution is not significant and it is justified by the reduction of the total number of communications shown in Figure~\ref{fig:requestsByStrategies}.


\InsertFig{requests_by_roles}{fig:requestsByRoles}{Requests between roles in our solution in a \Space{} with 100 \consumers{}}{}{0.75}{}



\subsection{Energy consumption}
\label{sec:energyConsumption}
% Idea: Lo de arriba está muy bien, pero específicamente, cómo afecta a los cacharros?
Our solution tries to save energy by making \providers{} handle fewer requests from \consumers{}.
These savings contrast to the overhead added by the communication with the \ac{wp}.
However, our results demonstrate that this overhead is small in comparison to the total number of communications.

The energy consumption in mobile and embedded devices increases each time a device needs to process something or communicate with another node (see Figure~\ref{fig:energy_consumption}).
To analyse how communications impact their energy autonomy, we have to consider not only the number of communications but also their time length (see Table~\ref{tab:measuresEmbedded}).
For example, a mobile phone will consume less energy asking clues to a server than asking them to an embedded device as it has to wait less for the response.


\InsertFig{energy_consumption}{fig:energy_consumption}{Average power consumption for FoxG20 during different activity periods}{}{0.6}{}


The experiment consists of 300 nodes joined to a \Space{} running on 1 server, 30 galaxy tabs, 75 FoxG20 and 194 XBees.
We increase the number of devices as their price and capacity decrease.
Using this approach, we mimic a typical \Space{} where cheap devices are more common.

As shown in Figure~\ref{fig:activity_measures}, our solution reduces the activity of each device by more than 5 times compared to \acl{nb}.
The diagram on the right details the average activity for each type of device.

In our solution, we can check how the load moves from the embedded devices (XBee and FoxG20) to the server (which is indeed chosen as a \ac{wp}).
The exceptional activity registered by the Galaxy Tabs is caused by their extremely high response time.
% This implies that handling a request in the Galaxy Tab takes much longer than the average.
However, we plan to reduce this response time changing the \acs{http} library used in our Android implementation.


\InsertFig{activity_measures}{fig:activity_measures}{Activity time for each strategy}{
  The first part shows the average active time a node spends on each strategy.
  The second one shows the active time classified by the type of device each node has run on.
}{1}{}
% TODO añadir a la derecha los valores en NB?

%%%%%%%%%%%%%%%%%%%%%%%%%%%%%%%%%%% QUIZA para 2da VERSION %%%%%%%%%%%%%%%%%%%%%%%%%%%%%%%%%%%%%%%%%
%  + Vamos a medir y comparar distintas situaciones entre sí:
%   Nota: cuando hablo de servidor, movil o dispositivo embebido, ejemplifico capacidades.
%         dispositivo embebido puede almacenar menos cosas y hace consultas más concretas.
%
%      - Situación 1: Negative broadcasting (o todo dispositivos embebidos)
%      - Situación 2: 1 servidor (WP), 20 moviles, 120 dispositivos embebidos (proporción 1:20:120)
%      - Situación 3: 20 moviles, 120 dispositivos embebidos (proporción 1:6, entre móviles 5 con el cargador enchufado)
%      - Situación 4: todo dispositivos embebidos
%  + Eje Y: Tiempo medio en ejecución
%  + Eje X: A parte de las situaciones (ver esquema de \emph{recall}) algún otro aspecto que también afecte al consumo de energía:
%      - número de consultas?
%      - frecuencia de las mismas?
%      - nodos consultores?



\subsection{Performance in dynamic environments}
\label{sec:dynamic}
We evaluate the network usage of our solution in ordinary situations in sections~\ref{sec:NetworkUsage} and \ref{sec:energyConsumption}.
Nevertheless, we do not evaluate scenarios where the nodes frequently join and leave the \Space{}.
In such situation, the communication needed to manage the clues might be a burden.


\InsertFig{dynamism}{fig:dynamic_situations}{Effects of dynamic scenarios in our solution}{
  Note that the last interval in the x-axis represents a simulation with no drops.
}{1}{}


To assess the effect of dynamic networks on the performance of our solution, we used the scenario presented in Section~\ref{sec:energyConsumption},
Then, we simulate nodes joining and leaving the \Space{} at different intervals: 30 seconds, 1 minute, 5 minutes, 10 minutes, 20 minutes, 30 minutes, 45 minutes.
Particularly, for our solution, we tested the most harmful situation: the node leaving the \Space{} abruptly is always the \ac{wp}.
We also added an scenario with no drops as a baseline.
Note that we represent this scenario by configuring the drop-interval with a greater value than the simulation time.

In Figure~\ref{fig:dynamic_situations}, we see the results of these simulations.
These results show that even in such dynamic situations, our solution requires fewer communications than \acl{nb}.
In our solution, most of the communications are between \consumers{} and \providers{}.
To evaluate the overhead added by our solution, the graphic on the right hand side shows the communications involving \acp{wp}.

We can appreciate that the updates on \consumers{} are independent of the number of times the \ac{wp} changes.
We can also see a minimal change between the scenario where the \ac{wp} is always available and the one with 5 minutes drop-interval.
In this case, when the \ac{wp} drops, many \consumers{} have the latest version of the aggregated clue.
% Dado que se parte del que tenia el WP o del que le ha dado el consumer que le ha elegido. Ambos tienen más opciones de tener una versión actualizada.
This situation increases the chances of getting an updated version for the initialization of the new \ac{wp} (see Section~\ref{sec:selection}).
Thus, it reduces the number of messages from \providers{} to the new \ac{wp}.


\subsection{Effects on discovery mechanisms}
\label{sec:mdns}

We want to prove the feasibility of our solution using a common discovery mechanism.
To that end, we simulate the behaviour of the \ac{mdns} and \ac{dns-sd} \citeweb{dnssd2013} protocols.
Both protocols are based on the well known and widely accepted \ac{dns}.
\ac{dns} associates different pieces of information (i.e., records) with domain names in a distributed manner.

On the one hand, \ac{dns-sd} proposes a new use for \ac{dns}'s TXT records.
The TXT record was originally intended to associate an arbitrary human-readable text with a domain name.
\ac{dns-sd} proposes to use this type of record, to share key-value pairs in its data field.
We use these key-value pairs to share the information needed by the selection algorithm among the nodes.

On the other hand, \ac{mdns} defines how this and other records are shared through UDP multicast (or unicast in certain situations).
We ignore the cost of browsing the nodes to discover new nodes because it is the same for both strategies.
However, note that as explained above, our strategy does differ from \acl{nb} in the use of TXT records.

The nodes announce records during the start up or whenever they have a resource record with new data.
Therefore, each time a record is updated, we send a multicast message that increases the network traffic.
In our solution the TXT record may change 
\begin{enumerate*}[label=\itshape(\arabic*\upshape)]
  \item when a new \ac{wp} is selected or
  \item when we update the time elapsed since it joined the \Space{} and its battery charge level.
\end{enumerate*}
The last two parameters need to be updated to select an appropriate \ac{wp} but they do not need to change too frequently.

% Regarding the first case,  % no pillo la introduccion del first case
In the most static scenario the TXT record is written only once.
The more dynamic scenario from the previous section, on the contrary, updates that record 126 times after writing it for the first time.
This demonstrates that the overhead generated on the discovery system by our solution is minimal even in the worst-case scenario.
\section{Summary}
\label{sec:actuation_summary}

This chapter presented two ways to actuate on the physical environment.
The first is the usual way to operate through spaces and provides a higher degree of decoupling.
However, it requires participants to use our middleware's primitives. % requiere la cooperación de los proveedores...
In other words, our middleware is not able to reuse third applications \ac{rest} services.


The second actuation mechanism directly consumes \ac{rest}ful \acs{http} \acsp{api}.
This mechanism relies in the semantic description of the services, additional knowledge and in a reasoning process. % additional knowledge: background + initial
With that information, it is able to generate executions plans towards a goal.
Following these plans implies different calls to the different services.


We implemented the same scenario using two actuation mechanisms.
Besides, since interoperability is one of our middleware's guiding principles, we sketched how to reuse these \ac{rest}ful \acs{http} \acsp{api} in our \Space{} model in a third implementation.
This reuse avoids any alteration on the space-based consumer or the \ac{http} provider.
Instead, it improves the \Space{} implementation with an Agent in charge of generating execution plans.
To aim this generation, the agent which reuses the information from the space-based actuation.
Doing so, it avoids requiring any additional information from the developer.


There are other design alternatives to promote the reuse of \acs{http} \acsp{api} from a space-computing middleware.
We described these alternatives and analyse their advantages and weaknesses.
% regarding lo que sea?
However, some questions remain still unsolved: will both methods be triggered indistinctly or will the first prevail over the second?
% uso de nuestra actuación por parte de apps WoT
Finally, the question of how to reuse actuation mechanisms of nodes using \ac{ts} patterns from \ac{wot} has not been addressed.
However, this chapter has exposed, described and compared the key points towards the actuation through a \ac{tsc} middleware for \ac{ubicomp}.
\section{Guiding Principles} % o problems
\label{sec:guiding_principles}

% cómo y porqué hacer un TSC distribuído
In Section~\ref{sec:soa_tsc_discussion} we showed how \ac{tsc} can comply with all the constraints of the \ac{rest} style.
One of these constraints is that the access to a space is done in a client-server basis.
The easiest way to achieve this constraint is by centralizing all the content of each space in the server itself.


% Pero un entorno Ubicomp es eminentemente distribuído
However, data in a \ac{ubicomp} environment is not generated at a single point. %  intelligent environment
%   La información se genera en fuentes separadas, ¿cómo accedo a los últimos datos?
In fact, the plethora of sensors where data are generated can be enormous. % are => data es plural
Besides, each of these sensors may generate data continuously.
This creates a trade off between efficiency and freshness difficult to overcome:
\begin{itemize}
  \item The more frequently each sensor sends contents to the server, the more inefficient is the solution in terms of network usage.
  \item The less frequent this communication is, the less updated is the data in the server.
        This leads to spaces which misrepresent the environments.
\end{itemize}



%   ademas no practico:
%        las fuentes del conocimiento son las que mejor saben como gestionar una informacion (crearla, acualizarla o eliminarla)
%        la información se transporta en dispositivos
%        reusar datos en distintas apps
Consequently, a sensible solution is to let these sensors manage their own information.
This strategy is not only useful to ensure the access to up-to-date data, but also for the following causes: % beneficios o razones?
% This strategy == delegating responsibilities between the devices
\begin{itemize}
  \item The devices directly connected to the sensors and actuators will know how to represent these contents better than others. % ejemplo en grados centigrados o celsius?
	For example, the unit of a temperature measure.
  \item They know when to create, update or delete data. % ejemplo if a server receives 2 contradictory measures... algo
        Furthermore, they can opt for creating data on demand.
  \item Data may be reused by other spaces or even other applications. % Interop!
        These applications would not be required to use a space-based approach.
        Therefore, they will not depend on the correct functioning of our whole system. % del servidor central
  \item Carrying the information can be useful in certain mobility situations.
        For example, let us imagine a person which stores her user profile in her personal smartphone.
        She could share it with different spaces or applications as it moves along the city.
\end{itemize}


% Aún así, que los dispositivos se coordinen a través de una entidad central es conveniente.
%       Modo: caching (para reliability) o modo ver qué hay en el espacio ahora mismo (visión en tiempo real)
%       -->> INDEPENDENCIA DE LOCALIZACIÓN!!!
Still, most of the \ac{iot} devices or mobile phones are unreliable: they can join and leave at any moment.
As a result, distributing the shared space among unreliable nodes comes with a number of drawbacks:
\begin{itemize}
  \item Devices rely on the data written and read from the space to coordinate themselves.
	Therefore, the access to these pieces of information must be guaranteed regardless of dynamic devices' availability. % dynamic == unreliable
        % i.e. que no desaparezca una tarea en el espacio
  % Sencillo de implementar: notificaciones. En temas de razonamiento no he entrado.
  \item A blocking mechanism is important in space-based computing. 
        For example, a worker node may block until some tasks become available in the space.
        A way to implement it in a distributed space is by means of a notification system.
        However, the efficient implementation of this system using unreliable devices is challenging. % no sé si esto último no sonará demasiado a mofa...
\end{itemize}
Overcoming the previous difficulties, usually implies a high network traffic.
This traffic negatively influences the energy consumption of nodes whose energy autonomy might be limited.

\bigskip

In conclusion, our design must face two apparently contradictory principles:
\begin{itemize}
  \item To consider data from independent and limited data providers.
  \item To rely on the providers of the data which enables the coordination of the space participants.
\end{itemize}
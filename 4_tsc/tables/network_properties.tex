\InsertTab{tab:network_properties}{Properties of different architectural styles for network-based applications}{
  This table is an adaptation of the one originally conceived by \citet{fielding_architectural_2000}.
  These adaptations are remarked inside the table. % using several footnotes.

  Each plus symbol (+) represents a positive influence and each minus symbol (-) a negative one.
  Plus-minus (±) denotes that it depends on some aspect of the problem domain.
  
  The leftmost column contains each of the network substyles \acs{rest} derives from:
  \acf{restcs}, \acf{rests}, \acf{restcache}, \acf{restu}, \acf{restl} and \acf{restcod}.
  
  The horizontal line indicates that the immediate row below is composed by all the rows from the upper level.
}{
  % Tablita de cómo hereda propiedades de los estilos anteriores?
  % (y si quieren más información, que miren en la tesis de Fielding)
  \footnotesize
  \begin{tabular}{lccccccccccccc}
    Style &
    \rotatebox{90}{Net Perform} &
    \rotatebox{90}{UP Perform} &
    \rotatebox{90}{Efficiency} &
    \rotatebox{90}{Scalability} &
    \rotatebox{90}{Simplicity} &
    \rotatebox{90}{Evolvability} &
    \rotatebox{90}{Extensibility} &
    \rotatebox{90}{Customiz.} &
    \rotatebox{90}{Configur.} &
    \rotatebox{90}{Reusability} &
    \rotatebox{90}{Visibility} &
    \rotatebox{90}{Portability} &
    \rotatebox{90}{Reliability} \\
    \hline
    \acs{restcs} & ~ & ~ & ~ & $+$ & $+$ & $+$ & ~ & ~ & ~ & ~ & ~ & ~ & ~ \\
    \acs{rests}\footnote{\emph{S} represents the difference between \emph{CSS} and \emph{CS} in \citep{fielding_architectural_2000}.}
      & $-$ & ~ & ~ & $+$ & ~ & ~ & ~ & ~ & ~ & ~ & $+$ & ~ & $+$ \\ % = CSS - CS
    \acs{restcache} & ~ & $+$ & $+$ & $+$ & $+$ & ~ & ~ & ~ & ~ & ~ & ~ & ~ & ~ \\
    \hline
    Early web\footnote{Corresponds to the \emph{C\$SS} style in \citep{fielding_architectural_2000}.}
      & $-$ & $+$ & $+$ & $++$ & $+$ & $+$ & ~ & ~ & ~ & ~ & $+$ & ~ & $+$ \\ % = C$SS
    \acs{restl} & ~ & $-$ & ~ & $+$ & ~ & $+$ & ~ & ~ & ~ & $+$ & ~ & $+$ & ~ \\ % = LS
    \acs{restcod} & ~ & $+$ & $+$ & $+$ & $\pm$ & ~ & $+$ & ~ & $+$ & ~ & $-$ & ~ & ~ \\
    \hline
    LCODC\$SS & $-$ & $++$ & $++$ & $+4+$ & $+\pm+$ & $++$ & $+$ & ~ & $+$ & $+$ & $\pm$ & $+$ & $+$ \\
    \acs{restu}\footnote{Although it is not explicitly included in the original table, \emph{U} has been derived from \citeauthor{fielding_architectural_2000}'s description.}
      & ~ & ~ & ~ & ~ & $+$ & ~ & ~ & ~ & ~ & $+$ & $+$ & ~ & ~ \\ % Uniform Interface (simple, visible, reusable) && en el texto dice que degrada efficiency
    \hline
    \acs{rest}\footnote{Derived from the addition of \emph{U} to \emph{LCODC\$SS}.} % esto se podría intuír por la línea, pero quién sabe..
      & $-$ & $++$ & $++$ & $+4+$ & $+\pm++$ & $++$ & $+$ & ~ & $+$ & $++$ & $+\pm$ & $+$ & $+$ \\ % = LCODC$SS + U
%	TSC (own) & × & × & × & × & × & × & × & × & × & × & × & × & ×\\ % mejor describirlo sólo con palabras que meterme en berenjenales
    \hline
  \end{tabular}
}{htbp}
\section{A Hybrid solution} % o "A Halfway Solution"
\label{sec:hybrid_solution}

We can distinguish two different usage patterns of a space:
\begin{enumerate*}[label=\itshape(\arabic*\upshape)]
  %  escribir y extraer para coordinarse con otros
  \item coordinating with other devices by writing and extracting content; and % MISMA EXPLICACIÓN QUE EN 2_1
  %  consultar que pasa en un entorno
  \item checking what happens in the environment by reading.
\end{enumerate*}
So far, other semantic space-based works have always tried to integrate both strategies within the same space.
However, the needs of both usages are different.
The first one demands availability of the data regardless of its generator availability.
The second does not. % se ha visto en el anterior
Therefore, we propose a novel approach: to treat them independently.
% To conserve the desirable uncoupling properties of a space



% el hecho de que sea REST => para poner un mínimo contrato entre entidades
As explained in the previous section, limited devices are not reliable enough to manage data for the first usage.
Nevertheless, they can contribute to an enriched view of the space (i.e. the second usage pattern).
Accordingly, we propose a dual model where:
\begin{enumerate*}[label=\itshape(\arabic*\upshape)]
  \item the content needed to coordinate devices will be managed by independent machines; and
  \item readings in the space will consider not only the previous content, but also the knowledge provided by autonomous providers.
\end{enumerate*}
Summarizing, we propose to enrich the \ac{ts} view with autonomous federated subspaces.


Figure~\ref{fig:new_model} presents the key elements of this new model.
The \coordspace{} is where the graphs can be written, read and taken by any participant.
The \coordspace{} is held by one or many devices called \coordinators{}.
The \outerspace{} is composed by the graphs managed by independent and probably limited devices.
We call these devices \asteroids{} and their graphs \selfgraphs{}.

\InsertFig{new_model}{fig:new_model}{
  Key concepts of the new \ac{tsc} model presented.
}{}{1}{}


At networking level, we define a minimal contract between the \asteroids{} and the \coordinators{} through two \acp{api}.
Both are \ac{rest}-like \acp{api}.
In our view, this choice's appropriateness for resource constrained devices is endorsed by \ac{wot} initiative.


For the \coordspace{}, we propose a uniform access to the space through the \acs{http} \ac{api} described in Section~\ref{sec:align_tsc_http}.
This \ac{api} can provide data from a distributed space like some approaches in Section~\ref{sec:soa_tsc_distribution} do.
However, this dissertation does not cope with this problem. % TODO cite
Hence, the reader can assume that each space is managed by a unique server for the sake of clarity.


The \coordspace{} will be enriched with data from the \outerspace{}. % enrichment or extension?
This data is exposed with the \osapi{}, which must be implemented by any node willing to share \selfgraphs{}. % TODO no se presenta?
The \outerspace{} demands an extension of the classical \ac{tsc} model presented in Section~\ref{sec:align_tsc_http}.
Section~\ref{sec:federated_space} presents this extension's new concepts, primitives and behaviour.
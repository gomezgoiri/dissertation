\documentclass[a4paper,10pt]{article}
\usepackage[utf8]{inputenc}

%opening
\title{}
\author{}

\begin{document}

\maketitle

\begin{abstract}

\end{abstract}


% 3. Adaptar TSC para que todo esto no sea un cristo para el programador. Ocultar la complejidad de todo esto.
%    Esto implicaría una evaluación con personas, lo cual sería jodido (y no estoy acostumbrado a ello).
% IG> Una interfaz para programarlos a todos? Y que segun el dispositivo se actue diferente? Esto puede molar. Tu programas algo en general y segun el dispositivo se hace de una forma o de otra. Si se te ocurren ejemplos...
% AG> Hablamos de 2 cosas distintas. No se si pillo tu idea :-)
% IG> lenguaje de alto nivel, traducción a algo, etc.
%	"obtener temperatura" => retorno 1 valor
%	"obtener temperatura" => retorno una media
%	"escribo tarea de alto nivel" => middleware distribuye
%	"enseñar mensaje al usuario" => evaluar 
%	tarea de forma general => el espacio se encarga de interpretarla y distribuirla
%	reglas prestablecidas, o que las facilite el usuario
%	qué actuador tiene más energía, cúal es más grande, etc.



% Goal with the paper: Concise, not too ambitious but yet formal in the evaluation and findings.
% General question: How to Act on a distributed space over HTTP?
% Personal interest: test Restdesc on a Raspberry Pi, advance in Section 5
% Main idea: compare the 2 actuation approaches described in Section 5
%	1) a node which wants to actuate over the space, gets Restdescs of others, reasons and performs a HTTP request.
%	2) a node writes a task a space (in its own memory), another is notified about that writing, gets the task, performs the task and writes the results.
%		Problem: This second approach is quite complicated (you need a subscription system, to model tasks and agree on that model)
%		Otsopack has a subscription system, but it is quite limited and might be difficult to explain together with so many other things (because of the space)
%	3) Another solution?
% General problems:
%	+ What if a node wants to actuate over a light and 2 different nodes manage that parameter. What would be an acceptable change in the environment?
%	+ Contradictory tasks?
% Evaluation (ideas):
%	+ Implement a simple scenario in both ways.
%		Too much work (silly problems which take time). I may need help from a colleague. Not sure if anybody will accept to do it.
%	+ How much time, energy and/or memory does the first approach take depending on how many Restdesc we use as input.
%	+ Total communications needed in each solution (the subscription adds overhead, but also getting all the restdescs from other nodes)
%	+ Time required to act into the space with each solution (for different size of the network?).
%	+ Lines of code for each solution,.
%	+ Other subjective features.
% Contributions:
%	+ Act on a completely distributed space build on top of HTTP (REST?) <= indirect communication model to ease developer
%	+ Check feasibility of reasoning in current platforms.


% ----------------------------------------------------------------------------------
% Más ideas:
% (primo, vete decidiendo inline si tienen scientific soundness para tí o cuales tienen potencial)


% ------------ Pueden convencer ------------

% 2. Abordar "autonomía temporal"
%    No tanto desde el pto de vista del punto 1, sino que una orden persista en el espacio por un tiempo.
%      a. Nodo A escribe "quiero temperatura a 4 grados"
%      b. Actuador actúa
%      c. 1 segundo después Nodo B escribe "a 10 grados"
%      d. Actuador actúa jodiendo las preferencias del Nodo A
%    Algo así como asociar una marca de tiempo y que los actuadores hagan media o lo que sea.
%    Tirando del hilo, si almacenan las ordenes, es posible que las propaguen a otros sensores interesados si estos se unen al espacio más tarde.
% IG> puede ser...alguna propuesta mas concreta?
% AG> más?

% Locking, lock remoto, lock temporal, etc.
% Qué abstracción añadir a esto?

% 4. Posibilitar la cooperación entre sensores
%     Usando patron de workflow, o haciendo que ellos se regulen entre sí de alguna forma.
% IG> Puede estar interesante. Algo mas concreto?
% AG> No lo tengo muy claro.


% ------------ No se entienden ------------ 

% 1. Abordar el reto de la autonomía temporal (que se ha dejado de lado en todo momento)
%      a. Nodo A escribe
%      b. Nodo A se va
%      c. Nodo B se une
%      d. Nodo B usa la información de A para hacer algo
% IG> Esto no lo tratas al escribir en el espacio?
% AG> No.
%    Problemas:
%      + no parece muy específico de IoT o UbiComp
%      + complejidad
%      + no sé si tiene sentido en este contexto que venga un móvil con unas preferencias al espacio, se vaya y luego el entorno se ajuste a sus necesidades
%       b) ¿qué pasa si los objetivos se describen muy vagamente?
% IG> que tipo de objetivos?
% AG> objetivos de Restdesc. Simplificando, tú dices "quiero luz" y el te busca un plan para conseguirlo invocando servicios REST.

% 3. Actuar a través de los actuadores con más "energía"
%     Posibles actuadores (para encontrar escenario inspirador en el que encaje):
%         + motores (e.g. aspirador autónomo tipo rumba, microbots colaboradores)
%         + calefactor
%         + diferentes pantallas (e.g. la del portatil, móvil, un OLED conectado a un dispositivo embebido)
%         ( también tenemos un relé preparado, con lo que cualquier cosa que se conecte a la red eléctrica podría ser un actuador, lo malo que no autónomo )
%     Que si aparece otro en el espacio se turnen o lo que sea.
% IG> No pillo esta mucho. Envias el trabajo a bichos gordos?


% ------------ No convencen ------------ 

% 1. Actuar en un espacio distribuido usando descripciones REST semánticas (interoperabilidad).
%    Aquí lo de la comparación era para aportar algo distinto al trabajo a los del ETH. Que en una master tesis proponían un escenario en el que usaban Restdesc para actuar en el espacio según las preferencias del usuario.
% IG> esto seria basicamente hacer REST semantico y probarlo en varios bichos. No lo veo muy research.

% 2. Abordar alguno de los problemas que quedaban sueltos en el ETH (pero eso podría no ser cosa de un mes)
%       a) ¿qué hacer cuando 2 actuadores actuan sobre la misma propiedad? => resolución de conflictos
% IG> esto puede tener sentido si mueves alguna cosa de otro ambito y cambia significativamente. However, no veo que haya un cambio en este escenario. Si lo hay, mola.
%       c) si 2 sensores miden distintas temperaturas, ¿cómo saber si se ha actuado correctamente? => incertidumbre (hay compañeros trabajando en esto)
% IG> no se que podrias usar para tratar incertidumbre. Esto es mas razonamiento. Lo veo un poco alejado


\section{}

\end{document}

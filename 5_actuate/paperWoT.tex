\documentclass[a4paper,10pt]{article}
\usepackage[utf8]{inputenc}

%opening
\title{}
\author{}

\begin{document}

\maketitle

\begin{abstract}

\end{abstract}


% Goal with the paper: Concise, not too ambitious but yet formal in the evaluation and findings.
% General question: How to Act on a distributed space over HTTP?
% Personal interest: test Restdesc on a Raspberry Pi, advance in Section 5
% Main idea: compare the 2 actuation approaches described in Section 5
%	1) a node which wants to actuate over the space, gets Restdescs of others, reasons and performs a HTTP request.
%	2) a node writes a task a space (in its own memory), another is notified about that writing, gets the task, performs the task and writes the results.
%		Problem: This second approach is quite complicated (you need a subscription system, to model tasks and agree on that model)
%		Otsopack has a subscription system, but it is quite limited and might be difficult to explain together with so many other things (because of the space)
%	3) Another solution?
% General problems:
%	+ What if a node wants to actuate over a light and 2 different nodes manage that parameter. What would be an acceptable change in the environment?
%	+ Contradictory tasks?
% Evaluation (ideas):
%	+ Implement a simple scenario in both ways.
%		Too much work (silly problems which take time). I may need help from a colleague. Not sure if anybody will accept to do it.
%	+ How much time, energy and/or memory does the first approach take depending on how many Restdesc we use as input.
%	+ Total communications needed in each solution (the subscription adds overhead, but also getting all the restdescs from other nodes)
%	+ Time required to act into the space with each solution (for different size of the network?).
%	+ Lines of code for each solution,.
%	+ Other subjective features.
% Contributions:
%	+ Act on a completely distributed space build on top of HTTP (REST?) <= indirect communication model to ease developer
%	+ Check feasibility of reasoning in current platforms.

\section{}

\end{document}

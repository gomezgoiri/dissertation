
\section{Consumption of \acs{rest} services}

% por qué? motivación: permitir a otras app integrarse en nuestro espacio

The actuation mechanism of the previous section requires participants to be aware of the content of the space.
For instance, a heater should check the space to find if a new desired temperature was written.
This presents a limitation: what if an actuator was not designed to work with the space?
Following the coordination patterns presented, the middleware will not be able to directly reuse other \ac{wot} solutions' actuation capacities. % interoperate
% lo podrá reusar a través de un intermediario
% interop era una de las cosas que queriamos cuidar
% y qué pasa con las soluciones REST que quieren usar lo nuestro?


In this section we propose to integrate a rule-based mechanism in the middleware.
To do that, we first explain this mechanism (Section~\ref{sec:background_restdesc}) and then we answer two key questions to enable that integration:
(1) Which node is responsible for this mechanism? (Section~\ref{}), and
(2) Where are the inputs for this mechanism read from? (Section~\ref{})


\subsection{Background}
\label{sec:background_restdesc}

In \ac{wot} physical changes in the environment are performed by manipulating \ac{http} resources.
According to the \ac{rest} principles, a client should navigate through these resources with no prior knowledge of the \ac{api}.
% Copiar esta explicación mejor de algún lado:
The client should (1) interpret the representations provided by the server and then (2) choose the appropriate state transition from the hypertext according to its intention. % y su conocimiento básico del protocolo: CRUD


% es HATEOAS?
As explained in Section~\ref{}, semantic representations do not include a native way to express the hypertext.
To solve this, three solutions can be adopted:
% Unos proponen extender con ontologías
(1) to use an ontology to represent the hypertext,
(2) to embed the hypertext independently to the representations on the \ac{http} headers, and % TODO citar lo de HTTP header de Wilde
(3) to provide a description of the resources using the \ac{http} OPTIONS verb.
The latter two enable to discover resources and state transitions without adding metadata to the representations.
This allows not only to describe semantic representations, but any type of formats.
Furthermore, the third option represents the most flexible one.


Different ways exist to describe \ac{rest} services. % o mejor resources? % mencionar WADL, etc.?
% citar a donde se hable de RESTdesc y así ya se empieza a explicar la solución de forma discreta.
\emph{RESTdesc} \cite{verborgh_wsrest_2012} proposes a fully semantic description.
These descriptions are rules expressed in the Notation 3 (N3)\footnote{\url{http://www.w3.org/TeamSubmission/n3/}} language with a premise and a conclusion.
The premise expresses the requirements to invoke a REST service.
The conclusion expresses both the REST call that needs to be made and the description of that invocation result.


Verborgh et al. \cite{verborgh_ijcs_2014} propose a proof-based composition mechanism for Web \acp{api} using \emph{RESTdesc}.
This mechanism uses as inputs:
(1) an initial state,
(2) a goal state,
(3) Web \ac{api} descriptions using RESTdesc, and
(4) optional background knowledge.
Each of these inputs are semantically expressed and therefore, they can be processed by standard N3 reasoners.
The reasoners generate proofs about how to achieve the goal starting from the initial state using the rest of the inputs.
These proofs can be seen as steps that need to be made to reach a desired state.


In \cite{verborgh_ijcs_2014}, they distinguish between pre-proof and post-proof.
The first, are those which assume that the execution of all \acs{api} calls will behave as expected.
The latter, can be seen as a \emph{revision} of the pre-proof.
It executes the Web \acs{api} of the pre-proofs and uses actual execution results to generate a new proof.



\subsection{Inputs for the Proof-based Actuation Mechanism}

As has been seen, the reasoning process uses as input:
(1) an initial state,
(2) a goal state,
(3) Web \ac{api} descriptions using RESTdesc, and
(4) optional background knowledge.
% inputs: clues
% goal: task


\paragraph{The descriptions} must be read previously by an agent and integrate them in the \clues{} presented in Chapter~\ref{XXX} or write in the coordination space.
The latter option requires the node triggering the process read from the space.
The first option ensures that they will be available in each \consumer{}.
The static nature of these descriptions does not break the stability assumption of the \clues{}.


\paragraph{The initial state and background knowledge} can be acquired both from the coordination space and the outer space.
To represent the knowledge in the outer space instead of reading it, the \clues{} can be used.
We propose to create \emph{actuation rules} from \clues{}.


A node which wants to actuate over the space needs to obtain the \emph{clues} from the \ac{wp}.
These clues, as explained before, tell the node what other nodes provide.
In this work, we have assumed that these \emph{clues} are composed by the predicates used by the nodes which provide content.


The existence of a predicate used in a premise does not necessarily imply that this rule can be used.
Nevertheless, its absence does imply that it will not be used (see Figure~\ref{fig:bad_rule}).
Therefore, at this stage, we create \emph{activation rules} which activate those potential rules. % latter rules las que se han mencionado primero


% Un tanto mierden a la hora de intentar poner dos imagenes en el mismo marginpar
% La última la pone cortada
% Por eso he optado por ahorrar espacio poniendo la más grande en el margen y la otra en el texto
% \begin{figure}
%   \begin{center}
%   \includegraphics[width=\marginparwidth]{img/activation_rules/activatable_rule.pdf}
%   \caption{ A rule which, according to the clues shown, may be invoked.
% 	  The third point shows an example of activation rule for the previous rule. }
%   \label{fig:activatable_rule}
%   \end{center}  
% \end{figure}
% 
% \begin{figure}
%   \begin{center}
%   \includegraphics[width=\marginparwidth]{img/activation_rules/bad_rule.pdf}
%   \caption{ A rule which, according to the clues shown in Figure~\ref{fig:activatable_rule}, will never be invoked.}
%   \label{fig:bad_rule}
%   \end{center}  
% \end{figure}


An \emph{activation rule} for a rule R contains a \emph{true} in the premise.
The conclusion is made by R's premise substituting the variables with fictitious \acsp{uri} with a common prefix (see Figure~\ref{fig:activatable_rule}).
These fictitious \acsp{uri} are used in the third stage to distinguish when a triple should be replaced by actual knowledge from the space. % o obtained from the space?
Note that the nature of the \emph{activation rules} is temporary since they are only used during the second stage.



Finally, \paragraph{the goal state} may be created from the task.
For instance, from a task of \emph{regulate temperature to 6ºC} we can deduce the goal state of \emph{temperature of 6ºC}.
This translation is out of the scope of the dissertation.
Note that this alignment will allow not to introduce new primitives for the proof-based mechanism.



\subsection{Node Responsible for Proof-based Actuation Mechanism}

In the Section~\ref{sec:background_restdesc}, two coarse-grained steps were described:
\begin{itemize}
  \item Reasoning over the descriptions, background knowledge, an initial state and a goal state.
        The result of the reasoning process if a pre-proof, which can be seen as a tentative \emph{execution plan} to achieve the goal.
  \item Check the execution plan by following it.
\end{itemize}


This mechanism can be performed by any node.
% no obligamos a implementar una u otra, pero recomendamos esto: XXX
In this thesis, we present three different alternatives and their trade offs.
However, we do not adhere to any of them.
We leave as a future work to implement and empirically compare them.


First, it can be triggered by an agent which resides in the same machine as the \emph{coordination space}.
Doing so, it can locally consume the knowledge available in the \emph{coordination space}.


Second, each \Consumer{} interested on changing the environment can trigger it.
If these \Consumers{} use the search mechanism present in Chapter~\ref{XXX}, they will background knowledge about other nodes.
This reduces the dependency on the node providing the \emph{coordination space} as the search mechanism from the Chapter~\ref{XXX} does.
However, it requires them to carry tasks such as reasoning and checking the pre-proof.
While the latter increases the network usage, the first increases the computation.
As we already mentioned in previous chapters, these tasks severely affect to the energy consumption.
Furthermore, some resource platforms will not be even able to reason.


% proponer otra solución: workers que lleven a cabo este proceso.
Third, we could delegate on any node able to perform such tasks.
In fact, these nodes could follow the \emph{replicated-worker pattern}.
Any node able to perform the process will read from the space goals to trigger the process (i.e. \emph{reasoning tasks}).
Apart from balancing the load between all the worker nodes, any node can stop being worker at any time by not taking more \emph{tasks} (e.g. if it has low energy).
Preferably, these nodes must be \Consumers{} to use the \clues{} from the search mechanism as background knowledge.



% Poner ejemplo adaptado del paper de WoT2013?


%Proceso de suggest:

%\begin{itemize}
% \item obtener todo el conocimiento necesario para el proceso?
% \item razonamiento sobre conocimiento
% \item parsear el resultado para invocar servicios HTTP
% \item monitorizar que se ha complido el cambio?
%\end{itemize}
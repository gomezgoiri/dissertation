
\section{REST services consumption}

% por qué? motivación: permitir a otras app integrarse seamlessly en nuestro espacio

% explicar cómo encajaría esto con RESTdesc

% Objetivo => que cuando se escriba una tarea en el espacio, se pueda traducir en cambio de estado de actuadores sin requerir su colaboración específica
%    ventajas:
%       + interop (el middleware trabaja más por tí)
%       + flexibilidad
%       + autonomía ref y temporal

% Cómo?
%    lanzando proceso de razonamiento deductivo
%        cuando?
%            siempre que se escribe algo? (mucho cristo, depende del tiempo que lleve)
%            cuando se escribe una tarea
%            periodicamente: para tareas no cumplidas en el pasado
%            periodicamente siempre que se hayan escrito nuevas cosas

%    proveyendole inputs
%         RESTdesc => por el mecanismo que sea
%               los clientes lo escriben en el espacio
%               mejor: un agente en el espacio descubre nuevos RESTdescs
%         info de base: la info del espacio
%                             + clues falseadas? => puede requerir tiempo y habría que explicar de nuevo el proceso!
%         goal: ir probando con cada

% Problemas:
%     datos contradictorios (e.g. intereses contradictorios)
%     en operación compuesta (rollo mashup), ¿cómo hacerlo bien? (e.g. rollback y commit)
%     eficiencia y consumo por razonamiento


Nueva primitiva denominada suggest.

Explicar RESTdesc.

Entrada => goal

Proceso de suggest:

\begin{itemize}
 \item obtener todo el conocimiento necesario para el proceso?
 \item razonamiento sobre conocimiento
 \item parsear el resultado para invocar servicios HTTP
 \item monitorizar que se ha complido el cambio?
\end{itemize}
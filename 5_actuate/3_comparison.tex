
\section{Comparison}

The first actuation mechanism requires a subscription mechanism and provides space and time autonomy.
However, it does that by creating a dependency on the \space{}.
Two nodes will not be able to communicate with each other without the \space{}.


The second approach offers this independence of the \space{}: any node can directly invoke a service to act over the environment.
Remarkably, it would allow to integrate third \ac{wot} applications in our middleware.
However, it also implies that a node needs to have prior knowledge about the environment to reason over it.
Since this knowledge must be acquired from remote nodes somehow, this approach demands more network usage than the first one.
% de pre-proof a post-proof se generará más networking activity seguro? por qué? por invocar servicios REST? Es equivalente a andar leyendo tareas y escribiendo resultados...
Besides, the proof-based actuation itself also generates more computation activity on the node responsible of reasoning.
Both aspects have a negative impact in the energy consumption.
Consequently, it will most likely not be an adequate mechanism for resource constrained devices.
In a more practical aspect, reasoning demands a reasoner which may not be available in many computing platforms. % aspecto práctico, pragmático o qué? % afectando negativamente su implantación



% Tabla: XXX
% autonomía: las de space-based vs. 
% dependencia en spacio
% requirements: subscription, reasoning
% interoperability: con otros servicios, no alreves


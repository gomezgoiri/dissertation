
\section{Comparison}

The first actuation mechanism requires a subscription mechanism and provides space and time autonomy.
However, it does that by creating a dependency on the \space{}.
Two nodes will not be able to communicate with each other without the \space{}.


The second approach offers that independence of the \space{}.
Any node can directly invoke a service to act over the environment.
However, it also implies that a node needs to have prior knowledge about the environment to reason over it.
This knowledge must be adquired from the other nodes somehow (i.e. demands more network usage).
In the technical aspect, it requires a reasoner, which is not necessarily available in all computing platforms.
Besides, it generates more computation and networking activity on the node responsible of the process.
Remarkably, this second approach would allow to integrate third \ac{wot} applications in our middleware.


% Tabla: XXX
% autonomía: las de space-based vs. 
% dependencia en spacio
% requirements: subscription, reasoning
% interoperability: con otros servicios, no alreves



\section{Comparison}

The first actuation mechanism requires a subscription mechanism and provides space and time autonomy.
However, it does that by creating a dependency on the \space{}.
Two nodes will not be able to communicate with each other without the \space{}.


The second approach offers that autonomy on the \space{}.
However, it also implies that a node needs to have prior knowledge about the environment to reason over it.
This knowledge must be adquired from the other nodes in one way or another.
In the technical aspect, it requires a reasoner, which is not necessarily available in all the computing platforms.
Besides, it generates more computation and networking activity on the node responsible of the process.
On the contrary, it possibly discovers new actuation plans.


% Tabla: XXX
% autonomía: las de space-based vs. 
% dependencia en spacio
% requirements: subscription, reasoning
% interoperability: con otros servicios, no alreves


% Future work o conclusion?
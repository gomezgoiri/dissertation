\section{Conclusion}

In this chapter we presented two ways to actuate on the physical environment.
The first is the usual way to operate through spaces and provides a higher degree of decoupling.
However, it requires participants to use our middleware's primitives.
In other words, our middleware is not able to reuse third applications \ac{rest} services.



Since the interoperability is one of the guiding principles of our middleware, we presented an additional actuation mechanism.
This mechanism relies in the semantic description of the services, additional knowlege and in a reasoning process. % additional knowledge: background + initial
With that information, it is able to generate executions plans towards a goal.
Following these plans implies different calls to the different services.


We sketched an allignment of our middleware with this actuation mechanism.
The tasks written in the space can be translated into goals.
The knowledge from the space can be used as an input for the reasoning process.
Different nodes are able to trigger the reasoning and check the plan.
Remarkably, we suggest the use of \ac{ts} patterns to promote load balancing between the devices able to perform that process.


% Future work o conclusion?
The second actuation alternative requires further work to check some of the assumptions made.
For example, will always be possible to translate a task into a goal.
Additionally, it opens the door to solve interesting questions: when there are two or more paths to a goal, how to discern which one to follow.


Furthermore, the middleware must deal with the coexistence of both mechanisms.
Will both methods be triggered indistinctly or will the first prevail over the second?
% uso de nuestra actuación por parte de apps WoT
Finally, the question of how to reuse actuation mechanisms of nodes using \ac{ts} patterns from \ac{wot} has not been even addressed.
However, this chapter has exposed, described and compared the key points towards the actuation through a \ac{tsc} middleware for \ac{ubicomp}.
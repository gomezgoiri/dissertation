\section{Comparison}
\label{sec:actuation_comparison}

The actuation procedure explained in Section~\ref{sec:actuation_space} requires a subscription mechanism, but in exchange it provides space and time autonomy.
Obviously, it does that by creating a dependency on the \Space{}.
Two nodes will not be able to communicate with each other without the \Space{}.


The second approach offers independence of the \Space{}: any node can directly invoke a service to act over the environment.
This approach allows reusing \ac{wot} applications' actuation capacities, providing they are properly described. % apps, APIs o services? % no digo integration/interop porque es sólo one-way
To enable this reuse, it demands a rule-based reasoning engine. % reasoning engine ya queda bien?
The unavailability of these engines in many computing platforms may impede its massive adoption. % como apunte de su practicalidad



% Tabla: XXX
% autonomía: poner o simplemente indirect vs direct?
% requirements in consumers: estar pendiente del espacio vs razonar
% requirements in providers: estar pendiente del espacio vs (describir semánticamente servicio => equivalente a tasks)
\begin{table}[htbp]
  \caption{Characteristics of the discussed actuation mechanisms.}
  \begin{center}
    \footnotesize
    \begin{tabular}{llll} 
      \hline
      \multirow{2}{*}{Actuation} &
      Communication & % o poner autonomias y demás para enfatizar?
      \multirow{2}{*}{Benefits} &
      Required \\
      &
      style &
      ~ &
      features \\
      \hline
      Space-based & Indirect & Decoupled communication & Subscriptions \\[0.2cm]
      REST-based & Direct & Reuse of third \ac{wot} & Rule-based \\ % mencionar interop? % TODO ampliar la noción de WoT a REST-like?
      & & applications & reasoning \\ % no digo "wot actuation capacities porque ya está verbalizado en la explicación"
      \hline
    \end{tabular}
  \end{center}
  \label{tab:actuation_mechanisms}
\end{table}


\bigskip

The impact of both mechanisms in resource constrained devices will be affected by their computing and networking activities.
\ac{ts}-based patterns require all the participants to be aware of what is written in the space to react (i.e. be proactive).
Both consumers and providers read and write from the space, subscribe to specific changes and receive notifications.
Thanks to this specificity, they are only affected by the contents they are interested in. % reduciendo la actividad
Besides, they only perform trivial computing task: interpreting results and pattern matching. % pattern matching sólo en el caso del coordination space!


The proof-based actuation mechanism requires the consumer to have prior knowledge about the environment to reason over it. % frase susceptible de simplificar
Since this knowledge must be acquired from remote nodes somehow, this approach might demand more network usage than the first one.
% de pre-proof a post-proof se generará más networking activity seguro? por qué? por invocar servicios REST? Es equivalente a andar leyendo tareas y escribiendo resultados...
Besides, the proof-based actuation itself also generates more computation activity on the node responsible of reasoning.
Both aspects have a negative impact in the energy consumption.
Consequently, from its consumer perspective, it will most likely not be an adequate mechanism for resource constrained devices.
As a contrast, this mechanism demands few things to the provider: to serve \ac{rest}-like services and provide their descriptions. % aclarar quien es el provider/actuator?



% Tabla: XXX
% autonomía: poner o simplemente indirect vs direct?
% requirements in consumers: estar pendiente del espacio vs razonar
% requirements in providers: estar pendiente del espacio vs (describir semánticamente servicio => equivalente a tasks)
\begin{table}[htbp]
  \caption{Foreseeable networking and computing impact on the nodes involved in the actuation mechanism.}
  \begin{center}
    \footnotesize
    \begin{tabular}{llp{4cm}p{4.4cm}}
      \hline
      \multirow{2}{*}{Actuation} &
      \multirow{2}{*}{Perspective} &
      \multicolumn{2}{c}{Activity} \\
      ~ &
      ~ &
      \multicolumn{1}{c}{Networking} &
      \multicolumn{1}{c}{Computation} \\
      \hline
      \multirow{3}{*}{\ac{ts} patterns} & Provider & Proactive, limited activity & Limited: Results Interpretation \\
				    ~ & Consumer & Proactive, limited activity & Limited: Results Interpretation \\
				    ~ & Space & Reactive, high activity & Varies with the implementation \\[0.2cm]
      \ac{rest} \acp{api} & Provider & Reactive, limited activity & Limited: Handling requests \\
        consumption       & Consumer & Proactive, high activity &  Demanding: Reasoning \\ % Y comprobar planes...
      \hline
    \end{tabular}
  \end{center}
  \label{tab:actuation_networking_computing}
\end{table}



% TODO podríamos decir que el proof-based es más genérico?
%  + Space-based es muy dependiente del tipo de datos escritos en el espacio.
%    Proof-based también lo es a decir verdad.
%  + El proveedor del Space-based no tiene forma de saber cómo activar un mecanismo concreto.
%    Salvo que tenga posibilidad de saber a qué se han suscrito otros...
%    No sé como expresar esa diferencia.
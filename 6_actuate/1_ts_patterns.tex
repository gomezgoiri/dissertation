
\section{Actuating through the Space}


% explicar brevemente patrones
% requirements: sistema de suscripciones (out of the scope)
% escenario de ejemplo en Ubicomp


In space-based computing, participants coordinate by reading and writing in a shared Space. % TODO usar define a nivel general para Space!
This encourages an uncoupled communication between applications using the same Space.

% IG: esto lo he homogeneizado porque tenia 3 estilos distintos :)
This section analyzes how to achieve this uncoupled communication following a top-down approach.
First, Section~\ref{sec:ts_patterns} briefly explains \acl{ts}'s most common application patterns. % according to the classification of \citet{freeman_javaspaces_1999}.
Second, Section~\ref{sec:envisaged_scenarios} uses some of these patterns to show how to build an \ac{ubicomp} scenario.
The scenario shows some requirements the middleware must fulfill.
Finally, Section~\ref{sec:notification} analyzes these requisites.



\subsection{\aclp{ts}' patterns}
\label{sec:ts_patterns}

According to \citet{freeman_javaspaces_1999}, there are four main application patterns which can be used with \ac{ts}: % IG: with o in?

\begin{description}
  \item[Replicated-worker pattern.] In this pattern, there is a master process and many worker processes able to compute the same task.
				    %% REDUCIENDO esto: %%
%				    The master takes a problem and divides it in tasks which are solved by any of the workers.
%				    More concretely the pattern is composed by the following steps:
% 				    \begin{enumerate}
% 				      \item The master divides a problem up into smaller tasks.
% 				      \item It writes them in the space.
% 				      \item Any of the many workers takes a task.
% 				      \item That worker computes the task.
% 				      \item It writes the result of that computation.
% 				      \item The master collects the results for the tasks it wrote.
% 				      \item Once the master has all the results, it combines them into a meaningful overall solution.
% 				    \end{enumerate}  
				    The master takes a problem, divides it into smaller tasks, and writes them tasks into the space.
				    Any available worker takes a task, processes it, and writes the result back into the space.
				    When all the workers have written their results, the master takes the results and combines them into a meaningful overall solution.
				    %Note that the workers accept new tasks whenever they are available and able to work.
				    %Therefore, while a worker is busy computing a big task, another one can solve many small tasks.
				    %In other words, this pattern naturally \emph{balances the load} on the space.
				    %Besides, it is \emph{scalable} since we can add more workers running in more machines without rewriting our code.
				    This pattern is scalable and naturally balances the load on the space.
  \item[Command pattern.] This pattern encapsulates the behavior into the tasks shared in the space.
			  Therefore, it requires (1) to share behavior through the space, and (2) any generic worker to be able to compute the behavior.
			  For instance, a graph in \ac{tsc} could include code in an interpreted programming language.
			  % nosotros pasamos de este jaleo
			  % hacer referencia a on-demand-code
			  %In that way, a generic worker application can compute whatever code other processes define and share through the space.
			  %This pattern is only possible in \aclp{ts} where the behavior can be described generically (e.g. in object-oriented \acp{ts}).
			  %Furthermore, the tasks to be performed over \ac{ubicomp} environments are not generically processable by any node.
			  %In contrast, each device is responsible of managing its own actuators on behalf of the rest.
			  %Therefore, this pattern will not be considered.
  \item[Marketplace pattern.] In this pattern, producers (or sellers) and consumers (or buyers) of resources interact to find the best deal. % TODO cita TripCom?
			  %It should be noted that the resources are products or services which can be bought and sold.
			  %Therefore, it is not applicable to the environments considered. % o use cases
			  %esto está relacionado con el paper aquel de TripCom y con el de Simon de hace un año en WoT
  \item[Specialist patterns.] In contrast to the replicated-worker pattern, in this pattern, each worker is specialized.
                              Therefore, each worker performs a particular task.
                              \citeauthor{freeman_javaspaces_1999} enumerate three subtypes:
			      \begin{description}
				  \item[Blackboard pattern.]
					It associates the concept of the space to a \emph{blackboard}.
					Following this analogy, the master is associated with a teacher, tasks with \emph{problems} and the specialized workers with \emph{students}.
					The blackboard pattern starts when the \emph{teacher} writes a \emph{problem} in the \emph{blackboard}.
					The \emph{students} observe the space and write their intention to contribute to solve the problem (i.e. \emph{raise their hands}).
					The \emph{teacher} selects to the expert which will make a modification.
					After the modification, the \emph{teacher} decides if it found the solution or another \emph{student} should contribute.
				  \item [Trellis pattern.]
					In this pattern, the master arranges the problem into low-level, mid-level, and high-level pieces.
					The workers of each level benefit from the refined data provided by the immediate level below.
				  \item [Collaborative patterns.]
					It encompasses all the patterns which allow nodes to collaborate to complete a greater task (i.e. by creating a workflow).
					%They are specific to the domain so they cannot be generalized, but it includes all the patterns where the nodes collaborate to complete a greater task
					%(e.g. by creating a workflow).
			      \end{description}
\end{description}


% Voy preparando el terreno para que se vea que los primeros son más de computación y el otro de negociación
To summarize, the \emph{replicated-worker pattern} is centered in optimizing the computation by parallelizing tasks.
The the \emph{command pattern} can be seen as an abstraction of the latter where the behavior is shipped in the tuples.
The \emph{marketplace pattern} allows negotiation of two entities through the space.
% Coño! pero si el especialista te permite colaborar...
Finally, the \emph{specialist pattern} allows nodes with distinct capacities to cooperate towards a common goal.


% Posible TODO Un esquema dividido en 4 en el que aparecen todos los patrones explicados.
% No es fácil mostrar el 2do y el 3ero.
% Una intentona de estos y otros patrones inventados en Lancaster:
%     https://docs.google.com/a/deusto.es/document/d/1QXGsQ4-wAByc_ByQ-Tts-TSphSQAzfO_c3mcgoZdloc/edit


% IG: ves... esta seccion parece q sale de la nada y creo que es el objetivo principal del capitulo, no? a ver si puedes orientar la intro del capitulo para mostrar esto mas... basicament que se vea que esto es de implementar movidas.
\subsection{Envisaged Scenarios} % IG: he aprendido palabro nuevo...
\label{sec:envisaged_scenarios}
% 1era iteración en reescritura, se puede mejorar mucho

This section devises two stereotypical scenarios for home automation.
Both scenarios emphasize how devices can coordinate in a decoupled mode using the patterns seen in the previous section. % más de uno?
%The scenarios are formed by mobile phones, and embedded platforms with sensors and/or actuators.
Specifically, we believe that the \emph{specialist patterns} fit the best to the needs of \ac{ubicomp} environments:
\begin{itemize}
  \item The devices in \ac{ubicomp} often serve to very specific needs. % IG: comentario general de esto de las comas, que me las pones todo al reves... el sujeto y el predicado sin coma entre medio (a noser que tenga de esas cosas descriptivas, en plan "jose, que es negro, se ha quemado"). Y la coma se pone cuando es algo complementario del tipo de: "en este capitulo, ..."
        For example, let us imagine a mobile phone showing a message or an embedded device turning on the lights of a room. % IG: lo de mobile es el estandard de la tesis? lo de cell no, no?
        %Although there may exist some redundancy in the tasks which different nodes can be accomplish, the task are not nodes are not as interchangeable as in... son tareas simples, no necesitan optimizar ni lanzar muchas
  \item These tasks are usually lightweight. % TODO primera vez que se habla de "task", comentar que para TripCom eran servicios web y que crearon un sistema (WXSW?)
        They do not require large computation resources, but to achieve a concrete and simple goal. % IG: comor? frase rara
\end{itemize}
Therefore, \ac{ubicomp} generally faces a problem of collaboration between nodes with different capacities.
In this problem, computation or negotiation aspects are secondary. % se entenderá a qué me refiero con computation??


%In this pattern, a master writes a task into the space and waits for its result, which is performed by some of the workers specialized in this particular task (e.g. show a message or regulate the temperature).

% IG: hala! y esto de sopeton... lo que decia, preparame el terreno un poco
\subsubsection{Scenario 1: Temperature Regulation}

The first scenario presents a room populated with several kind of sensors such as Oracle's SunSPOTs \citeweb{sunspot},
% TODO TODO TODO referenciar a la parte donde se hayan referenciado bien los dispositivos! (e.g. see ~\ref{environment})
% Para el resto proveer al menos una URL!
Digi's XBee sensors with an IP gateway,
the sensors on a KNX domotic bus \citeweb{knx}, and a fan connected to a FoxG20 \citeweb{foxg20} embedded platform as an actuator (see Figure \ref{fig:devices_scenario}).
In addition, an Android application \citeweb{android} semantically stores the temperature preferences of the user.


\InsertFig{devices_scenario}{fig:devices_scenario}{
  The devices used in the devised scenarios.
}{
}{0.7}{}


An independent node (i.e. the master node) continuously reads from the space % using \emph{read primitive}
(1) the room's temperature, and
(2) the user's desired temperature.
Note that the master node does not care about who exactly provides the temperature information.
It just takes the first available graph from the space.
When the second one is below the first one, it generates a ``\emph{decrease temperature during a certain period}'' task which can be consumed by different independent worker nodes.
In this case, the FoxG20 periodically checks for orders it can fulfill extracting them from the space. % consumes them with a \textit{take primitive}.
% zasca, acabamos de poner en bandeja que nos rechine lo de comprobar periodicamente



\subsubsection{Scenario 2: Sedentary Lifestyle Checker}

The second scenario presents an application which helps the user to avoid a sedentary lifestyle by giving him different warnings.
We consider the recommendation that states that an adult should walk at least 10.000 steps in an ordinary day \citep{tudor2002taking}.
Taking into account the expected steps which should have been completed at each moment of the day, the application generates different priority level messages.

There are several nodes involved in this task.
Each node runs on a device. % which belongs to a user.
First, an Android phone periodically updates the number of steps covered by a user that day \citeweb{pedometer}.
%Besides, it writes her profile, more important to this problem: her age. % simplificado un poquillo
Second, there is an undefined number of devices which know how to warn the user about her unhealthy behavior.
Third, there is a node in charge of generating the warnings.


\medskip


The application follows the \emph{blackboard pattern}.
The node which generates the warnings is the \emph{teacher}.
The \emph{problem} is the warning for the user.
The \emph{students} are the devices which can warn the user. % IG: no es nada intuitivo. Leyendo despues lo entiendo, pero... es raro


When the \emph{teacher} writes a warning for a user, the devices able to warn the user write their characteristics into the space. % IG: alguna cosa mejor que characteristics?
In this example, these characteristics are the device ownership and the intrusiveness of the warning method.
The \emph{teacher} reads these characteristics, chooses the most appropriate \emph{student} to solve the \emph{problem} and updates the \emph{problem} to indicate the selection.
%If no device writes their characteristics, the \emph{teacher} can wait longer.
Finally, the selected device takes any warning for it and warns the user. % no tiene respuesta
If the chosen device does not take messages, the \emph{teacher} can update the \emph{problem} with a new selection.


With this application, if the priority level is low, the user receives the warning in a less intrusive way.
For example, room's light brightness can be increased for low priority notifications, a Chumby \citeweb{chumby} can show an icon for normal priority ones and the user's mobile phone show directly the message when it is a high priority warning. % IG: yo anyadiria una tablita para los dos scenarios mostrando los actuadores y los niveles. Aqui esto me salta un poco de repente.
The remarkable characteristic of this mechanism is that the devices present in the environment at each moment can vary and the application will still fulfill its goal.




\subsection{Notification mechanism}
\label{sec:notification}


In the scenarios described in the previous section, some writings in the space trigger other node's action.
For example, a device must show a message whenever a new warning is written in the space. % e.g. a task or a result
To be aware of these writings, the node can either poll the space or rely on a notification mechanism. % async or blocking
Obviously, the later encourages a more efficient use of the network. % is more optimal and more scalable approach.
% IG: encourage no creo que sea la palabra...

As a consequence, a notification mechanism is highly advisable to fulfill \ac{ts}'s patterns in a distributed environment.
Although the implementation of this notification mechanism is out of the scope of this thesis,
% Hablar de requisitos, describir nuevas primitivas de suscripciones o alternativas de implementación
it should comply with the following aspects:
\begin{itemize}
  %  Debe permitir polling (sync), por ello proponemos nuevas primitivas: read_async y take_async.
  \item Do not substitute the polling mechanism. % porque todavía puede ser interesante en otros casos
        The middleware must provide additional \emph{read} and \emph{take} primitives.
  % Se evalúan sobre el espacio, no sobre el outer-space. 
  \item Since the notification mechanism intends to allow coordination patterns,
        it must consider the knowledge from the coordination space. % TODO comprobar que lo llamé así en el anterior capítulo
        In other words, knowledge from the \emph{outer-space} will not trigger notifications. % TODO usar constante para esto
  % Evaluarse cada X, no siempre que se escriba algo o se vuelve loco.
  \item The evaluation of the subscriptions must not interfere with the writing process (e.g. introducing a delay).
        Therefore, it must run asynchronously.
  % Debe ser sencillo de implementar por cualquier nodo. Para ello proponemos un mínimo contrato: callback URL.
  \item Ease its adoption by any type of computing platform.
	This can be achieved by reducing the requirements on the \emph{clients}.
        For instance, a callback \ac{url} passed during the subscription can represent a minimal contract between the client and the server. % no viceversa
  % Deben caducar para no mantener suscripciones de dispositivos que ya no existen.
  \item Provide additional subscription removal mechanisms.
	\ac{ubicomp} scenarios are composed by unreliable devices which may frequently join and leave the space.
	In this situation, the correct use of unsubscription primitives cannot be guaranteed.
	This may worsen the performance of the system with useless subscriptions from absent devices.
	Therefore, the device managing the subscriptions should adopt more proactive mechanisms.
	For example, it may let the subscriptions expire after a lifetime or remove them when it discovers the unavailability of a callback \ac{url}. % algo así como garbage collection
	% A) The subscriptions must expire.
        % The expiration will allow to delete subscriptions from absent devices. % no longer present devices
        % This also implies that clients are responsible for periodically updating their subscriptions.
        % B) Additional mechanism
        % For instance, a \emph{garbage collection} agent can check the availability of callback \acp{url}.
        % For unavailable callback \acp{url}, the subscriptions can be removed.
\end{itemize}


The main drawback of any subscription mechanism is that it breaks the \ac{rests} property of the \ac{rest} style.
This implies that network performance will improve at the cost of scalability, simplicity and reliability. % TODO poner un see Section~\ref{X} ???


% hasta AQUI!
% Therefore, although it is not the main focus of this thesis, in this section we present a subscription mechanism which can be used on top of the \ac{tsc} middleware proposed.
% The requirements that this mechanism needs to fulfill are the following ones:
% \begin{itemize}
%   \item It should be independent of \ac{tsc}'s writing and reading primitives.
% 	This requirement ensures that frequent writings do not lead to excessive processing in resource constrained nodes.
% 	The main drawback of this requirement is that a node does not now when relevant data is written in the space \emph{per se}.
% 	A developer needs to advertise when it writes relevant data.
%   \item Any node running our solution must be able to implement this mechanism.
% 	In other words, no new dependencies should be added.
%   \item The new primitives should use elements the current developer is used to (i.e. templates or RDF triples).
% \end{itemize}
% 
% explicar cómo funciona
% https://otsopack.readthedocs.org/en/latest/subscriptions.html
% \subsubsection{Subscription primitives}
% 
% The subscription primitives the developer should use are the following ones:
% \begin{itemize}
%   \item \emph{Subscribe}. The node subscribes to the given template returning an URI which identifies the subscription.
%     \begin{minted}{java}
% URI subscribe(space_uri, template, listener)
%     \end{minted}
%   \item \emph{Unsubscribe}. Unsubscribes to a subscription given its subscription URI.
%     \begin{minted}{java}
% void unsubscribe(space_uri, String subscriptionURI)
%     \end{minted}
%   \item \emph{Notify}.
%     \begin{minted}{java}
% void notify(space_uri, template)
%     \end{minted}
% \end{itemize}
% 
% 
% \subsubsection{Deployment}
% 
% The nodes responsible of handling subscriptions and notifications are called \emph{bulletin boards}.
% Other nodes belonging to the same space, discover them using a out of scope method and publish their subscriptions and notifications using their HTTP API.
% 
% Each \emph{bulletin board}:
% \begin{itemize}
%   \item Belongs to a space.
%   \item Exposes a subscription API.
%   \item Shares its subscriptions with other bulletin boards which belong to the same space.
%   \item Propagates the notifications to the relevant nodes using the \emph{callback url} provided by them.
% \end{itemize}
% 
% 
% The Figure~\ref{} describes a simple subscription use case.
% \begin{enumerate}
%   \item N1 subscribes to BB1 with a template t1.
%   \item BB1 propagates the subscription provided by N1 to BB2 and BB3
%   \item N3 notifies to BB3 about t2.
%   \item Since t1 matches t2, BB3 tries to notify to N1 using the callback URI provided during the subscription process.
%   \item Unfortunately, the BB3 cannot notify to N1 due to unexpected network problems.
%   \item BB3 propagates the notification of t2 to BB2.
%   \item BB2 reaches N1, so it notifies it about t2 using the callback URI.
% \end{enumerate}
% 
% TODO poner la foto esquemática de https://otsopack.readthedocs.org/en/latest/subscriptions.html
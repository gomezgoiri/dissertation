\section{Summary}
\label{sec:actuation_summary}

This chapter presented two techniques to actuate on the physical environment.
The first technique is the usual way to operate through spaces and provides a higher degree of decoupling.
However, it requires actuator nodes to use our middleware's primitives. % requiere la cooperación de los proveedores...


The second actuation mechanism directly consumes \ac{rest}ful \acs{http} \acsp{api}.
This mechanism relies in the semantic description of the services, additional background knowledge and a goal state. % additional knowledge: background + initial
With that information, it is able to generate executions plans towards a goal.
These plans detail which resources to change and how to reach the desired goal.


We implemented the same scenario using these two actuation mechanisms.
In addition, since interoperability is one of our middleware's guiding principles, we sketched a third implementation which reuses these \ac{rest}ful \acs{http} \acsp{api} in our \Space{} model.
This hybrid actuation technique avoids any alteration on the space-based consumer or the \ac{http} provider.
Instead, it improves the \Space{} implementation with an agent in charge of generating execution plans.
This agent reuses the information from the space-based actuation not to require any additional information to the developer.


This implementation alignment between our space-based actuation and the direct web \acsp{api} consumption-based one presents some limitations.
For some of these limitations, we discussed other design alternatives and compared them with the chosen one.
The rest of the limitations only appear in more complex scenarios which are beyond the scope of this dissertation.
For our future work, we will implement these complex scenarios where advanced conflicts between the \ac{rest} and space-based computing worlds can arise.
%Furthermore, this chapter does not answer how to reuse the actuation mechanisms of the nodes using \ac{ts} patterns from applications using \ac{rest}ful architectures.
\section{Interoperation Example Between both Approaches}

This section aims to experimentally proof the validity of the assumptions made in the allignment of the proof-based actuation mechanism with space-based computing one.
To this end, we have designed a simple scenario to act on the light of a lamp.
Then, we have implemented the nodes which participate in this scenario in different manners.


On the one hand, the lamp node implements two different mechanisms to change this light's value:
\begin{enumerate}[label=\itshape(\alph*\upshape)]
  \item It provides a \ac{rest} \ac{api} and describe it using \restdesc{}. % REST API seguro 100%? por si acaso decir HTTP API?
        When a client invokes changes the value of the resource which describes the light's value, the node will physically change it.
  \item It is aware of the tasks written into the space and change the light's value accordingly. % esto es, se suscribe a ellas
\end{enumerate}


On the other hand, we have implemented the node which wants to change the light's value in the following ways:
\begin{enumerate}[resume,label=\itshape(\alph*\upshape)]
  \item It reasons over the knowledge and descriptions to get a plan to fulfill a goal.
	With this plan, it invokes the needed \ac{rest} services.
  \item It writes a task into the space describing its desire to change the light's value.
\end{enumerate}


Using these nodes, we implemented three scenarios.
In the first one is fully based on the proof-based mechanism and is composed by a \emph{(a)} provider and a \emph{(c)} consumer.
The second scenario presents two nodes using the space-based computing actuation patterns and therefore it is composed by a \emph{(b)} provider and a \emph{(d)} consumer.
The third scenario presents a mixed scenario where we have an \emph{(a)} provider and a \emph{(d)} consumer.
Sections \ref{sec:actuation_scn1}, \ref{sec:actuation_scn2} and \ref{sec:actuation_scn3} describe these scenarios respectively.
% TODO The implementation is publicly available in URL.


\subsection{Scenario 1}
\label{sec:actuation_scn1}

% poner un diagrama que presente el escenario
% dar detalles de cómo se ha implementado
%   Añadir descripciones y demás o ponerlas como anexo?
%   Pasos a seguir por parte del consumidor


\subsection{Scenario 2}
\label{sec:actuation_scn2}

% poner un diagrama que presente el escenario
% dar detalles de cómo se ha implementado
%   poner tarea de ejemplo


\subsection{Scenario 3}
\label{sec:actuation_scn3}

% poner un diagrama que presente el escenario
% dar detalles de cómo se ha implementado
%    comentar que asunciones de las del anterior capítulo se han tomado
%    por simplificar: agente que reside en el coordination space
% Pasos a seguir por el agente


% TODO añadir apartado de discussion?
% luego posiblemente se podrían medir algunos indicadores de los mismos
%   e.g. cuando código extra ha hecho falta añadir en el tercero para que se hablen entre sí
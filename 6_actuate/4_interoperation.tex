\section{Interoperation Between both Approaches}
\label{sec:actuation_interoperation}


\ac{rest} and \ac{rest}-like services constitute an emerging mechanism to directly actuate in the physical environment.
Therefore, their seamless integration opens the door to reuse the capacities of many existing devices. % interoperate
Specifically, we aim to integrate \ac{rest} services in our space-based model. % integration, reuse, alignment, close the gap
% we first need to close the gap between both approaches.
%This section aims to experimentally proof the validity of the alignment presented in the previous section.


This section explores this integration helped by a simple real-world scenario.
The goal of the scenario is to remotely change the light of a lamp.
Sections \ref{sec:actuation_scn1}, \ref{sec:actuation_scn2} and \ref{sec:actuation_scn3} present three different implementations of this scenario. % which guide the reader to the mentioned integration


The first implementation uses the proof-based actuation mechanism presented in Section~\ref{sec:restdesc}.
The second implementation uses the space-based actuation mechanism presented in Section~\ref{sec:actuation_space}.
Then, we mix nodes from the previous implementations in the third implementation.
This lead us to analyse the adjustments required to enable their interoperation.
% TODO TODO TODO The implementation is publicly available at THIS URL.
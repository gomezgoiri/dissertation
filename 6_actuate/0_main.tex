% ----------------------------------------------------------------------

\begin{savequote}[50mm]
If you live among wolves you have to act like a wolf.
\qauthor{Nikita Khrushchev}
\end{savequote}


\chapter{Remote Actuation}
\label{cha:actuate}


\newcommand{\restdesc}{\emph{RESTdesc}}
\newcommand{\spaceActuation}{\emph{Space-based actuation}}
\newcommand{\restActuation}{\emph{REST actuation}}
\newcommand{\hybridActuation}{\emph{Hybrid actuation}}


% the code below specifies where the figures are stored
\ifpdf
    \graphicspath{{\pathchapsix/figures/PNG/}{\pathchapsix/figures/PDF/}{\pathchapsix/figures/JPG/}{\pathchapsix/figures/}}
\else
    \graphicspath{{\pathchapsix/figures/EPS/}{\pathchapsix/figures/}}
\fi


%------------------------------------------------------------------------- 


In the previous chapter, we presented an energy-aware technique to search on the completely distributed semantic space. % technique, mechanism???
%This search mechanism promotes the end-to-end interaction between the objects.
However, acting over an \ac{ubicomp} environment is as important as observing what happens on it.
Although this dissertation describes this problem less thoroughly, we consider interesting to discuss it to provide the complete story. %not leave loose ends. % transmit a broader vision. % the big picture

\bigskip

This chapter presents and compares two techniques to change the physical environment.
The first technique is based on common \ac{ts} usage patterns while the second one relies on semantically described \ac{rest} services to create a plan to fulfil a given goal.
The rest of this chapter refers to them as \spaceActuation{} and \restActuation{} respectively.

Note that the second actuation technique is out the scope of the space-based communication covered in this dissertation.
However, the seamless integration of these \ac{rest} services opens the door to reuse the capacities of many existing devices.
Therefore, we propose reusing the \acs{http} \acsp{api} of already existing semantically described providers.
% without adapting them to the space-based computation. % a nivel practico hacemos más fácil empezar a usar nuestro middleware

This chapter explores this reuse helped by a simple real-world scenario.
The goal of the scenario is to remotely change the light of a lamp.

\bigskip

This chapter is organized as follows.
Sections \ref{sec:actuation_space} and \ref{sec:direct_actuation} present the two techniques and their corresponding implementations of the baseline scenario\footnote{
The implementations of the baseline scenario have been made public at \url{XXX}. % TODO TODO TODO Update the URL.
\label{fn:impl_available}
}.
Section~\ref{sec:hybrid_actuation} compares these techniques and explores how they can interoperate. % no funciona en ambos sentidos, igual habría que especificarlo!
This exploration is done by means of a new implementation\footref{fn:impl_available} which reuses much of the implementations from sections \ref{sec:actuation_space} and \ref{sec:direct_actuation}.
%Specifically, it presents a consumer willing to indirectly actuate through a space in a semantically described \acs{http} provider. % no sé si queda muy claro
After that, Section~\ref{sec:actuation_discussion} discusses the advantages and limitations of the previous implementation presenting further implementation alternatives.
Finally, Section~\ref{sec:actuation_summary} concludes the chapter.


% ----------------------------------------------------------------------



\section{Actuating through the Space}


% explicar brevemente patrones
% requirements: sistema de suscripciones (out of the scope)
% escenario de ejemplo en Ubicomp


In space-based computing, participants coordinate by reading and writing in a shared Space. % TODO usar define a nivel general para Space!
This encourages an uncoupled communication between applications using the same Space.

% IG: esto lo he homogeneizado porque tenia 3 estilos distintos :)
This section analyzes how to achieve this uncoupled communication following a top-down approach.
First, Section~\ref{sec:ts_patterns} briefly explains \acl{ts}'s most common application patterns. % according to the classification of \citet{freeman_javaspaces_1999}.
Second, Section~\ref{sec:envisaged_scenarios} uses some of these patterns to show how to build an \ac{ubicomp} scenario.
The scenario shows some requirements the middleware must fulfill.
Finally, Section~\ref{sec:notification} analyzes these requisites.



\subsection{\aclp{ts}' patterns}
\label{sec:ts_patterns}

According to \citet{freeman_javaspaces_1999}, there are four main application patterns which can be used with \ac{ts}: % IG: with o in?

\begin{description}
  \item[Replicated-worker pattern.] In this pattern, there is a master process and many worker processes able to compute the same task.
				    %% REDUCIENDO esto: %%
%				    The master takes a problem and divides it in tasks which are solved by any of the workers.
%				    More concretely the pattern is composed by the following steps:
% 				    \begin{enumerate}
% 				      \item The master divides a problem up into smaller tasks.
% 				      \item It writes them in the space.
% 				      \item Any of the many workers takes a task.
% 				      \item That worker computes the task.
% 				      \item It writes the result of that computation.
% 				      \item The master collects the results for the tasks it wrote.
% 				      \item Once the master has all the results, it combines them into a meaningful overall solution.
% 				    \end{enumerate}  
				    The master takes a problem, divides it into smaller tasks, and writes them tasks into the space.
				    Any available worker takes a task, processes it, and writes the result back into the space.
				    When all the workers have written their results, the master takes the results and combines them into a meaningful overall solution.
				    %Note that the workers accept new tasks whenever they are available and able to work.
				    %Therefore, while a worker is busy computing a big task, another one can solve many small tasks.
				    %In other words, this pattern naturally \emph{balances the load} on the space.
				    %Besides, it is \emph{scalable} since we can add more workers running in more machines without rewriting our code.
				    This pattern is scalable and naturally balances the load on the space.
  \item[Command pattern.] This pattern encapsulates the behavior into the tasks shared in the space.
			  Therefore, it requires (1) to share behavior through the space, and (2) any generic worker to be able to compute the behavior.
			  For instance, a graph in \ac{tsc} could include code in an interpreted programming language.
			  % nosotros pasamos de este jaleo
			  % hacer referencia a on-demand-code
			  %In that way, a generic worker application can compute whatever code other processes define and share through the space.
			  %This pattern is only possible in \aclp{ts} where the behavior can be described generically (e.g. in object-oriented \acp{ts}).
			  %Furthermore, the tasks to be performed over \ac{ubicomp} environments are not generically processable by any node.
			  %In contrast, each device is responsible of managing its own actuators on behalf of the rest.
			  %Therefore, this pattern will not be considered.
  \item[Marketplace pattern.] In this pattern, producers (or sellers) and consumers (or buyers) of resources interact to find the best deal. % TODO cita TripCom?
			  %It should be noted that the resources are products or services which can be bought and sold.
			  %Therefore, it is not applicable to the environments considered. % o use cases
			  %esto está relacionado con el paper aquel de TripCom y con el de Simon de hace un año en WoT
  \item[Specialist patterns.] In contrast to the replicated-worker pattern, in this pattern, each worker is specialized.
                              Therefore, each worker performs a particular task.
                              \citeauthor{freeman_javaspaces_1999} enumerate three subtypes:
			      \begin{description}
				  \item[Blackboard pattern.]
					It associates the concept of the space to a \emph{blackboard}.
					Following this analogy, the master is associated with a teacher, tasks with \emph{problems} and the specialized workers with \emph{students}.
					The blackboard pattern starts when the \emph{teacher} writes a \emph{problem} in the \emph{blackboard}.
					The \emph{students} observe the space and write their intention to contribute to solve the problem (i.e. \emph{raise their hands}).
					The \emph{teacher} selects to the expert which will make a modification.
					After the modification, the \emph{teacher} decides if it found the solution or another \emph{student} should contribute.
				  \item [Trellis pattern.]
					In this pattern, the master arranges the problem into low-level, mid-level, and high-level pieces.
					The workers of each level benefit from the refined data provided by the immediate level below.
				  \item [Collaborative patterns.]
					It encompasses all the patterns which allow nodes to collaborate to complete a greater task (i.e. by creating a workflow).
					%They are specific to the domain so they cannot be generalized, but it includes all the patterns where the nodes collaborate to complete a greater task
					%(e.g. by creating a workflow).
			      \end{description}
\end{description}


% Voy preparando el terreno para que se vea que los primeros son más de computación y el otro de negociación
To summarize, the \emph{replicated-worker pattern} is centered in optimizing the computation by parallelizing tasks.
The the \emph{command pattern} can be seen as an abstraction of the latter where the behavior is shipped in the tuples.
The \emph{marketplace pattern} allows negotiation of two entities through the space.
% Coño! pero si el especialista te permite colaborar...
Finally, the \emph{specialist pattern} allows nodes with distinct capacities to cooperate towards a common goal.


% Posible TODO Un esquema dividido en 4 en el que aparecen todos los patrones explicados.
% No es fácil mostrar el 2do y el 3ero.
% Una intentona de estos y otros patrones inventados en Lancaster:
%     https://docs.google.com/a/deusto.es/document/d/1QXGsQ4-wAByc_ByQ-Tts-TSphSQAzfO_c3mcgoZdloc/edit


% IG: ves... esta seccion parece q sale de la nada y creo que es el objetivo principal del capitulo, no? a ver si puedes orientar la intro del capitulo para mostrar esto mas... basicament que se vea que esto es de implementar movidas.
\subsection{Envisaged Scenarios} % IG: he aprendido palabro nuevo...
\label{sec:envisaged_scenarios}
% 1era iteración en reescritura, se puede mejorar mucho

This section devises two stereotypical scenarios for home automation.
Both scenarios emphasize how devices can coordinate in a decoupled mode using the patterns seen in the previous section. % más de uno?
%The scenarios are formed by mobile phones, and embedded platforms with sensors and/or actuators.
Specifically, we believe that the \emph{specialist patterns} fit the best to the needs of \ac{ubicomp} environments:
\begin{itemize}
  \item The devices in \ac{ubicomp} often serve to very specific needs. % IG: comentario general de esto de las comas, que me las pones todo al reves... el sujeto y el predicado sin coma entre medio (a noser que tenga de esas cosas descriptivas, en plan "jose, que es negro, se ha quemado"). Y la coma se pone cuando es algo complementario del tipo de: "en este capitulo, ..."
        For example, let us imagine a mobile phone showing a message or an embedded device turning on the lights of a room. % IG: lo de mobile es el estandard de la tesis? lo de cell no, no?
        %Although there may exist some redundancy in the tasks which different nodes can be accomplish, the task are not nodes are not as interchangeable as in... son tareas simples, no necesitan optimizar ni lanzar muchas
  \item These tasks are usually lightweight. % TODO primera vez que se habla de "task", comentar que para TripCom eran servicios web y que crearon un sistema (WXSW?)
        They do not require large computation resources, but to achieve a concrete and simple goal. % IG: comor? frase rara
\end{itemize}
Therefore, \ac{ubicomp} generally faces a problem of collaboration between nodes with different capacities.
In this problem, computation or negotiation aspects are secondary. % se entenderá a qué me refiero con computation??


%In this pattern, a master writes a task into the space and waits for its result, which is performed by some of the workers specialized in this particular task (e.g. show a message or regulate the temperature).

% IG: hala! y esto de sopeton... lo que decia, preparame el terreno un poco
\subsubsection{Scenario 1: Temperature Regulation}

The first scenario presents a room populated with several kind of sensors such as Oracle's SunSPOTs \citeweb{sunspot},
% TODO TODO TODO referenciar a la parte donde se hayan referenciado bien los dispositivos! (e.g. see ~\ref{environment})
% Para el resto proveer al menos una URL!
Digi's XBee sensors with an IP gateway,
the sensors on a KNX domotic bus \citeweb{knx}, and a fan connected to a FoxG20 \citeweb{foxg20} embedded platform as an actuator (see Figure \ref{fig:devices_scenario}).
In addition, an Android application \citeweb{android} semantically stores the temperature preferences of the user.


\InsertFig{devices_scenario}{fig:devices_scenario}{
  The devices used in the devised scenarios.
}{
}{0.7}{}


An independent node (i.e. the master node) continuously reads from the space % using \emph{read primitive}
(1) the room's temperature, and
(2) the user's desired temperature.
Note that the master node does not care about who exactly provides the temperature information.
It just takes the first available graph from the space.
When the second one is below the first one, it generates a ``\emph{decrease temperature during a certain period}'' task which can be consumed by different independent worker nodes.
In this case, the FoxG20 periodically checks for orders it can fulfill extracting them from the space. % consumes them with a \textit{take primitive}.
% zasca, acabamos de poner en bandeja que nos rechine lo de comprobar periodicamente



\subsubsection{Scenario 2: Sedentary Lifestyle Checker}

The second scenario presents an application which helps the user to avoid a sedentary lifestyle by giving him different warnings.
We consider the recommendation that states that an adult should walk at least 10.000 steps in an ordinary day \citep{tudor2002taking}.
Taking into account the expected steps which should have been completed at each moment of the day, the application generates different priority level messages.

There are several nodes involved in this task.
Each node runs on a device. % which belongs to a user.
First, an Android phone periodically updates the number of steps covered by a user that day \citeweb{pedometer}.
%Besides, it writes her profile, more important to this problem: her age. % simplificado un poquillo
Second, there is an undefined number of devices which know how to warn the user about her unhealthy behavior.
Third, there is a node in charge of generating the warnings.


\medskip


The application follows the \emph{blackboard pattern}.
The node which generates the warnings is the \emph{teacher}.
The \emph{problem} is the warning for the user.
The \emph{students} are the devices which can warn the user. % IG: no es nada intuitivo. Leyendo despues lo entiendo, pero... es raro


When the \emph{teacher} writes a warning for a user, the devices able to warn the user write their characteristics into the space. % IG: alguna cosa mejor que characteristics?
In this example, these characteristics are the device ownership and the intrusiveness of the warning method.
The \emph{teacher} reads these characteristics, chooses the most appropriate \emph{student} to solve the \emph{problem} and updates the \emph{problem} to indicate the selection.
%If no device writes their characteristics, the \emph{teacher} can wait longer.
Finally, the selected device takes any warning for it and warns the user. % no tiene respuesta
If the chosen device does not take messages, the \emph{teacher} can update the \emph{problem} with a new selection.


With this application, if the priority level is low, the user receives the warning in a less intrusive way.
For example, room's light brightness can be increased for low priority notifications, a Chumby \citeweb{chumby} can show an icon for normal priority ones and the user's mobile phone show directly the message when it is a high priority warning. % IG: yo anyadiria una tablita para los dos scenarios mostrando los actuadores y los niveles. Aqui esto me salta un poco de repente.
The remarkable characteristic of this mechanism is that the devices present in the environment at each moment can vary and the application will still fulfill its goal.




\subsection{Notification mechanism}
\label{sec:notification}


In the scenarios described in the previous section, some writings in the space trigger other node's action.
For example, a device must show a message whenever a new warning is written in the space. % e.g. a task or a result
To be aware of these writings, the node can either poll the space or rely on a notification mechanism. % async or blocking
Obviously, the later encourages a more efficient use of the network. % is more optimal and more scalable approach.
% IG: encourage no creo que sea la palabra...

As a consequence, a notification mechanism is highly advisable to fulfill \ac{ts}'s patterns in a distributed environment.
Although the implementation of this notification mechanism is out of the scope of this thesis,
% Hablar de requisitos, describir nuevas primitivas de suscripciones o alternativas de implementación
it should comply with the following aspects:
\begin{itemize}
  %  Debe permitir polling (sync), por ello proponemos nuevas primitivas: read_async y take_async.
  \item Do not substitute the polling mechanism. % porque todavía puede ser interesante en otros casos
        The middleware must provide additional \emph{read} and \emph{take} primitives.
  % Se evalúan sobre el espacio, no sobre el outer-space. 
  \item Since the notification mechanism intends to allow coordination patterns,
        it must consider the knowledge from the coordination space. % TODO comprobar que lo llamé así en el anterior capítulo
        In other words, knowledge from the \emph{outer-space} will not trigger notifications. % TODO usar constante para esto
  % Evaluarse cada X, no siempre que se escriba algo o se vuelve loco.
  \item The evaluation of the subscriptions must not interfere with the writing process (e.g. introducing a delay).
        Therefore, it must run asynchronously.
  % Debe ser sencillo de implementar por cualquier nodo. Para ello proponemos un mínimo contrato: callback URL.
  \item Ease its adoption by any type of computing platform.
	This can be achieved by reducing the requirements on the \emph{clients}.
        For instance, a callback \ac{url} passed during the subscription can represent a minimal contract between the client and the server. % no viceversa
  % Deben caducar para no mantener suscripciones de dispositivos que ya no existen.
  \item Provide additional subscription removal mechanisms.
	\ac{ubicomp} scenarios are composed by unreliable devices which may frequently join and leave the space.
	In this situation, the correct use of unsubscription primitives cannot be guaranteed.
	This may worsen the performance of the system with useless subscriptions from absent devices.
	Therefore, the device managing the subscriptions should adopt more proactive mechanisms.
	For example, it may let the subscriptions expire after a lifetime or remove them when it discovers the unavailability of a callback \ac{url}. % algo así como garbage collection
	% A) The subscriptions must expire.
        % The expiration will allow to delete subscriptions from absent devices. % no longer present devices
        % This also implies that clients are responsible for periodically updating their subscriptions.
        % B) Additional mechanism
        % For instance, a \emph{garbage collection} agent can check the availability of callback \acp{url}.
        % For unavailable callback \acp{url}, the subscriptions can be removed.
\end{itemize}


The main drawback of any subscription mechanism is that it breaks the \ac{rests} property of the \ac{rest} style.
This implies that network performance will improve at the cost of scalability, simplicity and reliability. % TODO poner un see Section~\ref{X} ???


% hasta AQUI!
% Therefore, although it is not the main focus of this thesis, in this section we present a subscription mechanism which can be used on top of the \ac{tsc} middleware proposed.
% The requirements that this mechanism needs to fulfill are the following ones:
% \begin{itemize}
%   \item It should be independent of \ac{tsc}'s writing and reading primitives.
% 	This requirement ensures that frequent writings do not lead to excessive processing in resource constrained nodes.
% 	The main drawback of this requirement is that a node does not now when relevant data is written in the space \emph{per se}.
% 	A developer needs to advertise when it writes relevant data.
%   \item Any node running our solution must be able to implement this mechanism.
% 	In other words, no new dependencies should be added.
%   \item The new primitives should use elements the current developer is used to (i.e. templates or RDF triples).
% \end{itemize}
% 
% explicar cómo funciona
% https://otsopack.readthedocs.org/en/latest/subscriptions.html
% \subsubsection{Subscription primitives}
% 
% The subscription primitives the developer should use are the following ones:
% \begin{itemize}
%   \item \emph{Subscribe}. The node subscribes to the given template returning an URI which identifies the subscription.
%     \begin{minted}{java}
% URI subscribe(space_uri, template, listener)
%     \end{minted}
%   \item \emph{Unsubscribe}. Unsubscribes to a subscription given its subscription URI.
%     \begin{minted}{java}
% void unsubscribe(space_uri, String subscriptionURI)
%     \end{minted}
%   \item \emph{Notify}.
%     \begin{minted}{java}
% void notify(space_uri, template)
%     \end{minted}
% \end{itemize}
% 
% 
% \subsubsection{Deployment}
% 
% The nodes responsible of handling subscriptions and notifications are called \emph{bulletin boards}.
% Other nodes belonging to the same space, discover them using a out of scope method and publish their subscriptions and notifications using their HTTP API.
% 
% Each \emph{bulletin board}:
% \begin{itemize}
%   \item Belongs to a space.
%   \item Exposes a subscription API.
%   \item Shares its subscriptions with other bulletin boards which belong to the same space.
%   \item Propagates the notifications to the relevant nodes using the \emph{callback url} provided by them.
% \end{itemize}
% 
% 
% The Figure~\ref{} describes a simple subscription use case.
% \begin{enumerate}
%   \item N1 subscribes to BB1 with a template t1.
%   \item BB1 propagates the subscription provided by N1 to BB2 and BB3
%   \item N3 notifies to BB3 about t2.
%   \item Since t1 matches t2, BB3 tries to notify to N1 using the callback URI provided during the subscription process.
%   \item Unfortunately, the BB3 cannot notify to N1 due to unexpected network problems.
%   \item BB3 propagates the notification of t2 to BB2.
%   \item BB2 reaches N1, so it notifies it about t2 using the callback URI.
% \end{enumerate}
% 
% TODO poner la foto esquemática de https://otsopack.readthedocs.org/en/latest/subscriptions.html

\section{Direct Consumption of \acs{rest} services}

% por qué? motivación: permitir a otras app integrarse en nuestro espacio
\ac{rest}-like services constitute an emerging mechanism to actuate in the physical environment. % REST-like, no compliant with Fielding
Therefore, their seamless integration would open the door to reuse the capacities of many existing devices. % interoperate
However, this requires applications to directly use the actuator's node \ac{http} \ac{api}.


In contrast, the actuation mechanism presented in the previous section uses the \Space{} (i.e. it is indirect).
The mechanism implies that the actuator must be aware of the content of the space.
For instance, a heater must check the space to find if a new desired temperature was written.
% lo podrá reusar a través de un intermediario
% interop era una de las cosas que queriamos cuidar
% y qué pasa con las soluciones REST que quieren usar lo nuestro?


In this section we propose to close the gap between both approaches.
To do that, we propose to transparently invoke \ac{rest}-like services on behalf of the developer.
From her perspective, she will still use the space-based patterns presented in the previous section.
However, the middleware will be enriched with the actuation capacities of third \ac{http} \acp{api}.


To define that integration, we first explain the mechanism used to describe and select the appropriate \ac{rest} services (Section~\ref{sec:background_restdesc}).
Then we sketch this integration by answering two key elements:
\begin{itemize}
  \item How do the inputs for this mechanism relate to the ones used in the \ac{ts} patterns? (Section~\ref{sec:inputs_proof})
  \item Which node is responsible for triggering this mechanism? (Section~\ref{sec:responsible_proof})
\end{itemize}



\subsection{Background}
\label{sec:background_restdesc}

In \ac{wot} physical changes in the environment are performed by manipulating \ac{http} resources.
According to the \ac{rest} principles, a client should navigate through these resources with no prior knowledge of the \ac{api}.
% Copiar esta explicación mejor de algún lado:
The client should 
\begin{enumerate*}[label=\itshape(\arabic*\upshape)]
  \item interpret the representations provided by the server and then
  \item choose the appropriate state transition from the hypertext according to its intention. % y su conocimiento básico del protocolo: CRUD
\end{enumerate*}


% es HATEOAS?
As explained in Section~\ref{sec:network_properties}, semantic representations do not include a native way to express the hypertext. % TODO realmente se explica?
To solve this, three solutions can be adopted:
% Unos proponen extender con ontologías
\begin{enumerate}
  \item To use an ontology to represent the hypertext \citep{kjernsmo_necessity_2012},
  \item To embed the hypertext independently to the representations on the \ac{http} headers \citep{mark_web_2010}, and
  \item To provide a description of the resources using the \ac{http} OPTIONS verb \citep{verborgh_functional_2012,verborgh_ijcs_2014}.
\end{enumerate}

The latter two enable to discover resources and state transitions without adding metadata to the representations.
This allows not only to describe semantic representations, but any type of formats.

\citeauthor{mark_web_2010}'s \citep{mark_web_2010} approach is extended by \citet{erik_profile_2013} to define how to embed additional semantics to process a resource representation.
These additional semantics are called profiles and are identified by an \acs{uri}.


\bigskip


\citet{verborgh_ijcs_2014} present a more expressive solution which goes beyond simply describing the resource's type.
It also allows to semantically describe the knowledge needed to use a concrete \acs{http} verb on a resource, and the content this call returns. % o precondition
The materialization of this proposal is called \restdesc{} \citep{verborgh_functional_2012}.


% TODO mirar si tiene cabida la mención de otros enfoques para describir recursos que no son muy RESTful
%Different ways exist to describe \ac{rest} services. % mencionar WADL, etc.?
% citar a donde se hable de \restdesc{} y así ya se empieza a explicar la solución de forma discreta.
\restdesc{} describes \acs{http} methods using rules expressed in the \ac{n3}\footnote{\url{http://www.w3.org/TeamSubmission/n3/}} language.
% he evitado explicar que las reglas tienen premisa y conclusión, porque me parece demasiado obvio
A rule's \emph{premise} expresses the requirements to invoke a \ac{rest} service.
A rule's \emph{conclusion} expresses both the \ac{rest} call that needs to be made and the description of that invocation result.


\citet{verborgh_ijcs_2014} propose a proof-based composition mechanism for Web \acp{api} using \restdesc{}.
This mechanism uses as inputs:
\begin{enumerate*}[label=\itshape(\arabic*\upshape)]
  \item an initial state,
  \item a goal state,
  \item Web \ac{api} descriptions using \restdesc{}, and
  \item optional background knowledge.
\end{enumerate*}
Each of these inputs are semantically expressed and therefore, they can be processed by standard \ac{n3} reasoners.
The reasoners generate proofs about how to achieve the goal starting from the initial state using the rest of the inputs.
These proofs can be seen as steps that need to be made to reach a desired state.


Additionally, \citet{verborgh_ijcs_2014} distinguish between pre-proof and post-proof.
The first, are those which assume that the execution of all \acs{api} calls will behave as expected.
The latter, can be seen as a \emph{revision} of the pre-proof.
It executes the Web \acs{api} of the pre-proofs and uses actual execution results to generate a new proof.



\subsection{Inputs for the Proof-based Actuation Mechanism}
\label{sec:inputs_proof}

The reasoning process presented by \citet{verborgh_ijcs_2014} uses as inputs: % presented by en vez de posesivo, porque "X et al.'s" queda como el culo
an initial state,
a goal state,
Web \ac{api} descriptions using \restdesc{}, and
optional background knowledge.
% inputs: clues
% goal: task


\begin{description}
  \item[The descriptions] must be read from the actuation nodes prior to reasoning. % obvio
        They can be read with a \emph{ad hoc} discovery mechanisms.
        However, in order to simplify the design of the middleware, it is sensible to use one of the already existing mechanisms:
        
        \begin{itemize}
	  \item An agent could discover the descriptions and simply write them in the coordination \Space{}. % y ahora habrá que definir que es un agente?
		Then, any node willing to reason will read them from the \Space{}.
	  \item The descriptions could be part of the \clues{} presented in Chapter~\ref{cha:searching}.
		This option ensures that they will be available in any \consumer{}.
		Besides, the static nature of these descriptions does not break the stability assumption of the \clues{}. % explicar más?
        \end{itemize}

  % TODO TODO TODO especificar que este mecanismo adicional de usar clues es optional y no el usado en nuestro escenario simple
  \item[The initial state and background knowledge] can be acquired both from the coordination space and the outer space.
	However, reading all the space would be highly inefficient.
	To reduce readings on the space (i.e. network usage), we propose a procedure composed by two reasoning steps.
	In the first one we only use local incomplete knowledge derived from the \clues{}.
	Then, we read from the \Space{} just the knowledge needed to confirm the pre-proofs obtained in the first reasoning.
	The second reasoning uses this knowledge to get real pre-proofs. % real o actual pre-proofs
	
	% Y ahora explicado en mayor detalle:
	Therefore, a node which wants to actuate over the space will need to obtain the \emph{clues} from the \ac{wp}.
	%These clues, as explained before, tell what kind of content other nodes provide. % un poco redundante
	Let us assume that these \emph{clues} are composed by the predicates used by the nodes which provide content. % es una de las alternativas que se planteaban
	The existence of a predicate used in a premise does not necessarily imply that this rule can be used.
	Nevertheless, its absence does imply that it will not be used (see Figure~\ref{fig:activation_rules}).
	Therefore, we can create temporary \emph{activation rules} from \clues{} which activate those potential rules. % latter rules las que se han mencionado primero

	\InsertFig{activation_rules}{fig:activation_rules}{
	  Sample clues, two rules and the activation rule created from the first rule.
	}{
	  According to the clues shown, the second rule will never be invoked.
	}{0.8}{}

	An \emph{activation rule} for a rule R contains a \emph{true} in the premise.
	The conclusion is made by R's premise substituting the variables with fictitious \acsp{uri} with a common prefix (see Figure~\ref{fig:activation_rules}).
	These fictitious \acsp{uri} are used to distinguish when a triple should be replaced by actual knowledge from the space. % explicar mejor?


  \item[The goal state] should be created from the task to reuse the primitives and the data from the \ac{ts}-based actuation mechanism.
	For instance, from a task of \emph{regulate temperature to 6ºC} we can deduce the goal state of \emph{temperature of 6ºC}.
	This translation is out of the scope of the dissertation.
\end{description}


\subsection{Responsibility for Triggering the Proof-based Actuation Mechanism}
\label{sec:responsible_proof}

Section~\ref{sec:background_restdesc} described two coarse-grained steps:
\begin{itemize}
  \item Reasoning over the descriptions, background knowledge, an initial state and a goal state.
        The result of the reasoning process if a pre-proof, which can be seen as a tentative \emph{execution plan} to achieve the goal.
  \item Check the execution plan by following it.
\end{itemize}


This mechanism can be performed by any node.
% no obligamos a implementar una u otra, pero recomendamos esto: XXX
In this thesis, we briefly analyze three different alternatives and their trade offs.
However, we do not adhere to any of them.
We leave as a future work to implement and quantitatively compare them.

\begin{enumerate}
  \item An agent which resides in the same machine as the \emph{coordination space} can trigger the process. % TODO buscar su nombre bueno
	Doing so, it can locally consume the knowledge available in the \emph{coordination space}.
  \item A more flexible approach would consist of letting any \consumer{} interested on changing the environment to trigger it.
	If these \consumers{} use the search mechanism presented in the Chapter~\ref{cha:searching}, they will have background knowledge about other nodes.
	This reduces the dependency on the node providing the \emph{coordination space}. % TODO buscar su nombre bueno
	However, it requires them to perform tasks such as reasoning and checking the pre-proofs.
	While the latter increases the network usage, the first increases the computation.
	As we already mentioned in previous chapters, these tasks severely affect to the energy consumption.
	Furthermore, some resource platforms will not be even able to reason.
  \item To mitigate that problem, we could delegate this task only on the nodes able to perform such tasks.
	In fact, these nodes could follow the \emph{replicated-worker pattern}.
	They could read from the space goals to trigger the process (i.e. \emph{reasoning tasks}).
	Apart from balancing the load between all the worker nodes, any node can stop being worker at any time by not taking more \emph{tasks} (e.g. if it has low energy).
	These nodes must be \consumers{} to use the \clues{} from the search mechanism as background knowledge.
\end{enumerate}


% Poner ejemplo adaptado del paper de WoT2013?


%Proceso de suggest:

%\begin{itemize}
% \item obtener todo el conocimiento necesario para el proceso?
% \item razonamiento sobre conocimiento
% \item parsear el resultado para invocar servicios HTTP
% \item monitorizar que se ha complido el cambio?
%\end{itemize}
\section{Hybrid actuation}
\label{sec:hybrid_actuation}

The actuation technique presented in Section~\ref{sec:actuation_space} is the natural way of acting using a \Space{}.
However, \ac{rest} and \ac{rest}-like services are well-accepted mechanisms to expose limited devices' actuation capabilities \citep{guinard_internet_2011}. %  in the physical environment
Therefore, their seamless integration (i.e., interoperation) opens the door to reuse the capabilities of many existing devices.
This section aims to integrate semantically described \ac{rest} services in our space-based model. % integration, reuse, alignment, close the gap
That is, mix \spaceActuation{} with \restActuation{}.
% we first need to close the gap between both approaches.
%This section aims to experimentally proof the validity of the alignment presented in the previous section.

Section~\ref{sec:actuation_comparison} compares \spaceActuation{} and \restActuation{} and argues the need of this integration.
Then, Section~\ref{sec:actuation_scn3} presents a new implementation of the baseline scenario.
This implementation shows how the \Space{} can transparently invoke \ac{rest} services on behalf of any consumer following the \spaceActuation{}.
To this end, it reuses nodes from the implementations presented in sections \ref{sec:actuation_scn1} and \ref{sec:actuation_scn2}.
%This lead us to analyse the adjustments required to enable their interoperation.


\subsection{Comparison}
\label{sec:actuation_comparison}

The actuation technique explained in Section~\ref{sec:actuation_space} requires a subscription mechanism, but in exchange, it provides space and time decoupling.
Obviously, this decoupling comes at the price of dependency on the \Space{}.
That is, two nodes will not be able to communicate with each other without the \Space{}.


The second approach presented in Section~\ref{sec:direct_actuation} offers independence of the \Space{}: any node can directly invoke a service to act over the environment.
This approach allows reusing \ac{wot} applications' actuation capabilities, providing they are properly described. % apps, APIs o services? % no digo integration/interop porque es sólo one-way
This reuse is enabled by a rule-based reasoning engine. % reasoning engine ya queda bien?
% The unavailability of these engines in many computing platforms may an obstacle for its massive adoption. % como apunte de su practicalidad, pero no sé si venía muy a cuento



% Tabla: XXX
% autonomía: poner o simplemente indirect vs direct?
% requirements in consumers: estar pendiente del espacio vs razonar
% requirements in providers: estar pendiente del espacio vs (describir semánticamente servicio => equivalente a tasks)
\begin{table}[htbp]
  \caption{Characteristics of the discussed actuation mechanisms.}
  \begin{center}
    \footnotesize
    \begin{tabular}{llll} 
      \hline
      \multirow{2}{*}{Actuation} &
      Communication & % o poner autonomias y demás para enfatizar?
      \multirow{2}{*}{Benefits} &
      Required \\
      &
      style &
      ~ &
      features \\
      \hline
      Space-based & Indirect & Decoupled communication & Subscriptions \\[0.2cm]
      REST-based & Direct & Reuse of third \ac{wot} & Rule-based \\ % mencionar interop? % TODO ampliar la noción de WoT a REST-like?
      & & applications & reasoning \\ % no digo "wot actuation capacities porque ya está verbalizado en la explicación"
      \hline
    \end{tabular}
  \end{center}
  \label{tab:actuation_mechanisms}
\end{table}


\bigskip

The effect of both mechanisms in resource constrained devices is affected by their computing and networking activities.
In order to analyse their impact, we measure two metrics for each implementation: the time and total amount of requests.
In addition, as the number of additional actuators in the physical environment affects these measurements,
we consider four additional variations of the scenarios previously explained. % to capture how they scale to more complex scenarios?
Apart from the scenario with 1 actuator, we contemplate scenarios with 200, 400, 600, 800 and 1000 actuators.


In order to measure the networking activity, we calculate the total number of requests performed in each scenario.
We assume that each new actuator will behave exactly as each of the actuators described in \implSpace{} and in \implRest{}.
Consequently, in \spaceActuation{} each new actuator writes 2 graphs into the space and in the \restActuation{} each \ac{api} is crawled once.
The crawler checks each of the 5 different calls clients can make to each \ac{api} (1 \acs{http} OPTIONS request and 4 \acs{http} GET requests).
Figure~\ref{fig:requestsTechnique} shows the estimations for these scenarios.


\InsertFig{requests_by_techniques}{fig:requestsTechnique}{Total amount of requests per technique}{}{0.7}{}


% Dado que depende de diseño, lo importante por tanto no es el número ese, sino quienes se ven afectados.
The number of requests which need to be made per new provider depends on the parameters detailed above (i.e. graphs written, times each \ac{api} is crawled and calls needed to discover a whole \ac{api}).
As these parameters are design-dependent, the slopes shown by the Figure~\ref{fig:requestsTechnique} might vary from one implementation to another.
% p.e. podriamos reducir el número de recursos por API o escribir un grafo en vez de dos
In any case, the figure does show that none of the techniques behave in a scalable manner.
However, if we analyse the nature of this network traffic, we can see that
\begin{enumerate*}[label=\itshape\bfseries(\arabic*\upshape)]
  \item in the \spaceActuation{} the \Space{} is involved in all the communications, and
  \item in the \restActuation{} \nodeConsRest{} is the source of every request
\end{enumerate*}.


In \spaceActuation{}, all the participants must be aware of what is written into the space to react (i.e., they are proactive).
Both consumers and providers read and write from the space, subscribe to specific changes and receive notifications.
Thanks to this specificity, they are only affected by the contents they are interested in. % reduciendo la actividad
On the contrary, the \Space{} will be involved in any networking activity.
Therefore, although it depends on the number of providers and consumers' interactions, the networking activity will presumably be high for the node hosting the \Space{}.


On the other hand, \restActuation{} requires consumers to have prior knowledge about the environment to reason over it. % frase susceptible de simplificar
Since this knowledge must be acquired from remote nodes before reasoning, this approach demands an extra network usage that the first does not.
Therefore, depending on the network size, this technique might not be appropriate for resource-constrained consumers.


\bigskip


In order to analyse the computation activity in the scenarios' implementations, we measure the time needed to run both implementations of the scenario in a \emph{Raspberry Pi} (Model B\footnote{
RAM Memory: 512MB.

CPU: 700 MHz Low Power ARM1176JZ-F Applications Processor.
}).
In this case, we also contemplate scenarios with 1, 200, 400, 600, 800 and 1000 actuators.
The additional actuators represent smart-heaters and mimic \nodeProvSpace{} and \nodeProvRest{} in each implementation.
That is, in the \spaceActuation{} executions each actuator writes 2 graphs (5 triples in total) and subscribes to temperature preference changes.
In the \restActuation{} executions each actuator has an \ac{api} equivalent to \nodeProvRest{}'s \ac{api} (i.e., 2 rules and 14 triples are exposed).
For example in the \nodeProvRest{} executions with 1000 actuators, the \nodeConsRest{} reasons over 2000 additional rules and 14000 additional triples.
Figure~\ref{fig:timeTechnique} shows the results of these executions (each case is executed N=50 times).
Note that the time measured 
\begin{enumerate*}[label=\itshape\bfseries(\arabic*\upshape)]
  \item only includes the time \nodeConsSpace{} and \nodeConsRest{} need to make a change in the physical environment (i.e., it excludes actuators' previous writings and subscriptions), and
  \item ignores the delays added by the communication between nodes % networking time?
\end{enumerate*}.


\InsertFig{performance_by_techniques}{fig:timeTechnique}{Time needed to make a change in the environment per technique}{}{0.7}{}


% There is much room for performance improvement in the mechanism
Figure~\ref{fig:timeTechnique} shows that \spaceActuation{} needs much less time than \restActuation{} to make a change.
Furthermore, consumers and providers in \spaceActuation{} only perform trivial computing tasks: interpreting results. % pattern matching sólo en el caso del coordination space!
The space, which corresponds to our solution's \coordspace{}, will be in charge of querying and notification mechanisms.
The computation activity in the space varies depending on the complexity of these mechanisms' implementation, which is beyond the scope of this dissertation.
However, note that due to the prototype nature of the subscription mechanism implemented, for the \spaceActuation{} implementation there is still room for a great performance improvement in this metric. % prototype, prototyping?


Finally, from the time needed to complete the \restActuation{} executions, most of the time is spent reasoning in the consumer.
% Furthermore, this actuation technique also generates additional computation activity on the node responsible for reasoning.
%Both aspects have a negative impact in the energy consumption.
Consequently, from the data consumer perspective, this approach will generally be more energy demanding than the first one. % might porque no está medido, pero es algo intuitivo
In contrast, this mechanism demands few things to the provider: to serve \ac{http} resources and provide their descriptions. % aclarar quien es el provider/actuator?


\bigskip



% Tabla: XXX
% autonomía: poner o simplemente indirect vs direct?
% requirements in consumers: estar pendiente del espacio vs razonar
% requirements in providers: estar pendiente del espacio vs (describir semánticamente servicio => equivalente a tasks)
\begin{table}[htbp]
  \caption{Foreseeable networking and computing impact on the nodes involved in the actuation mechanism.}
  \begin{center}
    \footnotesize
    \begin{tabular}{llp{4cm}p{4.4cm}}
      \hline
      \multirow{2}{*}{Actuation} &
      \multirow{2}{*}{Perspective} &
      \multicolumn{2}{c}{Activity} \\
      ~ &
      ~ &
      \multicolumn{1}{c}{Networking} &
      \multicolumn{1}{c}{Computation} \\
      \hline
      \multirow{3}{*}{\ac{ts} patterns} & Provider & Proactive, limited activity & Limited: Results Interpretation \\
				    ~ & Consumer & Proactive, limited activity & Limited: Results Interpretation \\
				    ~ & Space & Reactive, high activity & Varies with the implementation \\[0.2cm]
      \ac{rest} \acp{api} & Provider & Reactive, limited activity & Limited: Handling requests \\
        consumption       & Consumer & Proactive, high activity &  Demanding: Reasoning \\ % Y comprobar planes...
      \hline
    \end{tabular}
  \end{center}
  \label{tab:actuation_networking_computing}
\end{table}


Table~\ref{tab:actuation_networking_computing} summarizes the previous discussion.
As can be appreciated, the providers in the second actuation mechanism are more lightweight.
They just attend to the request received using HTTP.
As a consequence of these few requirements, exposing the actuation capabilities of the limited devices with HTTP is a consolidated trend.
This tendency is backed by the \ac{wot} initiative.
To make these web-enabled actuators automatically reusable by consumers, the second mechanism only requires them to describe their resources semantically.
This can be done before deploying them and does not affect to their usual operation.


From the perspective of our space-based computing solution, reusing the existing HTTP providers will potentially open a new world of actuation possibilities.
Preferably, we should do it keeping \spaceActuation{} consumers' simplicity. % Simplicidad: networking y computing que aparecen en la tabla.


% ¿Podríamos decir que el proof-based es más genérico?
%  + Space-based es muy dependiente del tipo de datos escritos en el espacio.
%    Proof-based también lo es a decir verdad.
%  + El proveedor del Space-based no tiene forma de saber cómo activar un mecanismo concreto.
%    Salvo que tenga posibilidad de saber a qué se han suscrito otros...
%    No sé como expresar esa diferencia.


\subsection{Baseline scenario: Implementation 3}
\label{sec:actuation_scn3}
\newcommand{\implMix}{\emph{Implementation 3}}

This implementation aims to prove that our space-based middleware can easily reuse third-party applications' providers capabilities. % easily? transparently?
It presents the space and the consumer (i.e., \nodeConsSpace{}) from the \implSpace{}, but it replaces the provider with the one from the \implRest{} (i.e., \nodeProvRest{}).
Both nodes have a good foundation for the interoperation because they use the same vocabularies. % to describe the knowledge they manage.
However, the \nodeProvRest{} and the \nodeConsSpace{} rely on different communication mechanisms: direct communication and indirect communication.


To close the gap between these two worlds, we avoid changing the participant node's implementations.
Instead, we create an agent which acts on the consumers' behalf. % TODO Yo he entendido que se sobreentiende qué es un agente
% WIKIPEDIA: "which derives from the Latin agere (to do): an agreement to act on one's behalf"
This agent resides in the same machine as the \Space{}, but its existence must not interfere with the \Space{} one. % dificil explicar en qué terminos: es otro proceso
Therefore, it should run on an independent process.


The agent takes care of the tasks that \nodeConsRest{} does in the \implRest:
\begin{enumerate*}[label=\itshape(\arabic*\upshape)]
  \item crawls the discovered \acsp{api}\footnote{The discovery process is out of the scope of this implementation.},
  \item reasons about their data to get a plan, and
  \item follows the resulting plan performing \acs{http} requests.
\end{enumerate*}


To trigger the reasoning, the agent awaits for new tasks written into the space.
Listing~\ref{lst:generic_task_subscription} shows the subscription template used by the agent.
Providing that after reasoning the agent does not find a plan to achieve a task, it will write it into the space again.
This way, another node which may know how to process it may take the task. % TODO discutir porque reasoning va a tener preferencia sobre actuación indirecta!


\begin{listing}
  \expandafter\def\csname PY@tok@err\endcsname{}
\begin{Verbatim}[commandchars=\\\{\},numbers=left,firstnumber=1,stepnumber=1]
\PY{k}{prefix }\PY{n+nv}{frap:}\PY{n+nn}{ \PYZlt{}http://purl.org/frap/\PYZgt{}}

\PY{k}{select }\PY{n+nv}{?pref }\PY{k}{where}\PY{p}{\PYZob{}}
\PY{n+nv}{	?pref}\PY{o}{ a }\PY{n+na}{frap:Preference }\PY{p}{.}
\PY{p}{\PYZcb{}}
\end{Verbatim}

  \caption{Subscription to any task written into the space.}
  \label{lst:generic_task_subscription}
\end{listing}


Demanding new data from the developer would impede the transparent reuse of the nodes from \implSpace{} and \implRest{}.
Therefore, the agent reuses all the information pieces that it needs:
\begin{itemize}
  \item It uses the \nodeConsSpace{}'s subscription to the task result as a goal for the reasoning.
	In our implementation, this correspondence needs a minimal mapping between N3QL \citeweb{n3ql2004} and SPARQL \citeweb{sparql2008}.
	The reason why we use both languages are the underlying frameworks: EYE \citeweb{euler} and RDFLib \citeweb{rdflib}.
	% TODO Justificar?
	
  \item The agent uses all the content written into the space as \emph{additional knowledge} for the reasoning process.
	This is feasible because the agent resides in the same machine as the space. % i.e. it's a local reading, does not demand any network usage
	Otherwise, acquiring this knowledge through the network would be too consuming both in bandwidth and in time.
\end{itemize}


% TODO poner aquí algunos indicadores de la implementación de escenarios?
%   e.g. cuando código extra ha hecho falta añadir en el tercero para que se hablen entre sí
\section{Discussion}
\label{sec:actuation_discussion}


In the previous section we presented how the space-based approach can interoperate with semantically described \ac{rest} services.
The space-mediated communication transparently invokes \ac{rest} services on behalf of the consumer which follows the \ac{ts} usage patterns.
This section scrutinizes the strengths and weaknesses of the presented design, in conjunction with some other design alternatives. % design, alternative, etc.


% TODO poner aquí algunos indicadores de la implementación de escenarios?
%   e.g. cuando código extra ha hecho falta añadir en el tercero para que se hablen entre sí


\subsection{Actuation Abstraction Level}

% explain why like this? why not directly sensors/light?
The most abstract way to act on the environment would be to directly actuate on a sensed value (e.g. stating ``set the \emph{room A} light-level to 19 luxes'').
This would require the coordination of all the actuators which directly or indirectly affect this value (e.g. all the lamps of a room).
Furthermore, even in this case, many other physical aspects would affect the value (e.g. the daylight at each time).
As a consequence, we opt for clearly distinguishing between actuators and sensors.
In our scenarios, the data measured by the sensors can only be indirectly affected by the actuators.


This indirect effects on the sensed values could be modelled for each actuator.
For example, to state that turning on a lamp increases the light measured by a sensor.
However, precisely anticipating the exact value of the new measure is difficult if not impossible due to the mentioned external physical conditions.
Considering this difficulty to predict their relationship and its tangential importance for the scenario, we simply ignore these relationships.


\subsection{Knowledge Model}

% Ontologies used
% explicar por qué hemos modelado así los escenarios?
The \emph{glue} between the two actuation worlds presented in Section~\ref{sec:actuation_interoperation} is the data they use. % o between both approaches, o yo qué sé
Therefore, it must be modelled according to the same ontologies.
This section provides the rationale behind the selection of the ontologies we used.
Note that this selection does not pretend to be best or unique alternative.

% por qué con medidas?
To represent the value of a lamp actuator, we use the SSN ontology \citeweb{ssn}.
The SSN ontology is intended to describe measures for sensors, so its use for actuators may sound contradictory.
However, for the best of our knowledge, there is no more widely-accepted ontology to specifically describe actuators.
Therefore, with this model we transmit things like ``the lamp actuator has a light value of \emph{N} luxes''.
Note the difference with an order like the following ``the light sensed by a sensor located in the lamp has a value of \emph{N} luxes''.


% por qué uso preferencias?
Finally, there is the other core ontology in our scenario implementations: the preferences ontology.
It is used to clearly distinguish between the value of an actuator and a consumer's intention or desire to change it.
A preference can turn into an actuator's new state, but it is not always necessarily true.
The actuator can discard a preference because of some policies, a malfunction or any other reason.

% por qué no se tienen en cuenta otras características como la localización o lo que sea => por simplificar
%   en el goal se podrían describir


\subsection{Obtaining Resource Descriptions}

The core of the proof-based actuation mechanisms are the descriptions.
They must be read from the actuation nodes prior to reasoning. % obvio
This action corresponds to the consumer in the first scenario and to the agent which resides in the \Space{} in the third scenario.

In both cases, they crawl over a given \ac{api} starting from a \ac{url}.
The discovery of the \ac{url} is out of the scope.
% Otra alternativa: An agent could discover the descriptions and simply write them in the coordination \Space{}.


Another alternative to discover this descriptions is to make them part of the \clues{} presented in Chapter~\ref{cha:searching}.
This option ensures that they will be available in any \consumer{}.
Besides, the static nature of these descriptions does not break the stability assumption of the \clues{}. % explicar más?


In any case, this chapter focuses on the interoperability problem, not on how to obtain the description.
Consequently, we opted for the most simple alternative: crawl the descriptions in an independent process. % simpler / more intuitive / easier to understand



\subsection{Obtaining Background Knowledge}

Besides resource descriptions, \citeauthor{verborgh_ijcs_2014}'s proof-based mechanism also requires an initial state and background knowledge as inputs (see Section~\ref{sec:restdesc}).
In the first scenario the consumer obtains this knowledge crawling over all the possible \acsp{api}.
In the third scenario we also add all the knowledge from the space.
This is feasible because it is located in the same machine as the agent which needs it, so does not demand any networking. % no es costoso obtenerla


However, as explained in Chapter~\ref{cha:searching}, the data of the providers in \ac{ubicomp} is too dynamic to simply crawl it from time to time.
% Reading all the space would be highly inefficient.
Crawling all the \acsp{api} each time a change needs to be done is also highly inefficient.
Hence, the proposed approach to obtain knowledge is a clear simplification.
This simplification is justified because this chapter is centred on the interoperability rather than in the efficient communication mechanisms.


\bigskip


A possible optimization would be to benefit from the search architecture presented in Chapter~\ref{cha:searching}.
To reduce readings on the space (i.e. network usage), we propose a procedure composed by two reasoning steps.
In the first one we only use local incomplete knowledge derived from the \clues{}.
Then, we read from the \Space{} just the knowledge needed to confirm the pre-proofs obtained in the first reasoning.
The second reasoning uses this knowledge to get real pre-proofs. % real o actual pre-proofs


% Y ahora explicado en mayor detalle:
Therefore, a node which wants to actuate over the space will need to obtain the \emph{clues} from the \ac{wp}.
%These clues, as explained before, tell what kind of content other nodes provide. % un poco redundante
Let us assume that these \emph{clues} are composed by the predicates used by the nodes which provide content. % es una de las alternativas que se planteaban
The existence of a predicate used in a premise does not necessarily imply that this rule can be used.
Nevertheless, its absence does imply that it will not be used (see Figure~\ref{fig:activation_rules}).
Therefore, we can create temporary \emph{activation rules} from \clues{} which activate those potential rules. % latter rules las que se han mencionado primero


\InsertFig{activation_rules}{fig:activation_rules}{
  Sample clues, two rules and the activation rule created from the first rule.
}{
  According to the clues shown, the second rule will never be invoked.
}{0.8}{}


An \emph{activation rule} for a rule R contains a \emph{true} in the premise.
The conclusion is made by R's premise substituting the variables with fictitious \acsp{uri} with a common prefix (see Figure~\ref{fig:activation_rules}).
These fictitious \acsp{uri} are used to distinguish when a triple should be replaced by actual knowledge from the space. % explicar mejor?



\subsection{Responsibility for Triggering the Proof-based Actuation Mechanism}
\label{sec:responsible_proof}

Section~\ref{sec:restdesc} described two coarse-grained steps:
\begin{itemize}
  \item Reasoning over the descriptions, background knowledge, an initial state and a goal state.
        The result of the reasoning process if a pre-proof, which can be seen as a tentative \emph{execution plan} to achieve the goal.
  \item Check the execution plan by following it.
\end{itemize}


% no obligamos a implementar una u otra, pero recomendamos esto: XXX
The third scenario presented in the previous section opts for triggering the reasoning process when an agent receives a notification.
Previously, it subscribes to any task written into the space.


The reasoning can be performed in any node apart from the one which holds the \Space{}:
%However, we do not adhere to any of them.
%We leave as a future work to implement and quantitatively compare them.

\begin{enumerate}
  \item Any \consumer{} interested on changing the environment could trigger the process.
	If these \consumers{} use the search mechanism presented in the Chapter~\ref{cha:searching}, they will have background knowledge about other nodes.
	This reduces the dependency on the node providing the \emph{coordination space}. % TODO buscar su nombre bueno
	However, it requires them to perform tasks such as reasoning and checking the pre-proofs.
	While the latter increases the network usage, the first increases the computation.
	As we already mentioned in previous chapters, these tasks severely affect to the energy consumption.
	Furthermore, some resource platforms will not be even able to reason.
	
  \item To mitigate that problem, we could delegate this task only on the nodes able to perform such tasks.
	In fact, these nodes could follow the \emph{replicated-worker pattern}.
	They could read from the space goals to trigger the process (i.e. \emph{reasoning tasks}).
	Apart from balancing the load between all the worker nodes, any node can stop being worker at any time by not taking more \emph{tasks} (e.g. if it has low energy).
	These nodes must be \consumers{} to use the \clues{} from the search mechanism as background knowledge.
\end{enumerate}


Although both alternatives avoid the dependency on the \Space{}, the space-based actuation mechanism intrinsically depends on the \Space{}.
Therefore, it makes sense that the unavailability of the space will cause the unavailability of actuating on the space.
On the contrary, it simplifies the consumers' responsibilities, which just need to worry about writing a task in the space.


% TODO DISCUTIR ESTO
However, some questions remain still unsolved: will both methods be triggered indistinctly or will the first prevail over the second?


\subsection{Interoperation Weakness}

The previous sections presented various alternative designs and their likely impacts on the actuation performance.
However, none of them addresses the interoperability flaws of the third scenario.


In this regard, the conversion of a consumer subscription into a goal is probably the most evident interoperability flaw of the third scenario.
Although in the scenario both aspects match, there is no guarantee that the consumer will use a subscription which matches with a goal.
For instance, she could use a more general subscription and then just treat the concrete tasks.
Even worse, there is no guarantee that the consumer would subscribe to any result.
Thus, the universality of the proposed alignment can be easily affected.


% qué pasa si el consumidor del escenario 2 no se suscribe a los resultados?
% qué pasa si se suscribe a un patrón más general?
A more universal approach would be to deduce the goal from the task. % universal, generalizable
For instance, from a task of \emph{regulate temperature to 6ºC} the \Space{} could deduce the goal state of \emph{temperature of 6ºC}.
In this case, the mapping should be either
\begin{enumerate*}[label=\itshape(\arabic*\upshape)]
  \item provided by the consumer or
  \item pre-set in the space.
\end{enumerate*}
The first choice demands to provide additional information to the \Space{}. % to the middleware
This is very inflexible and differs little from manually programming a gateway with each provider.
The second choice assumes a concrete ontology must used or extended by the user to represent tasks.
Therefore, it would limit the freedom of choosing any vocabulary to define a task.


Due to that reason and taking into account that this chapter simply wants to remark the potential interoperability of the presented approaches,
we opted for selecting the automatic translation from a subscription to a goal.



\subsection{Advanced Limitations}

The scenario presented is quite limited.
Consequently, the interoperation example shown requires further work to check its feasibility in more advanced scenarios.
We anticipate the following problems:

\begin{itemize}
  \item When there are two or more paths to a goal, how can we discern which one to follow?
	This problem is specific from the proof-based mechanism.
  \item How does the middleware deal with the coexistence of both mechanisms.
	When both methods can be applied, which one is triggered? Will one of them prevail over the second?
\end{itemize}
\section{Summary and Future Work}
\label{sec:actuation_summary}

This chapter presented two ways to actuate on the physical environment.
The first is the usual way to operate through spaces and provides a higher degree of decoupling.
However, it requires participants to use our middleware's primitives. % requiere la cooperación de los proveedores...
In other words, our middleware is not able to reuse third applications \ac{rest} services.


The second actuation mechanism directly consumes \ac{rest}ful \acs{http} \acsp{api}.
This mechanism relies in the semantic description of the services, additional knowledge and in a reasoning process. % additional knowledge: background + initial
With that information, it is able to generate executions plans towards a goal.
Following these plans implies different calls to the different services.


We implemented the same scenario using these two actuation mechanisms.
Besides, since interoperability is one of our middleware's guiding principles, we sketched how to reuse these \ac{rest}ful \acs{http} \acsp{api} in our \Space{} model in a third implementation.
This reuse avoids any alteration on the space-based consumer or the \ac{http} provider.
Instead, it improves the \Space{} implementation with an agent in charge of generating execution plans.
This agent reuses the information from the space-based actuation not to require any additional information to the developer.


This implementation alignment between our space-based actuation and the direct web \acsp{api} consumption one presents some limitations.
For some of these limitations, we discuss other design alternatives and compare them with the chosen one.
The rest of the limitations only appear in more complex scenarios.


For our future work, we will implement these complex scenarios where advanced conflicts between the \ac{rest} and space-based computing worlds can arise.
Besides, this dissertation does not answer how to reuse actuation mechanisms of the nodes using \ac{ts} patterns from \ac{rest}ful \acs{http} \acsp{api}.
Specifically, it would be interesting to experimentally test what would be necessary to seamlessly use our middleware's capacities from  third \ac{wot} solutions.
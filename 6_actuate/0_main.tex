% ----------------------------------------------------------------------

\begin{savequote}[50mm]
If you live among wolves you have to act like a wolf.
\qauthor{Nikita Khrushchev}
\end{savequote}


\chapter{Remotely Actuating}
\label{cha:actuate}


\newcommand{\restdesc}{\emph{RESTdesc}}



% the code below specifies where the figures are stored
\ifpdf
    \graphicspath{{\pathchapsix/figures/PNG/}{\pathchapsix/figures/PDF/}{\pathchapsix/figures/JPG/}{\pathchapsix/figures/}}
\else
    \graphicspath{{\pathchapsix/figures/EPS/}{\pathchapsix/figures/}}
\fi


%------------------------------------------------------------------------- 

In the previous chapter, we presented an energy aware technique to search on the completely distributed semantic space.
%This search mechanism promotes the end-to-end interaction between the objects.
However, as important as observing what happens in an \ac{ubicomp} environment, is to act over it.
Although this dissertation describes less thoroughly this problem, we considered interesting to discuss about it not to leave loose ends. % transmit a broader vision. % the big picture


Therefore, this chapter presents and compares two techniques to change the environment by \emph{writing} into the space. % no se si se entiende que la segunda también emula "escribir en el espacio"
The first one is based on \ac{ts} patterns and uses tasks and a subscription mechanism.
The second one starts with a goal and knowledge about the environment.
Then, it reasons to create a plan to fulfill the goal.
This plan comprises calls to \ac{rest} services.
%These services are semantically annotated and are read from the space.


% ----------------------------------------------------------------------



\section{Actuating through the Space}


% explicar brevemente patrones
% requirements: sistema de suscripciones (out of the scope)
% escenario de ejemplo en Ubicomp


In space-based computing, participants coordinate by reading and writing in a shared Space. % TODO usar define a nivel general para Space!
This encourages an uncoupled communication between applications using the same Space.

% IG: esto lo he homogeneizado porque tenia 3 estilos distintos :)
This section analyzes how to achieve this uncoupled communication following a top-down approach.
First, Section~\ref{sec:ts_patterns} briefly explains \acl{ts}'s most common application patterns. % according to the classification of \citet{freeman_javaspaces_1999}.
Second, Section~\ref{sec:envisaged_scenarios} uses some of these patterns to show how to build an \ac{ubicomp} scenario.
The scenario shows some requirements the middleware must fulfill.
Finally, Section~\ref{sec:notification} analyzes these requisites.



\subsection{\aclp{ts}' patterns}
\label{sec:ts_patterns}

According to \citet{freeman_javaspaces_1999}, there are four main application patterns which can be used with \ac{ts}: % IG: with o in?

\begin{description}
  \item[Replicated-worker pattern.] In this pattern, there is a master process and many worker processes able to compute the same task.
				    %% REDUCIENDO esto: %%
%				    The master takes a problem and divides it in tasks which are solved by any of the workers.
%				    More concretely the pattern is composed by the following steps:
% 				    \begin{enumerate}
% 				      \item The master divides a problem up into smaller tasks.
% 				      \item It writes them in the space.
% 				      \item Any of the many workers takes a task.
% 				      \item That worker computes the task.
% 				      \item It writes the result of that computation.
% 				      \item The master collects the results for the tasks it wrote.
% 				      \item Once the master has all the results, it combines them into a meaningful overall solution.
% 				    \end{enumerate}  
				    The master takes a problem, divides it into smaller tasks, and writes them tasks into the space.
				    Any available worker takes a task, processes it, and writes the result back into the space.
				    When all the workers have written their results, the master takes the results and combines them into a meaningful overall solution.
				    %Note that the workers accept new tasks whenever they are available and able to work.
				    %Therefore, while a worker is busy computing a big task, another one can solve many small tasks.
				    %In other words, this pattern naturally \emph{balances the load} on the space.
				    %Besides, it is \emph{scalable} since we can add more workers running in more machines without rewriting our code.
				    This pattern is scalable and naturally balances the load on the space.
  \item[Command pattern.] This pattern encapsulates the behavior into the tasks shared in the space.
			  Therefore, it requires (1) to share behavior through the space, and (2) any generic worker to be able to compute the behavior.
			  For instance, a graph in \ac{tsc} could include code in an interpreted programming language.
			  % nosotros pasamos de este jaleo
			  % hacer referencia a on-demand-code
			  %In that way, a generic worker application can compute whatever code other processes define and share through the space.
			  %This pattern is only possible in \aclp{ts} where the behavior can be described generically (e.g. in object-oriented \acp{ts}).
			  %Furthermore, the tasks to be performed over \ac{ubicomp} environments are not generically processable by any node.
			  %In contrast, each device is responsible of managing its own actuators on behalf of the rest.
			  %Therefore, this pattern will not be considered.
  \item[Marketplace pattern.] In this pattern, producers (or sellers) and consumers (or buyers) of resources interact to find the best deal. % TODO cita TripCom?
			  %It should be noted that the resources are products or services which can be bought and sold.
			  %Therefore, it is not applicable to the environments considered. % o use cases
			  %esto está relacionado con el paper aquel de TripCom y con el de Simon de hace un año en WoT
  \item[Specialist patterns.] In contrast to the replicated-worker pattern, in this pattern, each worker is specialized.
                              Therefore, each worker performs a particular task.
                              \citeauthor{freeman_javaspaces_1999} enumerate three subtypes:
			      \begin{description}
				  \item[Blackboard pattern.]
					It associates the concept of the space to a \emph{blackboard}.
					Following this analogy, the master is associated with a teacher, tasks with \emph{problems} and the specialized workers with \emph{students}.
					The blackboard pattern starts when the \emph{teacher} writes a \emph{problem} in the \emph{blackboard}.
					The \emph{students} observe the space and write their intention to contribute to solve the problem (i.e. \emph{raise their hands}).
					The \emph{teacher} selects to the expert which will make a modification.
					After the modification, the \emph{teacher} decides if it found the solution or another \emph{student} should contribute.
				  \item [Trellis pattern.]
					In this pattern, the master arranges the problem into low-level, mid-level, and high-level pieces.
					The workers of each level benefit from the refined data provided by the immediate level below.
				  \item [Collaborative patterns.]
					It encompasses all the patterns which allow nodes to collaborate to complete a greater task (i.e. by creating a workflow).
					%They are specific to the domain so they cannot be generalized, but it includes all the patterns where the nodes collaborate to complete a greater task
					%(e.g. by creating a workflow).
			      \end{description}
\end{description}


% Voy preparando el terreno para que se vea que los primeros son más de computación y el otro de negociación
To summarize, the \emph{replicated-worker pattern} is centered in optimizing the computation by parallelizing tasks.
The the \emph{command pattern} can be seen as an abstraction of the latter where the behavior is shipped in the tuples.
The \emph{marketplace pattern} allows negotiation of two entities through the space.
% Coño! pero si el especialista te permite colaborar...
Finally, the \emph{specialist pattern} allows nodes with distinct capacities to cooperate towards a common goal.


% Posible TODO Un esquema dividido en 4 en el que aparecen todos los patrones explicados.
% No es fácil mostrar el 2do y el 3ero.
% Una intentona de estos y otros patrones inventados en Lancaster:
%     https://docs.google.com/a/deusto.es/document/d/1QXGsQ4-wAByc_ByQ-Tts-TSphSQAzfO_c3mcgoZdloc/edit


% IG: ves... esta seccion parece q sale de la nada y creo que es el objetivo principal del capitulo, no? a ver si puedes orientar la intro del capitulo para mostrar esto mas... basicament que se vea que esto es de implementar movidas.
\subsection{Envisaged Scenarios} % IG: he aprendido palabro nuevo...
\label{sec:envisaged_scenarios}
% 1era iteración en reescritura, se puede mejorar mucho

This section devises two stereotypical scenarios for home automation.
Both scenarios emphasize how devices can coordinate in a decoupled mode using the patterns seen in the previous section. % más de uno?
%The scenarios are formed by mobile phones, and embedded platforms with sensors and/or actuators.
Specifically, we believe that the \emph{specialist patterns} fit the best to the needs of \ac{ubicomp} environments:
\begin{itemize}
  \item The devices in \ac{ubicomp} often serve to very specific needs. % IG: comentario general de esto de las comas, que me las pones todo al reves... el sujeto y el predicado sin coma entre medio (a noser que tenga de esas cosas descriptivas, en plan "jose, que es negro, se ha quemado"). Y la coma se pone cuando es algo complementario del tipo de: "en este capitulo, ..."
        For example, let us imagine a mobile phone showing a message or an embedded device turning on the lights of a room. % IG: lo de mobile es el estandard de la tesis? lo de cell no, no?
        %Although there may exist some redundancy in the tasks which different nodes can be accomplish, the task are not nodes are not as interchangeable as in... son tareas simples, no necesitan optimizar ni lanzar muchas
  \item These tasks are usually lightweight. % TODO primera vez que se habla de "task", comentar que para TripCom eran servicios web y que crearon un sistema (WXSW?)
        They do not require large computation resources, but to achieve a concrete and simple goal. % IG: comor? frase rara
\end{itemize}
Therefore, \ac{ubicomp} generally faces a problem of collaboration between nodes with different capacities.
In this problem, computation or negotiation aspects are secondary. % se entenderá a qué me refiero con computation??


%In this pattern, a master writes a task into the space and waits for its result, which is performed by some of the workers specialized in this particular task (e.g. show a message or regulate the temperature).

% IG: hala! y esto de sopeton... lo que decia, preparame el terreno un poco
\subsubsection{Scenario 1: Temperature Regulation}

The first scenario presents a room populated with several kind of sensors such as Oracle's SunSPOTs \citeweb{sunspot},
% TODO TODO TODO referenciar a la parte donde se hayan referenciado bien los dispositivos! (e.g. see ~\ref{environment})
% Para el resto proveer al menos una URL!
Digi's XBee sensors with an IP gateway,
the sensors on a KNX domotic bus \citeweb{knx}, and a fan connected to a FoxG20 \citeweb{foxg20} embedded platform as an actuator (see Figure \ref{fig:devices_scenario}).
In addition, an Android application \citeweb{android} semantically stores the temperature preferences of the user.


\InsertFig{devices_scenario}{fig:devices_scenario}{
  The devices used in the devised scenarios.
}{
}{0.7}{}


An independent node (i.e. the master node) continuously reads from the space % using \emph{read primitive}
(1) the room's temperature, and
(2) the user's desired temperature.
Note that the master node does not care about who exactly provides the temperature information.
It just takes the first available graph from the space.
When the second one is below the first one, it generates a ``\emph{decrease temperature during a certain period}'' task which can be consumed by different independent worker nodes.
In this case, the FoxG20 periodically checks for orders it can fulfill extracting them from the space. % consumes them with a \textit{take primitive}.
% zasca, acabamos de poner en bandeja que nos rechine lo de comprobar periodicamente



\subsubsection{Scenario 2: Sedentary Lifestyle Checker}

The second scenario presents an application which helps the user to avoid a sedentary lifestyle by giving him different warnings.
We consider the recommendation that states that an adult should walk at least 10.000 steps in an ordinary day \citep{tudor2002taking}.
Taking into account the expected steps which should have been completed at each moment of the day, the application generates different priority level messages.

There are several nodes involved in this task.
Each node runs on a device. % which belongs to a user.
First, an Android phone periodically updates the number of steps covered by a user that day \citeweb{pedometer}.
%Besides, it writes her profile, more important to this problem: her age. % simplificado un poquillo
Second, there is an undefined number of devices which know how to warn the user about her unhealthy behavior.
Third, there is a node in charge of generating the warnings.


\medskip


The application follows the \emph{blackboard pattern}.
The node which generates the warnings is the \emph{teacher}.
The \emph{problem} is the warning for the user.
The \emph{students} are the devices which can warn the user. % IG: no es nada intuitivo. Leyendo despues lo entiendo, pero... es raro


When the \emph{teacher} writes a warning for a user, the devices able to warn the user write their characteristics into the space. % IG: alguna cosa mejor que characteristics?
In this example, these characteristics are the device ownership and the intrusiveness of the warning method.
The \emph{teacher} reads these characteristics, chooses the most appropriate \emph{student} to solve the \emph{problem} and updates the \emph{problem} to indicate the selection.
%If no device writes their characteristics, the \emph{teacher} can wait longer.
Finally, the selected device takes any warning for it and warns the user. % no tiene respuesta
If the chosen device does not take messages, the \emph{teacher} can update the \emph{problem} with a new selection.


With this application, if the priority level is low, the user receives the warning in a less intrusive way.
For example, room's light brightness can be increased for low priority notifications, a Chumby \citeweb{chumby} can show an icon for normal priority ones and the user's mobile phone show directly the message when it is a high priority warning. % IG: yo anyadiria una tablita para los dos scenarios mostrando los actuadores y los niveles. Aqui esto me salta un poco de repente.
The remarkable characteristic of this mechanism is that the devices present in the environment at each moment can vary and the application will still fulfill its goal.




\subsection{Notification mechanism}
\label{sec:notification}


In the scenarios described in the previous section, some writings in the space trigger other node's action.
For example, a device must show a message whenever a new warning is written in the space. % e.g. a task or a result
To be aware of these writings, the node can either poll the space or rely on a notification mechanism. % async or blocking
Obviously, the later encourages a more efficient use of the network. % is more optimal and more scalable approach.
% IG: encourage no creo que sea la palabra...

As a consequence, a notification mechanism is highly advisable to fulfill \ac{ts}'s patterns in a distributed environment.
Although the implementation of this notification mechanism is out of the scope of this thesis,
% Hablar de requisitos, describir nuevas primitivas de suscripciones o alternativas de implementación
it should comply with the following aspects:
\begin{itemize}
  %  Debe permitir polling (sync), por ello proponemos nuevas primitivas: read_async y take_async.
  \item Do not substitute the polling mechanism. % porque todavía puede ser interesante en otros casos
        The middleware must provide additional \emph{read} and \emph{take} primitives.
  % Se evalúan sobre el espacio, no sobre el outer-space. 
  \item Since the notification mechanism intends to allow coordination patterns,
        it must consider the knowledge from the coordination space. % TODO comprobar que lo llamé así en el anterior capítulo
        In other words, knowledge from the \emph{outer-space} will not trigger notifications. % TODO usar constante para esto
  % Evaluarse cada X, no siempre que se escriba algo o se vuelve loco.
  \item The evaluation of the subscriptions must not interfere with the writing process (e.g. introducing a delay).
        Therefore, it must run asynchronously.
  % Debe ser sencillo de implementar por cualquier nodo. Para ello proponemos un mínimo contrato: callback URL.
  \item Ease its adoption by any type of computing platform.
	This can be achieved by reducing the requirements on the \emph{clients}.
        For instance, a callback \ac{url} passed during the subscription can represent a minimal contract between the client and the server. % no viceversa
  % Deben caducar para no mantener suscripciones de dispositivos que ya no existen.
  \item Provide additional subscription removal mechanisms.
	\ac{ubicomp} scenarios are composed by unreliable devices which may frequently join and leave the space.
	In this situation, the correct use of unsubscription primitives cannot be guaranteed.
	This may worsen the performance of the system with useless subscriptions from absent devices.
	Therefore, the device managing the subscriptions should adopt more proactive mechanisms.
	For example, it may let the subscriptions expire after a lifetime or remove them when it discovers the unavailability of a callback \ac{url}. % algo así como garbage collection
	% A) The subscriptions must expire.
        % The expiration will allow to delete subscriptions from absent devices. % no longer present devices
        % This also implies that clients are responsible for periodically updating their subscriptions.
        % B) Additional mechanism
        % For instance, a \emph{garbage collection} agent can check the availability of callback \acp{url}.
        % For unavailable callback \acp{url}, the subscriptions can be removed.
\end{itemize}


The main drawback of any subscription mechanism is that it breaks the \ac{rests} property of the \ac{rest} style.
This implies that network performance will improve at the cost of scalability, simplicity and reliability. % TODO poner un see Section~\ref{X} ???


% hasta AQUI!
% Therefore, although it is not the main focus of this thesis, in this section we present a subscription mechanism which can be used on top of the \ac{tsc} middleware proposed.
% The requirements that this mechanism needs to fulfill are the following ones:
% \begin{itemize}
%   \item It should be independent of \ac{tsc}'s writing and reading primitives.
% 	This requirement ensures that frequent writings do not lead to excessive processing in resource constrained nodes.
% 	The main drawback of this requirement is that a node does not now when relevant data is written in the space \emph{per se}.
% 	A developer needs to advertise when it writes relevant data.
%   \item Any node running our solution must be able to implement this mechanism.
% 	In other words, no new dependencies should be added.
%   \item The new primitives should use elements the current developer is used to (i.e. templates or RDF triples).
% \end{itemize}
% 
% explicar cómo funciona
% https://otsopack.readthedocs.org/en/latest/subscriptions.html
% \subsubsection{Subscription primitives}
% 
% The subscription primitives the developer should use are the following ones:
% \begin{itemize}
%   \item \emph{Subscribe}. The node subscribes to the given template returning an URI which identifies the subscription.
%     \begin{minted}{java}
% URI subscribe(space_uri, template, listener)
%     \end{minted}
%   \item \emph{Unsubscribe}. Unsubscribes to a subscription given its subscription URI.
%     \begin{minted}{java}
% void unsubscribe(space_uri, String subscriptionURI)
%     \end{minted}
%   \item \emph{Notify}.
%     \begin{minted}{java}
% void notify(space_uri, template)
%     \end{minted}
% \end{itemize}
% 
% 
% \subsubsection{Deployment}
% 
% The nodes responsible of handling subscriptions and notifications are called \emph{bulletin boards}.
% Other nodes belonging to the same space, discover them using a out of scope method and publish their subscriptions and notifications using their HTTP API.
% 
% Each \emph{bulletin board}:
% \begin{itemize}
%   \item Belongs to a space.
%   \item Exposes a subscription API.
%   \item Shares its subscriptions with other bulletin boards which belong to the same space.
%   \item Propagates the notifications to the relevant nodes using the \emph{callback url} provided by them.
% \end{itemize}
% 
% 
% The Figure~\ref{} describes a simple subscription use case.
% \begin{enumerate}
%   \item N1 subscribes to BB1 with a template t1.
%   \item BB1 propagates the subscription provided by N1 to BB2 and BB3
%   \item N3 notifies to BB3 about t2.
%   \item Since t1 matches t2, BB3 tries to notify to N1 using the callback URI provided during the subscription process.
%   \item Unfortunately, the BB3 cannot notify to N1 due to unexpected network problems.
%   \item BB3 propagates the notification of t2 to BB2.
%   \item BB2 reaches N1, so it notifies it about t2 using the callback URI.
% \end{enumerate}
% 
% TODO poner la foto esquemática de https://otsopack.readthedocs.org/en/latest/subscriptions.html

\section{Direct Consumption of \acs{rest} services}

% por qué? motivación: permitir a otras app integrarse en nuestro espacio
\ac{rest}-like services constitute an emerging mechanism to actuate in the physical environment. % REST-like, no compliant with Fielding
Therefore, their seamless integration would open the door to reuse the capacities of many existing devices. % interoperate
However, this requires applications to directly use the actuator's node \ac{http} \ac{api}.


In contrast, the actuation mechanism presented in the previous section uses the \Space{} (i.e. it is indirect).
The mechanism implies that the actuator must be aware of the content of the space.
For instance, a heater must check the space to find if a new desired temperature was written.
% lo podrá reusar a través de un intermediario
% interop era una de las cosas que queriamos cuidar
% y qué pasa con las soluciones REST que quieren usar lo nuestro?


In this section we propose to close the gap between both approaches.
To do that, we propose to transparently invoke \ac{rest}-like services on behalf of the developer.
From her perspective, she will still use the space-based patterns presented in the previous section.
However, the middleware will be enriched with the actuation capacities of third \ac{http} \acp{api}.


To define that integration, we first explain the mechanism used to describe and select the appropriate \ac{rest} services (Section~\ref{sec:background_restdesc}).
Then we sketch this integration by answering two key elements:
\begin{itemize}
  \item How do the inputs for this mechanism relate to the ones used in the \ac{ts} patterns? (Section~\ref{sec:inputs_proof})
  \item Which node is responsible for triggering this mechanism? (Section~\ref{sec:responsible_proof})
\end{itemize}



\subsection{Background}
\label{sec:background_restdesc}

In \ac{wot} physical changes in the environment are performed by manipulating \ac{http} resources.
According to the \ac{rest} principles, a client should navigate through these resources with no prior knowledge of the \ac{api}.
% Copiar esta explicación mejor de algún lado:
The client should 
\begin{enumerate*}[label=\itshape(\arabic*\upshape)]
  \item interpret the representations provided by the server and then
  \item choose the appropriate state transition from the hypertext according to its intention. % y su conocimiento básico del protocolo: CRUD
\end{enumerate*}


% es HATEOAS?
As explained in Section~\ref{sec:network_properties}, semantic representations do not include a native way to express the hypertext. % TODO realmente se explica?
To solve this, three solutions can be adopted:
% Unos proponen extender con ontologías
\begin{enumerate}
  \item To use an ontology to represent the hypertext \citep{kjernsmo_necessity_2012},
  \item To embed the hypertext independently to the representations on the \ac{http} headers \citep{mark_web_2010}, and
  \item To provide a description of the resources using the \ac{http} OPTIONS verb \citep{verborgh_functional_2012,verborgh_ijcs_2014}.
\end{enumerate}

The latter two enable to discover resources and state transitions without adding metadata to the representations.
This allows not only to describe semantic representations, but any type of formats.

\citeauthor{mark_web_2010}'s \citep{mark_web_2010} approach is extended by \citet{erik_profile_2013} to define how to embed additional semantics to process a resource representation.
These additional semantics are called profiles and are identified by an \acs{uri}.


\bigskip


\citet{verborgh_ijcs_2014} present a more expressive solution which goes beyond simply describing the resource's type.
It also allows to semantically describe the knowledge needed to use a concrete \acs{http} verb on a resource, and the content this call returns. % o precondition
The materialization of this proposal is called \restdesc{} \citep{verborgh_functional_2012}.


% TODO mirar si tiene cabida la mención de otros enfoques para describir recursos que no son muy RESTful
%Different ways exist to describe \ac{rest} services. % mencionar WADL, etc.?
% citar a donde se hable de \restdesc{} y así ya se empieza a explicar la solución de forma discreta.
\restdesc{} describes \acs{http} methods using rules expressed in the \ac{n3}\footnote{\url{http://www.w3.org/TeamSubmission/n3/}} language.
% he evitado explicar que las reglas tienen premisa y conclusión, porque me parece demasiado obvio
A rule's \emph{premise} expresses the requirements to invoke a \ac{rest} service.
A rule's \emph{conclusion} expresses both the \ac{rest} call that needs to be made and the description of that invocation result.


\citet{verborgh_ijcs_2014} propose a proof-based composition mechanism for Web \acp{api} using \restdesc{}.
This mechanism uses as inputs:
\begin{enumerate*}[label=\itshape(\arabic*\upshape)]
  \item an initial state,
  \item a goal state,
  \item Web \ac{api} descriptions using \restdesc{}, and
  \item optional background knowledge.
\end{enumerate*}
Each of these inputs are semantically expressed and therefore, they can be processed by standard \ac{n3} reasoners.
The reasoners generate proofs about how to achieve the goal starting from the initial state using the rest of the inputs.
These proofs can be seen as steps that need to be made to reach a desired state.


Additionally, \citet{verborgh_ijcs_2014} distinguish between pre-proof and post-proof.
The first, are those which assume that the execution of all \acs{api} calls will behave as expected.
The latter, can be seen as a \emph{revision} of the pre-proof.
It executes the Web \acs{api} of the pre-proofs and uses actual execution results to generate a new proof.



\subsection{Inputs for the Proof-based Actuation Mechanism}
\label{sec:inputs_proof}

The reasoning process presented by \citet{verborgh_ijcs_2014} uses as inputs: % presented by en vez de posesivo, porque "X et al.'s" queda como el culo
an initial state,
a goal state,
Web \ac{api} descriptions using \restdesc{}, and
optional background knowledge.
% inputs: clues
% goal: task


\begin{description}
  \item[The descriptions] must be read from the actuation nodes prior to reasoning. % obvio
        They can be read with a \emph{ad hoc} discovery mechanisms.
        However, in order to simplify the design of the middleware, it is sensible to use one of the already existing mechanisms:
        
        \begin{itemize}
	  \item An agent could discover the descriptions and simply write them in the coordination \Space{}. % y ahora habrá que definir que es un agente?
		Then, any node willing to reason will read them from the \Space{}.
	  \item The descriptions could be part of the \clues{} presented in Chapter~\ref{cha:searching}.
		This option ensures that they will be available in any \consumer{}.
		Besides, the static nature of these descriptions does not break the stability assumption of the \clues{}. % explicar más?
        \end{itemize}

  % TODO TODO TODO especificar que este mecanismo adicional de usar clues es optional y no el usado en nuestro escenario simple
  \item[The initial state and background knowledge] can be acquired both from the coordination space and the outer space.
	However, reading all the space would be highly inefficient.
	To reduce readings on the space (i.e. network usage), we propose a procedure composed by two reasoning steps.
	In the first one we only use local incomplete knowledge derived from the \clues{}.
	Then, we read from the \Space{} just the knowledge needed to confirm the pre-proofs obtained in the first reasoning.
	The second reasoning uses this knowledge to get real pre-proofs. % real o actual pre-proofs
	
	% Y ahora explicado en mayor detalle:
	Therefore, a node which wants to actuate over the space will need to obtain the \emph{clues} from the \ac{wp}.
	%These clues, as explained before, tell what kind of content other nodes provide. % un poco redundante
	Let us assume that these \emph{clues} are composed by the predicates used by the nodes which provide content. % es una de las alternativas que se planteaban
	The existence of a predicate used in a premise does not necessarily imply that this rule can be used.
	Nevertheless, its absence does imply that it will not be used (see Figure~\ref{fig:activation_rules}).
	Therefore, we can create temporary \emph{activation rules} from \clues{} which activate those potential rules. % latter rules las que se han mencionado primero

	\InsertFig{activation_rules}{fig:activation_rules}{
	  Sample clues, two rules and the activation rule created from the first rule.
	}{
	  According to the clues shown, the second rule will never be invoked.
	}{0.8}{}

	An \emph{activation rule} for a rule R contains a \emph{true} in the premise.
	The conclusion is made by R's premise substituting the variables with fictitious \acsp{uri} with a common prefix (see Figure~\ref{fig:activation_rules}).
	These fictitious \acsp{uri} are used to distinguish when a triple should be replaced by actual knowledge from the space. % explicar mejor?


  \item[The goal state] should be created from the task to reuse the primitives and the data from the \ac{ts}-based actuation mechanism.
	For instance, from a task of \emph{regulate temperature to 6ºC} we can deduce the goal state of \emph{temperature of 6ºC}.
	This translation is out of the scope of the dissertation.
\end{description}


\subsection{Responsibility for Triggering the Proof-based Actuation Mechanism}
\label{sec:responsible_proof}

Section~\ref{sec:background_restdesc} described two coarse-grained steps:
\begin{itemize}
  \item Reasoning over the descriptions, background knowledge, an initial state and a goal state.
        The result of the reasoning process if a pre-proof, which can be seen as a tentative \emph{execution plan} to achieve the goal.
  \item Check the execution plan by following it.
\end{itemize}


This mechanism can be performed by any node.
% no obligamos a implementar una u otra, pero recomendamos esto: XXX
In this thesis, we briefly analyze three different alternatives and their trade offs.
However, we do not adhere to any of them.
We leave as a future work to implement and quantitatively compare them.

\begin{enumerate}
  \item An agent which resides in the same machine as the \emph{coordination space} can trigger the process. % TODO buscar su nombre bueno
	Doing so, it can locally consume the knowledge available in the \emph{coordination space}.
  \item A more flexible approach would consist of letting any \consumer{} interested on changing the environment to trigger it.
	If these \consumers{} use the search mechanism presented in the Chapter~\ref{cha:searching}, they will have background knowledge about other nodes.
	This reduces the dependency on the node providing the \emph{coordination space}. % TODO buscar su nombre bueno
	However, it requires them to perform tasks such as reasoning and checking the pre-proofs.
	While the latter increases the network usage, the first increases the computation.
	As we already mentioned in previous chapters, these tasks severely affect to the energy consumption.
	Furthermore, some resource platforms will not be even able to reason.
  \item To mitigate that problem, we could delegate this task only on the nodes able to perform such tasks.
	In fact, these nodes could follow the \emph{replicated-worker pattern}.
	They could read from the space goals to trigger the process (i.e. \emph{reasoning tasks}).
	Apart from balancing the load between all the worker nodes, any node can stop being worker at any time by not taking more \emph{tasks} (e.g. if it has low energy).
	These nodes must be \consumers{} to use the \clues{} from the search mechanism as background knowledge.
\end{enumerate}


% Poner ejemplo adaptado del paper de WoT2013?


%Proceso de suggest:

%\begin{itemize}
% \item obtener todo el conocimiento necesario para el proceso?
% \item razonamiento sobre conocimiento
% \item parsear el resultado para invocar servicios HTTP
% \item monitorizar que se ha complido el cambio?
%\end{itemize}
\section{Interoperation Example Between both Approaches}

This section aims to experimentally proof the validity of the assumptions made in the allignment of the proof-based actuation mechanism with space-based computing one.
To this end, we have designed a simple scenario to act on the light of a lamp.
Then, we have implemented the nodes which participate in this scenario in different manners.


On the one hand, the lamp node implements two different mechanisms to change this light's value:
\begin{enumerate}[label=\itshape(\alph*\upshape)]
  \item It provides a \ac{rest} \ac{api} and describe it using \restdesc{}. % REST API seguro 100%? por si acaso decir HTTP API?
        When a client invokes changes the value of the resource which describes the light's value, the node will physically change it.
  \item It is aware of the tasks written into the space and change the light's value accordingly. % esto es, se suscribe a ellas
\end{enumerate}


On the other hand, we have implemented the node which wants to change the light's value in the following ways:
\begin{enumerate}[resume,label=\itshape(\alph*\upshape)]
  \item It reasons over the knowledge and descriptions to get a plan to fulfill a goal.
	With this plan, it invokes the needed \ac{rest} services.
  \item It writes a task into the space describing its desire to change the light's value.
\end{enumerate}


Using these nodes, we implemented three scenarios.
In the first one is fully based on the proof-based mechanism and is composed by a \emph{(a)} provider and a \emph{(c)} consumer.
The second scenario presents two nodes using the space-based computing actuation patterns and therefore it is composed by a \emph{(b)} provider and a \emph{(d)} consumer.
The third scenario presents a mixed scenario where we have an \emph{(a)} provider and a \emph{(d)} consumer.
Sections \ref{sec:actuation_scn1}, \ref{sec:actuation_scn2} and \ref{sec:actuation_scn3} describe these scenarios respectively.
% TODO The implementation is publicly available in URL.


\subsection{Scenario 1}
\label{sec:actuation_scn1}

% poner un diagrama que presente el escenario
% dar detalles de cómo se ha implementado
%   Añadir descripciones y demás o ponerlas como anexo?
%   Pasos a seguir por parte del consumidor


\subsection{Scenario 2}
\label{sec:actuation_scn2}

% poner un diagrama que presente el escenario
% dar detalles de cómo se ha implementado
%   poner tarea de ejemplo


\subsection{Scenario 3}
\label{sec:actuation_scn3}

% poner un diagrama que presente el escenario
% dar detalles de cómo se ha implementado
%    comentar que asunciones de las del anterior capítulo se han tomado
%    por simplificar: agente que reside en el coordination space
% Pasos a seguir por el agente


% TODO añadir apartado de discussion?
% luego posiblemente se podrían medir algunos indicadores de los mismos
%   e.g. cuando código extra ha hecho falta añadir en el tercero para que se hablen entre sí

\section{Comparison}

The first actuation procedures require a subscription mechanism and provide space and time autonomy.
However, it does that by creating a dependency on the \Space{}.
Two nodes will not be able to communicate with each other without the \Space{}.


The second approach offers this independence of the \Space{}: any node can directly invoke a service to act over the environment.
Remarkably, it allows reusing third \ac{wot} applications' actuation capacities. % apps, APIs o services? % no digo integration/interop porque es sólo one-way
To do that, it requires a rule-based reasoning engine. % reasoning engine ya queda bien?
The unavailability of these engines in many computing platforms may impede its massive adoption. % como apunte de su practicalidad



% Tabla: XXX
% autonomía: poner o simplemente indirect vs direct?
% requirements in consumers: estar pendiente del espacio vs razonar
% requirements in providers: estar pendiente del espacio vs (describir semánticamente servicio => equivalente a tasks)
\begin{table}[htbp]
  \caption{Characteristics of the discussed actuation mechanisms.}
  \begin{center}
    \footnotesize
    \begin{tabular}{llll} 
      \hline
      \multirow{2}{*}{Actuation} &
      Communication & % o poner autonomias y demás para enfatizar?
      \multirow{2}{*}{Benefits} &
      Required \\
      &
      style &
      ~ &
      features \\
      \hline
      Space-based & Indirect & Decoupled communication & Subscriptions \\[0.2cm]
      REST-based & Direct & Reuse of third \ac{wot} & Rule-based \\ % mencionar interop? % TODO ampliar la noción de WoT a REST-like?
      & & applications & reasoning \\ % no digo "wot actuation capacities porque ya está verbalizado en la explicación"
      \hline
    \end{tabular}
  \end{center}
  \label{tab:actuation_mechanisms}
\end{table}


\bigskip

The impact of both mechanisms in resource constrained devices will be affected by computing and networking activities.
\ac{ts}-based patterns require all the participants to be aware of what is written in the space to react (i.e. be proactive).
Both consumers and providers read and write from the space, subscribe to specific changes and receive notifications.
Thanks to this specificity, they are only affected by the contents they are interested in. % reduciendo la actividad
Finally, they only perform trivial computing task: interpreting results and pattern matching. % pattern matching sólo en el caso del coordination space!


The proof-based actuation mechanism requires the consumer to have prior knowledge about the environment to reason over it. % frase susceptible de simplificar
Since this knowledge must be acquired from remote nodes somehow, this approach might demand more network usage than the first one.
% de pre-proof a post-proof se generará más networking activity seguro? por qué? por invocar servicios REST? Es equivalente a andar leyendo tareas y escribiendo resultados...
Besides, the proof-based actuation itself also generates more computation activity on the node responsible of reasoning.
Both aspects have a negative impact in the energy consumption.
Consequently, from its consumer perspective, it will most likely not be an adequate mechanism for resource constrained devices.
As a contrast, this mechanism requires few things from the provider perspective: to serve \ac{rest}-like services and provide their descriptions. % aclarar quien es el provider/actuator?




% Tabla: XXX
% autonomía: poner o simplemente indirect vs direct?
% requirements in consumers: estar pendiente del espacio vs razonar
% requirements in providers: estar pendiente del espacio vs (describir semánticamente servicio => equivalente a tasks)
\begin{table}[htbp]
  \caption{Foreseeable networking and computing impact on the nodes involved in the actuation mechanism.}
  \begin{center}
    \footnotesize
    \begin{tabular}{llp{4cm}p{4.4cm}}
      \hline
      \multirow{2}{*}{Actuation} &
      \multirow{2}{*}{Perspective} &
      \multicolumn{2}{c}{Activity} \\
      ~ &
      ~ &
      \multicolumn{1}{c}{Networking} &
      \multicolumn{1}{c}{Computation} \\
      \hline
      \multirow{3}{*}{\ac{ts} patterns} & Provider & Proactive, limited activity & Limited: Results Interpretation \\
				    ~ & Consumer & Proactive, limited activity & Limited: Results Interpretation \\
				    ~ & Space & Reactive, high activity & Varies with the implementation \\[0.2cm]
      \ac{rest} \acp{api} & Provider & Reactive, limited activity & Limited: Handling requests \\
        consumption       & Consumer & Proactive, high activity &  Demanding: Reasoning \\ % Y comprobar planes...
      \hline
    \end{tabular}
  \end{center}
  \label{tab:actuation_networking_computing}
\end{table} % o si queda muy corto, incluirlo directamente en conclusions
\section{Conclusion}

In this chapter we presented two ways to actuate on the physical environment.
The first is the usual way to operate through spaces and provides a higher degree of decoupling.
However, it requires participants to use our middleware's primitives. % requiere la cooperación de los proveedores...
In other words, our middleware is not able to reuse third applications \ac{rest} services.


Since the interoperability is one of the guiding principles of our middleware, we presented an additional actuation mechanism.
This mechanism relies in the semantic description of the services, additional knowledge and in a reasoning process. % additional knowledge: background + initial
With that information, it is able to generate executions plans towards a goal.
Following these plans implies different calls to the different services.


We sketched an alignment of our middleware with this actuation mechanism.
The tasks written in the space can be translated into goals.
The knowledge from the space can be used as an input for the reasoning process.
Different nodes are able to trigger the reasoning and check the plan.
Remarkably, we suggest the use of \ac{ts} patterns to promote load balancing between the devices able to perform that process.


% Future work o conclusion?
The second actuation alternative requires further work to check some of the assumptions made.
For example, will always be possible to translate a task into a goal?
Additionally, it opens the door to solve other interesting questions: when there are two or more paths to a goal, how can we discern which one to follow?


Furthermore, the middleware must deal with the coexistence of both mechanisms.
Will both methods be triggered indistinctly or will the first prevail over the second?
% uso de nuestra actuación por parte de apps WoT
Finally, the question of how to reuse actuation mechanisms of nodes using \ac{ts} patterns from \ac{wot} has not been addressed.
However, this chapter has exposed, described and compared the key points towards the actuation through a \ac{tsc} middleware for \ac{ubicomp}.
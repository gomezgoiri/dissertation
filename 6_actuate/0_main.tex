% ----------------------------------------------------------------------

\begin{savequote}[50mm]
If you live among wolves you have to act like a wolf.
\qauthor{Nikita Khrushchev}
\end{savequote}


\chapter{Remote Actuation}
\label{cha:actuate}


\newcommand{\restdesc}{\emph{RESTdesc}}



% the code below specifies where the figures are stored
\ifpdf
    \graphicspath{{\pathchapsix/figures/PNG/}{\pathchapsix/figures/PDF/}{\pathchapsix/figures/JPG/}{\pathchapsix/figures/}}
\else
    \graphicspath{{\pathchapsix/figures/EPS/}{\pathchapsix/figures/}}
\fi


%------------------------------------------------------------------------- 

In the previous chapter, we presented an energy-aware technique to search on the completely distributed semantic space.
%This search mechanism promotes the end-to-end interaction between the objects.
% However, as important as observing what happens in an \ac{ubicomp} environment, is to act over it.
However, acting over an \ac{ubicomp} environment is as important as observing what happens on it. % IG: cambia un poco la semantica, pero lo anterior me sonaba my raro
Although this dissertation describes this problem less thoroughly, we consider interesting to discuss it to provide the complete story. %not leave loose ends. % transmit a broader vision. % the big picture

%Therefore,
This chapter presents and compares two techniques to change the environment by \emph{writing} into the space. % no se si se entiende que la segunda también emula "escribir en el espacio"
The first one is based on \ac{ts} patterns and uses tasks and a subscription mechanism.
The second one has a goal and knowledge about the environment.
% IG: el nivel de detalle de la segunda es un poco alto para una introduccion y ademas desbalanceado con la primera.
Then, it reasons to create a plan to fulfil the goal.
This plan comprises calls to \ac{rest} services.
%These services are semantically annotated and are read from the space.

% IG: no entiendo mucho el giro entre la intro y lo siguiente

% ----------------------------------------------------------------------



\section{Coordination through the Space}



% explicar brevemente patrones
% requirements: sistema de suscripciones (out of the scope)
% escenario de ejemplo en Ubicomp


In space-based computing participants coordinate by reading and writing in a shared Space. % TODO usar define a nivel general para Space!
This encourages an uncoupled communication between applications using the same Space.

In this section we analyze how to achieve this uncoupled communication following a top-down approach.
First, we briefly explain \acl{ts}'s most common application patterns. % according to the classification of \citet{freeman_javaspaces_1999}.
Using some of these patterns, Section~\ref{sec:envisaged_scenarios} shows how to build an \ac{ubicomp} scenario.
The scenario evidences some requirements, analyzed in Section~\ref{sec:notification}, the middleware must comply with.



\subsection{\acl{ts}'s patterns}
\label{sec:ts_patterns}

According to \citet{freeman_javaspaces_1999} there are four main application patterns which can be used with \ac{ts}:

\begin{description}
  \item[Replicated-worker pattern.] In this pattern there is a master process and many worker processes which are able to compute the same task.
				    %% REDUCIENDO esto: %%
%				    The master takes a problem and divides it in tasks which are solved by any of the workers.
%				    More concretely the pattern is composed by the following steps:
% 				    \begin{enumerate}
% 				      \item The master divides a problem up into smaller tasks.
% 				      \item It writes them in the space.
% 				      \item Any of the many workers takes a task.
% 				      \item That worker computes the task.
% 				      \item It writes the result of that computation.
% 				      \item The master collects the results for the tasks it wrote.
% 				      \item Once the master has all the results, it combines them into a meaningful overall solution.
% 				    \end{enumerate}  
				    The master takes a problem, divides it in smaller tasks and writes these tasks in the space.
				    Any available worker takes a task, processes it and writes the result back on the space.
				    When all the workers have written their results, the master takes the results and combines them into a meaningful overall solution.
				    %Note that the workers accept new tasks whenever they are available and able to work.
				    %Therefore, while a worker is busy computing a big task, another one can solve many small tasks.
				    %In other words, this pattern naturally \emph{balances the load} on the space.
				    %Besides, it is \emph{scalable} since we can add more workers running in more machines without rewriting our code.
				    This pattern is scalable and naturally balances the load on the space.
  \item[Command pattern.] It encapsulates behavior in the tasks shared in the space.
			  Therefore, it requires (1) to share behavior through the space, and (2) any generic worker to be able to compute the behavior.
			  For instance, a graph in \ac{tsc} could include code in an interpreted programming language.
			  % nosotros pasamos de este jaleo
			  % hacer referencia a on-demand-code
			  %In that way, a generic worker application can compute whatever code other processes define and share through the space.
			  %This pattern is only possible in \aclp{ts} where the behavior can be described generically (e.g. in object-oriented \acp{ts}).
			  %Furthermore, the tasks to be performed over \ac{ubicomp} environments are not generically processable by any node.
			  %In contrast, each device is responsible of managing its own actuators on behalf of the rest.
			  %Therefore, this pattern will not be considered.
  \item[Marketplace pattern.] In this pattern producers (or sellers) and consumers (or buyers) of resources interact to find the best deal. % TODO cita TripCom?
			  %It should be noted that the resources are products or services which can be bought and sold.
			  %Therefore, it is not applicable to the environments considered. % o use cases
			  %esto está relacionado con el paper aquel de TripCom y con el de Simon de hace un año en WoT
  \item[Specialist patterns.] In opposition to the replicated-worker pattern, in the specialist pattern each worker is specialized.
                              Therefore each worker performs a particular task.
                              \citeauthor{freeman_javaspaces_1999} enumerates three subtypes:
			      \begin{description}
				  \item[Blackboard pattern.]
					This pattern associates the concept of the space with a \emph{blackboard}.
					Following the analogy, the master is associated with a teacher, the tasks with \emph{problems} and the specialized workers with \emph{students}.
					The blackboard pattern starts when the \emph{teacher} writes a \emph{problem} in the \emph{blackboard}.
					The \emph{students} observe the space and write their intention to contribute to solve the problem (i.e. \emph{raise their hands}).
					The \emph{teacher} selects to the expert which will make a modification.
					After the modification, the \emph{teacher} decides if they found the solution another \emph{student} should contribute.
				  \item [Trellis pattern.]
					In this pattern, the master arranges the problem into low-level, mid-level and high-level pieces.
					The workers of each level benefit from the refined data provided by the immediate level below.
				  \item [Collaborative patterns.]
					It encompasses all the patterns which allow nodes to collaborate to complete a greater task (i.e. by creating a workflow).
					%They are specific to the domain so they cannot be generalized, but it includes all the patterns where the nodes collaborate to complete a greater task
					%(e.g. by creating a workflow).
			      \end{description}
\end{description}


% Voy preparando el terreno para que se vea que los primeros son más de computación y el otro de negociación
To summarize, the replicated-worker is centered in optimizing the computation by parallelizing task.
Command pattern can be seen as an abstraction of the latter where the behavior is shipped in the tuples.
Marketplace pattern allows negotiation of two entities through the space.
% Coño! pero si el especialista te permite colaborar...
Finally, specialist patterns allow nodes with distinct capacities to cooperate towards a common goal.


% TODO un esquema dividido en 4 en el que aparecen todos los patrones explicados!!!



\subsection{Envisaged Scenarios}
\label{sec:envisaged_scenarios}

This section devises two stereotypical scenarios for home automation.
Both scenarios emphasize how devices can coordinate in a decoupled mode using some of patterns seen in the previous section. % más de uno?
%The scenarios are formed by mobile phones, and embedded platforms with sensors and/or actuators.
Specifically, we believe that the \emph{specialist patterns} best fit the needs of \ac{ubicomp} environments:
\begin{itemize}
  \item The devices in \ac{ubicomp}, often serve to very specific needs.
        For example, let us imagine a mobile phone showing a message or an embedded device turning on the lights of a room.
        %Although there may exist some redundancy in the tasks which different nodes can be accomplish, the task are not nodes are not as interchangeable as in... son tareas simples, no necesitan optimizar ni lanzar muchas
  \item These tasks are usually lightweight.
        They do not require huge computation resources, but to achieve a concrete and simple goal.
\end{itemize}
Therefore, \ac{ubicomp} generally faces a problem of collaboration between nodes with different capacities, not a computation or a negotiation problem. % se entenderá a qué me refiero con computation??


%In this pattern a master writes a task into the space and waits for its result, which is performed by some of the workers specialized in this particular task (e.g. show a message or regulate the temperature).

\bigskip

The first scenario presents a room populated with several kind of sensors such as Oracle's SunSPOTs\footnote{http://www.sunspotworld.com},
% TODO referenciado en el capítulo 4!!!!
Digi's XBee sensors with a IP gateway,
the sensors on a KNX20 domotic bus and a fan connected to a FoxG20 embedded platform to act as an actuator (see Figure \ref{fig:sensorsphoto}).
Besides, an Android application semantically stores the user's temperature preferences.
An independent node (i.e. the master node) continuously reads from the space % using \emph{read primitive}
(1) the room's temperature, and
(2) the user's desired temperature.
Note that the master node does not care about who exactly provides the temperature information.
It just takes the first available graph from the space.
When the second one is below the first one, it generates a ``\emph{decrease temperature during a certain period}'' task which can be consumed by different independent worker nodes.
In this case, the FoxG20 periodically checks for orders it can fulfill extracting them from the space. % consumes them with a \textit{take primitive}.
% zasca, acabamos de poner en bandeja que nos rechine lo de comprobar periodicamente

\medskip

%%%%% ME HE QUEDADO AQUI!!!!!

The second scenario presents an application which helps the user to avoid the sedentary lifestyle by giving him different warnings.
We considered the recommendation that states that an adult should daily walk at least 10.000 steps \citep{tudor2002taking}.
Taking into account the expected steps which should have been completed in each moment of the day, the application generates different priority level messages.

There are several nodes involved in this task.
First, an Android phone periodically writes the number of steps covered by a user in that day\footnote{http://code.google.com/p/pedometer/}.
%Besides, it writes her profile, more important to this problem: her age.
Second, there are an undefined number of devices which know how to warn the user about her unhealthy behavior.
Each device belongs to a user.


The application does not directly decides how to show these messages.
Instead, it follows the \emph{blackboard pattern} and it writes its intention to complete that \emph{problem}.

The particularity if the application is that 
These messages are shown to the user by different means to warn him about unhealthy behavior.

To achieve it, the semantic information is used to find the accelerometers embedded in a user mobile and his age.
Depending on the priority of the messages, different devices which belong to that user, defined in the ontology, look for those messages.
If the priority level is low, the user can be warned in a less intrusive way than if the priority is high. Hence, room's light brightness can be increased for low priority notifications, a chumby for normal priority ones or the message can be shown directly in his mobile phone when the user should have covered many more steps than he has walked (high priority).




\subsection{Notification mechanism}
\label{sec:notification}

All the patterns explained in the previous section are triggered when another process writes information in the space (e.g. a task or a result).
To be aware of these writings the node can either poll the space or rely on a notification mechanism.
The later is more optimal and more scalable approach.
Therefore, although it is not the main focus of this thesis, in this section we present a subscription mechanism which can be used on top of the \ac{tsc} middleware proposed.

The requirements that this mechanism needs to fulfill are the following ones:
\begin{itemize}
  \item It should be independent of \ac{tsc}'s writing and reading primitives.
	This requirement ensures that frequent writings do not lead to excessive processing in resource constrained nodes.
	The main drawback of this requirement is that a node does not now when relevant data is written in the space \emph{per se}.
	A developer needs to advertise when it writes relevant data.
  \item Any node running our solution must be able to implement this mechanism.
	In other words, it no new dependencies should be added.
  \item The new primitives should use elements the current developer is used to (i.e. templates or RDF triples).
\end{itemize}


% explicar cómo funciona
% https://otsopack.readthedocs.org/en/latest/subscriptions.html
\subsubsection{Subscription primitives}

The subscription primitives the developer should use are the following ones:
\begin{itemize}
  \item \emph{Subscribe}. The node subscribes to the given template returning an URI which identifies the subscription.
    \begin{minted}{java}
URI subscribe(space_uri, template, listener)
    \end{minted}
  \item \emph{Unsubscribe}. Unsubscribes to a subscription given its subscription URI.
    \begin{minted}{java}
void unsubscribe(space_uri, String subscriptionURI)
    \end{minted}
  \item \emph{Notify}.
    \begin{minted}{java}
void notify(space_uri, template)
    \end{minted}
\end{itemize}


\subsubsection{Deployment}

The nodes responsible of handling subscriptions and notifications are called \emph{bulletin boards}.
Other nodes belonging to the same space, discover them using a out of scope method and publish their subscriptions and notifications using their HTTP API.

Each \emph{bulletin board}:
\begin{itemize}
  \item Belongs to a space.
  \item Exposes a subscription API.
  \item Shares its subscriptions with other bulletin boards which belong to the same space.
  \item Propagates the notifications to the relevant nodes using the \emph{callback url} provided by them.
\end{itemize}


The Figure~\ref{} describes a simple subscription use case.
\begin{enumerate}
  \item N1 subscribes to BB1 with a template t1.
  \item BB1 propagates the subscription provided by N1 to BB2 and BB3
  \item N3 notifies to BB3 about t2.
  \item Since t1 matches t2, BB3 tries to notify to N1 using the callback URI provided during the subscription process.
  \item Unfortunately, the BB3 cannot notify to N1 due to unexpected network problems.
  \item BB3 propagates the notification of t2 to BB2.
  \item BB2 reaches N1, so it notifies it about t2 using the callback URI.
\end{enumerate}

% TODO poner la foto esquemática de https://otsopack.readthedocs.org/en/latest/subscriptions.html
\section{Direct Consumption of \acs{rest} services}
\label{sec:direct_actuation}


The actuation mechanism presented in the previous section uses the \Space{} (i.e. it is indirect).
This mechanism implies that the actuator must be aware of the content of the space.
For instance, a heater must check the space to find if a new desired temperature was written.
% lo podrá reusar a través de un intermediario
% interop era una de las cosas que queriamos cuidar
% y qué pasa con las soluciones REST que quieren usar lo nuestro?


% por qué? motivación: permitir a otras app integrarse en nuestro espacio
In contrast, \ac{rest} services constitute an emerging mechanism to directly actuate in the physical environment.
According to the \ac{rest} principles, a client should navigate through these resources with no prior knowledge of the \ac{api}.
% Copiar esta explicación mejor de algún lado:
The client should 
\begin{enumerate*}[label=\itshape(\arabic*\upshape)]
  \item interpret the representations provided by the server and then
  \item choose the appropriate state transition from the hypertext according to its intention. % y su conocimiento básico del protocolo: CRUD
\end{enumerate*}


\subsection{Background}


As Section~\ref{sec:network_properties} explained, semantic representations do not include a native way to express the hypertext. % TODO realmente se explica?
To solve this, three solutions can be adopted:
% Unos proponen extender con ontologías
\begin{enumerate}
  \item To use an ontology to represent the hypertext \citep{kjernsmo_necessity_2012},
  \item To embed the hypertext independently to the representations on the \ac{http} headers \citep{mark_web_2010}, and
  \item To provide a description of the resources using the \ac{http} OPTIONS verb \citep{verborgh_functional_2012,verborgh_ijcs_2014}.
\end{enumerate}


The latter two enable to discover resources and state transitions without adding metadata to the representations.
This allows not only to describe semantic representations, but any type of formats.


\citeauthor{mark_web_2010}'s \citep{mark_web_2010} approach is extended by \citet{erik_profile_2013} to define how to embed additional semantics to process a resource representation.
\citet{erik_profile_2013} calls these additional semantics \emph{profiles} and identifies them using \acsp{uri}.


\citet{verborgh_ijcs_2014} present a more expressive solution which goes beyond simply describing the resource's type.
It also allows to semantically describe the knowledge needed to use a concrete \acs{http} verb on a resource, and the content this request returns. % o precondition
The materialization of this proposal is called \restdesc{} \citep{verborgh_functional_2012}.
\citet{mayer_semantic_2013} use \restdesc{} in an environment populated by web-powered devices. % i.e. the \ac{wot}
The environment it presents is equivalent to the ones envisioned by this dissertation.


\bigskip


We consider \restdesc{} the best current solution which helps to achieve truly \ac{rest}ful \acsp{api}.
Therefore, this chapter assumes that the \ac{http} \acsp{api} whose capacities we want to reuse in our space model describe their \acsp{api} with \restdesc{}.


\subsection{\restdesc{}}
\label{sec:restdesc}

% TODO mirar si tiene cabida la mención de otros enfoques para describir recursos que no son muy RESTful
%Different ways exist to describe \ac{rest} services. % mencionar WADL, etc.?
% citar a donde se hable de \restdesc{} y así ya se empieza a explicar la solución de forma discreta.
\restdesc{} describes \acs{http} methods using rules expressed in the \ac{n3} language \citeweb{n32011}.
% he evitado explicar que las reglas tienen premisa y conclusión, porque me parece demasiado obvio
A rule's \emph{premise} expresses the requirements to invoke a \ac{rest} service.
A rule's \emph{conclusion} expresses both the \ac{rest} call that needs to be made and the description of that invocation result.


\citet{verborgh_ijcs_2014} propose a proof-based composition mechanism for Web \acp{api} using \restdesc{}.
This mechanism uses as inputs:
\begin{enumerate*}[label=\itshape(\arabic*\upshape)]
  \item an initial state,
  \item a goal state,
  \item Web \ac{api} descriptions using \restdesc{}, and
  \item optional background knowledge.
\end{enumerate*}
Each of these inputs are semantically expressed and therefore, they can be processed by standard \ac{n3} reasoners.
The reasoners generate proofs about how to achieve the goal starting from the initial state using the rest of the inputs.
These proofs can be seen as steps that need to be made to reach a desired state.


Additionally, \citet{verborgh_ijcs_2014} distinguish between pre-proof and post-proof.
The first, are those which assume that the execution of all \acs{api} calls will behave as expected.
The latter, can be seen as a \emph{revision} of the pre-proof.
It executes the Web \acs{api} of the pre-proofs and uses actual execution results to generate a new proof.


\section{Comparison}

The first actuation mechanism requires a subscription mechanism and provides space and time autonomy.
However, it does that by creating a dependency on the \space{}.
Two nodes will not be able to communicate with each other without the \space{}.


The second approach offers that independence of the \space{}.
Any node can directly invoke a service to act over the environment.
However, it also implies that a node needs to have prior knowledge about the environment to reason over it.
This knowledge must be adquired from the other nodes somehow (i.e. demands more network usage).
In the technical aspect, it requires a reasoner, which is not necessarily available in all computing platforms.
Besides, it generates more computation and networking activity on the node responsible of the process.
Remarkably, this second approach would allow to integrate third \ac{wot} applications in our middleware.


% Tabla: XXX
% autonomía: las de space-based vs. 
% dependencia en spacio
% requirements: subscription, reasoning
% interoperability: con otros servicios, no alreves

 % de los dos anteriores
\section{Interoperation Between both Approaches}
\label{sec:actuation_interoperation}


\ac{rest} and \ac{rest}-like services constitute an emerging mechanism to directly actuate in the physical environment.
Therefore, their seamless integration opens the door to reuse the capacities of many existing devices. % interoperate
Specifically, we aim to integrate \ac{rest} services in our space-based model. % integration, reuse, alignment, close the gap
% we first need to close the gap between both approaches.
%This section aims to experimentally proof the validity of the alignment presented in the previous section.


This section explores this integration helped by a simple real-world scenario.
The goal of the scenario is to remotely change the light of a lamp.
Sections \ref{sec:actuation_scn1}, \ref{sec:actuation_scn2} and \ref{sec:actuation_scn3} present three different implementations of this scenario. % which guide the reader to the mentioned integration


The first implementation uses the proof-based actuation mechanism presented in Section~\ref{sec:restdesc}.
The second implementation uses the space-based actuation mechanism presented in Section~\ref{sec:actuation_space}.
Then, we mix nodes from the previous implementations in the third implementation.
This lead us to analyse the adjustments required to enable their interoperation.
% TODO TODO TODO The implementation is publicly available at THIS URL.
\section{Discussion}
\label{sec:actuation_discussion}

The following sections scrutinize the strengths and weaknesses of the implementation presented in the previous section.
Besides, they discuss other design alternatives. % design, alternative, etc.

\subsection{Obtaining Resource Descriptions}

The core of the proof-based actuation mechanism are the resource descriptions.
They must be read from the nodes willing to actuate prior to reasoning. % obvio
This action is performed by the consumer in \implRest{} and by the agent in \implMix{}.
Both nodes crawl a given \ac{api} starting from an \ac{url} to obtain the descriptions.
The discovery of the initial \acp{url} is out of the scope of this chapter.
% Otra alternativa: An agent could discover the descriptions and simply write them in the coordination \Space{}.


Another alternative to discover these descriptions is to make them part of the \clues{} presented in Chapter~\ref{cha:searching}.
This option ensures that they will be available in any \consumer{}.
Besides, the static nature of these descriptions does not break the \clues{} stability assumption in which our architecture is founded. % explicar más?


In any case, this chapter focuses on the interoperability problem, not on how to obtain the descriptions.
Consequently, for the sake of clarity, we opted for the more intuitive and common alternative. % simpler / more intuitive / easier to understand
That is, we assume that a process crawls the descriptions in the background.



\subsection{Obtaining Background Knowledge}

Besides resource descriptions, \citeauthor{verborgh_ijcs_2014}'s actuation mechanism also requires an initial state and background knowledge as inputs (see Section~\ref{sec:restdesc}).
In \implRest{}, the consumer obtains this knowledge crawling over all the possible \acsp{api}.
In \implMix{}, we also add all the knowledge from the space.
This is feasible because it is located in the same machine as the agent which needs it, so does not demand any network usage. % no es costoso obtenerla


However, as explained in Chapter~\ref{cha:searching}, the data in \ac{ubicomp} changes too frequently to simply crawl it from time to time.
% Reading all the space would be highly inefficient.
Crawling all the \acsp{api} each time a change needs to be done is also highly inefficient.
Hence, the approach used to obtain knowledge is a clear simplification.
This simplification is justified because this chapter is centred on the interoperability rather than in the efficient communication mechanisms.


\bigskip


A possible optimization would be to benefit from the search architecture presented in Chapter~\ref{cha:searching}.
To reduce readings on the space (i.e. network usage), we propose a procedure composed by two reasoning steps.
In the first one, we only use local incomplete knowledge derived from the \clues{}.
Then, we read from the \Space{} just the knowledge needed to confirm the pre-proofs obtained in the first reasoning.
The second reasoning uses this knowledge to get real pre-proofs. % real o actual pre-proofs


% Y ahora explicado en mayor detalle:
Therefore, a node which wants to actuate over the \Space{} will need to obtain the \emph{clues} from the \ac{wp}.
%These clues, as explained before, tell what kind of content other nodes provide. % un poco redundante
Let us assume that these \emph{clues} are composed by the predicates used by the nodes which provide content. % es una de las alternativas que se planteaban
The existence of a predicate used in a premise does not necessarily imply that this rule can be used.
Nevertheless, its absence does imply that it will not be used (see Figure~\ref{fig:activation_rules}).
Therefore, we can create temporary \emph{activation rules} from \clues{} which activate those potential rules. % latter rules las que se han mencionado primero


\InsertFig{activation_rules}{fig:activation_rules}{
  Sample clues, two rules and the activation rule created from the first rule.
}{
  According to the clues shown, the second rule will never be invoked.
}{0.8}{}


An \emph{activation rule} for a rule R contains a \emph{true} in the premise.
The conclusion is made by R's premise substituting the variables with fictitious \acsp{uri} with a common prefix (see Figure~\ref{fig:activation_rules}).
These fictitious \acsp{uri} are used to distinguish when a triple should be replaced by actual knowledge from the space. % explicar mejor?



\subsection{Responsibility for Triggering the Proof-based Actuation Mechanism}
\label{sec:responsible_proof}

Section~\ref{sec:restdesc} describes two coarse-grained steps to learn how to use web \acsp{api} which use \restdesc{}:
\begin{enumerate}
  \item Reasoning over the descriptions, background knowledge, an initial state and a goal state.
        The result of the reasoning process if a pre-proof, which can be seen as a tentative \emph{execution plan} to achieve the goal.
  \item Check the execution plan by following it.
\end{enumerate}


% no obligamos a implementar una u otra, pero recomendamos esto: XXX
\implMix{} opts for triggering the reasoning process when an agent receives a notification.
Previously, it subscribes to any type of tasks written into the space.


The reasoning can be performed in any node apart from the one which holds the \Space{}:
%However, we do not adhere to any of them.
%We leave as a future work to implement and quantitatively compare them.

\begin{itemize}
  \item Any \consumer{} interested on changing the environment can trigger the process.
	If these \consumers{} use the search mechanism presented in Chapter~\ref{cha:searching}, they will have background knowledge about other nodes.
	This reduces the dependency on the node providing the \coordspace{}.
	However, it requires them to perform tasks such as reasoning and checking the pre-proofs.
	
	Unfortunately, the first task increases the computation and the second the network usage.
	As we already mentioned in previous chapters, these tasks severely affect to the energy consumption.
	Furthermore, some constrained platforms will not even be able to reason.
	
  \item To mitigate this problem, we can delegate this task only on the nodes able to perform such tasks.
	In fact, these nodes can follow the \emph{replicated-worker pattern}.
	That is, they can read from the space goals to trigger the process (i.e. \emph{reasoning tasks}).
	Apart from balancing the load between all the worker nodes, any node can stop being worker at any time by not taking more \emph{tasks} (e.g. if it has low energy).
	These nodes must be \consumers{} to use the \clues{} from the search mechanism as background knowledge.
\end{itemize}


Although both alternatives avoid the dependency on the \Space{}, the space-based actuation mechanism intrinsically depends on the \Space{}.
Therefore, it makes sense that the unavailability of the space will cause the unavailability of actuating on the space.
On the contrary, it simplifies the consumers' responsibilities, which just need to worry about writing a task in the space.



\subsection{Interoperation Weakness}

The previous sections presented various alternative designs and their likely impacts on the actuation performance.
However, none of them addresses the interoperability flaws of \implMix{}.
In this regard, the simple mapping between a consumer's subscription and a goal is probably its most evident interoperability flaw.


In \implMix{}, both the consumer subscription to a result and the goal for the reasoning match.
However, there is no guarantee that the consumer will always use a subscription which matches with a goal.
For instance, the consumer could use a more general subscription and then filter the concrete tasks it is interested in.
Even worse, there is no guarantee that the consumer will subscribe to any result.
Thus, the universality of the proposed alignment can be easily affected.
To avoid this undesired effect, any developer of a consumer node should need to take into account some good practices. % to increase the interoperability chances.


% qué pasa si el consumidor del escenario 2 no se suscribe a los resultados?
% qué pasa si se suscribe a un patrón más general?
A more universal approach would be to deduce the goal from the task. % universal, generalizable
For instance, from a task of \emph{regulate temperature to 6ºC} the \Space{} could deduce the goal state of \emph{temperature of 6ºC}.
In this case, the mapping should be either
\begin{enumerate*}[label=\itshape(\arabic*\upshape)]
  \item provided by the consumer or
  \item pre-set in the space.
\end{enumerate*}
The first choice demands to feed the \Space{} with additional information. % to the middleware
%This is very inflexible and differs little from manually programming a gateway with each provider. % REALMENTE?
The second choice assumes a concrete ontology must used or extended by the user to represent tasks.
Therefore, it would limit the freedom of choosing any vocabulary to define a task.


Since this chapter simply wants to remark the potential interoperability of the presented approaches,
we opted for selecting the automatic translation from a subscription to a goal.
The implementation of the rest of the approaches is left as future work.



\subsection{Advanced Challenges} % o Advanced Limitations

The scenario used as a guiding example is very basic.
Consequently, the interoperation example shown requires further work to check its feasibility in more advanced scenarios.
We anticipate the following challenges:
\begin{itemize}
  \item When there are two or more paths to a goal, how can we discern which one to follow?
	This problem is specific to the proof-based mechanism.
  \item How does the middleware deal with the coexistence of both mechanisms.
	When both methods can be applied, which one is triggered?
	Will one of them prevail over the second?
\end{itemize}
\section{Summary}
\label{sec:actuation_summary}

This chapter presented two ways to actuate on the physical environment.
The first is the usual way to operate through spaces and provides a higher degree of decoupling.
However, it requires participants to use our middleware's primitives. % requiere la cooperación de los proveedores...
In other words, our middleware is not able to reuse third applications \ac{rest} services.


The second actuation mechanism directly consumes \ac{rest}ful \acs{http} \acsp{api}.
This mechanism relies in the semantic description of the services, additional knowledge and in a reasoning process. % additional knowledge: background + initial
With that information, it is able to generate executions plans towards a goal.
Following these plans implies different calls to the different services.


We implemented the same scenario using two actuation mechanisms.
Besides, since interoperability is one of our middleware's guiding principles, we sketched how to reuse these \ac{rest}ful \acs{http} \acsp{api} in our \Space{} model in a third implementation.
This reuse avoids any alteration on the space-based consumer or the \ac{http} provider.
Instead, it improves the \Space{} implementation with an Agent in charge of generating execution plans.
To aim this generation, the agent which reuses the information from the space-based actuation.
Doing so, it avoids requiring any additional information from the developer.


There are other design alternatives to promote the reuse of \acs{http} \acsp{api} from a space-computing middleware.
We described these alternatives and analyse their advantages and weaknesses.
% regarding lo que sea?
However, some questions remain still unsolved: will both methods be triggered indistinctly or will the first prevail over the second?
% uso de nuestra actuación por parte de apps WoT
Finally, the question of how to reuse actuation mechanisms of nodes using \ac{ts} patterns from \ac{wot} has not been addressed.
However, this chapter has exposed, described and compared the key points towards the actuation through a \ac{tsc} middleware for \ac{ubicomp}.
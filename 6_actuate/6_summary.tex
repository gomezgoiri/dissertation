\section{Summary}
\label{sec:actuation_summary}

This chapter presented two ways to actuate on the physical environment.
The first is the usual way to operate through spaces and provides a higher degree of decoupling.
However, it requires participants to use our middleware's primitives. % requiere la cooperación de los proveedores...
In other words, our middleware is not able to reuse third applications \ac{rest} services.


The second actuation mechanism directly consumes \ac{rest}ful \acs{http} \acsp{api}.
This mechanism relies in the semantic description of the services, additional knowledge and in a reasoning process. % additional knowledge: background + initial
With that information, it is able to generate executions plans towards a goal.
Following these plans implies different calls to the different services.


We implemented the same scenario using two actuation mechanisms.
Besides, since interoperability is one of our middleware's guiding principles, we sketched how to reuse these \ac{rest}ful \acs{http} \acsp{api} in our \Space{} model in a third implementation.
This reuse avoids any alteration on the space-based consumer or the \ac{http} provider.
Instead, it improves the \Space{} implementation with an Agent in charge of generating execution plans.
To aim this generation, the agent which reuses the information from the space-based actuation.
Doing so, it avoids requiring any additional information from the developer.


There are other design alternatives to promote the reuse of \acs{http} \acsp{api} from a space-computing middleware.
We described these alternatives and analyse their advantages and weaknesses.
% regarding lo que sea?
However, some questions remain still unsolved: will both methods be triggered indistinctly or will the first prevail over the second?
% uso de nuestra actuación por parte de apps WoT
Finally, the question of how to reuse actuation mechanisms of nodes using \ac{ts} patterns from \ac{wot} has not been addressed.
However, this chapter has exposed, described and compared the key points towards the actuation through a \ac{tsc} middleware for \ac{ubicomp}.
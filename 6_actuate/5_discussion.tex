\section{Discussion}
\label{sec:actuation_discussion}


In the previous section we presented how the space-based approach can interoperate with semantically described \ac{rest} services.
The space-mediated communication transparently invokes \ac{rest} services on behalf of the consumer which follows the \ac{ts} usage patterns.
This section scrutinizes the strengths and weaknesses of the presented design, in conjunction with some other design alternatives. % design, alternative, etc.


% TODO poner aquí algunos indicadores de la implementación de escenarios?
%   e.g. cuando código extra ha hecho falta añadir en el tercero para que se hablen entre sí


\subsection{Actuation Modelling}

% explain why like this? why not directly sensors/light?
The most abstract way to actuate on the environment would be to invoke a change referring to the sensed values.
For example, stating ``set \emph{Room A}'s light-level to 19 luxes''.
This would require the coordination of all the actuators which directly or indirectly affect this value.
Following the example, the lamps in \emph{Room A} and the curtains of \emph{Room A}'s windows should coordinate to give the exact desired value.


Even in this case, many other physical aspects would affect the value.
For instance, the daylight at each time of the day or the location of the light sensor.
As a consequence, we opt for clearly distinguishing between actuators and sensors.
In our scenarios, the data measured by the sensors can only be indirectly affected by the actuators.


This indirect effects on the sensed values could be modelled for each actuator.
For example, to state that turning on a lamp increases the light measured by a sensor.
However, precisely anticipating the exact value of the new measure is difficult if not impossible due to the mentioned external physical conditions.
Considering this difficulty to predict their relationship and its tangential importance for the scenario, we simply ignore these relationships.
We assume that the consumer previously determined which actuators it should change to fulfil its initial abstract goal.


\subsection{Ontologies Used}

% Ontologies used
% explicar por qué hemos modelado así los escenarios?
The \emph{glue} between the two actuation worlds presented in Section~\ref{sec:actuation_interoperation} is the data they use. % o between both approaches, o yo qué sé
Therefore, it must be modelled according to the same ontologies.
This section provides the rationale behind the selection of the ontologies we used.
Note that these ontologies are not the unique selection which could be made.
Neither does it pretend to be the best one.


% por qué con medidas?
To represent the value of a lamp actuator, we use the SSN ontology \citeweb{ssn}.
For example, we express that ``the lamp actuator has a light value of \emph{N} luxes''.
The SSN ontology is intended to describe measures for sensors, so its use for actuators can sound contradictory.
Note how the previous statement differs from ``the light sensed by a sensor located near the lamp has a value of \emph{N} luxes''.
However, to the best of our knowledge, there is no more widely-accepted ontology to specifically describe actuators.


% por qué uso preferencias?
The preferences ontology is another of the core ontology used in our implementations.
It is used to clearly distinguish between the value of an actuator and a consumer's intention or desire to change it.
A preference can turn into an actuator's new state, but it is not always necessarily true.
The actuator can discard a preference due to an usage policy, a malfunction or any other reason.

% por qué no se tienen en cuenta otras características como la localización o lo que sea => por simplificar
%   en el goal se podrían describir



\subsection{Obtaining Resource Descriptions}

The core of the proof-based actuation mechanisms are the descriptions.
They must be read from the actuation nodes prior to reasoning. % obvio
This action corresponds to the consumer in the first scenario and to the agent which resides in the \Space{} in the third scenario.

In both cases, they crawl over a given \ac{api} starting from a \ac{url}.
The discovery of the \ac{url} is out of the scope.
% Otra alternativa: An agent could discover the descriptions and simply write them in the coordination \Space{}.


Another alternative to discover this descriptions is to make them part of the \clues{} presented in Chapter~\ref{cha:searching}.
This option ensures that they will be available in any \consumer{}.
Besides, the static nature of these descriptions does not break the stability assumption of the \clues{}. % explicar más?


In any case, this chapter focuses on the interoperability problem, not on how to obtain the description.
Consequently, we opted for the most simple alternative: crawl the descriptions in an independent process. % simpler / more intuitive / easier to understand



\subsection{Obtaining Background Knowledge}

Besides resource descriptions, \citeauthor{verborgh_ijcs_2014}'s proof-based mechanism also requires an initial state and background knowledge as inputs (see Section~\ref{sec:restdesc}).
In the first scenario the consumer obtains this knowledge crawling over all the possible \acsp{api}.
In the third scenario we also add all the knowledge from the space.
This is feasible because it is located in the same machine as the agent which needs it, so does not demand any networking. % no es costoso obtenerla


However, as explained in Chapter~\ref{cha:searching}, the data of the providers in \ac{ubicomp} is too dynamic to simply crawl it from time to time.
% Reading all the space would be highly inefficient.
Crawling all the \acsp{api} each time a change needs to be done is also highly inefficient.
Hence, the proposed approach to obtain knowledge is a clear simplification.
This simplification is justified because this chapter is centred on the interoperability rather than in the efficient communication mechanisms.


\bigskip


A possible optimization would be to benefit from the search architecture presented in Chapter~\ref{cha:searching}.
To reduce readings on the space (i.e. network usage), we propose a procedure composed by two reasoning steps.
In the first one we only use local incomplete knowledge derived from the \clues{}.
Then, we read from the \Space{} just the knowledge needed to confirm the pre-proofs obtained in the first reasoning.
The second reasoning uses this knowledge to get real pre-proofs. % real o actual pre-proofs


% Y ahora explicado en mayor detalle:
Therefore, a node which wants to actuate over the space will need to obtain the \emph{clues} from the \ac{wp}.
%These clues, as explained before, tell what kind of content other nodes provide. % un poco redundante
Let us assume that these \emph{clues} are composed by the predicates used by the nodes which provide content. % es una de las alternativas que se planteaban
The existence of a predicate used in a premise does not necessarily imply that this rule can be used.
Nevertheless, its absence does imply that it will not be used (see Figure~\ref{fig:activation_rules}).
Therefore, we can create temporary \emph{activation rules} from \clues{} which activate those potential rules. % latter rules las que se han mencionado primero


\InsertFig{activation_rules}{fig:activation_rules}{
  Sample clues, two rules and the activation rule created from the first rule.
}{
  According to the clues shown, the second rule will never be invoked.
}{0.8}{}


An \emph{activation rule} for a rule R contains a \emph{true} in the premise.
The conclusion is made by R's premise substituting the variables with fictitious \acsp{uri} with a common prefix (see Figure~\ref{fig:activation_rules}).
These fictitious \acsp{uri} are used to distinguish when a triple should be replaced by actual knowledge from the space. % explicar mejor?



\subsection{Responsibility for Triggering the Proof-based Actuation Mechanism}
\label{sec:responsible_proof}

Section~\ref{sec:restdesc} described two coarse-grained steps:
\begin{itemize}
  \item Reasoning over the descriptions, background knowledge, an initial state and a goal state.
        The result of the reasoning process if a pre-proof, which can be seen as a tentative \emph{execution plan} to achieve the goal.
  \item Check the execution plan by following it.
\end{itemize}


% no obligamos a implementar una u otra, pero recomendamos esto: XXX
The third scenario presented in the previous section opts for triggering the reasoning process when an agent receives a notification.
Previously, it subscribes to any task written into the space.


The reasoning can be performed in any node apart from the one which holds the \Space{}:
%However, we do not adhere to any of them.
%We leave as a future work to implement and quantitatively compare them.

\begin{enumerate}
  \item Any \consumer{} interested on changing the environment could trigger the process.
	If these \consumers{} use the search mechanism presented in the Chapter~\ref{cha:searching}, they will have background knowledge about other nodes.
	This reduces the dependency on the node providing the \emph{coordination space}. % TODO buscar su nombre bueno
	However, it requires them to perform tasks such as reasoning and checking the pre-proofs.
	While the latter increases the network usage, the first increases the computation.
	As we already mentioned in previous chapters, these tasks severely affect to the energy consumption.
	Furthermore, some resource platforms will not be even able to reason.
	
  \item To mitigate that problem, we could delegate this task only on the nodes able to perform such tasks.
	In fact, these nodes could follow the \emph{replicated-worker pattern}.
	They could read from the space goals to trigger the process (i.e. \emph{reasoning tasks}).
	Apart from balancing the load between all the worker nodes, any node can stop being worker at any time by not taking more \emph{tasks} (e.g. if it has low energy).
	These nodes must be \consumers{} to use the \clues{} from the search mechanism as background knowledge.
\end{enumerate}


Although both alternatives avoid the dependency on the \Space{}, the space-based actuation mechanism intrinsically depends on the \Space{}.
Therefore, it makes sense that the unavailability of the space will cause the unavailability of actuating on the space.
On the contrary, it simplifies the consumers' responsibilities, which just need to worry about writing a task in the space.


% TODO DISCUTIR ESTO
However, some questions remain still unsolved: will both methods be triggered indistinctly or will the first prevail over the second?


\subsection{Interoperation Weakness}

The previous sections presented various alternative designs and their likely impacts on the actuation performance.
However, none of them addresses the interoperability flaws of the third scenario.


In this regard, the conversion of a consumer subscription into a goal is probably the most evident interoperability flaw of the third scenario.
Although in the scenario both aspects match, there is no guarantee that the consumer will use a subscription which matches with a goal.
For instance, she could use a more general subscription and then just treat the concrete tasks.
Even worse, there is no guarantee that the consumer would subscribe to any result.
Thus, the universality of the proposed alignment can be easily affected.


% qué pasa si el consumidor del escenario 2 no se suscribe a los resultados?
% qué pasa si se suscribe a un patrón más general?
A more universal approach would be to deduce the goal from the task. % universal, generalizable
For instance, from a task of \emph{regulate temperature to 6ºC} the \Space{} could deduce the goal state of \emph{temperature of 6ºC}.
In this case, the mapping should be either
\begin{enumerate*}[label=\itshape(\arabic*\upshape)]
  \item provided by the consumer or
  \item pre-set in the space.
\end{enumerate*}
The first choice demands to provide additional information to the \Space{}. % to the middleware
This is very inflexible and differs little from manually programming a gateway with each provider.
The second choice assumes a concrete ontology must used or extended by the user to represent tasks.
Therefore, it would limit the freedom of choosing any vocabulary to define a task.


Due to that reason and taking into account that this chapter simply wants to remark the potential interoperability of the presented approaches,
we opted for selecting the automatic translation from a subscription to a goal.



\subsection{Advanced Limitations}

The scenario presented is quite limited.
Consequently, the interoperation example shown requires further work to check its feasibility in more advanced scenarios.
We anticipate the following problems:

\begin{itemize}
  \item When there are two or more paths to a goal, how can we discern which one to follow?
	This problem is specific from the proof-based mechanism.
  \item How does the middleware deal with the coexistence of both mechanisms.
	When both methods can be applied, which one is triggered? Will one of them prevail over the second?
\end{itemize}
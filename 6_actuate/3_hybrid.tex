\section{Hybrid actuation}
\label{sec:hybrid_actuation}

The actuation technique presented by Section~\ref{sec:actuation_space} constitutes the natural way of acting using a \Space{}.
However, \ac{rest} and \ac{rest}-like services constitute an emerging mechanism to directly actuate in the physical environment.
Therefore, their seamless integration (i.e., interoperation) opens the door to reuse the capabilities of many existing devices.
Consequently, we aim to integrate semantically described \ac{rest} services in our space-based model. % integration, reuse, alignment, close the gap
That is, mix \spaceActuation{} with \restActuation{}.
% we first need to close the gap between both approaches.
%This section aims to experimentally proof the validity of the alignment presented in the previous section.


This section explores this integration.
Section~\ref{sec:actuation_comparison} compares \spaceActuation{} and \restActuation{} and argues the need of this integration.
Then, Section~\ref{sec:actuation_scn3} presents a new implementation of the baseline scenario.
This implementation shows how the \Space{} can transparently invoke \ac{rest} services on behalf of any consumer following the \spaceActuation{}.
To this end, it reuses nodes from the implementations presented in sections \ref{sec:actuation_scn1} and \ref{sec:actuation_scn2}.
%This lead us to analyse the adjustments required to enable their interoperation.


\subsection{Comparison}
\label{sec:actuation_comparison}

The actuation technique explained in Section~\ref{sec:actuation_space} requires a subscription mechanism, but in exchange, it provides space and time decoupling.
Obviously, this decoupling comes at the price of dependency on the \Space{}.
That is, two nodes will not be able to communicate with each other without the \Space{}.


The second approach presented in Section~\ref{sec:direct_actuation} offers independence of the \Space{}: any node can directly invoke a service to act over the environment.
This approach allows reusing \ac{wot} applications' actuation capabilities, providing they are properly described. % apps, APIs o services? % no digo integration/interop porque es sólo one-way
This reuse is enabled by a rule-based reasoning engine. % reasoning engine ya queda bien?
% The unavailability of these engines in many computing platforms may an obstacle for its massive adoption. % como apunte de su practicalidad, pero no sé si venía muy a cuento



% Tabla: XXX
% autonomía: poner o simplemente indirect vs direct?
% requirements in consumers: estar pendiente del espacio vs razonar
% requirements in providers: estar pendiente del espacio vs (describir semánticamente servicio => equivalente a tasks)
\begin{table}[htbp]
  \caption{Characteristics of the discussed actuation mechanisms.}
  \begin{center}
    \footnotesize
    \begin{tabular}{llll} 
      \hline
      \multirow{2}{*}{Actuation} &
      Communication & % o poner autonomias y demás para enfatizar?
      \multirow{2}{*}{Benefits} &
      Required \\
      &
      style &
      ~ &
      features \\
      \hline
      Space-based & Indirect & Decoupled communication & Subscriptions \\[0.2cm]
      REST-based & Direct & Reuse of third \ac{wot} & Rule-based \\ % mencionar interop? % TODO ampliar la noción de WoT a REST-like?
      & & applications & reasoning \\ % no digo "wot actuation capacities porque ya está verbalizado en la explicación"
      \hline
    \end{tabular}
  \end{center}
  \label{tab:actuation_mechanisms}
\end{table}


\bigskip

The impact of both mechanisms in resource constrained devices will be affected by their computing and networking activities.
\ac{ts}-based patterns require all the participants to be aware of what is written into the space to react (i.e., be proactive).
Both consumers and providers read and write from the space, subscribe to specific changes and receive notifications.
Thanks to this specificity, they are only affected by the contents they are interested in. % reduciendo la actividad
Furthermore, they only perform trivial computing task: interpreting results. % pattern matching sólo en el caso del coordination space!


The space, which corresponds to our solution's \coordspace{}, will be in charge of the query and notification mechanisms.
The computation activity in the space varies depending on the complexity of these mechanisms' implementation, which is beyond the scope of this dissertation.
The space will be involved in any networking activity.
Therefore, although it depends on the number of providers and consumers' interactions, the networking activity will presumably be high.


\restActuation{} requires the consumer to have prior knowledge about the environment to reason over it. % frase susceptible de simplificar
Since this knowledge must be acquired from remote nodes before reasoning, this approach demands an extra network usage that the first does not.
% de pre-proof a post-proof se generará más networking activity seguro? por qué? por invocar servicios REST? Es equivalente a andar leyendo tareas y escribiendo resultados...
Furthermore, this actuation technique also generates additional computation activity on the node responsible for reasoning.
%Both aspects have a negative impact in the energy consumption.
Consequently, from the data consumer perspective, this approach will most likely be more energy demanding than the first one. % might porque no está medido, pero es algo intuitivo
In contrast, this mechanism demands few things to the provider: to serve \ac{http} resources and provide their descriptions. % aclarar quien es el provider/actuator?



% Tabla: XXX
% autonomía: poner o simplemente indirect vs direct?
% requirements in consumers: estar pendiente del espacio vs razonar
% requirements in providers: estar pendiente del espacio vs (describir semánticamente servicio => equivalente a tasks)
\begin{table}[htbp]
  \caption{Foreseeable networking and computing impact on the nodes involved in the actuation mechanism.}
  \begin{center}
    \footnotesize
    \begin{tabular}{llp{4cm}p{4.4cm}}
      \hline
      \multirow{2}{*}{Actuation} &
      \multirow{2}{*}{Perspective} &
      \multicolumn{2}{c}{Activity} \\
      ~ &
      ~ &
      \multicolumn{1}{c}{Networking} &
      \multicolumn{1}{c}{Computation} \\
      \hline
      \multirow{3}{*}{\ac{ts} patterns} & Provider & Proactive, limited activity & Limited: Results Interpretation \\
				    ~ & Consumer & Proactive, limited activity & Limited: Results Interpretation \\
				    ~ & Space & Reactive, high activity & Varies with the implementation \\[0.2cm]
      \ac{rest} \acp{api} & Provider & Reactive, limited activity & Limited: Handling requests \\
        consumption       & Consumer & Proactive, high activity &  Demanding: Reasoning \\ % Y comprobar planes...
      \hline
    \end{tabular}
  \end{center}
  \label{tab:actuation_networking_computing}
\end{table}


\bigskip


As can be appreciated in Table~\ref{tab:actuation_networking_computing}, the providers in the second actuation mechanism are more lightweight.
They just attend to the request received using HTTP.
As a consequence of these few requirements, exposing the actuation capabilities of the limited devices with HTTP is a consolidated trend.
This tendency is backed by the \ac{wot} initiative.
To make these web-enabled actuators automatically reusable by the consumers, the second mechanism only requires them to describe their resources semantically.
This can be done before deploying them and does not affect to their usual operation.


From the perspective of our space-based computing solution, reusing the existing HTTP providers will potentially open a new world of actuation possibilities.
For that, the simplicity of the consumers in \spaceActuation{} should be respected. % Simplicidad: networking y computing que aparecen en la tabla.


% TODO podríamos decir que el proof-based es más genérico?
%  + Space-based es muy dependiente del tipo de datos escritos en el espacio.
%    Proof-based también lo es a decir verdad.
%  + El proveedor del Space-based no tiene forma de saber cómo activar un mecanismo concreto.
%    Salvo que tenga posibilidad de saber a qué se han suscrito otros...
%    No sé como expresar esa diferencia.


\subsection{Baseline scenario: Implementation 3}
\label{sec:actuation_scn3}
\newcommand{\implMix}{\emph{Implementation 3}}

This implementation aims to prove that our space-based middleware can easily reuse third-party applications' providers capabilities. % easily? transparently?
It presents the space and the consumer (i.e., \nodeConsSpace{}) from the \implSpace{}, but it replaces the provider with the one from the \implRest{} (i.e., \nodeProvRest{}).
Both nodes have a good foundation for the interoperation because they use the same vocabularies. % to describe the knowledge they manage.
However, the \nodeProvRest{} and the \nodeConsSpace{} rely on different communication mechanisms: direct communication and indirect communication.


To close the gap between these two worlds, we avoid changing the participant node's implementations.
Instead, we create an agent which acts on the consumers' behalf. % TODO Yo he entendido que se sobreentiende qué es un agente
% WIKIPEDIA: "which derives from the Latin agere (to do): an agreement to act on one's behalf"
This agent resides in the same machine as the \Space{}, but its existence must not interfere with the \Space{} one. % dificil explicar en qué terminos: es otro proceso
Therefore, it should run on an independent process.


The agent takes care of the tasks that \nodeConsRest{} does in the \implRest:
\begin{enumerate*}[label=\itshape(\arabic*\upshape)]
  \item crawls the discovered \acsp{api}\footnote{The discovery process is out of the scope of this implementation.},
  \item reasons about their data to get a plan, and
  \item follows the resulting plan performing \acs{http} requests.
\end{enumerate*}


To trigger the reasoning, the agent awaits for new tasks written into the space.
Listing~\ref{lst:generic_task_subscription} shows the subscription template used by the agent.
Providing that after reasoning the agent does not find a plan to achieve a task, it will write it into the space again.
This way, another node which may know how to process it may take the task. % TODO discutir porque reasoning va a tener preferencia sobre actuación indirecta!


\begin{listing}
  \expandafter\def\csname PY@tok@err\endcsname{}
\begin{Verbatim}[commandchars=\\\{\},numbers=left,firstnumber=1,stepnumber=1]
\PY{k}{prefix }\PY{n+nv}{frap:}\PY{n+nn}{ \PYZlt{}http://purl.org/frap/\PYZgt{}}

\PY{k}{select }\PY{n+nv}{?pref }\PY{k}{where}\PY{p}{\PYZob{}}
\PY{n+nv}{	?pref}\PY{o}{ a }\PY{n+na}{frap:Preference }\PY{p}{.}
\PY{p}{\PYZcb{}}
\end{Verbatim}

  \caption{Subscription to any task written into the space.}
  \label{lst:generic_task_subscription}
\end{listing}


Demanding new data from the developer would impede the transparent reuse of the nodes from \implSpace{} and \implRest{}.
Therefore, the agent reuses all the information pieces that it needs:
\begin{itemize}
  \item It uses the \nodeConsSpace{}'s subscription to the task result as a goal for the reasoning.
	In our implementation, this correspondence needs a minimal mapping between N3QL \citeweb{n3ql2004} and SPARQL \citeweb{sparql2008}.
	The reason why we use both languages are the underlying frameworks: EYE \citeweb{euler} and RDFLib \citeweb{rdflib}.
	% TODO Justificar?
	
  \item The agent uses all the content written into the space as \emph{additional knowledge} for the reasoning process.
	This is feasible because the agent resides in the same machine as the space. % i.e. it's a local reading, does not demand any network usage
	Otherwise, acquiring this knowledge through the network would be too consuming both in bandwidth and in time.
\end{itemize}


% TODO poner aquí algunos indicadores de la implementación de escenarios?
%   e.g. cuando código extra ha hecho falta añadir en el tercero para que se hablen entre sí
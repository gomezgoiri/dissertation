\section{Discussion}
\label{sec:actuation_discussion}

The following sections scrutinize the strengths and weaknesses of the implementation presented in the previous section.
Besides, they discuss other design alternatives. % design, alternative, etc.

\subsection{Obtaining resource descriptions}

The core of the \restActuation{} are the descriptions of the resources.
They must be read by the nodes willing to actuate prior to reasoning. % obvio
This action is performed by the consumer in \implRest{} and by the agent in \implMix{}.
Both nodes crawl a given \ac{api} starting from an \ac{url} to obtain the descriptions.
The discovery of the initial \acp{url} is out of the scope of this chapter.
% Otra alternativa: An agent could discover the descriptions and simply write them in the coordination \Space{}.


Another alternative to discover these descriptions is to make them part of the \clues{} presented in Chapter~\ref{cha:searching}.
This option ensures that they will be available in any \consumer{}.
Besides, the static nature of these descriptions does not break the \clues{} stability assumption in which our architecture is founded. % explicar más?


In any case, this chapter focuses on the interoperability problem, not on how to obtain the descriptions.
Consequently, for the sake of clarity, we opted for the more intuitive and common alternative. % simpler / more intuitive / easier to understand
That is, we assume that a process crawls the descriptions in the background.



\subsection{Obtaining background knowledge}

Besides resource descriptions, \citeauthor{verborgh_ijcs_2014}'s actuation mechanism also requires an initial state and background knowledge as inputs (see Section~\ref{sec:restdesc}).
In \implRest{}, the consumer obtains this knowledge crawling all the possible \acsp{api}.
In \implMix{}, we also add all the knowledge from the space.
This is feasible because it is located in the same machine as the agent which needs it.
So, it does not demand any network usage. % no es costoso obtenerla


However, as explained in Chapter~\ref{cha:searching}, the data in \ac{ubicomp} changes too frequently to simply crawl it from time to time.
% Reading all the space would be highly inefficient.
Crawling all the \acsp{api} each time a change occurs, it is also highly inefficient.
Hence, the approach used to obtain knowledge is a clear simplification.
This simplification is justified because this chapter is centred on interoperability rather than on efficient communication mechanisms.


\bigskip


A possible optimization would be to benefit from the search architecture presented in Chapter~\ref{cha:searching}.
To reduce readings on the space (i.e., network usage), we propose a procedure composed by two reasoning steps.
In the first one, we only use local incomplete knowledge derived from the \clues{}.
Then, we read from the \Space{} just the knowledge needed to confirm the pre-proofs obtained in the first reasoning.
The second reasoning uses this knowledge to get real pre-proofs. % real o actual pre-proofs


% Y ahora explicado en mayor detalle:
Therefore, a node which wants to actuate will need to obtain the \emph{clues} from the \ac{wp}.
%These clues, as explained before, tell what kind of content other nodes provide. % un poco redundante
Let us assume that these \emph{clues} are composed by the predicates used by the nodes which provide content. % es una de las alternativas que se planteaban
The existence of a predicate used in a premise does not necessarily imply that this rule can be used.
Nevertheless, its absence does imply that it will not be used (see Figure~\ref{fig:activation_rules}).
Therefore, we can create temporary \emph{activation rules} from \clues{} which activate those potential rules. % latter rules las que se han mencionado primero


\InsertFig{activation_rules}{fig:activation_rules}{Sample clues, rules and the activation rule}{
According to the shown clues, the second rule will never be invoked.
Consequently, only an activation rule for the first one is created.
% Not explicitly attributed since the OK and Not OK icons are marked as "Public Domain":
% http://openclipart.org/detail/7983/red-+-green-ok-not-ok-icons-by-tzeeniewheenie
}{0.8}{}


An \emph{activation rule} for a rule R contains a \emph{true} in the premise.
The conclusion is made by R's premise substituting the variables with fictitious \acsp{uri} with a common prefix (see Figure~\ref{fig:activation_rules}).
These fictitious \acsp{uri} are used to distinguish when a triple should be replaced by actual knowledge from the space. % explicar mejor?



\subsection{Responsibility for triggering REST actuation}
\label{sec:responsible_proof}

Section~\ref{sec:restdesc} describes two coarse-grained steps to learn how to use web \acsp{api} which use \restdesc{}:
\begin{enumerate}
  \item Reasoning over the descriptions, background knowledge, an initial state and a goal state.
        The result of the reasoning process if a pre-proof, which can be seen as a tentative \emph{execution plan} to achieve the goal.
  \item Check the execution plan by following it.
\end{enumerate}


% no obligamos a implementar una u otra, pero recomendamos esto: XXX
\implMix{} opts for triggering the reasoning process when an agent receives a notification.
Previously, it subscribes to any type of tasks written into the space.


The reasoning can be performed in any node apart from the one which holds the \Space{}:
%However, we do not adhere to any of them.
%We leave as a future work to implement and quantitatively compare them.

\begin{itemize}
  \item Any \consumer{} interested in changing the environment can trigger the process.
	\begin{sloppypar}If these \consumers{} use the search mechanism presented in Chapter~\ref{cha:searching}, they will have background knowledge about other nodes.\end{sloppypar}
	This reduces the dependency on the node providing the \coordspace{}.
	However, it requires them to perform tasks such as reasoning and checking the pre-proofs.
	
	Unfortunately, the first task increases the computation and the second the network usage.
	As we already mentioned in previous chapters, these tasks severely affect to the energy consumption.
	Furthermore, some constrained platforms will not even be able to reason.
	
  \item To mitigate this problem, we can delegate this task only on the nodes able to perform such tasks.
	In fact, these nodes can follow the \emph{replicated-worker pattern}.
	That is, they can read from the space goals to trigger the process (i.e., \emph{reasoning tasks}).
	Apart from balancing the load between all the worker nodes, any node can stop being worker at any time by not taking more \emph{tasks} (e.g., if it has low energy).
	These nodes must be \consumers{} to use the \clues{} from the search mechanism as background knowledge.
\end{itemize}


Although both alternatives avoid the dependency on the \Space{}, the space-based actuation mechanism intrinsically depends on the \Space{}.
Therefore, it makes sense that the unavailability of the space will cause the unavailability of actuating on the space.
On the contrary, it simplifies the consumers' responsibilities, which just need to worry about writing a task into the space.



\subsection{Interoperation weakness}

The previous sections presented various alternative designs and their likely impacts on the actuation performance.
However, none of them addresses the interoperability flaws of \implMix{}.
In this regard, the simple mapping between a consumer's subscription and a goal is probably its most evident interoperability flaw.


In \implMix{}, the consumer subscription to a result and the goal for the reasoning must match.
However, there is no guarantee that the consumer will always use a subscription which matches with a goal.
For instance, the consumer could use a more general subscription and then filter the concrete tasks it is interested in.
Even worse, there is no guarantee that the consumer will subscribe to any result.
Thus, the universality of the proposed alignment can be easily affected.
To avoid this undesired effect, any developer of a consumer node should bear in mind some good practices. % to increase the interoperability chances.


% qué pasa si el consumidor del escenario 2 no se suscribe a los resultados?
% qué pasa si se suscribe a un patrón más general?
A more universal approach would be to deduce the goal from the task. % universal, generalizable
For instance, from a task of \emph{regulate temperature to 6ºC} the \Space{} could deduce the goal state of \emph{temperature of 6ºC}.
In this case, the mapping should be either
\begin{enumerate*}[label=\itshape(\arabic*\upshape)]
  \item provided by the consumer or
  \item pre-set in the space.
\end{enumerate*}
The first choice demands to feed the \Space{} with additional information. % to the middleware
%This is very inflexible and differs little from manually programming a gateway with each provider. % REALMENTE?
The second choice assumes a concrete ontology must be used or extended by the user to represent tasks.
Therefore, it would limit the freedom of choosing any vocabulary to define a task.


Since this chapter simply wants to remark the potential interoperability of the presented approaches,
we opted for selecting the automatic translation from a subscription to a goal.
The implementation of the rest of the approaches is left as future work.



\subsection{Advanced challenges} % o Advanced Limitations

The scenario used as a guiding example is very basic.
Consequently, the interoperation example shown requires further work to check its feasibility in more advanced scenarios.
We anticipate the following challenges:
\begin{itemize}
  \item Providing there are two or more paths to a goal, how can we discern which one to follow?
	This problem is specific to the \restActuation{}.
  \item How does the middleware deal with the coexistence of both mechanisms?
	When both methods can be applied, which one is triggered?
	Will one of them prevail over the second?
\end{itemize}
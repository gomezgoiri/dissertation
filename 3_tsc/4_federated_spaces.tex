\section{Federated Space}
\label{sec:halfway_solution}

% grafos autogestionados (no takeables por otros) => write en local
\subsection{Self-managed graphs}

These graphs are the ones which are managed by the device, called \emph{asteroid}, which provides them to the rest.
In other words, self-managed graphs enrich the space in the ways previously described, but they cannot be removed.
% poner ejemplo de por qué no tiene sentido que eliminen a través del espacio
Therefore, they are \emph{second-class graphs} which provide information about the environment but cannot be used for coordination purposes.

% write locally
% No device can remove them apart from their creator. % TODO o si puede serlo, esta no lo decide el sistema
The \emph{asteroid} makes these graphs accessible to others through \ac{http}.
% objetivo: que se pueda acceder a estos grafos incluso si no son parte de nuestra app => interop
The final goal is to potentially allow to reuse the data provided by any existing \ac{rest}ful service. % app level interop
Therefore, the \ac{api} should be or trend to be \ac{rest}ful.
% para ello es importante usar un RESTful approach, como sabemos que es complejo ofrecemos otra opción en el siguiente capítulo
However, we leave the \emph{hypermedia} \ac{api} as a future work.
Instead, we require a mandatory \emph{OSAPI} to be implemented in each \emph{asteroid} to guarantee access to the \emph{self-managed graphs}.
% lo sé, no está justificado 0:-)



\subsection{New primitives}

To make the most of the information in a space, we propose a new primitive to query all over the semantic information stored.
The \ac{rest}fulness of this primitive could be argued since it does not operate at resource level, but mixing several resources.
However, we believe that it is useful to have an endpoint for the queries which involve many graphs.
\citet{kjernsmo_necessity_2012} discusses this topic in depth. % TODO Mirar otras citas???

This new primitive is defined as follows:
\begin{itemize}
  \item The \textbf{query} primitive aims to see the space as a whole, returning all the triples matching the given template.
  
  \begin{lstlisting}
    query(space_URI,template): triples          [6]
  \end{lstlisting}
\end{itemize}


The Table~\ref{tab:queryAPI} extends Table~\ref{tab:tscAPI} to include this new primitive.
% TODO otra clave: debería no tener porqué pasar por el servidor

\begin{table} %http://en.wikibooks.org/wiki/LaTeX/Floats,_Figures_and_Captions#Wide_figures_in_two_column_documents
  \centering
  \caption {
    \acs{http} mapping for the primitives detailed in the Section~\ref{sec:primitives}. \textit{sp} is a space \acs{uri},
    \textit{g} is a graph \acs{uri}, \textit{s}, \textit{p} and \textit{o-uri} are subject, predicate and object \acsp{uri} or wildcards (represented with an as \textit{*}).
    When the template's object is a literal, it can be expressed specifying its value (\textit{o-val}) and its type (\textit{o-type}).
    \medskip
  }
  \begin{tabular}{c|l|c}
      \acs{http} request & \acs{url} & Returns \\
      \hline
      GET & \{sp\}/query/wildcards/\{s\}/\{p\}/\{o-uri\} &  [6] \\
      & \{sp\}/query/wildcards/\{s\}/\{p\}/\{o-type\}/\{o-val\} & \\
  \end{tabular}
  \label{tab:queryAPI}
\end{table}


A key point of the \ac{api} is that the \emph{asteroids} might not even follow the \ac{tsc} paradigm.
For instance, the \emph{OSAPI} can encapsulate data provided by a third middleware.
However, a primitive to ease that management can be a convenient for the developers which do not need a more customized behavior.
With that in mind, we propose another writing primitive.
This primitive only has local effects and therefore has no HTTP equivalent:
\begin{itemize}
  \item The \textbf{write\_self} primitive writes a \emph{self-managed graph} and returns an \ac{uri} which identifies it.
  
  % tiene sentido definir el space_URI???
  % mejor definir su propio espacio?
  % tiene sentido especificar su URI? write_self(graph_URI, triples): URI
  \begin{lstlisting}
    write_self(space_URI, triples): URI
  \end{lstlisting}
\end{itemize}



\subsection{New behaviors}

% explicar cómo se escribe y lee en el espacio
% solución enfoque híbrido:
%      cambios en actuadores => directamente a través de HTTP o indirectamente a través de tasks escritas en espacio
%                               o mejor: podrían esas tasks ser directamente esos servicios???
%      grafos que sí => write al "servidor HTTP"
%      query => en todos los dispositivos (capítulo 4) - Porque a veces es necesario a través de todos los nodos
%      take + read => sobre los grafos takeables (o inferencia con toda la info del espacio, cómo prefieras)
In this section we will try to clarify how the different behaviors coexist.

\subsubsection{Writing}

The most basic writing primitive allows a client to write a graph into the space hold by the server.
However, we also presented the \emph{write\_self} primitive.
\emph{Write\_self} writes into the local device a untakeable graph (i.e. self-managed graph).
% TODO comprobar una vez acabado que es así!
Finally, in Chapter~\ref{cha:actuate} we argue that a third \emph{writing} primitive is necessary.
This primitive should allow to \emph{suggest} physical changes in the space through actuators.


\subsubsection{Reading}

\emph{Query} performs a traversal query which aggregates all the graphs of the space.
\emph{read} and \emph{take} work at resource level.
However, we will also consider \emph{self-managed graphs} to enrich the \emph{read} primitive. % mediante redirect o lo que sea
%    propiedades aportadas por REST a IoT

\section{\acl{rest}}
\label{sec:rest}

\acf{rest} is an architectural style proposed by \citet{fielding_architectural_2000}.
% descripción de propiedades de REST: hipermedia distribuído
% que es hipertexto? http://roy.gbiv.com/untangled/2008/rest-apis-must-be-hypertext-driven#comment-718
It aims to cover certain properties, explained in Section~\ref{sec:network_properties}.
% Estilos de los que se deriva: ¿?
To achieve these properties, \ac{rest} establishes the following constraints from other network-based architectural styles:
% explicarlo o es demasiado obvio?
\begin{description}
 \item[Client-server (CS)]
 \item[Stateless (S)] the state is fully stored in the client and therefore each request has all the information needed to process it.
 \item[Cache (\$)] when added to the CS constraint, this style replicates content obtained from a server in the client.
 \item[Uniform interface (U)] is the key constraint which distinguishes \ac{rest} from other architectural styles.
                   This constraint is composed by the following ones:
    \begin{itemize}
	% explicados en sección 5.2 de Fielding, resumir?
	\item \emph{Identification of resources (ID)}.
		    Resources are the conceptual targets of hypertext references.
		    Their identification offers a generic interface to access and changes the values of a resource.
	\item \emph{Manipulation of resources through representations (REP)}
		    Representations are composed by a sequence of bytes and the metadata to describe those bytes.
	\item \emph{Self-descriptive messages (DESC)}
	% : interaction is stateless between requests,
	% standard methods and media types are used to indicate semantics and exchange
	% information, and responses explicitly indicate cacheability.
	\item \emph{Hypermedia as the engine of application state (HATEOAS)}
    \end{itemize}
 \item[Layered system (L)] each layer provides services to the top layer (e.g. TCP/IP). % TODO usar acrónimo!
 \item[Code-On-Demand (COD)] the client has a set of resources which does not know how to process.
       That \emph{know-how} is downloaded from the server.
\end{description}


% Furthermore, \citeauthor{fielding_architectural_2000} remarks the following features as \textbf{musts} a \ac{rest} solution contains\citep{fielding_rest_2008}:
% descripción de requisitos REST: untangled

% A REST API should not be dependent on any single communication protocol, though its successful mapping to a given protocol may be dependent on the availability of metadata, choice of methods, etc. In general, any protocol element that uses a URI for identification must allow any URI scheme to be used for the sake of that identification. [Failure here implies that identification is not separated from interaction.]
% A REST API should not contain any changes to the communication protocols aside from filling-out or fixing the details of underspecified bits of standard protocols, such as HTTP’s PATCH method or Link header field. Workarounds for broken implementations (such as those browsers stupid enough to believe that HTML defines HTTP’s method set) should be defined separately, or at least in appendices, with an expectation that the workaround will eventually be obsolete. [Failure here implies that the resource interfaces are object-specific, not generic.]
% A REST API should spend almost all of its descriptive effort in defining the media type(s) used for representing resources and driving application state, or in defining extended relation names and/or hypertext-enabled mark-up for existing standard media types. Any effort spent describing what methods to use on what URIs of interest should be entirely defined within the scope of the processing rules for a media type (and, in most cases, already defined by existing media types). [Failure here implies that out-of-band information is driving interaction instead of hypertext.]
% A REST API must not define fixed resource names or hierarchies (an obvious coupling of client and server). Servers must have the freedom to control their own namespace. Instead, allow servers to instruct clients on how to construct appropriate URIs, such as is done in HTML forms and URI templates, by defining those instructions within media types and link relations. [Failure here implies that clients are assuming a resource structure due to out-of band information, such as a domain-specific standard, which is the data-oriented equivalent to RPC's functional coupling].
% A REST API should never have “typed” resources that are significant to the client. Specification authors may use resource types for describing server implementation behind the interface, but those types must be irrelevant and invisible to the client. The only types that are significant to a client are the current representation’s media type and standardized relation names. [ditto]
% A REST API should be entered with no prior knowledge beyond the initial URI (bookmark) and set of standardized media types that are appropriate for the intended audience (i.e., expected to be understood by any client that might use the API). From that point on, all application state transitions must be driven by client selection of server-provided choices that are present in the received representations or implied by the user’s manipulation of those representations. The transitions may be determined (or limited by) the client’s knowledge of media types and resource communication mechanisms, both of which may be improved on-the-fly (e.g., code-on-demand). [Failure here implies that out-of-band information is driving interaction instead of hypertext.]



\section{Protocols to Follow the \ac{rest} Style}
\label{sec:protocols}

% hablar un poco de HTTP y CoAP y decir por qué no hablamos más a menudo de CoAP

Historically, \acf{http} has been considered a suitable protocol to follow the \ac{rest} style. % TODO cite
\ac{http} is a simple protocol whose adoption by computing platforms is massive.


However, in the last few years \acf{coap} has emerged as a specialized web transfer protocol for resource constrained devices. % has emerged, has arisen?
Some noteworthy features of \ac{coap} are
(1) the reduced message size,
(2) the use of UDP as a transport layer (with the possibility of using \emph{multicast} communication),
(3) similarity with \ac{http} (both to reuse its properties and to ease cross-protocol proxying),
(4) a resource discovery mechanism. % TODO cita a CORE
% mencionar seguridad?


One could defend that to design a lightweight \ac{tsc} solution \ac{coap} should be used as a baseline.
However, we have chosen to work with \ac{http} for the following reasons:
\begin{itemize}
  \item RDF-based media-formats can be rather verbosed.
	This contrasts with \ac{coap}'s message size limitations.
	Dealing with this limitations was not one main goals of the thesis.
	However, we have considered them at some points of the dissertation to avoid unrealistic assumptions.
  \item \ac{coap} is an ongoing standard.
        Therefore, its definition slightly changes in each draft version.
        A practical limitation of this is that at the moment there are few libraries
        and tools to work with \ac{coap}.
        This limits the range of platforms which could adopt any solution proposed.
  \item Due to its similarities with \ac{coap}, the future adoption of the latter should be relatively straightforward.
        % sin contar con los aspectos destacados anteriormente
\end{itemize}



\section{\acl{tsc}}
\label{sec:tsc_vs_rest}

\ac{tsc} consist of a shared space which is accessed using some known primitives.
\emph{Per se}, \ac{tsc} does not contradict in any sense the \ac{rest} principles:

\begin{description}
 \item[CS]. Providing the access to each space is centralized in a machine, clients can access to it using \emph{CS}.
           This does not prevent to use a distributed semantic repository in the backend.
 \item[S]. The primitives to access the space imply simple read and writes which do not store any state in the server.
 \item[\$] The semantic content stored in the space can be cached.
           However, the dynamism of the underlying network makes caching challenging to configure. % quería poner "tune up" pero "configure" me ha sonado más fino
 \item[U]
    \begin{itemize}
	\item \emph{ID}. There are types of identified resources in \ac{tsc}: spaces, RDF Graphs and certain elements of the RDF Triples.
	                 All of them are identified by URIs.
	                 The space can be seen as a coarse-grained view of the underlying graphs. % TODO coarse grained es lo contrario de fine-grained?
	                 The graphs have sets of triples which are usually related and describe a unit of knowledge. % A RDF triple by itself cannot transmit too much information
	                 Self-identified triple's subjects, predicates or objects are the source of concept linking.
	\item \emph{REP}. The RDF graphs and triples mentioned above can be represented using different standard serializations.
	\item \emph{DESC}. The messages derived from the primitives can be self-describing.
	\item \emph{HATEOAS}. This is the most challenging property to achieve with RDF-based representations.
	                      However, some relevant solutions propose means to fulfill this property.
    \end{itemize}
 \item[L] each layer provides services to the top layer (e.g. TCP/IP). % TODO usar acrónimo!
 \item[COD] the client has a set of resources which does not know how to process.
       That \emph{know-how} is downloaded from the server.
\end{description}



That is, considering

% acceso a fuentes de datos como si estuviesen en una BBDD, eso no debería impedir
This primitives imply writing and reading from that space.
% presentar interfaz HTTP














% presentar en qué consistirá la base de nuestra implementación

% compararlo en términos de las propiedades comentadas anteriormente
%    Cache => guay pero para datos que cambian tan a menudo es un tema complejo


% REST no dice nada, pero de facto el protocolo es HTTP
% durante el desarrollo de esta tesis en resource constrained ha surgido CoAP
%   muchas de las cosas aquí comentadas son iguales o mejor sobre CoAP
%   de todas formas, ha habido algo que ha guiado nuestro diseño: librerias existentes => afecta a plataformas
%         si, pero para eso está lo de layered approach! => motivos prácticos

% Para la comparativa final sólo

\subsection{Architectural Properties for Network-based Styles}
\label{sec:network_properties}

% Cómo afecta la app propuesta => ir actualizando al final de otras secciones???



In \citeauthor{fielding_architectural_2000}'s words, the relevant properties which describe a network-based system are the following ones \footnote{We refer to the reader to \citet{fielding_architectural_2000} for the complete thorough analysis.}:
% verificar que no se parece demasiado a sus definiciones o sino poner cursiva.
\begin{description}
  \item[Performance] is divided into network performance, user-perceived performance and efficiency.
		    The first is affected by the styles in the number of interactions and the granularity of data elements.
		    The second property refers to the impact a user in front of an application perceives.
		    The third is achieved by minimizing the use of the network.
  \item[Scalability] measures how an architecture supports a big amount of components and interactions between them.
  \item[Simplicity] is achieved through the separation of concerns for the components and the generality of architectural elements.
  \item[Modifiability] encompasses evolvability, extensibility and customization.
		      The first refers to the degree in which a component can be implemented without negatively impacting on others.
		      The second measures the ability to add functionality to the system.
		      The third is the ability to customize the behavior of an architectural element temporarily.
  \item[Configurability] is related with the extensibility and the reusability.
  \item[Reusability] is the ability to reuse components, connectors or data elements without modifying other apps.
  \item[Visibility] is the ability of a component of monitoring or mediating in the interaction between other two components.
  \item[Portability] is the ability of working in different environments.
  \item[Reliability] is the degree in which an architecture depends on the failures of the system or components, connectors or partially incorrect data.
\end{description}

\begin{center}
\begin{tabular}{lccccccccccccc}
Style &
\rotatebox{90}{Net Perform} &
\rotatebox{90}{UP Perform} &
\rotatebox{90}{Efficiency} &
\rotatebox{90}{Scalability} &
\rotatebox{90}{Simplicity} &
\rotatebox{90}{Evolvability} &
\rotatebox{90}{Extensibility} &
\rotatebox{90}{Customiz.} &
\rotatebox{90}{Configur.} &
\rotatebox{90}{Reusability} &
\rotatebox{90}{Visibility} &
\rotatebox{90}{Portability} &
\rotatebox{90}{Reliability} \\
\hline
Early web & $-$ & $\pm$ & $+$ & $+++$ & $++$ & $++$ & ~ & ~ & ~ & $+$ & $+$ & $+$ & $+$ \\
REST & $-$ & $++$ & $++$ & $++++$ & $+++\pm$ & $++$ & $+$ & ~ & $+$ & $+$ & $\pm$ & $+$ & $+$ \\ % LCODC$SS + Uniform Interface (simple, visible, reusable)
TSC & × & × & × & × & × & × & × & × & × & × & × & × & ×\\
\hline
\end{tabular}
\end{center}

% Requisitos de ealy web, propiedades deseables
%   Low entry-barrier
%   Extensibilidad
%   Distributed Hypermedia
%   Internet-scale

% una tablita de cómo hereda propiedades de los estilos anteriores?
% y si quieren más información, que miren en la tesis de fielding.



\subsection{Implementation of \ac{rest} style}
% Mención breve a HTTP y CoAP?
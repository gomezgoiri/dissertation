\section{Discussion}

In the previous sections we have get to the bottom of the strengths and weaknesses of the proposed model.
%  Ventajas/inconvenientes frente a WoT:
%      comunicación indirecta para cacharros
%          local+ref
%              de contenidos relativos a coordinación
%              podríamos cachear contenidos de otros en el espacio central (reliability+energy)
%          ref
%              sacar a colación los problemas aquellos de query sobre muchos recursos
As regard of other semantic \ac{http}-based solutions, the solution presented provides a better handling of mobility situations. % e.g. wot
The uncoupling properties which provide it come at the cost of simplicity to implement the middleware.
%   Para TSC (respecto a otras como Smart-M3):
%      limitamos qué gestiona la entidad central => sólo coordinación, el resto lo coge de los cacharros
%      se reconoce su autonomía permitiendo gestionarse su propio contenido => REST, escalabilidad
%      interop => SW y posibilidad de extender a otros servicios REST
%      multiplataforma => facilidad para portarlo
The main strength of our \ac{tsc} model is that it enriches the knowledge shared in a classical space model with content provided by autonomous embedded and mobile devices.
This opens de door to interoperate with other \ac{http}-based applications.


Within its weaknesses we found performance problems and a coupling between applications due to a no-hypermedia driven \ac{api}.
The performance problems may result of handling semantic data and are particularly important for resource constrained devices.
The coupling due to a mandatory \ac{http} \ac{api} complicates reusing third applications' data.


% Ventajas de meclar 2 cosas aparentemente separadas?
% what benefit will derive from the integration of both approaches?
The benefits of bringing together \ac{tsc} and \ac{http} is an unified and simple \ac{api} for the developer. % lo de simple no está demostrado
Due to the primitives' simplicity and to the standard and well accepted technologies used (i.e. \ac{http} and \ac{sw} standards), this \ac{api} can be ported to a range of platforms.
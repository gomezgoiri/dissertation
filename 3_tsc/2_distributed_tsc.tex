\section{Distributed \ac{tsc}}

% cómo y porqué hacer un TSC distribuído
In the previous subsection, we showed a basic alignment between \ac{tsc} and \ac{http} to stress the closeness between \ac{tsc} and \ac{rest}.
An important point of this alignment was the access to the data in client-server basis.
This implies that all the data of a space goes through a server. % para posiblemente almacenarlo en otras máquinas

% Pero un entorno Ubicomp es eminentemente distribuído
However, the data in a \ac{ubicomp} environment is not generated in a single point.
%   La información se genera en fuentes separadas, ¿cómo accedo a los últimos datos?
In fact, each sensor in an intelligent environment is potentially generating data constantly.
Besides, the magnitude of this sensors can be enormous.
This creates a trade off between efficiency and update difficult to overcome:
\begin{itemize}
  \item The more frequently each sensor sends contents to the server, the more inefficient is the solution in terms of network usage.
  \item The less frequently the communication is, the less updated is the data in the server.
        This leads to spaces which misrepresent the environments.
\end{itemize}
A sensitive solution is to let these sensor manage their own information.

%   ademas no practico:
%        las fuentes del conocimiento son las que mejor saben como gestionar una informacion (crearla, acualizarla o eliminarla)
%        la información se transporta en dispositivos
%        reusar datos en distintas apps
Furthermore, this strategy can not only be useful for devices which constantly generate new data.
Delegating responsibilities between the devices presents several other benefi y a for a number of reasons:
\begin{itemize}
  \item They will know how to represent these contents better than others. % ejemplo en grados centigrados o celsius?
	For example, the unit of a temperature measure.
  \item They know when to create, update or delete data. % ejemplo if a server receives 2 contradictory measures... algo
        In fact, they can opt for creating data on demand.
  % TODO esto es más un beneficio de REST que de otra cosa
  \item The data can be reused by other applications or even other spaces. % Interop!
        These applications would not be required to use a space-based approach.
  \item Carrying that information can be useful in certain mobility situations.
        For example, a person which stores her user profile in the smartphone.
        She can shares it with different spaces or applications as it moves along the city.
\end{itemize}


% Aún así, que los dispositivos se coordinen a través de una entidad central es conveniente:
%       Sencillo de implementar: notificaciones, razonamiento, etc.
%       Incluso si ese espacio luego está distribuído o cambia de máquina, acceso a través de dicha interfaz (HTTP o CoAP el día de mañana)
%       Modo: caching (para reliability) o modo ver qué hay en el espacio ahora mismo (visión en tiempo real)
%       -->> INDEPENDENCIA DE LOCALIZACIÓN!!!
Still, most of the \ac{iot} devices or mobile phones are unreliable: they can join and leave at any moment.
Therefore, distributing the shared space among unreliable nodes comes with a number of drawbacks:
\begin{itemize}
  \item This unreliability makes difficult to maintain the consistency of the space. % TODO CUIDADO, puedo estar metiendome en un fregado, la parte de los dispositivos no será consistente!
        Since this information drives the coordination between the devices, its consistency its important.
  \item Blocking mechanism is important in space-based computing.
        For example, a worker node may block until some task is become available in the space.
        A way to implement it in a distributed space is by means of a notification system.
        Again, implementing it using unreliable devices is hard to implement. % no sé si esto último no sonará demasiado a mofa...
  \item And last but not least, if the access to some data in the space cannot be guaranteed no matter if the device which wrote that information is available, there will not be \emph{time uncoupling}.
\end{itemize}
Overcoming the previous difficulties, usually implies a high network traffic.
However, these negatively influences the energy consumption of some nodes that whose energetic autonomy is limited.


\subsection{A Halfway solution}

% solución enfoque híbrido:
%      cambios en actuadores => directamente a través de HTTP o indirectamente a través de tasks escritas en espacio
%                               o mejor: podrían esas tasks ser directamente esos servicios???
%      grafos que sí => write al "servidor HTTP"
%      query => en todos los dispositivos (capítulo 4) - Porque a veces es necesario a través de todos los nodos
%      take + read => sobre los grafos takeables (o inferencia con toda la info del espacio, cómo prefieras)
In this thesis we propose to conserve the desirable characteristics of both \ac{rest} and \ac{tsc} by mixing them.

On the one hand, we propose a centralized access to the space by the \acs{http} \ac{api} described previously.
The space can then be distributed using any existing approach. % TODO cite
However, for the sake of clarity, we will assume that each space is managed by a unique server.

On the other hand, each space will be enriched by the data provided by various autonomous providers.
These providers might be any kind of mobile or embedded device, no matter how many constraints they have.
This \emph{enrichment} can be seen from different perspectives:
\begin{enumerate}[(a)]
  \item The consideration of that data.
	For example, to include it in a reasoning process triggered when a primitive is called.
  \item The inclusion of that data.
        For example, a graph read from an embedded device could be returned as a response to a primitive.
\end{enumerate}


To do that, we propose an extension of the already presented \ac{tsc} model by means of new concepts, new primitives and new behavior.


%      grafos autogestionados (no takeables por otros) => write en local
\subsubsection{Self-managed graphs}

These graphs are the ones which are managed by the device which provides them to the rest.
No device can remove them apart from their creator.
% write locally
This creator makes them accessible using an \acs{http} \ac{api}.
This \ac{api} should be or trend to be a \ac{rest}ful.

These graphs enrich the space and can be accessed through the space, but they cannot be removed.
Therefore, they are \emph{second-class graphs} which provide information about the environment but cannot be used for coordination purposes.

% objetivo: que se pueda acceder a estos grafos incluso si no son parte de nuestra app => interop
% para ello es importante usar un RESTful approach, como sabemos que es complejo ofrecemos otra opción en el siguiente capítulo


\subsubsection{New primitives}

% query

\subsubsection{New behaviors}

% explicar cómo se escribe y lee en el espacio


% presentar queries




% Ventajas de meclar 2 cosas aparentemente separadas?
%   Para WoT:
%      comunicación indirecta para cacharros
%          local+ref
%              de contenidos relativos a coordinación
%              podríamos cachear contenidos de otros en el espacio central (reliability+energy)
%          ref
%              sacar a colación los problemas aquellos de query sobre muchos recursos
%
%   Para TSC (respecto a otras como Smart-M3):
%      limitamos qué gestiona la entidad central => sólo coordinación, el resto lo coge de los cacharros
%      se reconoce su autonomía permitiendo gestionarse su propio contenido => REST, escalabilidad
%      interop => SW y posibilidad de extender a otros servicios REST
%      multiplataforma => facilidad para portarlo
%
%   Para desarrollador:
%      API integrada y simple

By doing so we retain the desirable properties of both approaches:



% presentar las 2 partes
% condiciones relajadas: consistencia, razonamiento distribuído

















The \acs{http} mapping is done among a \textit{client} and a \textit{server}.
However, as previously detailed, \ac{tsc} provides reference autonomy, so a consumer will query the space without knowing the particular addresses of the nodes composing that space.
This autonomy is managed at other upper and optional layers explained in the following section.
However, this \acs{http} mapping is all a data provider needs to serve information in the space.
% Quitado porque ni yo mismo lo entendía ahora...
%, and it is also what a data consumer needs to interrogate a multicast provider (a component provided by Otsopack to contact multiple nodes in a particular space).


\footnote{
  Note that the write primitive is explicitly excluded from the table.
  This primitive could be directly mapped to a \acs{http} POST,
  but we defend that each node should manage its own information
  (see Section~\ref{sec:knowledge_distribution}).
  Therefore, the \acs{http} \acs{api} does not allow remote writing.
}
% cómo y porqué hacer un TSC distribuído
% condiciones de REST que no cumpliría


% condiciones que hemos relajado de TSC (i.e. limitaciones)
%     1. autonomía espacial (dependes de un nodo)
%     2. consistencia de datos introducidos en el espacio
%     3. cómo asegurar un take?
%     4. subscripciones
%     + Hemos juntado como 2 mundos: TSC y WoT
%        La mayoría de las contribuciones son aplicables a WoT
%        El mecanismo de búsqueda no es Hypertext driven! (no?)
%        Por que sería muy costoso hacer rollo araña?!
%     5. Future work:
%          definir datos como consumibles por otros nodos o no
%          eliminar la necesidad de espacios?         
%     por qué creemos que sigue siendo útil?


% presentar en qué consistirá la base de nuestra implementación

% condiciones relajadas: consistencia, razonamiento distribuído


% SECCIÓN: CUÁL ES EL DISEÑO GENERAL QUE SE PODRÍA UTILIZAR DESDE FUERA Y QUE HA MOTIVADO NUESTRA IMPLEMENTACIÓN
% ALGO QUE DIGA QUE NO DEPENDE DE OTSOPACK
% TODO: other name? ``defined layers'' o así? cómo decir ``mira, lo importante es que puede haber diferentes grados de
% implementación aquí gracias al diseño que mostramos en esta sección en 5 palabras???

% Lo que quiero (yo Pablo) decir aquí es que sigue siendo distribuído aunque parezca que es uno a uno.


\subsection{Distributed \ac{tsc}}

% ahora vamos a empezar a joder el tema de TSC suponiendo que lo distribuimos a través de HTTP
% y lo justificamos xD

The \acs{http} mapping is done among a \textit{client} and a \textit{server}.
However, as previously detailed, \ac{tsc} provides reference autonomy, so a consumer will query the space without knowing the particular addresses of the nodes composing that space.
This autonomy is managed at other upper and optional layers explained in the following section.
However, this \acs{http} mapping is all a data provider needs to serve information in the space.
% Quitado porque ni yo mismo lo entendía ahora...
%, and it is also what a data consumer needs to interrogate a multicast provider (a component provided by Otsopack to contact multiple nodes in a particular space).


\footnote{
  Note that the write primitive is explicitly excluded from the table.
  This primitive could be directly mapped to a \acs{http} POST,
  but we defend that each node should manage its own information
  (see Section~\ref{sec:knowledge_distribution}).
  Therefore, the \acs{http} \acs{api} does not allow remote writing.
}
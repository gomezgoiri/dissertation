\section{Conclusion}
% 2. Comparativa
\label{sec:tsc_wot_conclusion}

% 5. otra forma de aunar las ventajas de ambos? => construir un TSC compatible con WoT (on top of that) <= integración profunda
%	Ventajas:
%		+ interoperable with other semantic WoT systems
%		+ syntactic services can still be provided using a different type of representation => WoT
%		+ promote indirect communication style => abstraction for the developer (does not care about manually discovering interfaces)
%	Problemas:
%		+ necesitas out-of-the-band conocimiento? API fija?
% esto trae problemas: cap 4 y cap 5


This chapter compares two different resource oriented approaches for the Internet of Things: Web of Things and Triple Space.
WoT seems to scale in a better way thanks to the underlying HTTP protocol, while TS performs the discovery process among locally available network-connected objects in a seamless way.
The first one is more human oriented and the second one relies on the Semantic Web capabilities to exchange a machine processable data.
Furthermore, the second one is ideal to easily configure intranets of network connected objects whilst the first one can easily bridge those intranets configuring global multi-site IoT ecosystems.

We deem that both approaches can win much from their combination since the weaknesses of one are outweighed by the strengths of the other.
Hence, they can be combined to offer a more scalable, machine and human processable solution that offers better cooperation possibilities among internet-connected objects and thus aid users in their daily activities.
As a simple proof of this hypothesis, a scenario employing WoT to export each space data to the outer world and TS to enable seamless and automatic configuration of heterogeneous devices on local networks has been presented.
\section{Guiding Principles} % o problems
\label{sec:guiding_principles}

% cómo y porqué hacer un TSC distribuído
In Section~\ref{sec:tsc_vs_rest} section we showed how \ac{tsc} can comply with all the constraints of the \ac{rest} style.
One of these constraints is that the access to a space is done in client-server basis.
The easiest way to achieve this constraint is by centralizing all the content of each space in the server itself.


% Pero un entorno Ubicomp es eminentemente distribuído
However, the data in a \ac{ubicomp} environment is not generated in a single point.
%   La información se genera en fuentes separadas, ¿cómo accedo a los últimos datos?
In fact, each sensor in an intelligent environment is potentially generating data constantly.
Besides, the magnitude of this sensors can be enormous.
This creates a trade off between efficiency and update difficult to overcome:
\begin{itemize}
  \item The more frequently each sensor sends contents to the server, the more inefficient is the solution in terms of network usage.
  \item The less frequent this communication is, the less updated is the data in the server.
        This leads to spaces which misrepresent the environments.
\end{itemize}
A sensitive solution is to let these sensor manage their own information.


%   ademas no practico:
%        las fuentes del conocimiento son las que mejor saben como gestionar una informacion (crearla, acualizarla o eliminarla)
%        la información se transporta en dispositivos
%        reusar datos en distintas apps
Furthermore, this strategy can not only be useful for devices which constantly generate new data.
Delegating responsibilities between the devices presents additional benefits:
\begin{itemize}
  \item They will know how to represent these contents better than others. % ejemplo en grados centigrados o celsius?
	For example, the unit of a temperature measure.
  \item They know when to create, update or delete data. % ejemplo if a server receives 2 contradictory measures... algo
        In fact, they can opt for creating data on demand.
  % TODO esto es más un beneficio de REST que de otra cosa
  \item The data can be reused by other applications or even other spaces. % Interop!
        These applications would not be required to use a space-based approach.
        Therefore, they will not depend on the correct functioning of our whole system. % del servidor central
  \item Carrying that information can be useful in certain mobility situations.
        For example, let a us imagine a person which stores her user profile in the smartphone.
        She could share it with different spaces or applications as it moves along the city.
\end{itemize}


% Aún así, que los dispositivos se coordinen a través de una entidad central es conveniente:
%       Sencillo de implementar: notificaciones, razonamiento, etc.
%       Incluso si ese espacio luego está distribuído o cambia de máquina, acceso a través de dicha interfaz (HTTP o CoAP el día de mañana)
%       Modo: caching (para reliability) o modo ver qué hay en el espacio ahora mismo (visión en tiempo real)
%       -->> INDEPENDENCIA DE LOCALIZACIÓN!!!
Still, most of the \ac{iot} devices or mobile phones are unreliable: they can join and leave at any moment.
Therefore, distributing the shared space among unreliable nodes comes with a number of drawbacks:
\begin{itemize}
  \item This unreliability makes difficult to maintain the consistency of the space. % TODO CUIDADO, puedo estar metiendome en un fregado, la parte de los dispositivos no será consistente!
        Since this information drives the coordination between the devices, its consistency is important.
  \item A blocking mechanism is important in space-based computing.
        For example, a worker node may block until some task is become available in the space.
        A way to implement it in a distributed space is by means of a notification system.
        Again, implementing it using unreliable devices is hard to implement. % no sé si esto último no sonará demasiado a mofa...
  \item And last but not least, if the access to some data in the space cannot be guaranteed no matter if the device which wrote that information is available, there will not be \emph{time uncoupling}.
\end{itemize}
Overcoming the previous difficulties, usually implies a high network traffic.
However, this negatively influences the energy consumption of nodes whose energetic autonomy might be limited.
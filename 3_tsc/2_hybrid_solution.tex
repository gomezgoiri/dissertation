\section{An Hybrid Solution} % o "A Halfway Solution"
\label{sec:hybrid_solution}

We can distinguish two different usage patterns of an space:
%  escribir y extraer para coordinarse con otros
(1) coordinating with other devices by writing and extracting content ; and % MISMA EXPLICACIÓN QUE EN 2_1
%  consultar que pasa en un entorno
(2) checking what happens in the environment by reading.
So far, other semantic space-based works have always tried to integrate both strategies within the same space.
However, the needs of both usages are different.
The first one demands availability and consistency on the data.
The second does not. % se ha visto en el anterior
Therefore, we propose a novel approach: to treat them independently.
% To conserve the desirable uncoupling properties of a space


Limited devices are not reliable enough to manage data for the first usage.
Nevertheless, they can contribute to an enriched view of the space.
According to this limitations, we propose the following dual model:
(1) the content needed to coordinate devices will be managed by independent machines; and
(2) reading in the space will consider not only the previous content, but also and the knowledge provided by autonomous providers.
Summarizing, we propose to enrich the \ac{ts} view with autonomous federated subspaces.


Figure~\ref{fig:new_model} presents the key elements of this new model.
The coordination space, is where the graphs can be written, read and taken by any participant.
The coordination space is hold by a device called \emph{coordinator}.
The current view of all the \emph{self-managed graphs} in the space forms the \emph{outer space}.
Therefore, the \emph{outer space} is hold by many devices (also called \emph{asteroids}).
The \ac{tsc} \ac{http} \ac{api} corresponds with the generic \ac{tsc} to access to the primitives presented in Section~\ref{sec:align_tsc_http}.
The \emph{OSAPI} is the \ac{api} which must be implemented by any node willing to share \emph{self-managed graphs}.


% TODO ver si reaprovechar esto!
%selecting the underlying network-based architectural style.
%At the networking level, this design is inspired by the \ac{rest} style.
%The \ac{wot} initiative endorses the appropriateness of its application to resource constrained devices.

\InsertFig{new_model}{fig:new_model}{
  Key concepts of the new \ac{tsc} model presented.
}{}{1}{}


For the coordination space, we propose a uniform access to the space through an \acs{http} \ac{api} described in Section~\ref{sec:align_tsc_http}.
This \ac{api} can provide data from a distributed space like some approaches in Section~\ref{sec:soa_tsc_distribution} do.
However, this dissertation does not cope with this problem. % TODO cite
Hence, the reader can assume that each space is managed by a unique server for the sake of clarity.


The \emph{coordination space} will be enriched with data from the \emph{outer space}. % enrichment or extension?
This demands an extension of the classical \ac{tsc} model presented in Section~\ref{sec:align_tsc_http}.
Section~\ref{halfway_solution} presents this extension's new concepts, primitives and behavior.
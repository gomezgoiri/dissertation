\section{An Hybrid Solution} % o "A Halfway Solution"
\label{sec:hybrid_solution}

In this thesis, we propose to conserve the desirable characteristics of both \ac{rest} and \ac{tsc} by mixing them.

On the one hand, we propose a uniform access to the space by the \acs{http} \ac{api} described previously.
The space can then be distributed using any existing approaches. % TODO cite
However, for the sake of clarity, we will assume that each space is managed by a unique server.

On the other hand, each space will be enriched by the data provided by various autonomous providers.
These providers might be any kind of mobile or embedded device, no matter how limited they are.
This \emph{enrichment} can be materialized:
\begin{enumerate}[(a)]
  \item Considering additional data in a reasoning process triggered when a primitive is called.
  \item Including additional data as a response.
        For example, a graph read from an embedded device could be returned as a response to a primitive.
\end{enumerate}
In this thesis, we have focused on the latter alternative.


To do that, we propose an extension of the already presented \ac{tsc} model by means of new concepts, new primitives and new behavior.
Figure~\ref{fig:new_model} presents the key elements of this new model.


\InsertFig{new_model}{fig:new_model}{
  Key concepts of the new \ac{tsc} model presented.
}{
  The coordination space, is where the graphs can be written, read and taken by any participant.
  The coordination space is hold by a device called \emph{coordinator}.
  The current view of all the \emph{self-managed graphs} in the space forms the \emph{outer space}.
  Therefore, the \emph{outer space} is hold by many devices (also called \emph{asteroids}).
  The \ac{tsc} \ac{http} \ac{api} corresponds with the generic \ac{tsc} to access to the primitives presented in Section~\ref{sec:align_tsc_http}.
  The \emph{OSAPI} is the \ac{api} which must be implemented by any node willing to share \emph{self-managed graphs}.
}{1}{}
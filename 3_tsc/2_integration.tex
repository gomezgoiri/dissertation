\section{First steps towards the integration of the \acs{wot} and \acs{tsc}}
% 2. Comparativa
\label{sec:integration_wot_tsc}

% TODO rehacer taaanto :-)


% existe gente que ha dicho que \ac{tsc} encaja con el pto de vista REST


% remarcar que en los primeros intentos se intentó esto, y luego lo otro

\subsection{Using \acs{wot} in a \acs{tsc} solution}
\label{sec:wotints}
To demonstrate the complete compatibility between \ac{tsc} and \ac{wot} approaches, we first used a \ac{wot} solution in a \ac{tsc} node. We used a gateway \citep{guinard_resource_2010} which provides access to sensors and actuators through RESTful services. To adapt it to the \ac{tsc} paradigm, we added to it a new data representation using a set of semantic triples (in this first approach, the solution is dependent on the ontology of the scenario).

In OtsoPack, each node is mainly made up of two parts: the network layer and the data access layer. While the first one has the responsibility of
keep a node communicated with the rest of the nodes of the space, the second one stores the triples managed by this node.
In OtsoSE we have replaced this data access layer to obtain the semantic information from the gateway instead of from a semantic repository.
Doing so, the \ac{tsc} primitives addressed to the gateway are translated into an HTTP request as summarized in Table \ref{tab:TS2WoT}.
\begin{table*}[t!] % http://en.wikibooks.org/wiki/LaTeX/Floats,_Figures_and_Captions#Wide_figures_in_two_column_documents
\centering
\caption {Mappings between OtsoPack's primitives and HTTP requests addressed to a \ac{wot} solution.}
\begin{tabular}{|c|p{10cm}|}
\hline
\acs{tsc} primitive & HTTP request \\
\hline \hline
read(spaceURI,[graphURI]) & HTTP GET over [graphURI] \\
\hline
read(spaceURI,[template]) & HTTP GET over http://gateway/read?template={''[template]``} \\
\hline
query(spaceURI,[template]) & HTTP GET over http://gateway/query?template={''[template]``}\\
\hline
write(spaceURI,[template]) & HTTP PUT over http://gateway/sunspots/[SpotName]/leds/led[0-6]

Parameters:
\begin{itemize}
  \item switch=[true/false]
  \item redColor=[0-255]
  \item blueColor=[0-255]
  \item greenColor=[0-255]
\end{itemize}

\\
\hline
\end{tabular}
\label{tab:TS2WoT}
\end{table*}

The Read primitive has two different types of implementations. In the most basic one, the graph is identified by a URI which in this case
coincides with the URL of the service that returns it (for example, in the case of temperature sensor
http://node/sunspots/\textit{SpotName}/sensors/temperature/). In the second implementation, the template is passed in as a query string parameter for
the GET command issued to a specific URL and the gateway checks all the graphs to facilitate a response. The Query operation has a similar behaviour.
The Write parses the contents of the triples extracting values and makes a POST request with them over a particular actuation service URL to change
its state.


% CONCLUSION: uso de gateways (a veces útil, pero no nuestro objetivo)


% mapping
% decir que ya hay muchos que intentaron esto antes: TSC?
\subsection{Making \acs{tsc} nodes part of the \acs{wot}}

\begin{table*}[t!] % http://en.wikibooks.org/wiki/LaTeX/Floats,_Figures_and_Captions#Wide_figures_in_two_column_documents
\centering
\caption {Examples of REST access to \ac{tsc} (\textit{sp:ex} is a space URI, \textit{sp:gr1} is a graph URI and templates are expressed between quotes)}
\begin{tabular}{|c|l|l|}
\hline
HTTP request & URL & Returns \\
\hline \hline
GET & http://nodeuri/prefixes & The list of prefixes used by the node \\
GET & http://nodeuri/prefixes/sp & The URI that ``sp'' prefix represents \\
POST & http://nodeuri/prefixes & Add a new prefix \\
 & \hspace{0.5cm}(parameters: \textit{URI} \& \textit{prefix name}) & \\
GET & http://nodeuri/sp:ex/query/any & \texttt{query(sp:ex,"?s ?p ?o ."): triples} \\
POST & http://nodeuri/sp:ex/graphs & \texttt{write(sp:ex,triples): URI} \\
 & \hspace{0.5cm}(parameter: \textit{triples}) & \\
GET & http://nodeuri/sp:ex/graphs & The list of graphs stored in this node \\
GET & http://nodeuri/sp:ex/graphs/sp:gr1 & \texttt{read(sp:ex,sp:gr1): triples} \\
GET & http://nodeuri/sp:ex/graphs/subject/sp:s1 & \texttt{read(sp:ex,"<sp:s1> ?p ?o ."): triples} \\
DELETE & http://nodeuri/sp:ex/graphs/sp:gr1 & \texttt{take(sp:ex,sp:gr1): triples} \\
DELETE & http://nodeuri/sp:ex/graphs/object/sp:o1 & \texttt{take(sp:ex,"?s ?p <sp:o1> ."): triples} \\
%GET & http://nodeuri/spaceuri/ontologies & The list of base ontologies used by the node \\
%GET & http://nodeuri/spaceuri/ontologies/rdfs & The triples which define RDF-Schema. \\
\hline
\end{tabular}
\label{tab:WoT2TS}
\end{table*}

In this section, a proposal to make any \ac{tsc} node \ac{wot} compliant is explained. To do that access to a \ac{tsc} should be provided through a RESTful service
(as shown in Table \ref{tab:WoT2TS}). As in \ac{tsc}, spaces, graphs, subjects, predicates and objects are identified by URIs, in order not to make the
requesting URLs too long, each node should provide a prefix mechanism to enable the URI shortening at \textit{http://nodeuri/prefixes/}.
This node will return all the prefixes used by this node, so they can be used inside any URL by simply using a name followed by '':`` and the last part of the URI.

To see the graphs available in a concrete node, \linebreak \textit{http://nodeuri/spaceuri/[graphs]} could be accessed. To access each graph
in a space, no matter if it is stored by the node responding to the HTTP request or not, we could access \textit{http://nodeuri/spaceuri/graphs/[graphuri]}.
It will be internally translated into a \primread primitive. To locate a graph giving a \textit{template} the accessed URL will be \linebreak
\textit{http://nodeuri/spaceuri/graphs/[template]}. The HTTP \linebreak DELETE verb should return the graph and delete it from the node where it was stored.

We propose specifying first a subject \textit{subject/[subj-uri]/} and concatenate \textit{/predicate/[pred-uri]/} and \textit{object/[obj-val]}\linebreak if needed to specify a \textit{template}. The order should be this, but any of them could be optional to express that any value could be ok (wildcard option). To express a \textit{?s ?p ?o .}-like template (any triple matches it), the URL ended by \textit{any} could be used.

To perform a \primquery \textit{http://node/spaceuri/query/[template]}-like URL should be accessed and to write a new graph the user should make a HTTP POST request to \linebreak \textit{http://nodeuri/spaceuri/graphs/} obtaining the new graph's URI as response.
%Additionally, some shortcuts could be proposed to explore the instance URIs of a given class (i.e. \linebreak 
%\textit{http://nodeuri/spaceuri/[classname]} would redirect to \linebreak \textit{http://nodeuri/spaceuri/query/p/rdf:type/o/[classname]}
%\footnote{For the sake of space ``predicate'' and ``object'' are represented by ``p'' and ``o''.}) \linebreak and an special URL to access to the base
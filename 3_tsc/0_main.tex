% ----------------------------------------------------------------------

\begin{savequote}[50mm]
The original idea of the web was that it should be a collaborative space where you can communicate through sharing information.
\qauthor{Tim Berners-Lee}
\end{savequote}


\newcommand{\codigo}[1]{``\texttt{#1}''}
\newcommand{\primquery}{\emph{query}}
\newcommand{\primread}{\emph{read}}
\newcommand{\primtake}{\emph{take}}
\newcommand{\primwrite}{\emph{write}}


\chapter{Triple Spaces for Limited Devices}
\label{cha:tsc}
\newcommand{\pathchapthree}{3_tsc}

% the code below specifies where the figures are stored
\ifpdf
    \graphicspath{{\pathchapthree/figures/PNG/}{\pathchapthree/figures/PDF/}{\pathchapthree/figures/}}
\else
    \graphicspath{{\pathchapthree/figures/EPS/}{\pathchapthree/figures/}}
\fi



% ----------------------------------------------------------------------



% TODO TODO TODO me he dado cuenta que si quiero interoperar, la clave sería meter en algún lado algo sobre LOD y SPARQL Update 1.1!!!
% Igual que quiero interoperar con Semántic WoT, debería interoperar con LOD !!!



In recent years, the \acf{iot} has become a reality due to the increasing number of everyday objects equipped with computing and networking capabilities.
The use of these objects together with the rising of mobile computing greatly contributes to the \ac{ubicomp} vision.
%In the beginning the community put more effort on those devices' connectivity issues, materialized in the spread of technologies such as Zigbee\footnote{http://www.zigbee.org}, 6LoWPAN\footnote{http://datatracker.ietf.org/wg/6lowpan/charter/} or Bluetooth\footnote{http://www.bluetooth.com}.
%However, nowadays it is focusing on the architectural solutions used by the applications built around these objects.
%A classification of these solutions has been already presented in the previous chapter. % Coulouris et al.
In this thesis, we defend that \acf{tsc} provides some valuable properties to \ac{ubicomp} environments. % mencionarlas?
However, some other properties are derived from how \ac{tsc} is implemented. % TODO implemented suena muy umpa-lumpa? mejor designed o así? o hablar de model mejor que de middleware?


% TODO problema: y por qué no otros? Habría que redactarlo de tal forma que no haga que un lector tiquismiquis requiera un puto survey de network-based architectural styles!
In this chapter, we focus on the design of the space model for our solution.
%selecting the underlying network-based architectural style.
Its application to resource constrained devices brought the successful \ac{wot} initiative.
\ac{wot} is probably the most prominent architectural solution used for smart-objects. % probably porque no tengo datos que lo defiendan
%The benefits that \ac{rest} brings to resource constrained devices have been deeply described in the previous chapter.


To that end, Section~\ref{sec:align_tsc_http} analyzes the characteristics of the \ac{rest} architectural style and the similarities it shares with \ac{tsc}.
Later, it describes a \ac{rest}-like \ac{api} to access a triplespace. % TODO ver cómo escribir triplespace
The goal pursued is two-fold: (1) inherit valuable the properties from \ac{rest} and (2) maintain a high degree of compatibility with the \ac{wot}.
Section~\ref{sec:distributed_tsc} presents the need and requirements of a distributed architecture for \ac{ubicomp}.
These requirements lead us to adapt the \ac{tsc} model.
Finally, Section~\ref{sec:middleware_qualitative_evaluation} evaluates the properties achieved with this model.


\section{Guiding Principles} % o problems
\label{sec:guiding_principles}

% cómo y porqué hacer un TSC distribuído
In Section~\ref{sec:soa_tsc_discussion} we showed how \ac{tsc} can comply with all the constraints of the \ac{rest} style.
One of these constraints is that the access to a space is done in a client-server basis.
The easiest way to achieve this constraint is by centralizing all the content of each space in the server itself.


% Pero un entorno Ubicomp es eminentemente distribuído
However, data in a \ac{ubicomp} environment is not generated at a single point. %  intelligent environment
%   La información se genera en fuentes separadas, ¿cómo accedo a los últimos datos?
In fact, the plethora of sensors where data are generated can be enormous. % are => data es plural
Besides, each of these sensors may generate data continuously.
This creates a trade off between efficiency and freshness difficult to overcome:
\begin{itemize}
  \item The more frequently each sensor sends contents to the server, the more inefficient is the solution in terms of network usage.
  \item The less frequent this communication is, the less updated is the data in the server.
        This leads to spaces which misrepresent the environments.
\end{itemize}



%   ademas no practico:
%        las fuentes del conocimiento son las que mejor saben como gestionar una informacion (crearla, acualizarla o eliminarla)
%        la información se transporta en dispositivos
%        reusar datos en distintas apps
Consequently, a sensible solution is to let these sensors manage their own information.
This strategy is not only useful to ensure the access to up-to-date data, but also for the following causes: % beneficios o razones?
% This strategy == delegating responsibilities between the devices
\begin{itemize}
  \item The devices directly connected to the sensors and actuators will know how to represent these contents better than others. % ejemplo en grados centigrados o celsius?
	For example, the unit of a temperature measure.
  \item They know when to create, update or delete data. % ejemplo if a server receives 2 contradictory measures... algo
        Furthermore, they can opt for creating data on demand.
  \item Data may be reused by other spaces or even other applications. % Interop!
        These applications would not be required to use a space-based approach.
        Therefore, they will not depend on the correct functioning of our whole system. % del servidor central
  \item Carrying the information can be useful in certain mobility situations.
        For example, let us imagine a person which stores her user profile in her personal smartphone.
        She could share it with different spaces or applications as it moves along the city.
\end{itemize}


% Aún así, que los dispositivos se coordinen a través de una entidad central es conveniente.
%       Modo: caching (para reliability) o modo ver qué hay en el espacio ahora mismo (visión en tiempo real)
%       -->> INDEPENDENCIA DE LOCALIZACIÓN!!!
Still, most of the \ac{iot} devices or mobile phones are unreliable: they can join and leave at any moment.
As a result, distributing the shared space among unreliable nodes comes with a number of drawbacks:
\begin{itemize}
  \item Devices rely on the data written and read from the space to coordinate themselves.
	Therefore, the access to these pieces of information must be guaranteed regardless of dynamic devices' availability. % dynamic == unreliable
        % i.e. que no desaparezca una tarea en el espacio
  % Sencillo de implementar: notificaciones. En temas de razonamiento no he entrado.
  \item A blocking mechanism is important in space-based computing. 
        For example, a worker node may block until some tasks become available in the space.
        A way to implement it in a distributed space is by means of a notification system.
        However, the efficient implementation of this system using unreliable devices is challenging. % no sé si esto último no sonará demasiado a mofa...
\end{itemize}
Overcoming the previous difficulties, usually implies a high network traffic.
This traffic negatively influences the energy consumption of nodes whose energy autonomy might be limited.

\bigskip

In conclusion, our design must face two apparently contradictory principles:
\begin{itemize}
  \item To consider data from independent and limited data providers.
  \item To rely on the providers of the data which enables the coordination of the space participants.
\end{itemize}
\section{An Hybrid Solution} % o "A Halfway Solution"
\label{sec:hybrid_solution}

In this thesis, we propose to conserve the desirable characteristics of both \ac{rest} and \ac{tsc} by mixing them.

On the one hand, we propose a uniform access to the space by the \acs{http} \ac{api} described previously.
The space can then be distributed using any existing approaches. % TODO cite
However, for the sake of clarity, we will assume that each space is managed by a unique server.

On the other hand, each space will be enriched by the data provided by various autonomous providers.
These providers might be any kind of mobile or embedded device, no matter how limited they are.
This \emph{enrichment} can be materialized:
\begin{enumerate}[(a)]
  \item Considering additional data in a reasoning process triggered when a primitive is called.
  \item Including additional data as a response.
        For example, a graph read from an embedded device could be returned as a response to a primitive.
\end{enumerate}
In this thesis, we have focused on the latter alternative.


To do that, we propose an extension of the already presented \ac{tsc} model by means of new concepts, new primitives and new behavior.
Figure~\ref{fig:new_model} presents the key elements of this new model.


\InsertFig{new_model}{fig:new_model}{
  Key concepts of the new \ac{tsc} model presented.
}{
  The coordination space, is where the graphs can be written, read and taken by any participant.
  The coordination space is hold by a device called \emph{coordinator}.
  The current view of all the \emph{self-managed graphs} in the space forms the \emph{outer space}.
  Therefore, the \emph{outer space} is hold by many devices (also called \emph{asteroids}).
  The \ac{tsc} \ac{http} \ac{api} corresponds with the generic \ac{tsc} to access to the primitives presented in Section~\ref{sec:align_tsc_http}.
  The \emph{OSAPI} is the \ac{api} which must be implemented by any node willing to share \emph{self-managed graphs}.
}{1}{} % o halfway solution?
\section{Aligning \acs{tsc} with \acs{http}}
\label{sec:align_tsc_http}

% Otros trabajos citables parecidos a esto:
%      posiblemente el primero que expuso explicitamente semejanzas entre HTTP y primitivas de TSC: Riemer 2006 (comprobar y citar!)
%      Otros papers de TSC y la WWW, que ya han sido citados en la sección 3_tsc

This section presents our materialization of a \ac{tsc} \acs{api} over \ac{http}.
This materialization gives a practical overview of \ac{tsc}'s \acs{rest}fulness.


\subsection{\acs{tsc} resources}

Three key concepts are important to understand the resources in our proposal: agents share information in a common \textbf{space}.
A space is identified by an \acs{uri}.
Therefore, all the operations in \ac{tsc} are performed against a particular space.
%By default, all applications connect to a common standard space, but they can optionally choose to connect to a particular private space.
Within a space, the information is stored in sets of \textbf{triples} called \textbf{graphs}.
Each graph can also be identified by an \acs{uri}.
% The \acs{rdf} \textbf{triples} are the underlying concept of all the \ac{sw} languages. % a estas alturas esto está explicadísimo
Each triple is composed by a subject (which is a \acs{uri} or a \emph{blank node}), a predicate (a \acs{uri}) and an object (which can be a \acs{uri}, a \emph{blank node} or a literal), as shown in the Figure~\ref{fig:triples_example}.
% TODO cuidado con los blank nodes!!! Los tenemos en cuenta? De ser así, cómo???

As detailed later, \ac{tsc}'s primitives add or remove graphs.
% este es un tema "delicado" desde el punto de vista de TSC, así que mejor si lo tratamos a parte:
% as well as to query for graphs or for sets of triples retrieved from different graphs.
To perform these operations, which enable the selection of a subset of the semantic content hold in a given space, a \textbf{template} is required. % these operations: add or remove
The wildcard templates used are special triples with optional wildcard subject, predicate and/or object.
For example, the template \texttt{?s foaf:knows gomezgoiri:aitor} could be employed to select instances which represent people who know Aitor (see Figure~\ref{fig:tsc_resources}).

% TODO Discussión profunda de SPARQL y REST???

% "Cartoon Cloud" by egyninja is marked as "public domain": http://openclipart.org/detail/25319/cartoon-cloud-by-egyninja
\InsertFig{tsc_resources}{fig:tsc_resources}{Schematic view of a space with four graphs and sample triples for one of these graphs}{
The figure uses prefixes, i.e., aliases for the beginning of the \acp{uri}, for the sake of clarity.
}{1}{}


Note that this thesis does not consider using more sophisticated query languages like SPARQL \citeweb{sparql2008}. % sophisticated o expressive?
The rationale behind this decision is that we wanted to avoid the complexity introduced by these languages in our \ac{api}.
While the advanced query languages need to be interpreted by parsers not available in many embedded platforms,
wildcard templates are straightforward to implement and to process.
This simplicity eases the adoption of the \ac{api} by as many platforms as possible.
However, since the wildcard templates are the base for the advanced query languages, our \ac{api} could be extended to allow their use.
In any case, this extension is left as a future work.



\subsection{Adopted \acs{tsc} primitives}
\label{sec:primitives}

\acs{tsc} derives some primitives originally defined in the Linda language \citep{gelernter_generative_1985} to access the semantic information hold in each graph.

\begin{itemize}
 \item The \textbf{write} primitive allows writing a graph into a given space (identified by its \acs{uri}).
	The set of triples received by this primitive will be stored together in the same graph, returning the \acs{uri} which identifies that graph.
	The graph \acs{uri} can be used to access directly to that graph later on, or to create new triples and relate contents.
       % Esto no encaja necesariamente con el concepto de "Browsable graph": http://www.w3.org/DesignIssues/LinkedData
  \begin{lstlisting}
    write(space_URI, triples): URI              [1]
  \end{lstlisting}


  \item The \textbf{read} returns a graph belonging to a given space which contains at least a triple matching the given template or has the given \acs{uri} as its identifier.
	If more than one graph fulfils one of these conditions, just one of them is returned (nondeterministically).
	It should be remarked that it has been designed as a non blocking operation.
  \begin{lstlisting}
    read(space_URI, graph_URI): triples         [2]
    read(space_URI, template): triples          [3]
  \end{lstlisting}
  
  
% why?
% TODO READ al ser no-deterministica puede afectar a caching???
  \item The \textbf{take} primitive behaves like a destructive \textbf{read}, deleting the graph returned from the space.
  \begin{lstlisting}
    take(space_URI, graph_URI): triples         [4]
    take(space_URI, template): triples          [5]
  \end{lstlisting}

%  \item Space management primitives.
%	A node can join or leave a space using \linebreak \texttt{joinSpace(space\_URI)} or \texttt{leaveSpace(space\_URI)}.
\end{itemize}



\subsection{\acs{http} \acs{api} for \acs{tsc}}
\label{sec:httpapi}

As both \ac{tsc} and \ac{http} are \ac{rest} compliant, their similarities are evident.
In the same way \ac{tsc} has the already explained primitives, \ac{http} has verbs to get, create, update or remove resources (GET, PUT, POST, DELETE).
%The main purpose of \acs{tsc} primitives is not to access data but to coordinate the nodes accessing to that data.
Consequently, the translation between these two worlds is straightforward.

\begin{sloppypar}
According to the \ac{rest} principles the interaction with an API must be hypertext-driven \citep{fielding_rest_2008}.
To ease its usage by developers, we provide a human-oriented \acs{html} representation of the \ac{api} which is completely \emph{browseable}.
% un poco repetido de la parte de HATEOAS
Regarding the machine-oriented representation of the \ac{api}, \citet{verborgh_functional_2012} and \citet{kjernsmo_necessity_2012} have proposed solutions to enable hypermedia driven semantic \acp{api}. % mencionar que son experimentales?
Independently of the mechanism adopted, in this section we propose a optional \ac{api} which stresses \ac{tsc}'s compliance with \ac{http}.
\end{sloppypar}

The list of spaces a node is joined to are available under \textit{/spaces}.
Each space is identified by an \acs{uri} (e.g., \url{http://space1}).
If this \acs{uri} is also a \acs{url}, $/spaces/\{space\_uri\}$ (summarized by \emph{sp} from now on) can simply redirect to it.
All the resources of that space, both real (i.e., graphs) or virtual (i.e., query) are listed under $\{sp\}/$.
Each graph is available on ${\{sp\}/graph/\{graph\_uri\}}$.
If we make an \acs{http} DELETE to that resource, under \ac{tsc}'s perspective, we take that graph from the space.
The rest of the mappings are shown in the Table~\ref{tab:tscAPI}.


\InsertTab{tab:tscAPI}{\acs{http} mapping for the primitives detailed in the Section~\ref{sec:primitives}}{
  \textit{sp} is a space \acs{uri},
  \textit{g} is a graph \acs{uri}, \textit{s}, \textit{p} and \textit{o-uri} are subject, predicate and object \acsp{uri} or wildcards (represented with an as \textit{*}).
  When the template's object is a literal, it can be expressed specifying its value (\textit{o-val}) and its type (\textit{o-type}).
}{
  \begin{tabular}{llc}
      \hline
      \acs{http} request & \acs{url} & Returns \\
      \hline
      POST & \{sp\}/graphs/ & [1] \\
      GET & \{sp\}/graphs/\{g\} & [2] \\
      GET & \{sp\}/graphs/wildcards/\{s\}/\{p\}/\{o-uri\} & [3] \\
      & \{sp\}/graphs/wildcards/\{s\}/\{p\}/\{o-type\}/\{o-val\} & \\
      DELETE & \{sp\}/graphs/\{g\} & [4] \\
      DELETE & \{sp\}/graphs/wildcards/\{s\}/\{p\}/\{o-uri\} & [5] \\
      & \{sp\}/graphs/wildcards/\{s\}/\{p\}/\{o-type\}/\{o-val\} & \\
      \hline
  \end{tabular}
}


\subsubsection{Status codes}
\label{sec:status_codes}
The \acs{api} should be compliant with the standardized \acs{http} status codes \citeweb{http2008status}.
These codes are sent back in the response as part of the header.
For instance, the \ac{tsc} middleware returns the \emph{404 error} when no significant result can be found for a primitive.
This adoption, apart from enhancing the compatibility with other web applications, can enable the modular adoption of the \ac{api}.
% Esto tiene más sentido para el outer space que para este API
For example, if a space does not offer a wildcard based \textit{read}, it can simply return a \emph{501 Not Implemented}.
The participants would then be aware of the problem and use an alternative primitive to obtain the data needed.
%Instead, they will interpret these cases as empty responses.
This modularity becomes crucial to ease the partial adoption on new platforms.


\subsubsection{Content negotiation}
Another key aspect of the \acs{http} protocol that our \acs{api} should take advantage of is the \textit{content negotiation}.
This mechanism allows to specify the desired representation for a content on the client side and to express what representation is sent as a response from the data provider side.
For that purpose, the client adds an \textit{Accept} field to the \acs{http} header with a weighted list of media types it knows how to interpret.
Then, the server will answer with the best possible format it understands, specifying the \textit{Content-type} in the response.

The benefits of using this mechanism in the \ac{tsc} middleware presented are two-fold.
Firstly, it enhances the browsability of the primitives with human understandable \acs{http} responses. % no sé si esto es de Content Negotiation per se
Secondly, it allows different semantic representations (e.g., RDF/XML \citeweb{rdfxml2004}, N-Triples \citeweb{ntriples2004} or \acs{n3} \citeweb{n32011}).
The latter characteristic becomes crucial since not all the nodes may understand all the formats (e.g., a mobile phone may not have a RDF/XML parser).
%This is true even if all the languages use the same basic concepts (i.e., \acs{rdf} Triples).
In these cases, the compatibility of both sides can be ensured through a conversion carried out in the server side.
Furthermore, expressing the preference for a semantic format can be useful too in other cases.
For example, to obtain the less verbose answer.

\section{Federated Space}
\label{sec:federated_space}

The \osapi{} extends the \ac{api} previously presented with the primitives and concepts explained in this section.


% grafos autogestionados (no takeables por otros) => write en local
\subsection{Self-managed graphs}

These graphs are shared with other participants, but can only be managed by the devices called \asteroids{}. % write and taken locally
In other words, \selfgraphs{} enrich the space but cannot be externally written or removed. % No device can remove them apart from their creator.
% poner ejemplo de por qué no tiene sentido que eliminen a través del espacio
Therefore, they are \emph{second-class graphs} which provide information about the environment but cannot be used for coordination purposes.


Each \asteroid{} makes these graphs accessible to others through \ac{http}.
% objetivo: que se pueda acceder a estos grafos incluso si no son parte de nuestra app => interop
The final goal is to potentially allow to reuse the data provided by any existing \ac{rest}ful service. % app level interop
Therefore, the \ac{api} should be or trend to be \ac{rest}ful.
% para ello es importante usar un RESTful approach, como sabemos que es complejo ofrecemos otra opción en el siguiente capítulo
However, we leave the \emph{hypermedia} \ac{api} as a future work.
Instead, we require a mandatory \osapi{} to be implemented in each \asteroid{} to guarantee access to the \selfgraphs{}.
% lo sé, no está justificado 0:-)


\subsection{New primitives}

To make the most of the information in a space, we propose a new primitive to query all over the semantic information stored.
% esto puede ser relativo, al final devolver una lista completa de recursos (o parte de sus atributos) no deja de ser REST según los libros de hypermedia design...
The \ac{rest}fulness of this primitive could be argued since it does not operate at resource level (i.e. returning graphs), but mixing several resources (i.e. triples from different graphs).
However, we believe that it is useful to have an endpoint for the queries which involve many graphs.
\citet{kjernsmo_necessity_2012} discusses this topic in depth. % TODO Mirar otras citas???

This new primitive is defined as follows:
\begin{itemize}
  \item The \textbf{query} primitive aims to see the space as a whole, returning all the triples matching the given template.
  
  \begin{lstlisting}
    query(space_URI,template): triples          [6]
  \end{lstlisting}
\end{itemize}


The Table~\ref{tab:queryAPI} extends Table~\ref{tab:tscAPI} to include this new primitive.
% TODO otra clave: debería no tener porqué pasar por el servidor

\begin{table} %http://en.wikibooks.org/wiki/LaTeX/Floats,_Figures_and_Captions#Wide_figures_in_two_column_documents
  \centering
  \caption {
    \acs{http} mapping for the \emph{query} primitive.
    \textit{sp} is a space \acs{uri}, \textit{s}, \textit{p} and \textit{o-uri} are subject, predicate and object \acsp{uri} or wildcards (represented with an as \textit{*}).
    When the template's object is a literal, it can be expressed specifying its value (\textit{o-val}) and its type (\textit{o-type}).
    \medskip
  }
  \begin{tabular}{c|l|c}
      \acs{http} request & \acs{url} & Returns \\
      \hline
      GET & \{sp\}/query/wildcards/\{s\}/\{p\}/\{o-uri\} &  [6] \\
      & \{sp\}/query/wildcards/\{s\}/\{p\}/\{o-type\}/\{o-val\} & \\
  \end{tabular}
  \label{tab:queryAPI}
\end{table}


A key point of the \ac{api} is that the \asteroids{} might not even follow the \ac{tsc} paradigm.
For instance, the \osapi{} can encapsulate data provided by a third middleware.
However, a primitive to ease that management can be a convenient for the developers which do not need a more customized behavior.
With that in mind, we propose another writing primitive.
This primitive only has local effects and therefore has no HTTP equivalent:
\begin{itemize}
  \item The \textbf{write\_self} primitive writes a \emph{self-managed graph} and returns an \ac{uri} which identifies it.
  
  % tiene sentido definir el space_URI???
  % mejor definir su propio espacio?
  % tiene sentido especificar su URI? write_self(graph_URI, triples): URI
  \begin{lstlisting}
    write_self(space_URI, triples): URI
  \end{lstlisting}
  
  \item The \textbf{read\_self} and \textbf{take\_self} primitives only affect to \selfgraphs{}.
  
  \begin{lstlisting}
    read_self(space_URI, graph_URI): triples
    read_self(space_URI, template): triples
    take_self(space_URI, graph_URI): triples
    take_self(space_URI, template): triples
  \end{lstlisting}
  
\end{itemize}



\subsection{New behaviors}

% explicar cómo se escribe y lee en el espacio
% solución enfoque híbrido:
%      cambios en actuadores => directamente a través de HTTP o indirectamente a través de tasks escritas en espacio
%                               o mejor: podrían esas tasks ser directamente esos servicios???
%      grafos que sí => write al "servidor HTTP"
%      query => en todos los dispositivos (capítulo 4) - Porque a veces es necesario a través de todos los nodos
%      take + read => sobre los grafos takeables (o inferencia con toda la info del espacio, cómo prefieras)
In this section we will try to clarify how the different behaviors coexist:

\begin{description}
 \item[Writing.]
      The most basic writing primitive allows a client to write a graph into the space hold by the server.
      However, we also presented the \emph{write\_self} primitive.
      \emph{Write\_self} writes into the local device an externally untakeable graph (i.e. \selfgraphs{}).
 \item[Reading.]
      \emph{Query} performs a traversal query which aggregates all the graphs of the space.
      \emph{Read} and \emph{take} work at resource level.
      \emph{Read\_self} and \emph{take\_self} are their equivalents for \selfgraphs{}.
      Finally, \emph{read} primitive's results will be enriched with \selfgraphs{} from the \outerspace{}. % mediante redirect o lo que sea
\end{description}
\section{Conclusions} % TODO llamarlo Evaluation?

With the solution presented in the previous section, we aimed to retain the desirable properties of both \ac{rest} and \ac{tsc}.
% analísis más largo en la siguiente sección
% 1. cómo sé que esto es mejor que usar REST y TSC por separado? Sinergia?
% 2. que aporta unirlo bajo un mismo middleware?
% 3. tiene beneficios?
However, some questions arise from that integration:
(1) does in fact retain all the properties of \ac{rest} and \ac{tsc} separately? Otherwise, which properties are affected?
(2) what other benefits carries comparing with using already existing \ac{tsc} and \ac{rest} middleware?

% Fielding distingue entre:
%      + "distributed-system": hace que el usuario vea el sistema como un todo, sin saber cuando se realizan llamadas en remoto y cuando no
%      + "networking-system": no necesariamente oculta las particularidades de la red <- el se centra en estos


To answer these questions, we analyze the presented solution from different point-of-views:
(1) its coordination properties (Section~\ref{sec:XXX_properties}),
(2) its network-level properties (Section~\ref{sec:network_properties}),
(3) how the latter work together to contribute to the challenges of an \ac{ubicomp} environment (Section~\ref{sec:middleware_properties}).
Section~\ref{sec:middleware_eval_summary} summarizes these analysis. % cómo dan respuesta a las preguntas del comienzo


% TODO cosas rescatables del paper original de WoT 2011:
%     Uniform API vs. specific usage of each API !!!
%     Coupling: lo comido (autonomies), por lo servidor (client-server coupling por tener API común)
%          In spite of the outlined decoupling nature of both approaches, the data definition can be considered a coupling mode indeed.
%          While in \ac{wot} each resource defines its own data formats and contents themselves, in \ac{tsc} the ontology in which the semantic
%          concepts are described must be known by each part of a distributed application to effectively cooperate among them.
%     Discovery: HATEOAS puede ser complejo para cacharros peques (parseo),
%                proponemos discovery basada en API uniforme y a nivel de cacharros con algún mecanismo en el que no entramos.
%          One of the main drawbacks in \ac{wot} is the lack of a discovery mechanism for new objects and the data they provide.
%          Even when this data can be linked in each object response (using HATEOAS) and microformats are sometimes included to ease the search-ability of these objects by search engines, it is difficult for an object which may change of location and context to be referred.
%          Thus, \ac{wot} may have a tendency to create isolated islands of data.
%          Several workarounds have been proposed to overcome this limitation, such as using a central repository\footnote{http://www.pachube.com/}, a framework which uses federated repositories responsible for different administrative domains \citep{stirbu_towards_2008} or making each connected sensor announce itself to let an intermediary know its presence \citep{kamilaris_smart_2010}.

%     Scalability: se ha hablado y ahora no es tan mala como antes ;-) Entre cacharros seguiría dando asco
%     Semantics: Microformats vs Full (hablar de esto)
%     Comprensión por parte de usuario: sencillo misma interfaz, no tienes que andar descubriendo nuevas APIs y viendo cómo funcionan, qué formato devuelven, etc.
%     Comprensión por parte de desarrollador: API sencillica siempre


\subsection{Coordination properties}
\label{sec:coordination_properties}

% TODO añadir la autonomía que se refería a la semántica?
As we have already discussed in Section~\ref{sec:integration}, indirect communication middleware can have two key properties:
\begin{itemize}
  \item The sender does not need to know the receiver or receivers and vice versa (i.e. \emph{space uncoupling}).
  \item Senders and receivers do not need to exists in the same time to communicate with each other (i.e. \emph{time uncoupling}).
\end{itemize}


The primitives used in our space-based computing implementation force it to be \emph{space uncoupled}.
On the contrary, it is not always \emph{time uncoupled}.
If two nodes write and read from the coordination space, they are \emph{time uncoupled};
but if a node accesses to others' content (i.e. \emph{self-managed graphs}), they are not.
In other words, \emph{time uncoupling} is not achieved by the extension of the model presented in Section~\ref{sec:halfway_solution}.
As a future work, this could be alleviated by a caching mechanism implemented in the space holder or holders. % TODO hablar con propiedad de esos elementos => nombrarlos antes
% see \ref{tab:middleware_coordinationprop}


% Tablita de cómo hereda propiedades de los estilos anteriores?
% (y si quieren más información, que miren en la tesis de Fielding)
\begin{savenotes} % to use footnotes inside
  \begin{table}[htbp]
    \caption{Uncoupling levels achieved by the different parts of the middleware presented.}
    \begin{center}
      \begin{tabular}{lcc}
	\hline
	~ & Space &	Time \\
	~ & uncoupling & uncoupling \\
	\hline
	% O mejor ponerlo con Yes y No?
	Coordination space & $\checkmark$ & $\checkmark$ \\
	Outer space & $\checkmark$ & × \\
	\hline
      \end{tabular}
    \end{center}
    \label{tab:middleware_coordinationprop}
  \end{table}
\end{savenotes}


\subsection{Networking properties} % Architectural Properties for Network-based Styles
\label{sec:network_properties}
% Cómo afecta la app propuesta => ir actualizando al final de otras secciones???

In Section~\ref{sec:tsc_vs_rest} we have described how \ac{tsc} does not contradict any of the \ac{rest} principles.
However, the adaptation presented in this thesis does not completely adhere to the \ac{rest} style.
In this section we present how this divergences affect the \ac{rest}'s properties.
% decir que el objetivo final es que pudiera ser REST

% TODO antes de entrar al lio, explicar cómo funciona esto de REST, en qué parte de nuestra solución lo necesitaremos
% decir que API para TSC mola
% decir que API para los cacharros deberia de ser independiente a lo que nosotros digamos y por eso estaria bien REST
%     de todas formas, como es dificil, nos quedamos en un nivel de madurez menor


In \citeauthor{fielding_architectural_2000}'s words, the relevant properties which describe a network-based system are the following ones \footnote{We refer to the reader to \citet{fielding_architectural_2000} for the complete thorough analysis.}:
% verificar que no se parece demasiado a sus definiciones o sino poner cursiva.
\begin{description}
  \item[Performance] is divided into network performance, user-perceived performance and efficiency.
    \begin{description}
      \item[Network performance] is affected by the styles in the number of interactions and the granularity of data elements.
      \item[User-perceived performance] refers to the impact a user in front of an application perceives. % UP Performance es una feature que no aporta mucho a nuestro caso
      \item[Efficiency] is achieved by minimizing the use of the network.
    \end{description}
  \item[Scalability] measures how an architecture supports a big amount of components and interactions between them.
  \item[Simplicity] is achieved through the separation of concerns for the components and the generality of architectural elements.
  \item[Modifiability] encompasses evolvability, extensibility, customization, configurability and reusability.
    \begin{description}
      \item[Evolvability] refers to the degree in which a component can be implemented without negatively impacting on others.
      \item[Extensibility] measures the ability to add functionality to the system.
      \item[Customization] is the ability to customize the behavior of an architectural element temporarily.
      \item[Configurability] is related with the extensibility and the reusability.
      \item[Reusability] is the ability to reuse components, connectors or data elements without modifying other apps.
    \end{description}
  \item[Visibility] is the ability of a component of monitoring or mediating in the interaction between other two components.
  \item[Portability] is the ability of working in different environments.
  \item[Reliability] is the degree in which an architecture depends on the failures of the system or components, connectors or partially incorrect data.
\end{description}

% Requisitos de ealy web, propiedades deseables
%   Low entry-barrier
%   Extensibilidad
%   Distributed Hypermedia
%   Internet-scale

The Table~\ref{tab:network_properties} summarizes how different architectural styles achieve these properties.
Particularly, it shows the styles from which \ac{rest} derives (see Section~\ref{sec:rest}).
The ultimate goal of the solution explained in the previous section is twofold:
\begin{enumerate}
  \item Provide a \ac{rest} access to a semantic space.
       This would ease its integration with the rest of the web.
  \item Enrich that space with the knowledge provided by other \ac{rest} \acp{api}.
\end{enumerate}
Therefore, ideally, its networking properties would be the sames as the \ac{rest} style.


However, defining a 100\% \ac{rest} compliant \ac{api} is not easy.
Indeed, most of the self-proclaimed \ac{rest}ful \acp{api} are not \citep{house_how_2012}. % la nueva buzz word para eso: hypermedia API
The main cause is the \emph{HATEOAS} constraint seen in the Section~\ref{sec:rest} \citep{fielding_rest_2008}.
Using \ac{http} as an application-level protocol forces a developer to comply the rest of the constraints \citet{moore_hypermedia_2010}, but not \emph{HATEOAS}.
% TODO hablar del Richardson Maturity model!


% semantic-based \ac{api} is not easy. => referencias a los que lo han intentado, a que no hay hipertexto, a que te puede interesar inferir con bastantes
Regarding the \ac{sw}, some recent efforts have tried to move closer to the \ac{hypermedia} constraint \citep{steiner_fulfilling_2011,kjernsmo_necessity_2012}.
However, these worlds remain quite isolated.
In fact, \ac{sw} API  usually present another main divergence with the \ac{rest} style: the use of query endpoints.
% i.e. no devuelven listas de enlaces, sino todo a lo bruto
% esto quiere decir que no se devuelve todo y ala, tu procesa esa burrada de datos, más bien se filtran
These endpoints intend to solve some inefficiency issues which \ac{rest} shows when working with a big amount of data. % TODO mencionar \ac{lod} en algún punto de aquí?
To that end, they allow expressive query languages and offer results where the boundaries of the different resources often blurs \citep{wilde_restful_2009}.
The most paradigmatic case may be to decide to which graph does a inferred content belongs. % ejemplo de lo anterior


% Aterrizar esto a IoT y a nuestro caso:
%        La eficiencia es importante => para evitar hacer 500 llamadas a un cliente
%        La reusabilidad es importante también => para potencialmente adaptarse a cualquier WoT semántico
%           en el futúro => más trabajo en este área
%       Solución de compromiso:
%           nivel de madurez X de Richardson. Suponemos que alguien debe serguir nuestro API mínimo en los objetos WoT

In \ac{iot} both the efficiency on the communications and reusability of the \ac{api} are important:
\begin{description}
  \item[Efficiency:] mobile and embedded devices' have restricted energy autonomy.
                    This autonomy is severely affected by network communications. % TODO cite
                    Particularly, the access to the data they provide by means of hyperlinks may result in many HTTP requests. % aunque eso se puede adaptar por lo visto
                    %  Por que sería muy costoso hacer rollo araña?!
		    %     lo que ganas de reusabilidad, lo aumentas en complejidad en el cliente
                    %     interpretar código del servidor no es tampoco super-sencillo (xHTML!)
  \item[Reusability:] due to the heterogeneity of the devices, assuming that they all share a common and unevolvable \ac{api} may not be realistic. % TODO paliar para que no parezca que no tiene sentido
\end{description}
The first solution together with the current difficulties on achieving a fully and standard \emph{HATEOAS} for machines in the \ac{sw}, lead us to opt for the second level on the Richardson Maturity level.
Besides, our \ac{tsc} \ac{api} will be \emph{HATEOAS} \ac{api} in its humans representation to ease their learning level.% pero aún así el API de TSC seguirá siendo HATEOAS para los usuarios, no para las máquinas
However, we do not discard to work towards a fully \ac{rest}ful \ac{api} as future work.


In conclusion, the properties of the properties for network communication style we adopted corresponds with \emph{LCODC\$SS} plus simplicity and visibility. % hacer una nueva fila con esto
% TODO comprobar que se explica eso tal cual en la tesis de Fielding: HATEOAS => sólo afecta a reusabilidad

% TODO Comprobar que no se salga por el borde derecho!

% Tablita de cómo hereda propiedades de los estilos anteriores?
% (y si quieren más información, que miren en la tesis de Fielding)
\begin{savenotes} % to use footnotes inside
  \begin{table}[htbp]
    \caption[Properties of different architectural styles for network-based applications as defined by \citet{fielding_architectural_2000}.]
            { Note that apart from including \ac{tsc}, the original table has been slightly adapted.
              These adaptations are defined inside the table. % using several footnotes.
            }
    \begin{center}
      \begin{tabular}{lccccccccccccc}
	Style &
	\rotatebox{90}{Net Perform} &
	\rotatebox{90}{UP Perform} &
	\rotatebox{90}{Efficiency} &
	\rotatebox{90}{Scalability} &
	\rotatebox{90}{Simplicity} &
	\rotatebox{90}{Evolvability} &
	\rotatebox{90}{Extensibility} &
	\rotatebox{90}{Customiz.} &
	\rotatebox{90}{Configur.} &
	\rotatebox{90}{Reusability} &
	\rotatebox{90}{Visibility} &
	\rotatebox{90}{Portability} &
	\rotatebox{90}{Reliability} \\
	\hline
	CS & ~ & ~ & ~ & $+$ & $+$ & $+$ & ~ & ~ & ~ & ~ & ~ & ~ & ~ \\
	S\footnote{\emph{S} represents the difference between \emph{CSS} and \emph{CS} in \citep{fielding_architectural_2000}.}
	  & $-$ & ~ & ~ & $+$ & ~ & ~ & ~ & ~ & ~ & ~ & $+$ & ~ & $+$ \\ % = CSS - CS
	\$ & ~ & $+$ & $+$ & $+$ & $+$ & ~ & ~ & ~ & ~ & ~ & ~ & ~ & ~ \\
	\hline
	Early web\footnote{Corresponds to the \emph{C\$SS} style in \citep{fielding_architectural_2000}.}
	 & $-$ & $+$ & $+$ & $++$ & $+$ & $+$ & ~ & ~ & ~ & ~ & $+$ & ~ & $+$ \\ % = C$SS
	L & ~ & $-$ & ~ & $+$ & ~ & $+$ & ~ & ~ & ~ & $+$ & ~ & $+$ & ~ \\ % = LS
	COD & ~ & $+$ & $+$ & $+$ & $\pm$ & ~ & $+$ & ~ & $+$ & ~ & $-$ & ~ & ~ \\
	\hline
	LCODC\$SS & $-$ & $++$ & $++$ & $++++$ & $+\pm+$ & $++$ & $+$ & ~ & $+$ & $+$ & $\pm$ & $+$ & $+$ \\
	U\footnote{Although it is not explicitely included in the original table, \emph{U} has been derived from \citeauthor{fielding_architectural_2000}'s description.}
	 & ~ & ~ & ~ & ~ & $+$ & ~ & ~ & ~ & ~ & $+$ & $+$ & ~ & ~ \\ % Uniform Interface (simple, visible, reusable) && en el texto dice que degrada efficiency
	\hline
	REST\footnote{Derived from the addition of \emph{U} to \emph{LCODC\$SS}.} % esto se podría intuír por la línea, pero quién sabe..
	 & $-$ & $++$ & $++$ & $++++$ & $+\pm++$ & $++$ & $+$ & ~ & $+$ & $++$ & $+\pm$ & $+$ & $+$ \\ % = LCODC$SS + U
	TSC (own) & × & × & × & × & × & × & × & × & × & × & × & × & ×\\
	\hline
      \end{tabular}
    \end{center}
    \label{tab:comparisonDistribution}
  \end{table}
\end{savenotes}



\subsection{Properties for \acs{ubicomp}}
\label{sec:middleware_properties}

A middleware is a software layer which provides a higher level of abstraction and masks the underlying heterogeneity.
The middleware presented in this thesis is oriented for \ac{ubicomp} environments and the devices which populate them (i.e. mobile and embedded devices).
Whereas the devices part of the \ac{iot} are a subset of the ones present in \ac{ubicomp}, we consider that the challenges they have to face are the same ones.
For instance, smartphones are not part of the \ac{iot}, but face similar energy and computational limitations as the embedded devices. % posiblemente no todas las propiedades!

% ¿Qué propiedades deseables para IoT añadiríamos?
Therefore, we will adhere to the challenges presented by the \emph{Internet-of-Things Architecture} European project \citep{walewski_project_2011} to analyze our solution.
\citeauthor{walewski_project_2011} state that the \ac{iot} must overcome the following challenges:
interoperability, scalability, manageability, mobility, security and privacy and reliability.
Furthermore, we also consider energetic and computational constraints which devices in \ac{ubicomp} have to face. % ponerlo de forma que sea menos pegote?


\subsubsection{Interoperability}

Interoperability is key to deal with a wide range of heterogeneous technologies.
For the \textbf{communication} between nodes, we rely on the widely supported \ac{http} protocol. % usar la palabra ubiquitous o pervasive???
Communication with devices using other protocols must be done by means of specialized gateways. % out-of-the-scope


At another level, we can consider the interoperability between applications built on the analyzed environments.
Our solution contributes to that goal by
(1) using semantic data,
(2) promoting the use of third applications' data and 
(3) making our data reusable by third applications. % estos 2 últimos son como redefinir interop

By using semantics, we describe data in a more rich and abstract way.
Two key mechanisms of the Semantic Web in this aspect are the inference of new data and the mapping of equivalent models.
These abstract way to annotate data allows third applications to reuse it.

% contribuimos a los otros 2 de la siguiente manera:
By using some basic primitives, we encourage sharing data in resource constrained devices through an \ac{http} \ac{api}.
This \ac{api} also promotes reusing data from other applications build on top of that middleware.
Furthermore, encapsulating semantic from third \ac{rest} applications is as straightforward as implementing a simple \ac{api} on top of them.

% Future work: eliminar la necesidad de espacios?
By using semantic data, we try to improve application-level interoperability.
However, the use of \emph{spaces} goes against this goal.
As explained in Section~\ref{sec:tsc_soa} their use is justified in terms of scalability.


% explicar por qué
% simplicidad siempre beneficia a alguno... :-S
Regarding the \emph{interoperability}, we defend that relates with the 

% Portability !!!
Besides, the Semantic Web is built on top of well-defined and bastly supported tools.



\subsubsection{Scalability}
to cope with a greater amount of devices.


\subsubsection{Manageability}
of the devices through autonomous behavior. % specially in environments where they cannot be controlled through centrality
                      % self-management, self-configuration, self-healing, self-optimisation and self-protection

\subsubsection{Mobility}
ability to face these situations.

\subsubsection{Security and privacy}
  % definición de reliability modificada yendo a la fuente de iot-a.eu

\subsubsection{Reliability}
to handle connectivity losses in various \emph{ad hoc}-like ways.



% TODO cómo las distintas propiedades vistas antes se combinan para aportar a estas
%      una valoración de las mismas?
As shown in the Table~\ref{tab:middleware_netprop}, some of these properties closely relate with some of the ones explained in the previous section.
Some of these properties are repeated (\emph{scalability} and \emph{reliability}).
Others relate in unself-explanatory ways (\emph{interoperability}, \emph{manageability} and \emph{energy and computational restrictions}).
Finally, some properties just relate indirectly.


% TODO Comprobar que no se salga por el borde derecho!

% Tablita de cómo hereda propiedades de los estilos anteriores?
% (y si quieren más información, que miren en la tesis de Fielding)
\begin{savenotes} % to use footnotes inside
  \begin{table}[htbp]
    \caption{Direct relations between network properties and properties for lightweight middleware.}
    \begin{center}
      \begin{tabular}{lccccccc}
	~ &
	\rotatebox{90}{Performance} &
	\rotatebox{90}{Scalability} &
	\rotatebox{90}{Simplicity} &
	\rotatebox{90}{Modifiability} &
	\rotatebox{90}{Visibility} &
	\rotatebox{90}{Portability} &
	\rotatebox{90}{Reliability} \\
	\hline
	Interoperability & ~ & ~ & × & × & × & × & ~\\
	Scalability & ~ & × & ~ & ~ & ~ & ~ & ~\\
	Manageability & ~ & ~ & ~ & × & ~ & ~ & ~\\
	Mobility & ~ & ~ & ~ & ~ & ~ & ~ & ~ \\
	Security &  ~ & ~ & ~ & ~ & ~ & ~ & ~ \\
	Reliability & ~ & ~ & ~ & ~ & ~ & ~ & ×\\
	Restricted energy \& comp. & × & ~ & ~ & ~ & ~ & ~ & ~\\ % también simplicity
	%XXX & × & × & × & × & × & × & × & × & × & × & × & × & ×\\
	\hline
      \end{tabular}
    \end{center}
    \label{tab:middleware_netprop}
  \end{table}
\end{savenotes}



% explicar cómo HTTP trataba de evitar complejidades
% explicar como al crear una capa por encima, complicas todo un poco
%       sencillez
%       user-based library vs a lo otro

% Propiedades que jodes de HTTP de cara al usuario: simplicidad...
% Requisitos de ealy web, propiedades deseables
%   Low entry-barrier
%   Extensibilidad
%   Distributed Hypermedia
%   Internet-scale


% drawbacks frente a HTTP: sencillez, etc.
A middleware is a software layer which provides a higher level of abstraction and masks the underlying heterogeneity.
\ac{http} is a middleware indeed, but sacrifices masking exceptions on behalf of the simplicity. % TODO checkear con la tesis de Fielding
% TODO hablar de user-based library vs a lo otro
The success of libraries and tools for \ac{http} can be found in this simplicity.


% TODO \subsection{Querying Distributedly over the Space} => optamos por no meternos en jardines y up to the middleware implementer to decide each ;-)
% condiciones relajadas: consistencia, razonamiento distribuído
% out of the scope: cómo implementarlo
% optaremos por lo más sencillo

% Additional remarks???


\subsection{Summary}
\label{sec:middleware_eval_summary}

In the previous sections we have get to the bottom of the strengths and weaknesses of our proposal.
To sum up, the main limitations are X and Y.
On the other hand, we can XXX.
Table~\ref{x} shows these and other aspects.

% Ventajas de meclar 2 cosas aparentemente separadas?
%   Para WoT:
%      comunicación indirecta para cacharros
%          local+ref
%              de contenidos relativos a coordinación
%              podríamos cachear contenidos de otros en el espacio central (reliability+energy)
%          ref
%              sacar a colación los problemas aquellos de query sobre muchos recursos
%
%   Para TSC (respecto a otras como Smart-M3):
%      limitamos qué gestiona la entidad central => sólo coordinación, el resto lo coge de los cacharros
%      se reconoce su autonomía permitiendo gestionarse su propio contenido => REST, escalabilidad
%      interop => SW y posibilidad de extender a otros servicios REST
%      multiplataforma => facilidad para portarlo
%
%   Para desarrollador:
%      API integrada y simple
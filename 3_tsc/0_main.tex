
% this file is called up by thesis.tex
% content in this file will be fed into the main document

%: ----------------------- introduction file header -----------------------
\begin{savequote}[50mm]
Historical methodology, as I see it, is a product of common sense applied to circumstances. 
\qauthor{Samuel E. Morison}
\end{savequote}


\newcommand{\primquery}{\emph{query}}
\newcommand{\primread}{\emph{read}}
\newcommand{\primtake}{\emph{take}}
\newcommand{\primwrite}{\emph{write}}


\chapter{Triple Space Computing for resource constrained devices}
\label{cha:tsc}

% the code below specifies where the figures are stored
\ifpdf
    \graphicspath{{3_overall_methodology/figures/PNG/}{3_overall_methodology/figures/PDF/}{3_overall_methodology/figures/}}
\else
    \graphicspath{{3_overall_methodology/figures/EPS/}{3_overall_methodology/figures/}}
\fi



% ----------------------------------------------------------------------

% 1. Intro (refiriendome descaradamente al estado del arte)
%  	Extensible a otros Ubicomp
In recent years, Internet of Things (IoT) is becoming a reality due to the increasing number of everyday objects containing embedded devices.
The use of these everyday objects together with the rising of mobile computing greatly contributes to the Ubicomp vision.
In the beginning the community put more effort on those devices' connectivity issues, materialized in the spread of technologies such as Zigbee, 6LoWPAN or Bluetooth.
However, nowadays the community is focusing on higher levels of the architecture.
These architectures define models in which the applications are built using the capabilities of these objects.
These models have been thoroughly described in the previous chapter.
In the present chapter we will focus on the comparison of the Web of Things (WoT) and Triple Space Computing (TSC).

On the one hand, the WoT is an approach which fully integrates real-world objects and devices in the web.
This is done by embedding web servers into them and adopting the REST architectural style to provide information on demand.
This style is highly decoupled because it focuses on accessing in a simple way to the data itself and not in complex protocols of communication between objects, making the services reusable and easily understandable by developers.

On the other hand, TSC is a paradigm where Semantic Web techniques are used to define the knowledge which is exchanged using a distributed shared space.
The idea of sharing information through a common space corresponds to the \textit{blackboard model}, used in context aware environments, which makes each process very autonomous from the rest.
The Semantic Web defines the information in a very expressive and machine understandable way (new data can be even inferred), and it has also been widely adopted in context-aware environments.
Thus, TSC can be seen as the conjunction of two well accepted data distribution and modelling standards.

Both WoT and TSC can be considered resource oriented solutions since they put emphasis in giving access to data resources.
However, their differences make also them suitable for different purposes.
% TODO reescribir esto TAAAAANTO
While TSC enables expressive queries, dynamic discovery and non human-mediated cooperation among objects, the WoT adopts the scalable properties of the World Wide Web and it is entirely based on web standards.
In this chapter we explain how to take advantage of both approaches, bringing together the best of both worlds.


\section{\acs{wot} and \acs{tsc} Comparison}
% 2. Comparativa
\label{sec:wot_tsc_comparison}

% TODO rehacer taaanto :-)

In this section, the strengths and weaknesses of both \ac{tsc} and REST approaches will be discussed and compared.
Then, their compatibility is shown by proposing two ways to use \ac{wot} solutions inside a \ac{tsc} environment and \textit{viceversa}.

The similarities and differences between \ac{wot} and \ac{tsc} which are summarized in the Table \ref{tab:comparison} are detailed below.

\begin{table}[ht!]
\centering
\caption {Comparison between \ac{wot} and \ac{tsc} (OtsoPack) resource-oriented approaches}
\begin{tabular}{|l|p{2,4cm}|p{2,6cm}|}
\hline
& \ac{wot} & \ac{tsc} (OtsoPack) \\
\hline
\textit{Architecture} & ROA \& C/S & ROA \& P2P \\
\textit{Communication} & HTTP & Jxta \\
\textit{Operations} & HTTP verbs & \ac{tsc} primitives \\
\textit{Format} & HTML, JSON \& XML & RDF (NTriples) \\ %content negotiation
\textit{Developer aid} & Less expressive data \& specific use & Expressiveness \& uniform API \\
\textit{Coupling} & Low & Very low \\ %data
\textit{Discovery} & Bad & Very good \\
\textit{Scalability} & Good & Bad \\
\textit{Semantics} & Microformats & Full \\
\hline
\end{tabular}
\label{tab:comparison}
\end{table}

\subsection{Architecture}
Both approaches are based on resource oriented architectures. \ac{wot} uses the REST architectural style, where the contents are directly addressable by
URLs to manipulate them, and \ac{tsc} creates, removes and modifies RDF triples or graphs, i.e. sets of interlinked triples, in a space formed by peers
in a P2P network.

Despite of these similarities, \ac{wot} uses a client-server architecture where each client has to know the URL where to address its operations, while
\ac{tsc} relies on a distributed shared memory which makes each process within a space completely autonomous. In \ac{tsc}, a node does not need to know where
information is stored, just the URI of the space it wants to query. Furthermore, the OtsoPack solution is fully distributed and conceived not to depend
on any centralized, previous known, intermediary server.

\subsection{Communication protocol}
Our \ac{tsc} solution currently uses Jxta\footnote{http://jxta.kenai.com/}, a language and platform independent \textit{Peer to Peer} protocol, to
interconnect the nodes and manage the groups (spaces) where they belong. Unfortunately, the mobile version relies in a Jxta gateway called Rendezvous
to propagate the messages to other nodes of the group, making some previous configuration necessary and potentially creating a bottleneck. Anyway,
some \ac{tsc} implementations such as TripCom\footnote{TripCom (IST-4-027324-STP, www.tripcom.org)} rely on HTTP, and one of our next steps will be the
reimplementation of our network communication through HTTP.

Even if it is not mandatory to comply with the RESTful style, \ac{wot} usually employs the HTTP protocol as a communication layer because of its
simplicity and wide adoption to ensure ample deployment.

\subsection{Operations}
\ac{wot} uses HTTP verbs to retrieve, create, modify or delete web resources. Retrieval is done using HTTP GET, creation by means of HTTP POST,
modifications with HTTP PUT and removals with HTTP DELETE. Finally, the HTTP OPTIONS command is used to introspectively find out what operations are
allowed for any URL.

\ac{tsc} and tuplespace solutions offer different primitives to write, read and delete content. In our solution, the previously explained \textit{write},
\textit{read} and \textit{take} primitives do that. Apart from these primitives, other ones are provided to manage spaces, to claim the manager role
for a type of content in a space, event notification or traditional request-reply style service consumption. Anyway, in this thesis we are going to focus in the
first ones because they are inherent to the \ac{tsc} paradigm. % innecesario mencionarlo?

\subsection{Format}
\ac{wot} usually returns a human readable HTML representation for a resource and XML or JSON common web representations to be used in mashups. One of the
key mechanisms that \ac{wot} inherits from HTTP is \textit{content negotiation}, which enables clients and servers to negotiate the requested and
provided representations for any given resource.

\ac{tsc} does not provide this response format negotiation mechanism. So, OtsoPack interchanges data in N-Triples format, which is the most primitive
representation of semantic data. In any case, other alternative semantic representations (e.g. RDF/XML, OWL/XML, Turtle or N3) could be
easily adopted as they basically describe the same RDF triples under different textual wrappers.

\subsection{Developer aid}%Actors}
In ROA an agreement about the data being exchanged and how it can be accessed is needed to develop applications.
On the one hand, \ac{wot} can define machine processable data using XML or JSON, but thanks to the Semantic Web \ac{tsc} can go one step
further formally defining them and making them also ``machine understandable'' (i.e. it is capable of inferring new knowledge and make
high-level queries over it).

On the other hand, both \ac{wot} and \ac{tsc} offer some operations and interfaces to access them. Whereas in \ac{tsc} this interface is the same in each node
and it is closely related with \ac{tsc} operations and resources, in \ac{wot} it changes in each application making a previous learning process necessary
to use them.

%Although applications which do not require human intervention can be built using RESTful services, one of the core concepts of \ac{wot} is the
%browse-ability of the object by the user using HTML. \ac{tsc} is a solution for the Semantic Web, which aims to provide a machine understandable
%web using ontologies and defining properties to enable inference.

\subsection{Coupling}
\label{sec:coupling}
The REST style describes loosely coupled services due to the platform independent, asynchronous and few self describing messages
\cite{pautasso_why_2009}. In a very similar manner, \ac{tsc} offers different kind of autonomies \cite{krummenacher_www_2005}:
\textit{time autonomy} (because of its asynchronous nature), \textit{location autonomy} (information providers and consumers are independent
from where the data is stored), \textit{reference autonomy} (nodes do not need to know each other) and
\textit{data schema autonomy} (it follows the RDF specification making it independent of nodes internal data schema).
% Quitado de la introducción para no repetirme
%Remarkably, \ac{tsc} guarantees four basic types of autonomies: \textit{space autonomy} (processes can be run in very different computational
%environments such as Android, Java ME devices or PCs, they just use a common coordination language), \textit{reference autonomy}
%(nodes do not need to know each other), \textit{time autonomy} (they communicate asynchronously) and \textit{data schema autonomy} (since
%written and read data is based on RDF triples, \ac{tsc} nodes are independent of their internal data schemas) \cite{fensel_triple-space_2004}.

In spite of the outlined decoupling nature of both approaches, the data definition can be considered a coupling mode indeed.
While in \ac{wot} each resource defines its own data formats and contents themselves, in \ac{tsc} the ontology in which the semantic
concepts are described must be known by each part of a distributed application to effectively cooperate among them.

\subsection{Discovery}
One of the main drawbacks in \ac{wot} is the lack of a discovery mechanism for new objects and the data they provide. Even when this data can be linked
in each object response (using HATEOAS) and microformats are sometimes included to ease the search-ability of these objects by search engines, it is difficult for an
object which may change of location and context to be referred. Thus, \ac{wot} may have a tendency to create isolated islands of data. Several workarounds
have been proposed to overcome this limitation, such as using a central repository\footnote{http://www.pachube.com/}, a framework which uses federated
repositories responsible for different administrative domains \cite{stirbu_towards_2008} or making each connected sensor announce itself to let an
intermediary know its presence \cite{kamilaris_smart_2010}.

\ac{tsc} provides a transparent data level discovery mechanism to the user since when a node joins a space (i.e. a group), its data become queriable by any other
node joint to this space. In OtsoPack the space management and communication relies on Jxta, a protocol which can be used at any level (even if we
have mainly used it on local environments using Jxta's discovery layer's capabilities).

\subsection{Scalability}
% TODO reescribir
The scalability of \ac{wot} is argued to be well proved since the World Wide Web is the most practical scalable system. As many objects as
necessary could be added to WWW without making it worse. 

Although Jxta scalability has been discussed and addressed in several works \cite{antoniu_performance_2007}, since our solution is fully decentralized
and uses flooding (each query is spread to the rest of the peers which belong to a space), its scalability is expected to be poor. To overcome this limitation,
we are working on implementing Semantic Overlay Networks (SON) to let the objects automatically divide the spaces into smaller ones and address queries
just to the appropriate nodes which conform a semantically related space.

% \begin{figure} [htbp]
% 	\centering
% 		\includegraphics[width=\linewidth]{./img/esquemaSON/arquitectura.png}
% 	\caption{Proposed distributed SON-based solution.}
% 	\label{fig:changeToSON}
% \end{figure}
% 
% As can be appreciated in Figure \ref{fig:changeToSON}, the proposed solution aims to split the spaces up into smaller subspaces.
% To do so, semantically similar contents are placed together in the same subspaces. E.g. \textit{subspaceA} contains \textit{graph1}
% and \textit{graph2} which describe ``light'' related knowledge. In the meantime, the nodes from the original space (\textit{space1}) can organize
% themselves to create another \textit{subspaceB} containing the relevant data for describing ``mobile phones''.

\subsection{Semantics}
On the one hand, \ac{wot} uses predefined microformats to embed semantic information into human readable pages. Doing so, the search process performed
by search engines is enhanced \cite{guinard_internet_2011}. On the other hand, \ac{tsc} allows the usage of full semantics, more expressive than microformats,
using RDF as a base. This makes \ac{tsc} capable of using standard query languages such as SPARQL and it becomes also independent of third parties' products to search data.



% 3. Ventajas ppales de cada uno
\subsection{Conclusion}
% TODO

In the previous section, the two-way compatibility between the \ac{tsc} and \ac{wot} approaches has been described. As has been shown, \ac{wot} is good due to its
scalability and usage of web standards, which makes it easy to be understood by potential developers (which encourages its usage). Our \ac{tsc}
solution, on the other hand, provides good local discovery of new resources and their information and allows more expressive semantic data
definition which leads to more expressive and sophisticated queries. Taking this into account, a scenario which takes advantage of the potential
combination of these approaches has been sketched (see Figure \ref{fig:scenario}).
\section{First steps towards the integration of the \acs{wot} and \acs{tsc}}
% 2. Comparativa
\label{sec:integration_wot_tsc}

% TODO rehacer taaanto :-)


% existe gente que ha dicho que \ac{tsc} encaja con el pto de vista REST


% remarcar que en los primeros intentos se intentó esto, y luego lo otro

\subsection{Using \acs{wot} in a \acs{tsc} solution}
\label{sec:wotints}
To demonstrate the complete compatibility between \ac{tsc} and \ac{wot} approaches, we first used a \ac{wot} solution in a \ac{tsc} node. We used a gateway \cite{guinard_resource_2010} which provides access to sensors and actuators through RESTful services. To adapt it to the \ac{tsc} paradigm, we added to it a new data representation using a set of semantic triples (in this first approach, the solution is dependent on the ontology of the scenario).

In OtsoPack, each node is mainly made up of two parts: the network layer and the data access layer. While the first one has the responsibility of
keep a node communicated with the rest of the nodes of the space, the second one stores the triples managed by this node.
In OtsoSE we have replaced this data access layer to obtain the semantic information from the gateway instead of from a semantic repository.
Doing so, the \ac{tsc} primitives addressed to the gateway are translated into an HTTP request as summarized in Table \ref{tab:TS2WoT}.
\begin{table*}[t!] % http://en.wikibooks.org/wiki/LaTeX/Floats,_Figures_and_Captions#Wide_figures_in_two_column_documents
\centering
\caption {Mappings between OtsoPack's primitives and HTTP requests addressed to a \ac{wot} solution.}
\begin{tabular}{|c|p{10cm}|}
\hline
\acs{tsc} primitive & HTTP request \\
\hline \hline
read(spaceURI,[graphURI]) & HTTP GET over [graphURI] \\
\hline
read(spaceURI,[template]) & HTTP GET over http://gateway/read?template={''[template]``} \\
\hline
query(spaceURI,[template]) & HTTP GET over http://gateway/query?template={''[template]``}\\
\hline
write(spaceURI,[template]) & HTTP PUT over http://gateway/sunspots/[SpotName]/leds/led[0-6]

Parameters:
\begin{itemize}
  \item switch=[true/false]
  \item redColor=[0-255]
  \item blueColor=[0-255]
  \item greenColor=[0-255]
\end{itemize}

\\
\hline
\end{tabular}
\label{tab:TS2WoT}
\end{table*}

The Read primitive has two different types of implementations. In the most basic one, the graph is identified by a URI which in this case
coincides with the URL of the service that returns it (for example, in the case of temperature sensor
http://node/sunspots/\textit{SpotName}/sensors/temperature/). In the second implementation, the template is passed in as a query string parameter for
the GET command issued to a specific URL and the gateway checks all the graphs to facilitate a response. The Query operation has a similar behaviour.
The Write parses the contents of the triples extracting values and makes a POST request with them over a particular actuation service URL to change
its state.


% CONCLUSION: uso de gateways (a veces útil, pero no nuestro objetivo)


% mapping
% decir que ya hay muchos que intentaron esto antes: TSC?
\subsection{Making \acs{tsc} nodes part of the \acs{wot}}

\begin{table*}[t!] % http://en.wikibooks.org/wiki/LaTeX/Floats,_Figures_and_Captions#Wide_figures_in_two_column_documents
\centering
\caption {Examples of REST access to \ac{tsc} (\textit{sp:ex} is a space URI, \textit{sp:gr1} is a graph URI and templates are expressed between quotes)}
\begin{tabular}{|c|l|l|}
\hline
HTTP request & URL & Returns \\
\hline \hline
GET & http://nodeuri/prefixes & The list of prefixes used by the node \\
GET & http://nodeuri/prefixes/sp & The URI that ``sp'' prefix represents \\
POST & http://nodeuri/prefixes & Add a new prefix \\
 & \hspace{0.5cm}(parameters: \textit{URI} \& \textit{prefix name}) & \\
GET & http://nodeuri/sp:ex/query/any & \texttt{query(sp:ex,"?s ?p ?o ."): triples} \\
POST & http://nodeuri/sp:ex/graphs & \texttt{write(sp:ex,triples): URI} \\
 & \hspace{0.5cm}(parameter: \textit{triples}) & \\
GET & http://nodeuri/sp:ex/graphs & The list of graphs stored in this node \\
GET & http://nodeuri/sp:ex/graphs/sp:gr1 & \texttt{read(sp:ex,sp:gr1): triples} \\
GET & http://nodeuri/sp:ex/graphs/subject/sp:s1 & \texttt{read(sp:ex,"<sp:s1> ?p ?o ."): triples} \\
DELETE & http://nodeuri/sp:ex/graphs/sp:gr1 & \texttt{take(sp:ex,sp:gr1): triples} \\
DELETE & http://nodeuri/sp:ex/graphs/object/sp:o1 & \texttt{take(sp:ex,"?s ?p <sp:o1> ."): triples} \\
%GET & http://nodeuri/spaceuri/ontologies & The list of base ontologies used by the node \\
%GET & http://nodeuri/spaceuri/ontologies/rdfs & The triples which define RDF-Schema. \\
\hline
\end{tabular}
\label{tab:WoT2TS}
\end{table*}

In this section, a proposal to make any \ac{tsc} node \ac{wot} compliant is explained. To do that access to a \ac{tsc} should be provided through a RESTful service
(as shown in Table \ref{tab:WoT2TS}). As in \ac{tsc}, spaces, graphs, subjects, predicates and objects are identified by URIs, in order not to make the
requesting URLs too long, each node should provide a prefix mechanism to enable the URI shortening at \textit{http://nodeuri/prefixes/}.
This node will return all the prefixes used by this node, so they can be used inside any URL by simply using a name followed by '':`` and the last part of the URI.

To see the graphs available in a concrete node, \linebreak \textit{http://nodeuri/spaceuri/[graphs]} could be accessed. To access each graph
in a space, no matter if it is stored by the node responding to the HTTP request or not, we could access \textit{http://nodeuri/spaceuri/graphs/[graphuri]}.
It will be internally translated into a \primread primitive. To locate a graph giving a \textit{template} the accessed URL will be \linebreak
\textit{http://nodeuri/spaceuri/graphs/[template]}. The HTTP \linebreak DELETE verb should return the graph and delete it from the node where it was stored.

We propose specifying first a subject \textit{subject/[subj-uri]/} and concatenate \textit{/predicate/[pred-uri]/} and \textit{object/[obj-val]}\linebreak if needed to specify a \textit{template}. The order should be this, but any of them could be optional to express that any value could be ok (wildcard option). To express a \textit{?s ?p ?o .}-like template (any triple matches it), the URL ended by \textit{any} could be used.

To perform a \primquery \textit{http://node/spaceuri/query/[template]}-like URL should be accessed and to write a new graph the user should make a HTTP POST request to \linebreak \textit{http://nodeuri/spaceuri/graphs/} obtaining the new graph's URI as response.
%Additionally, some shortcuts could be proposed to explore the instance URIs of a given class (i.e. \linebreak 
%\textit{http://nodeuri/spaceuri/[classname]} would redirect to \linebreak \textit{http://nodeuri/spaceuri/query/p/rdf:type/o/[classname]}
%\footnote{For the sake of space ``predicate'' and ``object'' are represented by ``p'' and ``o''.}) \linebreak and an special URL to access to the base
\section{Conclusion}
% 2. Comparativa
\label{sec:tsc_wot_conclusion}

% 5. otra forma de aunar las ventajas de ambos? => construir un TSC compatible con WoT (on top of that) <= integración profunda
%	Ventajas:
%		+ interoperable with other semantic WoT systems
%		+ syntactic services can still be provided using a different type of representation => WoT
%		+ promote indirect communication style => abstraction for the developer (does not care about manually discovering interfaces)
%	Problemas:
%		+ necesitas out-of-the-band conocimiento? API fija?
% esto trae problemas: cap 4 y cap 5


This chapter compares two different resource oriented approaches for the \acl{iot}: \acl{wot} and \acl{tsc}.
\ac{wot} seems to scale in a better way thanks to the underlying HTTP protocol, while \ac{tsc} performs the discovery process among locally available network-connected objects in a seamless way.
The first one is more human oriented and the second one relies on the Semantic Web capabilities to exchange a machine processable data.
Furthermore, the second one is ideal to easily configure intranets of network connected objects whilst the first one can easily bridge those intranets configuring global multi-site IoT ecosystems.

We deem that both approaches can win much from their combination since the weaknesses of one are outweighed by the strengths of the other.
Hence, they can be combined to offer a more scalable, machine and human processable solution that offers better cooperation possibilities among internet-connected objects and thus aid users in their daily activities.
As a simple proof of this hypothesis, a scenario employing \ac{wot} to export each space data to the outer world and \ac{tsc} to enable seamless and automatic configuration of heterogeneous devices on local networks has been presented.
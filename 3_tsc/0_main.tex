
% this file is called up by thesis.tex
% content in this file will be fed into the main document

%: ----------------------- introduction file header -----------------------
\begin{savequote}[50mm]
Historical methodology, as I see it, is a product of common sense applied to circumstances. 
\qauthor{Samuel E. Morison}
\end{savequote}


\chapter{Triple Space Computing for resource constrained devices}
\label{cha:tsc}

% the code below specifies where the figures are stored
\ifpdf
    \graphicspath{{3_overall_methodology/figures/PNG/}{3_overall_methodology/figures/PDF/}{3_overall_methodology/figures/}}
\else
    \graphicspath{{3_overall_methodology/figures/EPS/}{3_overall_methodology/figures/}}
\fi



% ----------------------------------------------------------------------

% 1. Intro (refiriendome descaradamente al estado del arte)
%  	Extensible a otros Ubicomp
In recent years, Internet of Things (IoT) is becoming a reality due to the increasing number of everyday objects containing embedded devices.
The use of these everyday objects together with the rising of mobile computing greatly contributes to the Ubicomp vision.
In the beginning the community put more effort on those devices' connectivity issues, materialized in the spread of technologies such as Zigbee, 6LoWPAN or Bluetooth.
However, nowadays the community is focusing on higher levels of the architecture.
These architectures define models in which the applications are built using the capabilities of these objects.
These models have been thoroughly described in the previous chapter.
In the present chapter we will focus on the comparison of the Web of Things (WoT) and Triple Space Computing (TSC).

On the one hand, the WoT is an approach which fully integrates real-world objects and devices in the web.
This is done by embedding web servers into them and adopting the REST architectural style to provide information on demand.
This style is highly decoupled because it focuses on accessing in a simple way to the data itself and not in complex protocols of communication between objects, making the services reusable and easily understandable by developers.

On the other hand, TSC is a paradigm where Semantic Web techniques are used to define the knowledge which is exchanged using a distributed shared space.
The idea of sharing information through a common space corresponds to the \textit{blackboard model}, used in context aware environments, which makes each process very autonomous from the rest.
The Semantic Web defines the information in a very expressive and machine understandable way (new data can be even inferred), and it has also been widely adopted in context-aware environments.
Thus, TSC can be seen as the conjunction of two well accepted data distribution and modelling standards.

Both WoT and TSC can be considered resource oriented solutions since they put emphasis in giving access to data resources.
However, their differences make also them suitable for different purposes.
% TODO reescribir esto TAAAAANTO
While TSC enables expressive queries, dynamic discovery and non human-mediated cooperation among objects, the WoT adopts the scalable properties of the World Wide Web and it is entirely based on web standards.
In this chapter we explain how to take advantage of both approaches, bringing together the best of both worlds.



% 2. Comparativa
% 3. Ventajas ppales de cada uno
% 4. Integración de WoT con TSC
%	a) uso de gateways (a veces útil, pero no nuestro objetivo)
%	b) interfaces HTTP (como de hecho hacen muchas soluciones)
% 5. otra forma de aunar las ventajas de ambos? => construir un TSC compatible con WoT (on top of that) <= integración profunda
%	esto trae problemas: cap 4 y cap 5